\documentclass[12pt,letterpaper]{article}
\usepackage{amsmath}
\usepackage{amsfonts}
\usepackage{amsthm}
\usepackage{cancel}
\usepackage[margin=1in]{geometry}
\usepackage{titling}

\setlength{\droptitle}{-10ex}

\preauthor{\begin{flushright}\large \lineskip 0.5em}
\postauthor{\par\end{flushright}}
\predate{\begin{flushright}\large}
\postdate{\par\end{flushright}}

\title{MAT 67 Homework 2\vspace{-2ex}}
\author{Hardy Jones\\
        999397426\\
        Professor Bandyopadhyay\vspace{-2ex}}
\date{Fall 2013}

\begin{document}

  \maketitle

  \begin{enumerate}
    \item Let $V$ be a vector space over the field $\mathbb{F}$ \\
    
      Given $a \in \mathbb{F}, \vec{v} \in V, a\vec{v} = \vec{0}$.
      Prove either $a = 0$ or $\vec{v} = \vec{0}$.
      
      \begin{proof}
        Assume $a \neq 0$.
        Then $a^{-1} \in \mathbb{F}$, and $a^{-1}a = 1$ by definition.
        
        \[\vec{v} = 1\vec{v} = (a^{-1}a)\vec{v} = a^{-1}(a\vec{v})\]
        
        But we are given $a\vec{v} = \vec{0}$, so
        
        \[a^{-1}(a\vec{v}) = a^{-1}\vec{0} = \vec{0}\]
        
        Thus, $\vec{v} = \vec{0}$. \\
        
        By contraposition, we have: if $\vec{v} \neq \vec{0}$, then $a = 0$. \\
        
        Thus we have $a = 0$ or $\vec{v} = \vec{0}$.
      \end{proof}
      
    \item Show that the set $V = \{(x_1, x_2. x_3) \in \mathbb{F} : x_1 + 2x_2 + 2x_3 = 0\}$
    is a vector space. \\
    
       We need to show that the 10 axioms hold. \\
       
       Given $\vec{u}, \vec{v}, \vec{w} \in V; a, b \in \mathbb{F}$
       where $\vec{u} = (u_1, u_2, u_3). \vec{v} = (v_1, v_2, v_3). \vec{w} = (w_1, w_2, w_3)$
       
       and the operators: \\
       \[\oplus: V \times V \to V\]
       \[\vec{u} \oplus \vec{v} = (u_1, u_2, u_3) \oplus (v_1, v_2, v_3) = (u_1 + v_1, u_2 + v_2, u_3 + v_3)\]
       \[*: \mathbb{F} \times V \to V\]
       \[a*\vec{u} = a*(u_1, u_2, u_3) = (a \cdot u_1, a \cdot u_2, a \cdot u_3)\]
       
       \begin{enumerate}
         \item Closure over vector addition.
           \begin{align*}
             \vec{u} \oplus \vec{v} &= (u_1, u_2, u_3) \oplus (v_1, v_2, v_3) \\
             &= (u_1 + v_1, u_2 + v_2, u_3 + v_3)
           \end{align*}
           
           Now we need to show that our equation is still valid.
           
           \begin{align*}
             (u_1 + v_1, u_2 + v_2, u_3 + v_3) &= (u_1 + v_1) + 2(u_2 + v_2) + 2(u_3 + v_3) \\
             &= u_1 + v_1 + 2u_2 + 2v_2 + 2u_3 + 2v_3) \\
             &= (u_1 + 2u_2 + 2u_3) + (v_1 + 2v_2 + 2v_3) \\
             &= 0 + 0 \\
             &= 0
           \end{align*}
           
           So, $\vec{u} \oplus \vec{v} \in V$.
           Thus, $V$ is closed over vector addition.
           
         \item Associativity over vector addition.
           \begin{align*}
             \vec{u} \oplus (\vec{v} \oplus \vec{w}) &= (u_1, u_2, u_3) \oplus ((v_1, v_2, v_3) \oplus (w_1, w_2, w_3)) \\
             &= (u_1, u_2, u_3) \oplus (v_1 + w_1, v_2 + w_2, v_3 + w_3) \\
             &= (u_1 + (v_1 + w_1), u_2 + (v_2 + w_2), u_3 + (v_3 + w_3)) \\
             &= ((u_1 + v_1) + w_1, (u_2 + v_2) + w_2, (u_3 + v_3) + w_3) \\
             &= (u_1 + v_1, u_2 + v_2, u_3 + v_3) \oplus (w_1, w_2, w_3) \\
             &= ((u_1, u_2, u_3) \oplus (v_1, v_2, v_3)) \oplus (w_1, w_2, w_3) \\
             &= (\vec{u} \oplus \vec{v}) \oplus \vec{w}
           \end{align*}
           
           Thus, association holds over vector addition.
           
         \item Commutativity over vector addition.
           \begin{align*}
             \vec{u} \oplus \vec{v} &= (u_1, u_2, u_3) \oplus (v_1, v_2, v_3) \\
             &= (u_1 + v_1, u_2 + v_2, u_3 + v_3) \\
             &= (v_1 + u_1, v_2 + u_2, v_3 + u_3) \\
             &= (v_1, v_2, v_3) \oplus (u_1, u_2, u_3) \\
             &= \vec{v} \oplus \vec{u}
           \end{align*}
           
           Thus, commutation holds over vector addition.
           
         \item Existence of identity vector.
           Let $\vec{0} = (0, 0, 0) \in V$
           \begin{align*}
             \vec{u} \oplus \vec{0} &= (u_1, u_2, u_3) \oplus (0, 0, 0) \\
             &= (u_1 + 0, u_2 + 0, u_3 + 0) \\
             &= (u_1, u_2, u_3) \\
             &= \vec{u}
           \end{align*}
           
           Thus, $\vec{0} = (0, 0, 0)$ exists as the identity vector.
           
         \item Existence of vector inverse.
           \begin{align*}
             \vec{u} \oplus (-\vec{u}) &= (u_1, u_2, u_3) \oplus (-u_1, -u_2, -u_3) \\
             &= (u_1 + (-u_1), u_2 + (-u_2), u_3 + (-u_3)) \\
             &= (u_1 - u_1, u_2 - u_2, u_3 - u_3) \\
             &= (0, 0, 0) \\
             &= \vec{0}
           \end{align*}
           
           Thus, there exists an inverse for all vectors.
           
         \item Closure over scalar multiplication.
           \begin{align*}
             a * \vec{u} &= a * (u_1, u_2, u_3) \\
             &= (a \cdot u_1, a \cdot u_2, a \cdot u_3)
           \end{align*}
           
           Now we need to ensure that our equation holds.
           
           \begin{align*}
             (a \cdot u_1, a \cdot u_2, a \cdot u_3) &= a \cdot u_1 + 2(a \cdot u_2) + 2(a \cdot u_3) \\
             &= a (u_1 + 2u_2 + 2u_3) \\
             &= a \cdot 0 \\
             &= 0
           \end{align*}
           
           So, $a * \vec{u} \in V$.
           Thus, $V$ is closed over scalar multiplication.
           
         \item Existence of scalar multiplication identity.  
           \begin{align*}
             1 * \vec{u} &= 1 * (u_1, u_2, u_3) \\
             &= (1 \cdot u_1, 1 \cdot u_2, 1 \cdot u_3) \\
             &= (u_1, u_2, u_3) \\
             &= \vec{u}
           \end{align*}
           
           Thus, $1$ exists as the scalar multiplication identity.
           
         \item Associativity over scalar multiplication.  
           \begin{align*}
             a * (b * \vec{u}) &= a * (b * (u_1, u_2, u_3)) \\
             &= a * (b \cdot u_1, b \cdot u_2, b \cdot u_3) \\
             &= (a (b \cdot u_1), a (b \cdot u_2), a (b \cdot u_3)) \\
             &= ((a b) \cdot u_1, (a b) \cdot u_2, (a b) \cdot u_3) \\
             &= (a b) * (u_1, u_2, u_3) \\
             &= (a b) * \vec{u}
           \end{align*}
           
           Thus, associativity holds over scalar multiplication.
           
         \item Distributivity of scalar multiplication over vector addition.
           \begin{align*}
             a * (\vec{u} \oplus \vec{v}) &= a * ((u_1, u_2, u_3) \oplus (v_1, v_2, v_3)) \\
             &= a * (u_1 + v_1, u_2 + v_2, u_3 + v_3) \\
             &= (a (u_1 + v_1), a (u_2 + v_2), a (u_3 + v_3)) \\
             &= (a \cdot u_1 + a \cdot v_1, a \cdot u_2 + a \cdot v_2, a \cdot u_3 + a \cdot v_3) \\
             &= (a \cdot u_1, a \cdot u_2, a \cdot u_3) \oplus (a \cdot v_1, a \cdot v_2, a \cdot v_3) \\
             &= a * (u_1, u_2, u_3) \oplus a * (v_1, v_2, v_3) \\
             &= a * \vec{u} \oplus a * \vec{v}
           \end{align*}  
           
           Thus, distributivity of scalar multiplication over vector addition holds.
           
         \item Distributivity of scalar addition over scalar multiplication.
           \begin{align*}
             (a + b) * \vec{u} &= (a + b) * (u_1, u_2, u_3) \\
             &= ((a + b) u_1, (a + b) u_2, (a + b) u_3) \\
             &= (a \cdot u_1 + b \cdot u_1, a \cdot u_2 + b \cdot u_2, a \cdot u_3 + b \cdot u_3) \\
             &= (a \cdot u_1, a \cdot u_2, a \cdot u_3) \oplus (b \cdot u_1, b \cdot u_2, b \cdot u_3) \\
             &= a * (u_1, u_2, u_3) \oplus b * (u_1, u_2, u_3) \\
             &= a * \vec{u} \oplus b * \vec{u}
           \end{align*}
           
           Thus, distributivity of scalar addition over scalar multiplication holds.
       \end{enumerate}
       
       So we conclude, since all 10 of the axioms hold, that $V$ is a vector space.
       
  \end{enumerate}
\end{document}
