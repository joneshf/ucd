\documentclass[12pt,letterpaper]{article}
\usepackage{amsmath}
\usepackage{amsfonts}
\usepackage{amsthm}
\usepackage{cancel}
\usepackage[margin=1in]{geometry}
\usepackage{titling}

\setlength{\droptitle}{-10ex}

\preauthor{\begin{flushright}\large \lineskip 0.5em}
\postauthor{\par\end{flushright}}
\predate{\begin{flushright}\large}
\postdate{\par\end{flushright}}

\title{MAT 67 Homework 8\vspace{-2ex}}
\author{Hardy Jones\\
        999397426\\
        Professor Bandyopadhyay\vspace{-2ex}}
\date{Fall 2013}

\begin{document}

  \maketitle

  \begin{enumerate}
    \item
      Let $V$ and $W$ be vector spaces over $\mathbb{F}$
      and suppose that $T \in \mathcal{L}(V,W)$ is injective.

      Given a linearly independent list $(v_1, \dots, v_n)$ of vectors in $V$,
      prove that the list $(T(v_1), \dots, T(v_n))$ is linearly independent in $W$.

      \begin{proof}
        Suppose $a_1, a_2, \dots, a_n \in \mathbb{F}$ and
        $a_1T(v_1) + a_2T(v_2) + \dots + a_nT(v_n) = 0$.

        Now since $T$ is a linear map,
        \begin{align*}
          0 &= a_1T(v_1) + a_2T(v_2) + \dots + a_nT(v_n)
          &= T(a_1v_1 + a_2v_2 + \dots + a_nv_n)
        \end{align*}

        And since $T$ is injective we have that there is one vector, namely $0$, in its kernel by proposition 6.2.6.

        That is:
        \[a_1v_1 + a_2v_2 + \dots + a_nv_n = 0\]

        Now since $(v_1, \dots, v_n)$ is linearly independent,
        $a_1 = a_2 = \dots = a_n = 0$.

        From this, we see that $a_1T(v_1) + a_2T(v_2) + \dots + a_nT(v_n) = 0$
        must be linearly independent.
      \end{proof}

    \item
      Let $V$ and $W$ be vector spaces over $\mathbb{F}$
      and suppose that $T \in \mathcal{L}(V,W)$ is surjective.

      Given a spanning list $(v_1, \dots, v_n)$ for $V$,
      prove that span$(T(v_1), \dots, T(v_n)) = W$.

      \begin{proof}
        Suppose $\exists w \in W$, then since $T$ is surjective,
        we have a $v \in V$ such that $T(v) = w$.

        Since $(v_1, \dots, v_n)$ spans $V$, we can make a linear combination for v
        \[a_1v_1 + a_2v_2 + \dots + a_nv_n = v, \text{ where } a_1, a_2, \dots, a_n \in \mathbb{F}\]

        Now, since $T$ is a linear map,
        \begin{align*}
          w &= T(v) \\
          &= T(a_1v_1 + a_2v_2 + \dots + a_nv_n) \\
          &= T(a_1v_1) + T(a_2v_2) + \dots + T(a_nv_n) \\
          &= a_1T(v_1) + a_2T(v_2) + \dots + a_nT(v_n)
        \end{align*}

        Since our choice for $w$ was arbitrary, we can find all $w \in W$ this way.

        So we have that $\forall w \in W, w \in \text{ span }(T(V))$.

        Or put another way, span$(T(v_1), \dots, T(v_n)) = W$
      \end{proof}
  \end{enumerate}

\end{document}
