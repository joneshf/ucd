\documentclass[12pt,letterpaper]{article}
\usepackage{amsmath}
\usepackage{amsfonts}
\usepackage{amsthm}
\usepackage{cancel}
\usepackage[margin=1in]{geometry}
\usepackage{titling}

\setlength{\droptitle}{-10ex}

\preauthor{\begin{flushright}\large \lineskip 0.5em}
\postauthor{\par\end{flushright}}
\predate{\begin{flushright}\large}
\postdate{\par\end{flushright}}

\title{MAT 67 Homework 5\vspace{-2ex}}
\author{Hardy Jones\\
        999397426\\
        Professor Bandyopadhyay\vspace{-2ex}}
\date{Fall 2013}

\begin{document}

  \maketitle

  \begin{enumerate}
    \item
      Let $V$ be a vector space over $\mathbb{F}$,
      and suppose that the list $(v_1, v_2, ..., v_n)$ of vectors spans $V$,
      where each $v_i \in V$.
      Prove that the list 
      \[(v_1 - v_2, v_2 - v_3, v_3 - v_4, ..., v_{n-2} - v_{n-1}, v_{n-1} - v_n, v_n)\]
      also spans $V$.

      \begin{proof}
        Each $v_j \in (v_1, v_2, ..., v_n)$ can be constructed from our new list.

        \begin{align*}
          v_1 &= (v_1 - v_2) + (v_2 - v_3) + (v_3 - v_4) + ... + (v_{n-2} - v_{n-1}) + (v_{n-1} - v_n) + v_n \\
          &= (v_1 - \cancel{v_2}) + (\cancel{v_2} - \cancel{v_3}) + (\cancel{v_3} - \cancel{v_4}) + ... + (\cancel{v_{n-2}} - \cancel{v_{n-1}}) + (\cancel{v_{n-1}} - \cancel{v_n}) + \cancel{v_n} \\
          &= v_! \\
          v_2 &= 0(v_1 - v_2) + (v_2 - v_3) + (v_3 - v_4) + ... + (v_{n-2} - v_{n-1}) + (v_{n-1} - v_n) + v_n \\
          &= (v_2 - \cancel{v_3}) + (\cancel{v_3} - \cancel{v_4}) + ... + (\cancel{v_{n-2}} - \cancel{v_{n-1}}) + (\cancel{v_{n-1}} - \cancel{v_n}) + \cancel{v_n} \\
          &= v_2 \\
          &\vdots \\
          v_n &= 0(v_1 - v_2) + 0(v_2 - v_3) + 0(v_3 - v_4) + ... + 0(v_{n-2} - v_{n-1}) + 0(v_{n-1} - v_n) + v_n \\
          &= v_n
        \end{align*}

        Since we see that we can generate each one of these,
        we can generate the entire list $(v_!, v_2, ..., v_n)$, which spans $V$.
        So, $span(v_1 - v_2, v_2 - v_3, ..., v_{n-1}- v_n, v_n) = V$
      \end{proof}

    \pagebreak

    \item
      Let $V$ be a finite-dimensional vector space over $\mathbb{F}$
      with $dim(V) = n$ for some $n \in \mathbb{Z}^+$.
      Prove that there are $n$ one-dimensional subspaces $U_1, U_2, ..., U_n$ of $V$ such that
      \[V = U_1 \oplus U_2 \oplus \cdots \oplus U_n\].

      \begin{proof}
        Since $V$ is a finite-dimensional vector space of dimension $n$,
        it has some basis $(v_1, v_2, ..., v_n)$.

        Let each $U_i$ be a one-dimensional subspace of $V$, where $i \in \mathbb{N}, 1 \le i \le n$, such that

        \begin{align*}
          U_1 &= \{v_1\} \\
          U_2 &= \{v_2\} \\
          &\vdots \\
          U_n &= \{v_n\}
        \end{align*}

        It is easy to see that each of these subspaces are also vector spaces.

        Since,
          \begin{enumerate}
            \item The zero vector exists in all of them: $0(v_i) = 0 \in U_i$
            \item They are closed under vector addition: $v_i + v_i = 2v_i \in U_i$
            \item They are closed under scalar multiplication: $cv_i \in U_i $
          \end{enumerate}

        Now, we can create any $v \in V$ by taking unique linear combinations of $v_1 + v_2 + ... + v_n$ with $v_1 \in U_1, v_2 \in U_2, ..., v_n \in U_n,$.

        First, we show that $v_i$ exists.

        \begin{align*}
          v_1 &= v_1 + 0v_2 + 0v_3 + ... + 0v_n = v_1 \\
          v_2 &= 0v_1 + v_2 + 0v_3 + ... + 0v_n = v_2 \\
          v_3 &= 0v_1 + 0v_2 + v_3 + ... + 0v_n = v_3 \\
          & \vdots \\
          v_n &= 0v_1 + 0v_2 + 0v_3 + ... + v_n = v_n \\
        \end{align*}

        Now, we show that $v_i$ is unique.

        Without loss of generality we examine $v_1$

        \[v_1 = a_1v_1 + a_2v_2 + ... + a_nv_n = b_1v_1 + b_2v_2 + ... + b_nv_n\]
        \[\forall a_j, b_j \in \mathbb{F}, j \in \mathbb{N}, i \le j \le n\]
        
        Since $v_1$ comes from the basis of $V$, it is part of a linearly independent set of vectors.
        This means that $v_1$ does not have any components of the other vectors.

        Symbolically,
        \begin{align*}
          a_1v_1 + 0v_2 + 0v_3 + ... + 0v_n &= b_1v_1 + 0v_2 + 0v_3 + ... + 0v_n \\
          a_1v_1 &= b_1v_1 \\
          a_1v_1 - b_1v_1 &= 0 \\
          (a_1 - b_1)v_1 &= 0
        \end{align*}

        Now, since we know that $v_1$ is a basis for $U_1$ and part of the basis for $V$,
        we know that $v_1 \neq 0$, so we must have:

        \begin{align*}
          (a_1 - b_1)v_1 &= 0 \\
          a_1 - b_1 &= 0 \\
          a_1 &= b_1 \\
        \end{align*}

        And so, there is only one unique way to create $v_1$.

        Through similar reasoning, we can show that each $v_i$ is unique.

        Thus, since each $v_i \in V$ can be uniquely represented as $v_1 + v_2 + ... + v_n$,
        where $v_1 \in U_i, v_2 \in U_2, ..., v_n \in U_n$,

        \[V = U_1 \oplus U_2 \oplus \cdots \oplus U_n\]

      \end{proof}

    \item
      Let $\mathbb{F}_m[z]$ denote the vector space of all polynomials with degree
      $\le m \in \mathbb{Z}^+$ and having coefficient over $\mathbb{F}$,
      and suppose that $p_0, p_1, ..., p_m \in \mathbb{F}_m[z]$ satisfy
      $p_j(2) = 0$.
      
      Prove that $(p_0, p_1, ..., p_m)$
      is a linearly dependent list of vectors in $\mathbb{F}_m[z]$.

      \begin{proof}
        It suffices to show $a_0p_0 + a_1p_1 + ... + a_mp_m = 0$ has at least one non-trivial solution.

        Choose $a_0 = 1$, $a_1 = ... = a_m = 0$.
        Then we have:
        \[p_0 + 0p_1 + ... + 0p_m = p_0\]

        Evaluating this at $z = 2$ we get:
        \[p_0(2) = 0\]

        So this list of vectors is linearly dependent.
      \end{proof}
  \end{enumerate}
\end{document}
