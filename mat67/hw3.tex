\documentclass[12pt,letterpaper]{article}
\usepackage{amsmath}
\usepackage{amsfonts}
\usepackage{amsthm}
\usepackage{cancel}
\usepackage[margin=1in]{geometry}
\usepackage{titling}

\setlength{\droptitle}{-10ex}

\preauthor{\begin{flushright}\large \lineskip 0.5em}
\postauthor{\par\end{flushright}}
\predate{\begin{flushright}\large}
\postdate{\par\end{flushright}}

\title{MAT 67 Homework 3\vspace{-2ex}}
\author{Hardy Jones\\
        999397426\\
        Professor Bandyopadhyay\vspace{-2ex}}
\date{Fall 2013}

\begin{document}

  \maketitle

  \begin{enumerate}
    \item
      Let $V$ be the set of all pairs $(x,y)$ of real numbers and suppose vector addition and scalar multiplication are defined in the following way:

      \begin{align*}
        (x_1,y_1) + (x_2,y_2) &= (x_1 + x_2, y_1 + y_2) \\
        a(x,y) &= (ax,y)
      \end{align*}

      for any scalar $a$ in the field of real numbers.

      Is the set $V$ a vector space over the field $\mathbb{R}$?

      The set $V$ is not a vector space over the field $\mathbb{R}$ for it fails to hold for distributivity of scalar addition over scalar multiplication.

      \begin{proof}

        Let $\vec{u} \in V$ and $ c, k \in \mathbb{R}$.  Let's check distributivity of scalar addition over scalar multiplication.  We should have $(c + k)\vec{u} = c\vec{u} + k\vec{u}$.

        \begin{align*}
          (c + k)\vec{u} &= ((c + k)u_1, u_2) \\
          &= (cu_1 + ku_1, u_2)
        \end{align*}
        \begin{align*}
          c\vec{u} + k\vec{u} &= (cu_1, u_2) + (ku_1, u_2) \\
          &= (cu_1 + ku_1, u_2 + u_2) \\
          &= (cu_1 + ku_1, 2u_2)
        \end{align*}

        But $(cu_1 + ku_1, u_2) \neq (cu_1 + ku_1, 2u_2)$,
        so $V$ does not hold for distributivity of scalar addition over scalar multiplication.

        Thus $V$ is not a vector space.
      \end{proof}

    \item
      Let $W_1$ and $W_2$ be subspaces of a vector space $V$ such that their union $W_1 \cup W_2$ is also a subspace of $V$.

      Prove that either $W_1$ is contained in $W_2$ or vice versa.

      \begin{proof}
        Without loss of generality, we arbitrarily choose to examine $W_1$ against $W_2$.

        $\forall w_1 \in W_1$ we assume $w_1 \notin W_2$ and so $W_1 \not\subseteq W2$.

        Since $W_1$ is a subspace, $-w_1 \in W_1$.

        Now since $W_1 \cup W_2$ is also a subspace, we should be able to take arbitrary vectors from this union and perform vector addition with them.
        Meaning, we can take one vector from $W_1$ and add it to another vector from $W_2$, for instance.

        So, $\forall w_2 \in W_2$ we have $w_1 + w_2 \in W_1 \cup W_2$.
        And since $W_2$ is a subspace, $-w_2 \in W_2$

        By the definition of union we must have one of the two:
        \begin{enumerate}
          \item $w_1 + w_2 \in W_1$
          \item $w_1 + w_2 \in W_2$
        \end{enumerate}

        Let's look at (b).

        If $w_1 + w_2 \in W_2$, then we have $(w_1 + w_2) + (-w_2) \in W_2$.
        Which means $w_1 \in W_2$; but we assumed that $w_1 \notin W_2$.
        This contradicts our assumption.
        So $\forall w_1 \in W_1, w_1 \in W_2$ and thus $W_1 \subseteq W_2$.
      \end{proof}
  \end{enumerate}
\end{document}
