\documentclass[12pt,letterpaper]{article}
\usepackage{amsmath}
\usepackage{amsfonts}
\usepackage{amsthm}
\usepackage{cancel}
\usepackage[margin=1in]{geometry}
\usepackage{titling}

\setlength{\droptitle}{-10ex}

\preauthor{\begin{flushright}\large \lineskip 0.5em}
\postauthor{\par\end{flushright}}
\predate{\begin{flushright}\large}
\postdate{\par\end{flushright}}

\title{MAT 67 Homework 6\vspace{-2ex}}
\author{Hardy Jones\\
        999397426\\
        Professor Bandyopadhyay\vspace{-2ex}}
\date{Fall 2013}

\begin{document}

  \maketitle

  We are asked to determine if there is a
  \[T: \mathbb{R}^2 \rightarrow \mathbb{R}^2\]
  such that $T\alpha_1 = \beta_1$, $T\alpha_2 = \beta_2$, and $T\alpha_3 = \beta_3$.

  \begin{proof}
    We can use theorem 6.1.3 to help in our proof.
    So we need to show two things:

    \begin{enumerate}
      \item $(\alpha_1, \alpha_2, \alpha_3)$ form a basis for $\mathbb{R}^2$
      \item $(\beta_1, \beta_2, \beta_3)$ is a list of vectors in $\mathbb{R}^2$
    \end{enumerate}

    \begin{enumerate}
      \item
        In order to show that $(\alpha_1, \alpha_2, \alpha_3)$ is a basis for $\mathbb{R}^2$,
        we need to show that $(\alpha_1, \alpha_2, \alpha_3)$ is linearly independent
        and $\mathbb{R}^2$ = span$(\alpha_1, \alpha_2, \alpha_3)$.

        By inspection, we can see that $(\alpha_1, \alpha_2, \alpha_3)$ is linearly dependent,
        since $\alpha_1 = -\alpha_2 - \alpha_3$, so we're better off using the basis reduction theorem.

        Let's start by showing that span$(\alpha_1, \alpha_2, \alpha_3)$ = $\mathbb{R}^2$.

        We simply need to show that
        \[\forall v \in \mathbb{R}^2; x, y, a_1, a_2, a_3 \in \mathbb{R}; v = (x, y) = a_1\alpha_1 + a_2\alpha_2 + a_3\alpha_3\]

        \begin{align*}
          v = (x, y) &= a_1(1, -1) + a_2(2, -1) + a_3(-3, 2) \\
          &= (a_1 + 2a_2 -3a_3, -a_1 - a_2 + 2a_3)
        \end{align*}

        This is equivalent to solving this system of equations.

        \begin{align*}
          a_1 & + & 2a_2 & - & 3a_3 & = x \\
          -a_1 & - & a_2 & + & 2a_3 & = y
        \end{align*}

        This reduces to:

        \begin{align*}
          a_1 & + & 0a_2 & - & a_3 = x \\
          0a_1 & + & a_2 & - & a_3 = y
        \end{align*}

        So, $v = (a_1\alpha_1 - a_3\alpha_3, a_2\alpha_2 - a_3\alpha_3)$

        So we can represent any vector in $\mathbb{R^2}$ as a linear combination
        of $(\alpha_1, \alpha_2, \alpha_3)$.
        Thus, span$(\alpha_1, \alpha_2, \alpha_3)$ = $\mathbb{R}^2$

        Now, we start our recursive reduction.

        $\alpha_1 \neq 0$, so we leave it alone, and our list is still $(\alpha_1, \alpha_2, \alpha_3)$
        $\alpha_2 \neq a_1\alpha_1$, so leave it in our list, which is still $(\alpha_1, \alpha_2, \alpha_3)$
        $\alpha_3 = -\alpha_1 - \alpha_2$, so remove it and now our list is $(\alpha_1, \alpha_2)$

        So, we're done, we now have a linearly independent list of vectors which span $\mathbb{R}^2$.
        Thus, we have our basis for $\mathbb{R}^2$, namely $(\alpha_1, \alpha_2)$

      \item
        It is trivial to show that $(\beta_1, \beta_2, \beta_3)$ is a list of vectors in $\mathbb{R}^2$

    \end{enumerate}

    From both of these, we can finally use theorem 6.1.3 to show that there exists some
    \[T: \mathbb{R}^2 \rightarrow \mathbb{R}^2\]
    such that $T\alpha_1 = \beta_1$, $T\alpha_2 = \beta_2$, and $T\alpha_3 = \beta_3$.
  \end{proof}

\end{document}
