\documentclass[12pt,letterpaper]{article}
\usepackage{amsmath}
\usepackage{amsfonts}
\usepackage{amsthm}
\usepackage{cancel}
\usepackage[margin=1in]{geometry}
\usepackage{titling}
\usepackage{multirow}
\usepackage{amssymb}
\usepackage{graphicx}
\usepackage{tikz}
\usepackage[mathrm,colour,cntbysection]{czt}

\setlength{\droptitle}{-10ex}

\preauthor{\begin{flushright}\large \lineskip 0.5em}
\postauthor{\par\end{flushright}}
\predate{\begin{flushright}\large}
\postdate{\par\end{flushright}}

\title{ECS 160 Homework 2 \vspace{-2ex}}
\author{Hardy Jones\\
        999397426\\
        Professor Levitt\vspace{-2ex}}
\date{Spring 2014}

\begin{document}
  \maketitle

  \begin{enumerate}
    \item
      \begin{enumerate}
        \item
          You cannot say much about two predicates not connected by an arrow.
          The only thing you can say is that neither is strong with relative to the other.
        \item
          The strongest thing that $\neg P$ can imply is $P \implies Q$.
          This is because, when $\neg P$ is True, $P$ is False.
          So, $P \implies Q$ is True.

          \begin{tikzpicture}[node distance=1.5cm]
            \node(false) {$False$};
            \node(pandq)   [below of=false] {$P \land Q$};
            \node(p)      [below left of=pandq]  {$P$};
            \node(q)      [below right of=pandq]  {$Q$};
            \node(qimpp)  [below left of=p]  {$Q \implies P$};
            \node(porq)  [below right of=p]  {$P \lor Q$};
            \node(pimpq)  [below right of=q]  {$P \implies Q$};
            \node(true)  [below of=porq]  {$True$};
            \node(negp)  [above right of=q]  {$\neg P$};
            \path[->]
              (false) edge (pandq)
                      edge [color=red] (negp)
              (pandq) edge (p)
                     edge (q)
              (p) edge (qimpp)
                  edge (porq)
              (q) edge (porq)
                  edge (pimpq)
              (qimpp) edge (true)
              (porq) edge (true)
              (pimpq) edge (true)
              (negp) edge [color=red] (pimpq)
              ;
          \end{tikzpicture}
      \end{enumerate}

    \item
      \begin{align*}
        \mathbb{P}(\mathbb{P}(\{1\})) &= \mathbb{P}(\{\{\}, \{1\}\}) \\
        &= \{\{\}, \{\{\}\}, \{\{1\}\}, \{\{\}, \{1\}\}\}
      \end{align*}
    \item
      \begin{zed}
        [People]
      \end{zed}

      \begin{axdef}
        knows : People \rel People \\
        livesAlone : \power People \\
      \where
        \forall x : People @ (\exists y : People @ x \mapsto y \in knows \land y \in livesAlone )
      \end{axdef}
    \item
      \begin{zed}
        PrimeNumbers == \{n, p : \nat | p > 1 \land 1 < n < p \land (p \mod n \neq 0) @ p\}
      \end{zed}
    \item
      \begin{zed}
        [ Student ]
      \end{zed}
      \begin{axdef}
        size : \nat
      \end{axdef}
      \begin{zed}
      Category ::= c1 | c2 | c3
      \end{zed}

      \begin{schema}{Class}
        enrolled, tested : \power Student
      \where
        \# enrolled \leq size \\
        tested \subseteq enrolled
      \end{schema}

      \begin{schema}{CanBeTested}
        \Xi Class \\
        s? : Student \\
        c? : Category \\
        testedIn : Student \rel Category
      \where
        s? \in enrolled \\
        s? \mapsto c? \notin testedIn \\
        testedIn = testedIn \cup \{s? \mapsto c?\}
      \end{schema}

      \begin{schema}{CanBeDeTested}
        \Xi Class \\
        s? : Student \\
        c? : Category \\
        testedIn : Student \rel Category
      \where
        s? \in enrolled \\
        s? \mapsto c? \in testedIn \\
        testedIn = testedIn \setminus \{s? \mapsto c?\}
      \end{schema}

    \item
      \begin{schema}{LeaveSetOk}
        \Delta Class \\
        students? : \power Student
      \where
        students? \cap enrolled = students? \\
        students? \cap tested = \{\} \\
        enrolled' = enrolled \setminus students?
      \end{schema}

    \item
      \begin{zed}
        [ Char ] \\
        Text == \seq Char \\
      \end{zed}

      \begin{axdef}
        maxsize : \nat \\
        Printing : \power Char
      \end{axdef}

      \begin{schema}{Editor}
        left, right, text : Text
      \where
        text = left \cat right
      \end{schema}

      \begin{schema}{Init}
        left, right, text : Text
      \where
        left = \{\}
        right = \{\}
        text = \{\}
      \end{schema}

      \begin{schema}{Insert}
        \Delta Editor \\
        ch? : Char
      \where
        \#left + \#right < maxsize \\
        left' = \{\#left + 1 \mapsto ch?\} \cup left
      \end{schema}

      \begin{schema}{Forward}
        \Delta Editor \\
        c : Char
      \where
        \#right > 0 \\
        1 \mapsto c \in right \\
        left' = left \cup \{\#left + 1 \mapsto c \} \\
        right' = \{n : \nat; char : Char | n \mapsto char \in right \land n > 1 @ n + 1 \mapsto char\}
      \end{schema}

    \pagebreak

    \item
      \begin{zed}
        \relation(\_ in \_)
      \end{zed}
      \begin{gendef}[X]
        \_ in \_ : (\seq X \rel \seq X)
      \where
        \forall u, v : \seq X \\
        @ (u \ in \ v \iff \\
        \t1 \exists x, y : \seq X | x \cat u \cat y = v)
      \end{gendef}

    \item
      $\mathbb{P}(X)$ has $2^n$ members

    \item
      \begin{zed}
        xs == \{n : \num | n \neq \negate1\} \cup \{1\}
      \end{zed}

    \item
      This set is exactly $\mathbb{Z}$.
      Since for every element of $\mathbb{Z}$,
      we can find two elements of $\mathbb{Z}$ that add to our chosen element.

    \item[13.]
      \begin{zed}
        R2R1 == \{1 \mapsto 3, 2 \mapsto 2, 2 \mapsto 3, 3 \mapsto 1\} \\
        R1R1R1 == \{1 \mapsto 4, 2 \mapsto 2, 2 \mapsto 3, 3 \mapsto 2, 3 \mapsto 3, 4 \mapsto 1\}
      \end{zed}

    \item[14.]
      \begin{zed}
        [CHAR]
      \end{zed}

      \begin{axdef}
        blank : \power CHAR
      \end{axdef}
      \begin{zed}
        TEXT == \seq CHAR \\
        SPACE == \seq blank \\
        WORD == \seq (CHAR \setminus blank)
      \end{zed}
      \begin{axdef}
        words : TEXT \fun \seq WORD
      \where
        \forall s : SPACE; w : WORD; l, r : TEXT \\
        @     words \langle \rangle     = \langle \rangle \\
        \land words \ s                 = \langle \rangle \\
        \land words \ w                 = \langle w \rangle \\
        \land words ( s \cat r )        = words \ r \\
        \land words ( r \cat s )        = words \ r \\
        \land words ( l \cat s \cat r ) = words \ l \cat words \ r
      \end{axdef}
  \end{enumerate}
\end{document}
