\documentclass[12pt,letterpaper]{article}
\usepackage{amsmath}
\usepackage{amsfonts}
\usepackage{amsthm}
\usepackage{cancel}
\usepackage[margin=1in]{geometry}
\usepackage{titling}
\usepackage{multirow}

\setlength{\droptitle}{-10ex}

\preauthor{\begin{flushright}\large \lineskip 0.5em}
\postauthor{\par\end{flushright}}
\predate{\begin{flushright}\large}
\postdate{\par\end{flushright}}

\title{ECS 120 Problem Set 1\vspace{-2ex}}
\author{Hardy Jones\\
        999397426\\
        Professor Rogaway\vspace{-2ex}}
\date{Spring 2014}

\begin{document}
  \maketitle

  \begin{enumerate}
    \item[Problem 1]
      We can gain some insight by looking at the first few $n$-digit palindromes.

      \begin{tabular}{l | l | l}
        \multicolumn{1}{l}{$n$} & \multicolumn{1}{l}{elements} & \multicolumn{1}{l}{$|n|$} \\
        \hline
        1 & 1,2,3,4,5,6,7,8,9 & 9 \\
        2 & 11,22,33,44,55,66,77,88,99 & 9 \\
        \multirow{4}{*}{3} & 101,202,303,404,505,606,707,808,909 & \multirow{4}{*}{90} \\
                           & 111,212,313,414,515,616,717,818,919 & \\
                           & $\vdots$ & \\
                           & 191,292,393,494,595,696,797,898,999 & \\
        \multirow{4}{*}{4} & 1001,2002,3003,4004,5005,6006,7007,8008,9009 & \multirow{4}{*}{90} \\
                           & 1111,2112,3113,4114,5115,6116,7117,8118,9119 & \\
                           & $\vdots$ & \\
                           & 1991,2992,3993,4994,5995,6996,7997,8998,9999 &
      \end{tabular}

      If we continue in this way we see that the number of palindromes increases by 10 for every 2 digits added.

      We can see a recurrence relation here:
      \begin{align*}
        D_1 &= 9 \\
        D_2 &= 9 \\
        D_n &= D_{n-2} \cdot 10 \text{, for n $>$ 2}
      \end{align*}

      From this we can manipulate it to a closed form:
      \[D_n = 9 \cdot 10^{\lfloor \frac{n-1}{2} \rfloor}\]

      We can prove this by induction.

      \begin{proof} \
        Base Cases:
        \begin{enumerate}
          \item[$n = 1$]
            \[D_1 = 9 \cdot 10^{\lfloor \frac{1-1}{2} \rfloor} = 9 \cdot 10^0 = 9\]
          \item[$n = 2$]
            \[D_2 = 9 \cdot 10^{\lfloor \frac{2-1}{2} \rfloor} = 9 \cdot 10^0 = 9\]
        \end{enumerate}

        Inductive Hypothesis:
        \[D_n = 9 \cdot 10^{\lfloor \frac{n-1}{2} \rfloor}\]

        Inductive Case:
        \begin{align*}
          D_{n+1} &= D_{n-1} \cdot 10 \\
          &= 9 \cdot 10^{\lfloor \frac{(n-1)-1}{2} \rfloor} \cdot 10 \\
          &= 9 \cdot 10^{\lfloor \frac{n-2}{2} \rfloor} \cdot 10 \\
          &= 9 \cdot 10^{\lfloor \frac{n}{2}-\frac{2}{2} \rfloor} \cdot 10 \\
          &= 9 \cdot 10^{\lfloor \frac{n}{2}-1 \rfloor} \cdot 10 \\
          &= 9 \cdot 10^{\lfloor \frac{n}{2}-1+1 \rfloor} \\
          &= 9 \cdot 10^{\lfloor \frac{n}{2} \rfloor} \\
        \end{align*}

        Thus, we have proved that our formula is correct.

      \end{proof}

      Now, we can calculate
      $D_{20} = 9 \cdot 10^{\lfloor \frac{20-1}{2} \rfloor} = 9 \cdot 10^9 = 9000000000$

    \item[Problem 2]
      Let's begin by enumerating some of the first few strings in lexographic order.

      \begin{tabular}{l | l}
        \hline
        $w_1$ & $\epsilon$ \\
        $w_2$ & 0 \\
        $w_3$ & 1 \\
        $w_4$ & 00 \\
        $w_5$ & 01 \\
        $w_6$ & 10 \\
        $w_7$ & 11 \\
        $w_8$ & 000
      \end{tabular}

      Interestingly, this looks like the binary representation of each $n$ in $w_n$ without the leading digit.
      In the case of $n = 1$, removing the leading (only) digit leaves the empty string.

      So we have a mapping: $n \in \mathbb{N} \rightarrow w_n$.

      This means that $w_{1234567}$ can be easily computed.
      We just find the binary representation for $1234567_{10}$ and remove the first digit.

      \[1234567_{10} = 100101101011010000111_2\]
      Removing the first digit, we end up with: $00101101011010000111$

    \item[Problem 3]
      \begin{enumerate}
        \item
          We can take some intuition from linear algebra for this.
          If we view each of the strings as a vector,
          concatenation as vector addition,
          and the number of characters in a string as the magnitude of the vector,
          then we can see that there are only two linearly independent vectors in $L$.
          For simplicity, we take $a$ and $b$ to be these two independent vectors.

          The question then becomes,
          how many different ways can we arrange these two vectors so that the resulting vector has magnitude 10?

          Let's enumerate the first few lengths.

          \begin{tabular}{l | l}
            \multicolumn{1}{l}{length} & \multicolumn{1}{l}{elements} \\
            \hline
            0 & $\epsilon$\\
            1 & $a, b$ \\
            2 & $aa, ab, ba, bb$ \\
            3 & $aaa, aab, aba, abb, baa, bab, bba, bbb$
          \end{tabular}

          So it looks like the number of elements is $2^{l}$, where $l$ is the length of the string.

          This means that there are $2^10 = 1024$ strings of length 10 from this language.

        \item
          We can start by enumerating some lengths

          \begin{tabular}{l | l}
            \multicolumn{1}{l}{length} & \multicolumn{1}{l}{elements} \\
            \hline
            0 & $\epsilon$\\
            1 & $a$ \\
            2 & $aa, bb$ \\
            3 & $aaa, abb, bba$ \\
            4 & $aaaa, aabb, abba, bbaa, bbbb$ \\
            5 & $aaaaa, aaabb, aabba, abbaa, abbbb, bbaaa, bbabb, bbbba$
          \end{tabular}

          Here it looks like the number of elements is $fib(l)$,
          where $l$ is the length of the string and $fib(l) = fib(l-1) + fib(l-2)$.

          This means there are $fib(10) = 89$ strings of length 10 from this language.

      \end{enumerate}

    \item[Problem 4]
      We can use the division formula here: $s = qN + r$,

      where $s \in \mathcal{S}, N = 314159265359, q, r \in \mathbb{Z}, q, r \ge 0$.

      Since $\mathcal{S}$ is infinite, we have more elements in $\mathcal{S}$ than there are congruence classes modulo N.
      So, by the pigeon hole principle, at least two elements in $\mathcal{S}$ must have the same remainder when divided by N.

      Using the division formula:

      \[s_1 = q_1N + r \text{ and } s_2 = q_2N + r\]

      If we manipulate and substitute for $r$:

      \begin{align*}
        s_1 &= q_1N + (s_2 - q_2N) \\
        s_1 - s_2 &= q_1N - q_2N \\
        &= (q_1 - q_2)N
      \end{align*}

      Thus, since $q_1 - q_2$ is an integer, $s_1 - s_2$ is a multiple of $N$.

      More succinctly, the difference of two elements of $\mathcal{S}$ is a multiple of $N$.
  \end{enumerate}
\end{document}
