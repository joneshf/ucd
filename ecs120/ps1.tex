\documentclass[12pt,letterpaper]{article}
\usepackage{amsmath}
\usepackage{amsfonts}
\usepackage{amsthm}
\usepackage{cancel}
\usepackage[margin=1in]{geometry}
\usepackage{titling}
\usepackage{multirow}

\setlength{\droptitle}{-10ex}

\preauthor{\begin{flushright}\large \lineskip 0.5em}
\postauthor{\par\end{flushright}}
\predate{\begin{flushright}\large}
\postdate{\par\end{flushright}}

\title{ECS 120 Problem Set 1\vspace{-2ex}}
\author{Hardy Jones\\
        999397426\\
        Professor Rogaway\vspace{-2ex}}
\date{Spring 2014}

\begin{document}
  \maketitle

  \begin{enumerate}
    \item[Problem 1]
      We can gain some insight by looking at the first few $n$-digit palindromes.

      \begin{tabular}{l | l | l}
        \multicolumn{1}{l}{$n$} & \multicolumn{1}{l}{elements} & \multicolumn{1}{l}{$|n|$} \\
        \hline
        1 & 1,2,3,4,5,6,7,8,9 & 9 \\
        2 & 11,22,33,44,55,66,77,88,99 & 9 \\
        \multirow{4}{*}{3} & 101,202,303,404,505,606,707,808,909 & \multirow{4}{*}{90} \\
                           & 111,212,313,414,515,616,717,818,919 & \\
                           & $\vdots$ & \\
                           & 191,292,393,494,595,696,797,898,999 & \\
        \multirow{4}{*}{4} & 1001,2002,3003,4004,5005,6006,7007,8008,9009 & \multirow{4}{*}{90} \\
                           & 1111,2112,3113,4114,5115,6116,7117,8118,9119 & \\
                           & $\vdots$ & \\
                           & 1991,2992,3993,4994,5995,6996,7997,8998,9999 &
      \end{tabular}

      If we continue in this way we see that the number of palindromes increases by 10 for every 2 digits added.

      We can see a recurrence relation here:
      \begin{align*}
        D_1 &= 9 \\
        D_2 &= 9 \\
        D_n &= D_{n-2} \cdot 10 \text{, for n $>$ 2}
      \end{align*}

      From this we can manipulate it to a closed form:
      \[D_n = 9 \cdot 10^{\lfloor \frac{n-1}{2} \rfloor}\]

      We can prove this by induction.

      \begin{proof} \
        Base Cases:
        \begin{enumerate}
          \item[$n = 1$]
            \[D_1 = 9 \cdot 10^{\lfloor \frac{1-1}{2} \rfloor} = 9 \cdot 10^0 = 9\]
          \item[$n = 2$]
            \[D_2 = 9 \cdot 10^{\lfloor \frac{2-1}{2} \rfloor} = 9 \cdot 10^0 = 9\]
        \end{enumerate}

        Inductive Hypothesis:
        \[D_n = 9 \cdot 10^{\lfloor \frac{n-1}{2} \rfloor}\]

        Inductive Case:
        \begin{align*}
          D_{n+1} &= D_{n-1} \cdot 10 \\
          &= 9 \cdot 10^{\lfloor \frac{(n-1)-1}{2} \rfloor} \cdot 10 \\
          &= 9 \cdot 10^{\lfloor \frac{n-2}{2} \rfloor} \cdot 10 \\
          &= 9 \cdot 10^{\lfloor \frac{n}{2}-\frac{2}{2} \rfloor} \cdot 10 \\
          &= 9 \cdot 10^{\lfloor \frac{n}{2}-1 \rfloor} \cdot 10 \\
          &= 9 \cdot 10^{\lfloor \frac{n}{2}-1+1 \rfloor} \\
          &= 9 \cdot 10^{\lfloor \frac{n}{2} \rfloor} \\
        \end{align*}

        Thus, we have proved that our formula is correct.

      \end{proof}

      Now, we can calculate
      $D_{20} = 9 \cdot 10^{\lfloor \frac{20-1}{2} \rfloor} = 9 \cdot 10^9 = 9000000000$

    \item[Problem 2]
      Let's begin by enumerating some of the first few strings in lexographic order.

      \begin{tabular}{l | l}
        \hline
        $w_1$ & $\epsilon$ \\
        $w_2$ & 0 \\
        $w_3$ & 1 \\
        $w_4$ & 00 \\
        $w_5$ & 01 \\
        $w_6$ & 10 \\
        $w_7$ & 11 \\
        $w_8$ & 000
      \end{tabular}

      Interestingly, this looks like the binary representation of each $n$ in $w_n$ without the leading digit.
      In the case of $n = 1$, removing the leading (only) digit leaves the empty string.

      So we have a mapping: $n \in \mathbb{N} \rightarrow w_n$.

      This means that $w_{1234567}$ can be easily computed.
      We just find the binary representation for $1234567_{10}$ and remove the first digit.

      \[1234567_{10} = 100101101011010000111_2\]
      Removing the first digit, we end up with: $00101101011010000111$
  \end{enumerate}
\end{document}
