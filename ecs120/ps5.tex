\documentclass[12pt,letterpaper]{article}
\usepackage{amsmath}
\usepackage{amsfonts}
\usepackage{amsthm}
\usepackage{cancel}
\usepackage[margin=1in]{geometry}
\usepackage{titling}
\usepackage{multirow}
\usepackage{amssymb}
\usepackage{tikz}
\usetikzlibrary{automata,positioning}

\setlength{\droptitle}{-10ex}

\preauthor{\begin{flushright}\large \lineskip 0.5em}
\postauthor{\par\end{flushright}}
\predate{\begin{flushright}\large}
\postdate{\par\end{flushright}}

\title{ECS 120 Problem Set 5\vspace{-2ex}}
\author{Hardy Jones\\
        999397426\\
        Professor Rogaway\vspace{-2ex}}
\date{Spring 2014}

\begin{document}
  \maketitle

  \begin{enumerate}
    \item[Problem 1]
      \begin{enumerate}
        \item
          Yes, this is a regular language.

          Since the distinct number of decimal digits is finite, i.e. less than or equal to $10$, we can construct an NFA to correspond to this.
          We need not know which digits appear infinitely often.
          We just need our NFA to have an arrow for each $d$ in $\{0,1,2,...,9\}$ where $d$ is a decimal digits that occurs infinitely often.

        \item
          No, this is not a regular language.

          Assume for contradiction that $L_b$ is regular.

          Then, there exists some $\rho$ such that,
          for all $s \in L_b, |s| \ge \rho$,
          there exists some $xyz = s, y \ne \varepsilon$ such that,
          for all $i \ge 0, xy^iz \in L_b$.

          Now, this implies that there is some repeating pattern within the decimal representation of $\pi$.
          However, since $\pi$ is irrational, there can be no such repeating pattern.

          From this contraction, we see that $L_b$ is not regular.
      \end{enumerate}
  \end{enumerate}
\end{document}
