\documentclass[12pt,letterpaper]{article}
\usepackage{amsmath}
\usepackage{amsfonts}
\usepackage{amsthm}
\usepackage{cancel}
\usepackage[margin=1in]{geometry}
\usepackage{titling}
\usepackage{multirow}
\usepackage{amssymb}
\usepackage{tikz}
\usetikzlibrary{automata,positioning}

\usepackage{listings}
\lstset{
  basicstyle=\itshape,
  literate={->}{$\rightarrow$}{2}
           {ε}{$\varepsilon$}{1}
           {≠}{$\ne$}{2}
}

\setlength{\droptitle}{-10ex}

\preauthor{\begin{flushright}\large \lineskip 0.5em}
\postauthor{\par\end{flushright}}
\predate{\begin{flushright}\large}
\postdate{\par\end{flushright}}

\title{ECS 120 Problem Set 5\vspace{-2ex}}
\author{Hardy Jones\\
        999397426\\
        Professor Rogaway\vspace{-2ex}}
\date{Spring 2014}

\begin{document}
  \maketitle

  \begin{enumerate}
    \item[Problem 1]
      \begin{enumerate}
        \item
          Yes, this is a regular language.

          Since the distinct number of decimal digits is finite, i.e. less than or equal to $10$, we can construct an NFA to correspond to this.
          We need not know which digits appear infinitely often.
          We just need our NFA to have an arrow for each $d$ in $\{0,1,2,...,9\}$ where $d$ is a decimal digits that occurs infinitely often.

        \item
          No, this is not a regular language.

          Assume for contradiction that $L_b$ is regular.

          Then, there exists some $\rho$ such that,
          for all $s \in L_b, |s| \ge \rho$,
          there exists some $xyz = s, y \ne \varepsilon$ such that,
          for all $i \ge 0, xy^iz \in L_b$.

          Now, this implies that there is some repeating pattern within the decimal representation of $\pi$.
          However, since $\pi$ is irrational, there can be no such repeating pattern.

          From this contraction, we see that $L_b$ is not regular.
      \end{enumerate}

    \item[Problem 3]

      It helps to see some of the strings in the language.

      \[L = \{\varepsilon, a, b, c, aa, ab, ac, ba, bb, bc, ca, cb, cc, aaa, abc, acb, bac, bbb, bca, \dots\}\]

      We see that this language is quite a bit more complex than it appears on the face.

      \begin{lstlisting}
        S -> T | U | V
        T -> aTbT | bTaT | abT | baT | cT | ε
        U -> aUcU | cUaU | acU | caU | bU | ε
        V -> bVcV | cVbV | bcV | cbV | aV | ε
      \end{lstlisting}

      So, we can see that each production has an equal number of two different characters.

    \item[Problem 4]
      It helps to see some of the strings in the language.

      \begin{align*}
        L = \{& 0\ne1, 1\ne0, \\
        & 0\ne00, 0\ne01, 0\ne10, 0\ne11, 1\ne00, 1\ne01, 1\ne10, 1\ne11, \\
        & 00\ne0, 01\ne0, 10\ne0, 11\ne0, 00\ne1, 01\ne1, 10\ne1, 11\ne1, \\
        & 00\ne01, 00\ne10, 00\ne11, 01\ne00, 01\ne10, \dots\}
      \end{align*}

      We can see that we'll have to look at at least two situation.

      One where $|x| = |y|$ and one where $|x| \ne |y|$.

      When $|x| = |y|$, we want to separate $x$ and $y$ into substrings $x = x_0x_1x_2$ and $y = y_0y_1y_2$ such that $|x_0| = |y_0|, |x_2| = |y_2|, x_1 \ne \varepsilon, y_1 \ne \epsilon, x_1 \ne y_1$.

      We can easily construct a CFG for this.
      Let's start with each side.

      \begin{lstlisting}
        X -> BXB | 1
        Y -> BYB | 0
        B -> 0 | 1
      \end{lstlisting}

      Now, we can connect both sides.

      \begin{lstlisting}
        E -> X≠Y | Y≠X
      \end{lstlisting}

      For the other case, where $|x| \ne |y|$ we don't care which digits are in which position, just that one side has more digits than the other.

      Let's start with each side.

      \begin{lstlisting}
        T -> BTB | U
        B -> 0 | 1
        U -> 0≠0 | 0≠1 | 1≠0 | 1≠1
      \end{lstlisting}

      Now, connect both sides, ensure that one has at least one more letter than the other.

      \begin{lstlisting}
        N -> VT | TV
        V -> 0V | 1V
      \end{lstlisting}

      All together we end up with the following CFG.

      \begin{lstlisting}
        S -> E | N
        E -> X≠Y | Y≠X
        N -> TV | VT
        X -> BXB | 1
        Y -> BYB | 0
        T -> BTB | U
        U -> 0≠0 | 0≠1 | 1≠0 | 1≠1
        V -> 0V | 1V
        B -> 0 | 1
      \end{lstlisting}

  \end{enumerate}
\end{document}
