\documentclass[12pt,letterpaper]{article}
\usepackage{amsmath}
\usepackage{amsfonts}
\usepackage{amsthm}
\usepackage{cancel}
\usepackage[margin=1in]{geometry}
\usepackage{titling}
\usepackage{multirow}
\usepackage{amssymb}
\usepackage{tikz}
\usetikzlibrary{automata,positioning}

\setlength{\droptitle}{-10ex}

\preauthor{\begin{flushright}\large \lineskip 0.5em}
\postauthor{\par\end{flushright}}
\predate{\begin{flushright}\large}
\postdate{\par\end{flushright}}

\title{ECS 120 Problem Set 3\vspace{-2ex}}
\author{Hardy Jones\\
        999397426\\
        Professor Rogaway\vspace{-2ex}}
\date{Spring 2014}

\begin{document}
  \maketitle

  \begin{enumerate}
    \item[Problem 1]
      \begin{tikzpicture}[shorten >=1pt,node distance=2cm,on grid,auto]
         \node[state,initial,accepting] (0)   {$0$};
         \node[state,accepting] (1) [right=of 0] {$1$};
         \node[state] (2) [right=of 1] {$2$};
          \path[->]
            (0) edge [bend left] node {a} (1)
            (1) edge [bend left] node {b,c} (0)
                edge [loop above] node {b} ()
                edge [bend left] node {c} (2)
            (2) edge [bend left] node {c} (1);
      \end{tikzpicture}

      We need to start by enumerating all sets of states,
      and their transitions for each input.

      \begin{tabular}{l | l | l | l}
        State & a & b & c \\
        \hline
        \{0\}     & \{1\} & \{\}    & \{\}      \\
        \{1\}     & \{\}  & \{0,1\} & \{0,2\}   \\
        \{0,1\}   & \{1\} & \{0,1\} & \{0,1,2\} \\
        \{0,2\}   & \{1\} & \{\}    & \{1\}     \\
        \{0,1,2\} & \{1\} & \{0,1\} & \{0,1,2\}
      \end{tabular}

      Now we need to generate new final states.
      These are any state sets that contain the original final states.
      In this case, all of the states in the first column are final states.

      Finally, we can generate our DFA.

      \begin{tikzpicture}[shorten >=1pt,node distance=3cm,on grid,auto]
         \node[state,initial,accepting] (0)   {$\{0\}$};
         \node[state,accepting] (1) [above right=of 0] {$\{1\}$};
         \node[state] (fail) [below right=of 1] {$\{\}$};
         \node[state,accepting] (01) [above right=of fail] {$\{01\}$};
         \node[state,accepting] (02) [below right=of 01] {$\{02\}$};
         \node[state,accepting] (012) [above right=of 1] {$\{012\}$};
          \path[->]
            (0) edge [bend left] node {a} (1)
                edge [bend right] node [below] {b,c} (fail)
            (1) edge node [below, pos=0.15] {a} (fail)
                edge [bend left] node [pos=0.15] {b} (01)
                edge [bend left=10] node [below, pos=0.15] {c} (02)
            (01)  edge node [pos=0.15] {a} (1)
                  edge [loop right] node {b} ()
                  edge [bend right=20] node {c} (012)
            (02)  edge [bend left=10] node [above, pos=0.15] {a,c} (1)
                  edge node {b} (fail)
            (012) edge [bend right] node [above] {a} (1)
                  edge [bend left=45] node {b} (01)
                  edge [loop above] node {c} ();
      \end{tikzpicture}

    \pagebreak

    \item[Problem 2]
      We have $\varepsilon$-arrows from 1 to 2, from 2 to 5, and from 3 to 1.

      From 1 to 2, we can get to state 4 following the path ``$\varepsilon$a''.
      So, we can eliminate this $\varepsilon$-arrow by creating a new path from 1 to 4 via a.

      From 1 to 2, we can also get to state 5 through an $\varepsilon$-arrow to state 2 following the path ``$\varepsilon\varepsilon$''.
      Since state 5 is a final state, we can eliminate these $\varepsilon$-arrows by making state 1 a final state.

      We can get from 2 to 5 via an $\varepsilon$-arrow and state 5 is a final state.
      So, we can eliminate this $\varepsilon$-arrow by making state 2 a final state.

      From 3 to 1, we can get to 1 again following the path ``$\varepsilon$a''.
      So, we can eliminate this $\varepsilon$-arrow by creating a new path from 3 to 1 via a.

      From 3 to 1, we can get to 1 again following the path ``$\varepsilon$c''.
      So, we can eliminate this $\varepsilon$-arrow by creating a new path from 3 to 1 via c.

      From 3 to 1, we can get to state 4 through an $\varepsilon$-arrow to state 2 following the path ``$\varepsilon\varepsilon$a''.
      So, we can eliminate this $\varepsilon$-arrow by creating a new path from 3 to 4 via a.

      Finally, from 3 to 1, we can also get to state 5 through an $\varepsilon$-arrow to state 2 and another $\varepsilon$-arrow to state 5 following the path ``$\varepsilon\varepsilon\varepsilon$''.
      Since state 5 is a final state, we can eliminate these $\varepsilon$-arrows by making state 3 a final state.

      Our $\varepsilon$-arrow-free NFA is:

      \begin{tikzpicture}[shorten >=1pt,node distance=3cm,on grid,auto]
         \node[state,initial,accepting] (1)   {$1$};
         \node[state,accepting] (2) [above right=of 1] {$2$};
         \node[state,accepting] (3) [below right=of 1] {$3$};
         \node[state,accepting] (4) [below right=of 2] {$4$};
         \node[state,accepting] (5) [right=of 4] {$5$};
          \path[->]
            (1) edge [loop above] node {a} ()
                edge node [pos=0.15] {a} (4)
                edge node {b} (2)
                edge node [pos=0.20] {d} (3)
            (2) edge node {a} (4)
            (3) edge [bend left] node {a,c} (1)
                edge node [below] {a,d} (4)
                edge node [pos=0.15] {b} (2)
            (4) edge node {b} (5)
            (5) edge [loop right] node {d} ();
      \end{tikzpicture}

    \item[Problem 3]
      \begin{proof}
        If $L_1,L_2,L_3$ are DFA-acceptable languages, then the intersection of these languages is also acceptable.
        We can reformulate the definition of \textbf{maj}:

        \textbf{maj}($L_1,L_2,L_3$) =

        $\{x \in \Sigma^* | (x \in L_1 \land x \in L_2) \lor (x \in L_1 \land x \in L_3) \lor (x \in L_2 \land x \in L_3) \lor (x \in L_1 \land x \in L_2 \land x \in L_3)\}$

        Or put more succinctly:

        \textbf{maj}($L_1,L_2,L_3$) =

        $\{x \in \Sigma^* | x \in (L_1 \cap L_2) \lor x \in (L_1 \cap L_3) \lor x \in (L_2 \cap L_3) \lor x \in (L_1 \cap L_2 \cap L_3)\}$

        But we know that each of these intersections is DFA-acceptable,
        so \textbf{maj}($L_1,L_2,L_3$) is DFA-acceptable.
      \end{proof}

    \item[Problem 4]
      \begin{proof}
        Given some accepting DFA $M = (Q, \Sigma, \delta, q_0, F)$ for $L$,
        we can construct a new accepting DFA $M' = (Q', \Sigma', \delta', q_0, F')$ for $\mathcal{Z}(L)$, where:

        \begin{align*}
          Q' &= Q \cup n \text{ additional states, where } n = |Q| \\
          \Sigma' &= \Sigma \cup \{0\} \\
          \delta' &: Q' \times \Sigma' \rightarrow Q' \\
          F' &= F \cup q_f\\
        \end{align*}

        Where we define
        \begin{align*}
          \delta'(q, \varepsilon) &= \delta(q, \varepsilon) \\
          \delta'(q_i, a0x) &= \delta(q_i, ax) \\
          & \text{In other words, we ``consume'' the } 0 \text{ by sending the DFA to an }\\
          & \text{intermediate state } q_{i_0} \text{ which needs a } 0 \\
          & \text{in order to continue accepting the string.} \\
          q_f &= \text{ The state with an arrow for the character } 0 \text{ coming from all final}\\
          & \text{states of } M \text{ and an arrow for every character in } \Sigma' \text{ going to a rejection state.}
        \end{align*}

        By construction, $M'$ accepts all strings from $M$ with $0$ interspersed and also suffixed by $0$.
        So, $M'$ is also DFA-acceptable.

        Thus, DFA-acceptable languages are closed under $\mathcal{Z}$.
      \end{proof}

      \begin{proof}
        We can also prove this in another way.

        Given some accepting DFA $M$ for $L$ and the fact that DFA's are closed under concatenation, we can decompose $M$ into smaller DFA's of a single character.
        I.e. $M = M_1M_2 \cdots M_n$.
        We can also construct a DFA $M_0$ that accepts the singleton $\{0\}$.
        Now we can intersperse $M_0$ through our decomposed $M$.
        We end up with $M' = M_1M_0M_2M_0 \cdots M_nM_0$.

        And since DFA-acceptable languages are closed under concatenation.
        Our new machine is an accepting DFA.

        If we define $\mathcal{Z}$ as this deconstruction/reconstruction operation,
        then we see that DFA-acceptable languages are closed under $\mathcal{Z}$.
      \end{proof}

    \pagebreak

    \item[Problem 5]
      The smallest DFA for $\{0,1\}^*{1^10}$ has 11 states.
      It must contain at least 10 individual states each counting one 1.

      \begin{proof}
        This DFA, let's call it $M$, could be described by the following table:

        \begin{tabular}{c | c | c}
          State & 0 & 1 \\
          \hline
          0   & 0 & 1   \\
          1   & 0 & 2   \\
          2   & 0 & 3   \\
          3   & 0 & 4   \\
          4   & 0 & 5   \\
          5   & 0 & 6   \\
          6   & 0 & 7   \\
          7   & 0 & 8   \\
          8   & 0 & 9   \\
          9   & 0 & 10  \\
          10  & 0 & 10
        \end{tabular}

        Where state 0 is the initial state, and the only final state is state 10.

        Now assume that we could construct a smaller DFA $M'$ with $i$ less states, where $0 < i < 11$.

        If $i \ge 11$, then $M'$ vacuously would not accept the language, and not the equivalent to $M$.
        If $i \le 0$, then $M'$ does not have less states.

        For other values of $i$, $M'$ will accept strings with $i$ less characters than $M$.
        E.g. if $i=10$, $M'$ will accept the string $1$, which is clearly not accepted by $M$.

        So, these smaller state DFA's are not equivalent to $M$, and we cannot make the assumption that we could construct a smaller DFA.

        Thus, the smallest DFA for $\{0,1\}^*{1^10}$ has 11 states.
      \end{proof}

    \pagebreak

    \item[Problem 6]
      It helps to describe what our language is.
      One interpretation is that $L_n$ is the language where $1$ is $n$ characters from the end of the string.
      Rephrasing like this can help us see, this is similar to the previous problem
      in the sense that we must have $n$ states to count each $\{0,1\}$ at the end of the accepted string.

      Let's look at some NFA's.

      $L_1 = $
      \begin{tikzpicture}[shorten >=1pt,node distance=2cm,on grid,auto]
         \node[state,initial] (1)   {$1$};
         \node[state] (2) [right=of 1] {$2$};
         \node[state,accepting] (3) [right=of 2] {$3$};
          \path[->]
            (1) edge [loop above] node {0,1} ()
                edge node {1} (2)
            (2) edge node {0,1} (3);
      \end{tikzpicture}

      $L_2 = $
      \begin{tikzpicture}[shorten >=1pt,node distance=2cm,on grid,auto]
         \node[state,initial] (1)   {$1$};
         \node[state] (2) [right=of 1] {$2$};
         \node[state] (3) [right=of 2] {$3$};
         \node[state,accepting] (4) [right=of 3] {$4$};
          \path[->]
            (1) edge [loop above] node {0,1} ()
                edge node {1} (2)
            (2) edge node {0,1} (3)
            (3) edge node {0,1} (4);
      \end{tikzpicture}

      $L_3 = $
      \begin{tikzpicture}[shorten >=1pt,node distance=2cm,on grid,auto]
         \node[state,initial] (1)   {$1$};
         \node[state] (2) [right=of 1] {$2$};
         \node[state] (3) [right=of 2] {$3$};
         \node[state] (4) [right=of 3] {$4$};
         \node[state,accepting] (5) [right=of 4] {$5$};
          \path[->]
            (1) edge [loop above] node {0,1} ()
                edge node {1} (2)
            (2) edge node {0,1} (3)
            (3) edge node {0,1} (4)
            (4) edge node {0,1} (5);
      \end{tikzpicture}

      We start to see that each $L_n$ has:
      \begin{itemize}
        \item an initial state for the prefix $\{0,1\}$
        \item one state to count a single $1$
        \item exactly $n$ states counting the suffix $\{0,1\}^n$
      \end{itemize}

      In other words, we suggest that an accepting NFA for $L_n$ has exactly $n + 2$ states.

      We can prove this by induction.

      \pagebreak

      \begin{proof}
        \textit{Base Case} $n = 1$

        \begin{align*}
          |states(L_1)| &= |\{0,1\}^*\{1\}\{0,1\}| \\
          &= |\{0,1\}^*| + |\{1\}| + |\{0,1\}| \\
          &= 1 + 1 + 1 \\
          &= 3 \\
          &= 1 + 2
        \end{align*}

        So our base case holds.

        \textit{Inductive Case} Assume $|states(L_n)| = n + 2$. Show $|states(L_{n+1})| = (n + 1) + 2 = n + 3$.

        \begin{align*}
          |states(L_{n+1}) &= |\{0,1\}^*\{1\}\{0,1\}^{n+1}| \\
          &= |\{0,1\}^*\{1\}\{0,1\}^n\{0,1\}| \\
          &= |\{0,1\}^*\{1\}\{0,1\}^n| + |\{0,1\}| \\
          &= |states(L_n)| + |\{0,1\}| \\
          &= (n + 2) + |\{0,1\}| \\
          &= (n + 2) + 1 \\
          &= n + 3 \\
        \end{align*}

        So our Inductive Hypothesis is true as well.

        Thus, we have shown that an accepting NFA for $L_n$ has exactly $n + 2$ states.
      \end{proof}

      We still need to show that an accepting DFA for $L_n$ has no less than $2^n$ states.

      \begin{proof}
        We know from the construction of a DFA, $M'$, from an NFA, $M$, that we will have to construct subsets of the possible states in an NFA.

        Assume that we can construct an arbitrary DFA $M'$ with fewer than $2^n$ states which still accepts our language.

        This means that we have left out at least one element of the powerset of states in our construction of $M'$.
        This element was a possible state set in $M'$
        Without this element we cannot guarantee that we have constructed a DFA that accepts the same language as our NFA.

        So, our assumption does not hold, and we must have at least $2^n$ states in $M'$.

        Thus, an accepting DFA for $L_n$ has no less than $2^n$ states.
      \end{proof}
  \end{enumerate}
\end{document}
