\documentclass[12pt,letterpaper]{article}
\usepackage{amsmath}
\usepackage{amsfonts}
\usepackage{amsthm}
\usepackage{cancel}
\usepackage[margin=1in]{geometry}
\usepackage{titling}
\usepackage{multirow}
\usepackage{amssymb}
\usepackage{tikz}
\usetikzlibrary{automata,positioning}

\setlength{\droptitle}{-10ex}

\preauthor{\begin{flushright}\large \lineskip 0.5em}
\postauthor{\par\end{flushright}}
\predate{\begin{flushright}\large}
\postdate{\par\end{flushright}}

\title{ECS 120 Problem Set 3\vspace{-2ex}}
\author{Hardy Jones\\
        999397426\\
        Professor Rogaway\vspace{-2ex}}
\date{Spring 2014}

\begin{document}
  \maketitle

  \begin{enumerate}
    \item[Problem 1]
      \begin{tikzpicture}[shorten >=1pt,node distance=2cm,on grid,auto]
         \node[state,initial,accepting] (0)   {$0$};
         \node[state,accepting] (1) [right=of 0] {$1$};
         \node[state] (2) [right=of 1] {$2$};
          \path[->]
            (0) edge [bend left] node {a} (1)
            (1) edge [bend left] node {b,c} (0)
                edge [loop above] node {b} ()
                edge [bend left] node {c} (2)
            (2) edge [bend left] node {c} (1);
      \end{tikzpicture}

      We need to start by enumerating all sets of states,
      and their transitions for each input.

      \begin{tabular}{l | l | l | l}
        State & a & b & c \\
        \hline
        \{0\}     & \{1\} & \{\}    & \{\}      \\
        \{1\}     & \{\}  & \{0,1\} & \{0,2\}   \\
        \{0,1\}   & \{1\} & \{0,1\} & \{0,1,2\} \\
        \{0,2\}   & \{1\} & \{\}    & \{1\}     \\
        \{0,1,2\} & \{1\} & \{0,1\} & \{0,1,2\}
      \end{tabular}

      Now we need to generate new final states.
      These are any state sets that contain the original final states.
      In this case, all of the states in the first column are final states.

      Finally, we can generate our DFA.

      \begin{tikzpicture}[shorten >=1pt,node distance=3cm,on grid,auto]
         \node[state,initial,accepting] (0)   {$\{0\}$};
         \node[state,accepting] (1) [above right=of 0] {$\{1\}$};
         \node[state] (fail) [below right=of 1] {$\{\}$};
         \node[state,accepting] (01) [above right=of fail] {$\{01\}$};
         \node[state,accepting] (02) [below right=of 01] {$\{02\}$};
         \node[state,accepting] (012) [above right=of 1] {$\{012\}$};
          \path[->]
            (0) edge [bend left] node {a} (1)
                edge [bend right] node [below] {b,c} (fail)
            (1) edge node [below, pos=0.15] {a} (fail)
                edge [bend left] node [pos=0.15] {b} (01)
                edge [bend left=10] node [below, pos=0.15] {c} (02)
            (01)  edge node [pos=0.15] {a} (1)
                  edge [loop right] node {b} ()
                  edge [bend right=20] node {c} (012)
            (02)  edge [bend left=10] node [above, pos=0.15] {a,c} (1)
                  edge node {b} (fail)
            (012) edge [bend right] node [above] {a} (1)
                  edge [bend left=45] node {b} (01)
                  edge [loop above] node {c} ();
      \end{tikzpicture}

    \item[Problem 2]
      We have $\varepsilon$-arrows from 1 to 2, from 2 to 5, and from 3 to 1.

      From 1 to 2, we can get to state 4 following the path ``$\varepsilon$a''.
      So, we can eliminate this $\varepsilon$-arrow by creating a new path from 1 to 4 via a.

      From 1 to 2, we can also get to state 5 through an $\varepsilon$-arrow to state 2 following the path ``$\varepsilon\varepsilon$''.
      Since state 5 is a final state, we can eliminate these $\varepsilon$-arrows by making state 1 a final state.

      We can get from 2 to 5 via an $\varepsilon$-arrow and state 5 is a final state.
      So, we can eliminate this $\varepsilon$-arrow by making state 2 a final state.

      From 3 to 1, we can get to 1 again following the path ``$\varepsilon$a''.
      So, we can eliminate this $\varepsilon$-arrow by creating a new path from 3 to 1 via a.

      From 3 to 1, we can get to 1 again following the path ``$\varepsilon$c''.
      So, we can eliminate this $\varepsilon$-arrow by creating a new path from 3 to 1 via c.

      From 3 to 1, we can get to state 4 through an $\varepsilon$-arrow to state 2 following the path ``$\varepsilon\varepsilon$a''.
      So, we can eliminate this $\varepsilon$-arrow by creating a new path from 3 to 4 via a.

      Finally, from 3 to 1, we can also get to state 5 through an $\varepsilon$-arrow to state 2 and another $\varepsilon$-arrow to state 5 following the path ``$\varepsilon\varepsilon\varepsilon$''.
      Since state 5 is a final state, we can eliminate these $\varepsilon$-arrows by making state 3 a final state.

      Our $\varepsilon$-arrow-free NFA is:

      \begin{tikzpicture}[shorten >=1pt,node distance=3cm,on grid,auto]
         \node[state,initial,accepting] (1)   {$1$};
         \node[state,accepting] (2) [above right=of 1] {$2$};
         \node[state,accepting] (3) [below right=of 1] {$3$};
         \node[state,accepting] (4) [below right=of 2] {$4$};
         \node[state,accepting] (5) [right=of 4] {$5$};
          \path[->]
            (1) edge [loop above] node {a} ()
                edge node [pos=0.15] {a} (4)
                edge node {b} (2)
                edge node [pos=0.20] {d} (3)
            (2) edge node {a} (4)
            (3) edge [bend left] node {a,c} (1)
                edge node [below] {a,d} (4)
                edge node [pos=0.15] {b} (2)
            (4) edge node {b} (5)
            (5) edge [loop right] node {d} ();
      \end{tikzpicture}
  \end{enumerate}
\end{document}
