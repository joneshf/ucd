\documentclass[12pt,letterpaper]{article}
\usepackage{amsmath}
\usepackage{amsfonts}
\usepackage{amsthm}
\usepackage{cancel}
\usepackage[margin=1in]{geometry}
\usepackage{titling}
\usepackage{multirow}
\usepackage{amssymb}
\usepackage{algorithm2e}

\newcommand{\lb}[0]{\text{lg}}

\setlength{\droptitle}{-10ex}

\preauthor{\begin{flushright}\large \lineskip 0.5em}
\postauthor{\par\end{flushright}}
\predate{\begin{flushright}\large}
\postdate{\par\end{flushright}}

\title{ECS 122A Homework 2\vspace{-2ex}}
\author{Hardy Jones\\
        999397426\\
        Professor Bai\vspace{-2ex}}
\date{Spring 2014}

\begin{document}
  \maketitle

  \begin{enumerate}
    \item
      \begin{enumerate}
        \item
          We can expand some of the polynomial
          \[(n + 2)^{10} = n^{10} + 20n^9 + \dots + 1024\]

          Looking at everything except the first term, we see that the greatest power is $n^9$.
          We can set a \textit{little-oh} bound of $n^{10}$,
          because $\forall c > 0, \exists n_0 > 0$ such that $0 \le 20n^9 + \dots + 1024 < cn^{10}, \forall n \ge n_0$.
          That is:
          \[(n + 2)^{10} = n^{10} + o(n^{10})\]

          Now, this is $\Theta(n^{10})$ because

          $\exists c_1 > 0, c_2 > 0, n_0 > 0$ such that $0 \le c_1n^{10} \le n^{10} + o(n^{10}) \le c_2n^{10}, \ \forall n \ge n_0$

          This simplifies to $0 \le c_1 \le 1 + \frac{o(n^{10})}{n^{10}} \le c_2$.

          Now, since $o(n^{10})$ is an upper bound we know the largest the term $\frac{o(n^{10})}{n^{10}}$ can be is 1.
          So, this further siplifies to $0 \le c_1 \le 2 \le c_2$.

          So this is $\Theta(n^{10})$ if we choose appropriate constants.
          For instance we can choose $c_1 = c_2 = 2$.

          Thus, $(n + 2)^{10} = \Theta(n^{10})$

        \item
          We can apply reasoning similar to the previous problem.

          We can expand some of the polynomial
          \[(n + a)^b = n^b + \binom{b}{1}n^{b-1}a + \dots + a^b\]

          Looking at everything except the first term, we see that the greatest power is $n^{b-1}$.
          We can set a \textit{little-oh} bound of $n^b$,
          because $\forall c > 0, \exists n_0 > 0$ such that $0 \le \binom{b}{1}n^{b-1}a + \dots + a^b < cn^b, \forall n \ge n_0$.
          That is:
          \[(n + a)^b = n^b + o(n^b)\]

          Now, this is $\Theta(n^b)$ because

          $\exists c_1 > 0, c_2 > 0, n_0 > 0$ such that $0 \le c_1n^b \le n^b + o(n^b) \le c_2n^b, \ \forall n \ge n_0$

          This simplifies to $0 \le c_1 \le 1 + \frac{o(n^b)}{n^b} \le c_2$.

          Now, since $o(n^b)$ is an upper bound we know the largest the term $\frac{o(n^b)}{n^b}$ can be is 1.
          So, this further siplifies to $0 \le c_1 \le 2 \le c_2$.

          So this is $\Theta(n^b)$ if we choose appropriate constants.
          For instance we can choose $c_1 = c_2 = 2$.

          Thus, $(n + a)^b = \Theta(n^b)$
      \end{enumerate}

    \item
      ``The running time of algorithm A is at least $O(n^2)$,'' states that the running time of ``A'' can be greater than $O(n^2)$.
      This has no meaning because \textit{big-oh} notation states an asymptotic upper bound on a function.
      Asserting that the running time can be greater than this value negates the reason for using a bound in the first place.

    \item
      \begin{enumerate}
        \item Yes.
          We need to find some positive constants $c, n_0$ such that $0 \le 2^{n+1} \le c2^n \forall n \ge n_0$.
          \[0 \le 2^{n + 1} = 2 \cdot 2^n\]. So we have our $c = 2, n = 1$.

          Thus $2^{n+1} = O(2^n)$.

        \item No.
          We need to show that no constants exist that satisfy the definition.
          \[2^{2n} = 2^n \cdot 2^n \le c2^n\].

          It's easy to see that no constant will make this equation true for all values of $n$.

          Thus $2^{2n} \ne O(2^n)$.
      \end{enumerate}

    \item
      \begin{enumerate}
        \item
          From smallest to largest:

          \begin{tabular}{c | l}
            Function & Notes \\
            \hline
            lg$n$ \\
            $n$lg$n$ \\
            n \\
            $n^2, n^2 + $lg$n$ & These are the same asymptotically because the lg$n$ term will ``drop-out'' \\
            $n^3$ \\
            $n - n^3 + 7n^5$ \\
            $2^n$
          \end{tabular}

        \item
          From smallest to largest:

          \begin{tabular}{c | l}
            Function & Notes \\
            \hline
            1 \\
            lg lg$n$ \\
            lg$n$, ln$n$ & The change of base merely changes the value by a constant factor \\
            $(\text{lg}n)^2$ \\
            $n$lg$n$ \\
            $\sqrt{n}$, $\sqrt{2}^{\text{lg}n}$, $n^{1+\epsilon}$ & $\sqrt{n} = \sqrt{2}^{\text{lg}n}$, $n^{1+\epsilon}$ differs by  constant factor \\
            n \\
            $n^2, n^2 + $lg$n$ & These are the same asymptotically because the lg$n$ term will ``drop-out'' \\
            $n^3$ \\
            $n - n^3 + 7n^5$ \\
            $2^n$, $2^{n-1}$ & See Problem 3.a \\
            $e^n$ \\
            $n!$
          \end{tabular}
      \end{enumerate}

    \pagebreak

    \item
      \begin{enumerate}
        \item
          $a = 2, b = 4, f(n) = 1, n^{log_4 2} = n^\frac{1}{2} = \sqrt{n}$

          We can see this fits case 1 of the master theorem, where we let $\epsilon = \frac{1}{2}$.

          So we have $f(n) = O(n^{log_4 2 - \frac{1}{2}}) = O(n^{\frac{1}{2} - \frac{1}{2}}) = O(n^0) = O(1)$.

          Thus, $T(n) = \Theta(n^\frac{1}{2}) = \Theta(\sqrt{n})$
        \item
          $a = 2, b = 4, f(n) = \sqrt{n}, n^{log_4 2} = n^\frac{1}{2} = \sqrt{n}$

          This is exactly case 2 of the master theorem.

          Thus, $T(n) = \Theta(\sqrt{n}\text{lg}n)$

        \item
          $a = 2, b = 4, f(n) = n, n^{log_4 2} = n^\frac{1}{2} = \sqrt{n}$

          This might fit case 3 of the master theorem (with $\epsilon = \frac{1}{2}$) if we can find an appropriate constant $c$.

          \begin{align*}
            af\left(\frac{n}{b}\right) &= 2f\left(\frac{n}{4}\right) \\
            &= 2\left(\frac{n}{4}\right) \\
            &= \frac{1}{2}n
          \end{align*}

          So, if we let $c = \frac{1}{2}$, then we satisfy the conditions of case 3 of the master theorem.

          Thus, $T(n) = \Theta(n)$

        \item
          $a = 2, b = 4, f(n) = n^2, n^{log_4 2} = n^\frac{1}{2} = \sqrt{n}$

          This might fit case 3 of the master theorem (with $\epsilon = \frac{3}{2}$) if we can find an appropriate constant $c$.

          \begin{align*}
            af\left(\frac{n}{b}\right) &= 2f\left(\frac{n}{4}\right) \\
            &= 2\left(\frac{n}{4}\right)^2 \\
            &= 2\left(\frac{n}{16}\right) \\
            &= \frac{1}{8}n
          \end{align*}

          So, if we let $c = \frac{1}{8}$, then we satisfy the conditions of case 3 of the master theorem.

          Thus, $T(n) = \Theta(n^2)$

      \end{enumerate}
  \end{enumerate}
\end{document}
