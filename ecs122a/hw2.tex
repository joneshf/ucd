\documentclass[12pt,letterpaper]{article}
\usepackage{amsmath}
\usepackage{amsfonts}
\usepackage{amsthm}
\usepackage{cancel}
\usepackage[margin=1in]{geometry}
\usepackage{titling}
\usepackage{multirow}
\usepackage{amssymb}
\usepackage{algorithm2e}

\newcommand{\lb}[0]{\text{lg}}

\setlength{\droptitle}{-10ex}

\preauthor{\begin{flushright}\large \lineskip 0.5em}
\postauthor{\par\end{flushright}}
\predate{\begin{flushright}\large}
\postdate{\par\end{flushright}}

\title{ECS 122A Homework 2\vspace{-2ex}}
\author{Hardy Jones\\
        999397426\\
        Professor Bai\vspace{-2ex}}
\date{Spring 2014}

\begin{document}
  \maketitle

  \begin{enumerate}
    \item
      \begin{enumerate}
        \item
          We can expand some of the polynomial
          \[(n + 2)^{10} = n^{10} + 20n^9 + \dots + 1024\]

          Looking at everything except the first term, we see that the greatest power is $n^9$.
          We can set a \textit{little-oh} bound of $n^{10}$,
          because $\forall c > 0, \exists n_0 > 0$ such that $0 \le 20n^9 + \dots + 1024 < cn^{10}, \forall n \ge n_0$.
          That is:
          \[(n + 2)^{10} = n^{10} + o(n^{10})\]

          Now, this is $\Theta(n^{10})$ because

          $\exists c_1 > 0, c_2 > 0, n_0 > 0$ such that $0 \le c_1n^{10} \le n^{10} + o(n^{10}) \le c_2n^{10}, \ \forall n \ge n_0$

          This simplifies to $0 \le c_1 \le 1 + \frac{o(n^{10})}{n^{10}} \le c_2$.

          Now, since $o(n^{10})$ is an upper bound we know the largest the term $\frac{o(n^{10})}{n^{10}}$ can be is 1.
          So, this further siplifies to $0 \le c_1 \le 2 \le c_2$.

          So this is $\Theta(n^{10})$ if we choose appropriate constants.
          For instance we can choose $c_1 = c_2 = 2$.

          Thus, $(n + 2)^{10} = \Theta(n^{10})$

        \item
          We can apply reasoning similar to the previous problem.

          We can expand some of the polynomial
          \[(n + a)^b = n^b + \binom{b}{1}n^{b-1}a + \dots + a^b\]

          Looking at everything except the first term, we see that the greatest power is $n^{b-1}$.
          We can set a \textit{little-oh} bound of $n^b$,
          because $\forall c > 0, \exists n_0 > 0$ such that $0 \le \binom{b}{1}n^{b-1}a + \dots + a^b < cn^b, \forall n \ge n_0$.
          That is:
          \[(n + a)^b = n^b + o(n^b)\]

          Now, this is $\Theta(n^b)$ because

          $\exists c_1 > 0, c_2 > 0, n_0 > 0$ such that $0 \le c_1n^b \le n^b + o(n^b) \le c_2n^b, \ \forall n \ge n_0$

          This simplifies to $0 \le c_1 \le 1 + \frac{o(n^b)}{n^b} \le c_2$.

          Now, since $o(n^b)$ is an upper bound we know the largest the term $\frac{o(n^b)}{n^b}$ can be is 1.
          So, this further siplifies to $0 \le c_1 \le 2 \le c_2$.

          So this is $\Theta(n^b)$ if we choose appropriate constants.
          For instance we can choose $c_1 = c_2 = 2$.

          Thus, $(n + a)^b = \Theta(n^b)$
      \end{enumerate}
  \end{enumerate}
\end{document}
