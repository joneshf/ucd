\documentclass[12pt,letterpaper]{article}
\usepackage{amsmath}
\usepackage{amsfonts}
\usepackage{amsthm}
\usepackage{cancel}
\usepackage[margin=1in]{geometry}
\usepackage{titling}
\usepackage{multirow}
\usepackage{amssymb}
\usepackage{algorithm2e}
\usepackage{wasysym}

\newcommand{\lb}[0]{\text{lg}}

\setlength{\droptitle}{-10ex}

\preauthor{\begin{flushright}\large \lineskip 0.5em}
\postauthor{\par\end{flushright}}
\predate{\begin{flushright}\large}
\postdate{\par\end{flushright}}

\title{ECS 122A Homework 1\vspace{-2ex}}
\author{Hardy Jones\\
        999397426\\
        Professor Gysel\vspace{-2ex}}
\date{Fall 2014}

\begin{document}
  \maketitle

  \begin{enumerate}
    \item
      \begin{enumerate}
        \item
          The most number of pennies an optimal solution can have is 9.
          The most number of dimes an optimal solution can have is 4.
          We can see this by looking at some cases:
          \begin{itemize}
            \item $0 \le n < 10 :$ We have no choice but to use pennies, as any other denomination will be more than the change required. This has a maximum of 9 coins.
            \item $10 \le n < 20 :$ We can use 1 dime to get down to the first case (rather than all pennies), and the rest returned in pennies. This has a maximum of 1 dime + 9 pennies = 10 coins.
            \item $20 \le n < 25 :$ We can use 1 dime to get down to the previous case, then continue following the previous reasoning. This is less overall coins than using pennies to get to the previous case. This has a maximum of 2 dimes + 4 pennies = 6 coins.
            \item $25 \le n < 30 :$ We can use 1 quarter to get to the case where $0 \le n < 5$, then use pennies from there. This is less overall coins than using pennies and dimes to get to any previous cases. This has a maximum of 1 quarter + 4 pennies = 5 coins.
            \item $30 \le n < 35 :$ We can use 1 dime to get down to the case where $20 \le n < 25$, then continue with the reasoning from there.
            This is less overall coins than using a quarter to get the the previous case and pennies. This has a maximum of 3 dimes + 4 pennies = 7 coins.
            \item $35 \le n < 40 :$ We can use 1 quarter to get to the case where $10 \le n < 20$, then continue with the reasoning from there. This is less overall coins than using dimes and pennies alone. This has a maximum of 1 quarter + 1 dime + 9 pennies = 11 coins.
            \item $40 \le n < 45 :$ We can use 1 dime to get to the case where $30 \le n < 35$, then continue with the reasoning from there. This is less overall coins than using a quarter to get to another case. This has a maximum of 4 dimes + 4 pennies = 8 coins.
            \item $45 \le n < 50 :$ We can use 1 quarter to get to the case where $20 \le n < 25$, then continue with the reasoning from there. This is less overall coins than using a dime to get to another case. This has a maximum of 1 quarter + 2 dimes + 4 pennies = 7 coins.
            \item $50 \le n:$ We can use 2 quarters to get down to one of the above cases, then continue with the reasoning from there. This is less overall coins than using dimes or pennies alone to get down to any of the above cases.
          \end{itemize}

        \item
          Assume that it is not the case that using quarters initially when $n \ge 50$ provides the optimal solution.

          This means we can provide the optimal solution by first using dimes and/or pennies until the change needed is less than 50\cent.
          We can see that once the change needed is below 50\cent, we can use the cases in part \textit{a} to continue reasoning.
          So, we need 5 dimes for any $n \ge 50$ to get it down to the cases used above.
          However, we know that we can use just 2 quarters to get any $n \ge 50$ down to the cases above.
          Since 2 quarters is less overall coins than 5 dimes, we have found a solution better than what we assumed to be the optimal.
          Similar reasoning works with using pennies, however, we would need 50 pennies rather than 5 dimes and this is clearly non-optimal.

          Thus, our assumption was wrong.

          Therefore, the optimal solution uses quarters until the amount is less than 50\cent.

        \item
          One algorithm for this is to do a direct translation of the cases above.

          We first check to see if $n \ge 50$.

          If so we know we need at least 2 quarters.
          We also subtract $50$ from $n$ to get a new amount of change needed, say $n_1$.
          If $n < 50$ we can still say the new amount of change needed is $n_1$.

          Then, we take $n_1$ and find which case it fits.

          Since we have already enumerated the number and denomination of quarters and dimes for each case,
          we subtract that amount from whichever case $n_1$ falls under.
          E.g. if $n_1 = 36\text{\cent}$, when we would subtract the total of 1 quarter and 1 dime (or 35\cent) from $n_1$.
          We call this new amount of change needed $n_2$.
          We also add the respective number and denomination of quarters and dimes to the coins needed.

          Finally, we return $n_2$ pennies.

          This algorithm is $O(1)$ because it only performs standard arithmetic and comparisons, each of which have $O(1)$ cost.
          The algorithm is not dependent on the size of $n$.
      \end{enumerate}
  \end{enumerate}
\end{document}
