\documentclass[12pt,letterpaper]{article}
\usepackage{amsmath}
\usepackage{amsfonts}
\usepackage{amsthm}
\usepackage{cancel}
\usepackage[bottom=1in,left=0.5in,right=1in,top=1in]{geometry}
\usepackage{titling}
\usepackage{multirow}
\usepackage{amssymb}
\usepackage{wasysym}
\usepackage{qtree}
\usepackage{algorithm}
\usepackage{algpseudocode}
\usepackage{tikz}
\usepackage{mathtools}
\usetikzlibrary{automata,positioning}

\newcommand{\lb}[0]{\text{lg}}

\setlength{\droptitle}{-10ex}

\preauthor{\begin{flushright}\large \lineskip 0.5em}
\postauthor{\par\end{flushright}}
\predate{\begin{flushright}\large}
\postdate{\par\end{flushright}}

\title{ECS 122A Homework 6\vspace{-2ex}}
\author{Hardy Jones\\
        999397426\\
        Professor Gysel\vspace{-2ex}}
\date{Fall 2014}

\begin{document}
  \maketitle

  \begin{enumerate}
    \item

    \item
      \begin{enumerate}
        \item
          Using the following graph:

          \begin{tikzpicture}[shorten >=1pt,node distance=3cm,on grid,auto]
            \node[state] (a)   {$a$};
            \node[state] (b) [right=of a] {$b$};
            \node[state] (c) [below=of a] {$c$};
            \node[state] (d) [right=of c] {$d$};
            \path[-]
              (a) edge node {$12$} (b)
                  edge node {$1$} (c)
                  edge [pos=0.25] node {$1$} (d)
              (b) edge [pos=0.25] node {$1$} (c)
                  edge node {$1$} (d)
              (c) edge node {$1$} (d)
              ;
          \end{tikzpicture}

          You can construct a tour starting at node $a$ using the greedy approach: $a, c, d, b, a$.
          This tour has a cost of $1 + 1 + 1 + 12 = 15$.
          Whereas an optimal tour is $a, d, b, c, a$ with a cost of $1 + 1 + 1 + 1 = 4$.

          So the greedy approach did not find an optimal tour.

        \item
          We can use the above graph and change the edge between $a$ and $b$ to have a value of $k + 1$.
          Now, the greedy approach will always find a tour of cost $1 + 1 + 1 + (k + 1) = k + 4$.
          This tour is exactly $k$ greater than the optimal tour.
      \end{enumerate}
    \item
      \begin{enumerate}
        \item
          One benefit of linear programming is that
          you can describe any problem in \textbf{P} and solve it,
          even if you do not explicitly know an algorithm for the problem.
          One drawback of linear programming is that
          you have no control over the runtime of the algorithm;
          you may be able to solve the problem,
          but the LP solver may take $O(n^5)$ time when an $O(n)$ solution is possible.
        \item
          One benefit of 01-integer programming is that
          you can solve problems in \textbf{NP} now,
          even without an explicit algorithm for the original problem.
          One drawback of 01-integer programming is that
          the solver now does not run in polynomial time for sure.
        \item
          One benefit of linear programming relaxation is that
          you can take a problem in \textbf{NP} and
          find ``some'' local optimal solution in polynomial time.
          One drawback of linear programming relaxation is that
          you have to interpret the results of the output in order to use the solution;
          this interpretation may not be trivial.
      \end{enumerate}
  \end{enumerate}
\end{document}
