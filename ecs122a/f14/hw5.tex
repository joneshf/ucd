\documentclass[12pt,letterpaper]{article}
\usepackage{amsmath}
\usepackage{amsfonts}
\usepackage{amsthm}
\usepackage{cancel}
\usepackage[bottom=1in,left=0.5in,right=1in,top=1in]{geometry}
\usepackage{titling}
\usepackage{multirow}
\usepackage{amssymb}
\usepackage{wasysym}
\usepackage{qtree}
\usepackage{algorithm}
\usepackage{algpseudocode}
\usepackage{tikz}
\usepackage{mathtools}
\usetikzlibrary{automata,positioning}

\newcommand{\lb}[0]{\text{lg}}

\setlength{\droptitle}{-10ex}

\preauthor{\begin{flushright}\large \lineskip 0.5em}
\postauthor{\par\end{flushright}}
\predate{\begin{flushright}\large}
\postdate{\par\end{flushright}}

\title{ECS 122A Homework 5\vspace{-2ex}}
\author{Hardy Jones\\
        999397426\\
        Professor Gysel\vspace{-2ex}}
\date{Fall 2014}

\begin{document}
  \maketitle

  \begin{enumerate}
    \item
      We can use the Floyd-Warshall algorithm to do the brunt of the work here.
      We just need to modify some parts of it, and perform some post computation.

      Given a graph $G$ where
      \[
        G.V = \{C_1, C_2, \dots, C_n\}
      \]
      \[
        G.E = \{H_{i, j} \ | \ C_i \text{ is connected to } C_j \text{ by a highway}\}
      \]

      We can construct most of the algorithm the same,
      and just change the final case in the recursive formula.


      \[
        D^{(k)}[i, j] =
        \begin{dcases*}
          0           & if $k = 0$ and $i = j$ \\
          t(H_{i, j}) & if $k = 0$ and $H_{i, j} \in G.E$ \\
          \infty      & if $k = 0$ and $i \ne j$ and $H_{i, j} \notin G.E$ \\
          min(D^{(k-1)}[i, j], D^{(k-1)}[i, k] + t(C_k) + D^{(k-1)}[k, j]) & if $k \ge 1$
        \end{dcases*}
      \]

      Using this formulation, we construct FLOYD-WARSHALL-MODIFIED similar to the method for constructing FLOYD-WARSHALL in class -- taking care to use the modified cases where appropriate.
      The runtime of this algorithm has not changed, as the nested update function is still $O(1)$.
      So the complexity is still $\Theta(n^3)$

      \pagebreak

      \begin{algorithm}
        \begin{algorithmic}
          \Function{FLOYD-WARSHALL-MODIFIED}{$G, t$}
            \State $n \gets |G.V|$
            \State Create $n \times n$ array $D^{(0)}$
            \For{$i \gets 1$ to $n$}
              \For{$j \gets 1$ to $n$}
                \If{$i = j$}
                  \State $D^{(0)}[i, j] = 0$
                \ElsIf{$H_{i, j} \in G.E$}
                  \State $D^{(0)}[i, j] = t(H_{i, j})$
                \Else
                  \State $D^{(0)}[i, j] = \infty$
                \EndIf
              \EndFor
            \EndFor

            \For{$k \gets 1$ to $n$}
              \State Create $n \times n$ array $D^{(k)}$
              \For{$i \gets 1$ to $n$}
                \For{$j \gets 1$ to $n$}
                  \State $D^{(k)}[i, j] = min(D^{(k-1)}[i, j], D^{(k-1)}[i, k] + t(C_k) + D^{(k-1)}[k, j])$
                \EndFor
              \EndFor
            \EndFor
            \State \Return $D^{(n)}$
          \EndFunction
        \end{algorithmic}
      \end{algorithm}

      Since this gives us back an $n \times n$ array with just the times from one city to another (including travel times between cities), we still need to add on the time necessary to travel from each depot to the highway.

      For each $i, j$ entry in the array, where $i \ne j$, we need the time from the depot to the highway in $C_i$ and the time from the highway to the depot in $C_j$.
      Then we can just return the array.

      \pagebreak

      Our completed algorithm is:

      \begin{algorithm}
        \begin{algorithmic}
          \Function{MOVE-FREIGHT}{G, t, d}
            \State $n \gets |G.V|$
            \State $D \gets$ FLOYD-WARSHALL-MODIFIED(G, t)
            \For{$i \gets 1$ to $n$}
              \For{$j \gets 1$ to $n$}
                \If{$i \ne j$}
                  \State $D[i, j] = D[i, j] + d(C_i) + d(C_j)$
                \EndIf
              \EndFor
            \EndFor
          \State \Return $D$
          \EndFunction
        \end{algorithmic}
      \end{algorithm}

      Since the added loops only contribute $\Theta(n^2)$ time to the algorithm, the runtime of MOVE-FREIGHT is governed by the call to FLOYD-WARSHALL-MODIFIED.
      Thus, the runtime of this algorithm is $\Theta(n^3)$.

    \item
      \begin{enumerate}
        \item
          Given a graph $G$, a size $k$ and a certificate $S \subseteq G.V$,
          we need to check two things:
          \begin{enumerate}
            \item $|S| = k$ \label{vcd1}
            \item $\forall (i, j) \in G.E; i \in S, j \in S$, or $\{i, j\} \subseteq S$ \label{vcd2}
          \end{enumerate}

          \ref{vcd1} can be checked in $O(|S|)$ time by counting the size of $S$.

          \ref{vcd2} can be checked in $O(|G.E|)$ time by iterating over $G.E$ and checking each vertex for membership in $S$.

          These are both linear times, so our verification algorithm runs in polynomial time.
          Thus VERTEX-COVER-DEC $\in$ \textbf{NP}.

          \begin{algorithm}
            \begin{algorithmic}
              \Function{VERIFY-VERTEX-COVER}{G, k, S}
                \State $result \gets \textsc{True}$
                \If{$|S| \ne k$}
                  \State $result \gets$ \textsc{False}
                \Else
                  \ForAll{$(i, j) \in G.E$}
                    \If{$i \notin S$ or $j \notin S$}
                      \State $result \gets$ \textsc{False}
                    \EndIf
                  \EndFor
                \EndIf
                \State \Return $result$
              \EndFunction
            \end{algorithmic}
          \end{algorithm}

        \item
          Given a graph $G$, a size $k$ and a certificate $M \subseteq G.E$,
          we need to check two things:
          \begin{enumerate}
            \item $|M| = k$ \label{bmd1}
            \item $\forall v \in G.V$ at most one edge of $M$ is incident to $v$ \label{bmd2}
          \end{enumerate}

          \ref{bmd1} can be checked in $O(|M|)$ time by counting the size of $M$.

          \ref{bmd2} can be checked in $O(|M||G.V|)$ time by keeping track of the incident edges.

          These are both linear times, so our verification algorithm runs in polynomial time.
          Thus BIPARTITE-MATCHING-DEC $\in$ \textbf{NP}.

          \begin{algorithm}
            \begin{algorithmic}
              \Function{VERIFY-BIPARTITE-MATCHING}{G, k, M}
                \State $result \gets \textsc{True}$
                \If{$|M| \ne k$}
                  \State $result \gets$ \textsc{False}
                \Else
                  \ForAll{$v \in G.V$}
                    \State $incident \gets$ \textsc{False}
                    \ForAll{$(i, j) \in M$}
                      \If{($v = i$ or $v = j$) and $incident$}
                        \State $result \gets$ \textsc{False}
                      \ElsIf{$v = i$ or $v = j$}
                        \State $incident \gets$ \textsc{True}
                      \EndIf
                    \EndFor
                  \EndFor
                \EndIf
                \State \Return $result$
              \EndFunction
            \end{algorithmic}
          \end{algorithm}

      \end{enumerate}

    \item
      \begin{enumerate}
        \item True

          Assuming that the decision problem can verify in polynomial time,
          it can be used to verify possible solutions when computing sub problems.
        \item True

          Take a problem instance in $A$,
          reduce it to $B$ in polynomial time,
          then run the polynomial algorithm in $B$.
        \item False

          Since
          $\forall x \in \textbf{NP}, x \le_{P} 3SAT$,
          $3SAT \in \textbf{NP-Complete}$,
          $\forall x, y, z \in \textbf{NP}, x \le_{P} y \text{ and } y \le_{P} z \implies x \le_{P} z$ and
          $\textbf{P} \subseteq \textbf{NP}$ that conjecture would imply that
          any problem in \textbf{P} has no polynomial time algorithm.
          But this is clearly false, as $\textbf{P} \ne \emptyset$.
      \end{enumerate}
  \end{enumerate}
\end{document}
