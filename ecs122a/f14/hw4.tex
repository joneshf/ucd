\documentclass[12pt,letterpaper]{article}
\usepackage{amsmath}
\usepackage{amsfonts}
\usepackage{amsthm}
\usepackage{cancel}
\usepackage[margin=1in]{geometry}
\usepackage{titling}
\usepackage{multirow}
\usepackage{amssymb}
\usepackage{wasysym}
\usepackage{qtree}
\usepackage{algorithm}
\usepackage{algpseudocode}
\usepackage{tikz}
\usetikzlibrary{automata,positioning}

\newcommand{\lb}[0]{\text{lg}}

\setlength{\droptitle}{-10ex}

\preauthor{\begin{flushright}\large \lineskip 0.5em}
\postauthor{\par\end{flushright}}
\predate{\begin{flushright}\large}
\postdate{\par\end{flushright}}

\title{ECS 122A Homework 4\vspace{-2ex}}
\author{Hardy Jones\\
        999397426\\
        Professor Gysel\vspace{-2ex}}
\date{Fall 2014}

\begin{document}
  \maketitle

  \begin{enumerate}
    \item

      We can find a possible augmenting path if we modify both of the given algorithms.

      For DFS, we do not need to look at each vertex when starting the algorithm.
      We already know we want to find a path from $s$ to $t$, so we just need to check for paths from $s$.

      For DFS-VISIT, we should check if the current vertex is $t$.
      If it is, then we've constructed an $st$-path, and we can just return it.
      Otherwise, we continue as in the original algorithm.
      Each time we visit a new vertex, we add it to the end of the current path.
      If what was returned from our visitation is not an empty list,
      then we have an $st$-path, so we can return it.
      Otherwise, we continue searching for paths.
      If at the end of the search, we didn't find $t$, we return the empty list.

      \begin{algorithm}
        \begin{algorithmic}
          \Function{AUGMENTING-PATH}{$G_f$}
            \For{each $v \in G_f$}
              \State $v$.visited $\gets$ \textsc{False}
            \EndFor
            \State \Return ST-PATH($G_f$, $G_f.s$, APPEND(\textsc{Nil}, $G_f.s$))
          \EndFunction
        \end{algorithmic}
      \end{algorithm}

      \begin{algorithm}
        \begin{algorithmic}
          \Function{ST-PATH}{$G$, $v$, $l$}
            \State $v$.visited $\gets$ \textsc{True}
            \If{$v == G.t$}
              \State \Return $l$
            \EndIf
            \For{each $u \in G.$Adj($v$)}
              \If{$u$.visited $==$ \textsc{False}}
                \State $path? \gets$ ST-PATH($G$, $u$, APPEND($l$, $u$))
                \If{$path? \ne $ \textsc{Nil}}
                  \State \Return $path?$
                \EndIf
              \EndIf
            \EndFor
            \State \Return \textsc{Nil}
          \EndFunction
        \end{algorithmic}
      \end{algorithm}

    \item
      We can solve this following the maximum bipartite matching problem and modifying it a bit.

      We create the two sets, $L = \{d_1, d_2, \dots, d_n\}, R = \{p_1, p_2, \dots, p_m\}$.
      Create the edges $E = \{(d_i, p_j) \ | \ i \in 1, 2, \dots, n; j \in 1, 2, \dots, m; p_j \in S_i\}$.
      Give each edge in $E$ a weight of 1 to show that each doctor can treat one patient.

      Then, tack on two more vertices, $s, t$ as the source and sink.
      Construct an edge from $s$ to each $d_i \in L$ with weight $\infty$.
      Also construct an edge from each $p_j \in R$ to $t$ with weight $\infty$.

      This is exactly the construction for the maximum bipartite matching problem. However, we have an additional constraint. Each doctor can treat at most $k_i > 0$ patients.

      We can modify our graph to encode this information by taking each edge $(s, d_i) \forall d_i \in L$ and creating an additional vertex and edge between them. Give this new edge a weight of $k_i$.

      Now, we can run FORD-FULKERSON over our newly constructed graph,
      and we shall have a maximum flow which represents the most number of treated patients.

      An example is shown below:

      \begin{tikzpicture}[shorten >=1pt,node distance=2.25cm,on grid,auto]
         \node[state] (s)   {$s$};
         \node[state] (k_3) [below right=of s] {};
         \node[state] (k_2) [above=of k_3] {};
         \node[state] (k_4) [below=of k_3] {};
         \node[state] (k_1) [above=of k_2] {};
         \node[state] (d_1) [right=of k_1] {$d_1$};
         \node[state] (d_2) [right=of k_2] {$d_2$};
         \node[state] (d_3) [right=of k_3] {$d_3$};
         \node[state] (d_4) [right=of k_4] {$d_4$};
         \node[state] (p_1) [right=of d_1, xshift=2cm, yshift=2cm] {$p_1$};
         \node[state] (p_2) [below=2.75cm of p_1] {$p_2$};
         \node[state] (p_3) [below=2.25cm of p_2] {$p_3$};
         \node[state] (p_4) [below=2.25cm of p_3] {$p_4$};
         \node[state] (p_5) [below=2.25cm of p_4] {$p_5$};
         \node[state](t) [right=of p_3] {$t$};
          \path[->]
          (s) edge  node {$\infty$} (k_1)
              edge  node {$\infty$} (k_2)
              edge  node {$\infty$} (k_3)
              edge [below] node {$\infty$} (k_4)
          (k_1) edge  node {$k_1$} (d_1)
          (k_2) edge  node {$k_2$} (d_2)
          (k_3) edge  node {$k_3$} (d_3)
          (k_4) edge  node {$k_4$} (d_4)
          (d_1) edge  node {$1$} (p_1)
                edge  node {$1$} (p_3)
          (d_2) edge  node {$1$} (p_1)
                edge  node {$1$} (p_2)
                edge  node {$1$} (p_3)
                edge  node {$1$} (p_4)
          (d_3) edge  node {$1$} (p_1)
                edge  node {$1$} (p_3)
                edge  node {$1$} (p_5)
          (d_4) edge  node {$1$} (p_5)
          (p_1) edge  node {$\infty$} (t)
          (p_2) edge  node {$\infty$} (t)
          (p_3) edge  node {$\infty$} (t)
          (p_4) edge  node {$\infty$} (t)
          (p_5) edge [below] node {$\infty$} (t);
      \end{tikzpicture}

    \pagebreak

    \item
      \begin{enumerate}
        \item True.
          \[
            v(f) = f_{OUT}(s) - f_{IN}(s) = f_{OUT}(s) = f_{IN}(u_1) = f_{OUT}(u_1) = f_{IN}(u_2) = \dots = f_{IN}(t)
          \]
        \item False.
          Minimum cuts are equivalent to maximum flow
          by the Max-flow min-cut theorem.
        \item False.
          The while loop in FORD-FULKERSON has complexity $O(|E|)$ since the path can be found with either BFS or DFS, both of which have complexity $O(|E|)$.
        \item True.
          This satisfies both the capacity and the conservation constraints.
      \end{enumerate}
  \end{enumerate}
\end{document}
