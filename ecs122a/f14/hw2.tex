\documentclass[12pt,letterpaper]{article}
\usepackage{amsmath}
\usepackage{amsfonts}
\usepackage{amsthm}
\usepackage{cancel}
\usepackage[margin=1in]{geometry}
\usepackage{titling}
\usepackage{multirow}
\usepackage{amssymb}
\usepackage{wasysym}
\usepackage{qtree}
\usepackage{algpseudocode}

\newcommand{\lb}[0]{\text{lg}}

\setlength{\droptitle}{-10ex}

\preauthor{\begin{flushright}\large \lineskip 0.5em}
\postauthor{\par\end{flushright}}
\predate{\begin{flushright}\large}
\postdate{\par\end{flushright}}

\title{ECS 122A Homework 2\vspace{-2ex}}
\author{Hardy Jones\\
        999397426\\
        Professor Gysel\vspace{-2ex}}
\date{Fall 2014}

\begin{document}
  \maketitle

  \begin{enumerate}
    \item
      \begin{enumerate}
        \item
          \begin{algorithmic}
            \Function{SET-INTERSECTION}{$X$, $Y$}
              \State $X' \gets SORT(X)$
              \State $Y' \gets SORT(Y)$
              \State $i \gets 0$
              \State $j \gets 0$
              \State $intersection \gets$ EMPTY-SET

              \While{$i < LENGTH(X')$ AND $j < LENGTH(Y')$}
                \If{$X'[i] < Y'[j]$}
                  \State $i \gets i + 1$
                \ElsIf{$X'[i] = Y'[j]$}
                  \State $intersection \gets INSERT(intersection, X'[i])$
                \ElsIf{$X'[i] > Y'[j]$}
                  \State $j \gets j + 1$
                \EndIf
              \EndWhile

              \Return intersection

            \EndFunction
          \end{algorithmic}

          Assuming we have a sort function that runs in $O(n \lg n)$ time
          and an insertion function that runs in $O(1)$ time,
          this algorithm should take $O(n \lg n)$ time.

          We first sort the two sets in $O(n \lg n)$ time.

          When we're iterating over the two arrays, we take a maximum of $O(n)$ time
          as each iteration runs in $O(1)$ time.
          This is less than $O(n \lg n)$ time, so our upper bound has not changed.

          Thus, we have an intersection algorithm that runs in $O(n \lg n)$ time.

        \item
          \begin{algorithmic}
            \Function{SET-INTERSECTION}{$X$, $Y$}
              \State $intersection \gets$ EMPTY-SET
              \State $table \gets$ HASH-TABLE

              \ForAll{elements in $Y$}
                \State hash each element into $table$
              \EndFor

              \ForAll{elements in $X$}
                \If{SEARCH(table, element)}
                  \State $intersection \gets$ INSERT(intersection, element)
                \EndIf
              \EndFor

              \Return intersection

            \EndFunction
          \end{algorithmic}

          We need a table of size $k$ as this is the number of elements in the set $Y$.
          We should not need to worry about collision resolution as the set ensures that each element is distinct in $Y$,
          so no two elements should hash to the same location.

          Assuming we have a hashing function that runs in $O(1)$ time
          and an insert function that runs in $O(1)$ time,
          this algorithm should run $O(n+k)$ time.

          We need to iterate each element of $X = O(n)$ time,
          and each element of $Y = O(k)$.
          This combines to $O(n + k)$ time.
      \end{enumerate}

    \item
      Using the IRV defined in Theorem 11.2 $X_{ij} = I\{h(k_i) = h(k_j)\}$.

      We have $Pr\{h(k_i) = h(k_j)\} = \frac{1}{m}$,
      so $\text{E}[X_{ij}] = \frac{1}{m}$.

      So we have that the expected number of collisions is

      \begin{align*}
        E\left[\sum_{i=1}^n \sum_{j=1}^n X_{ij}\right] &= \sum_{i=1}^n E \left[\sum_{j=1}^n X_{ij} \right] \\
        &= \sum_{i=1}^n \sum_{j=1}^n E[X_{ij}] \\
        &= \sum_{i=1}^n \sum_{j=1}^n \frac{1}{m} \\
        &= \sum_{i=1}^n \frac{n}{m} \\
        &= \frac{n^2}{m} \\
      \end{align*}

      So the expected number of collisions is $\frac{n^2}{m}$.

    \item
      We can give a counter example with the following values:

      \begin{tabular}{r | c | c | c}
        $i$             & 1 & 2  & 3 \\
        \hline
        $p_i$           & 5 & 22 & 36 \\
        \hline
        $\frac{p_i}{i}$ & 5 & 11 & 12
      \end{tabular}

      If we wanted to maximize the profit on a rod that is $4$ inches long,
      the new algorithm would choose to cut the rod into two pieces:
      one of length $3$ inches, and one of length $1$ inch.

      This gives us $36 + 5 = 41$.

      However, the optimal solution is to cut the $4$ inch rod into two equal length pieces $2$ inches long.

      This gives us $22 + 22 = 44$.

      Since $44 > 41$, the new algorithm does not provide an optimal solution.

  \end{enumerate}
\end{document}
