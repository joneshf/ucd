\documentclass[12pt,letterpaper]{article}
\usepackage{amsmath}
\usepackage{amsfonts}
\usepackage{amsthm}
\usepackage{cancel}
\usepackage[margin=1in]{geometry}
\usepackage{titling}
\usepackage{multirow}
\usepackage{amssymb}
\usepackage{algorithm2e}
\usepackage{tikz}
\usetikzlibrary{automata,positioning}

\newcommand{\lb}[0]{\text{lg}}

\setlength{\droptitle}{-10ex}

\preauthor{\begin{flushright}\large \lineskip 0.5em}
\postauthor{\par\end{flushright}}
\predate{\begin{flushright}\large}
\postdate{\par\end{flushright}}

\title{ECS 122A Homework 1\vspace{-2ex}}
\author{Hardy Jones\\
        999397426\\
        Professor Bai\vspace{-2ex}}
\date{Spring 2014}

\begin{document}
  \maketitle

  \begin{enumerate}
    \item
      The entries of $BB^T$ are describe the edges in the graph.

      For each $i, j$ in $BB^T$
      \begin{itemize}
        \item if $i = j$ then it describes how many edges are at node $i$.
        \item if $i \ne j$ then it describes if there is an edge between node $i$ and node $j$.
      \end{itemize}
    \item
      \begin{enumerate}
        \item
          \begin{proof}
            Every edge $e$ in $G$ connects two vertices $v_i, v_j$.
            So each vertex $v_i$ adds $1$ to the total sum for every edge $e$ it has.
            This means that each edge $e$ is counted exactly twice in the sum.

            Thus,
              \[\sum_{v \in V}{d(v)} = 2|E|\]
          \end{proof}
        \item
          \begin{proof}
            Assume not, that is assume there is an odd number of vertices with odd degree.
            Call this set of vertices $v_o$, and the rest of the vertices $v_e$

            We can see that $\sum_{v \in v_e}{d(v)}$ is even,
            and that $\sum_{v \in v_o}{d(v)}$ is odd.
            Also we have, $\sum_{v \in (v_e \cup v_o)}{d(v)}$ is odd.
            N.B. $v_e \cup v_o = V$.
            but we know that $\sum_{v \in V}{d(v)} = 2|E|$,
            so we have a contradiction.

            So our assumption was wrong.

            Thus, there must be an even number of vertices with odd degree.
          \end{proof}
        \item
          No, there is no similar statement.
          We can have a simple directed graph from $A$ to $B$,
          but, there is only one $indegree$, which is odd.
      \end{enumerate}

    \item
      After running BFS on the graph, we have the following list as a result:
      \[[\infty, 3, 0, 2, 1, 1]\]
      where each entry corresponds to the list:
      \[[1,2,3,4,5,6]\]

      As a graph, the enumerated vertices are:

      \begin{tikzpicture}[shorten >=1pt,node distance=3cm,on grid,auto]
        \node[state] (1)   {$\infty$};
        \node[state] (2) [fill=black,text=white,right=of 1] {$3$};
        \node[state] (3) [fill=black,text=white,right=of 2] {$0$};
        \node[state] (4) [fill=black,text=white,below=of 1] {$2$};
        \node[state] (5) [fill=black,text=white,right=of 4] {$1$};
        \node[state] (6) [fill=black,text=white,right=of 5] {$1$};
        \path[->]
          (1) edge (2)
              edge (4)
          (2) edge (5)
          (3) edge (5)
              edge (6)
          (4) edge (2)
          (5) edge (4)
          (6) edge [loop right] ()
          ;
      \end{tikzpicture}

    \item
      After running BFS on the graph, we have the following list as a result:
      \[[4, 3, 1, 0, 5, 2, 2, 1]\]
      where each entry corresponds to the list:
      \[[r,s,t,u,v,w,x,y]\]

      As a graph, the enumerated vertices are:

      \begin{tikzpicture}[shorten >=1pt,node distance=3cm,on grid,auto]

        \tikzstyle{every state}=[fill=black,text=white]
        \node[state] (r)              {$4$};
        \node[state] (s) [right=of r] {$3$};
        \node[state] (t) [right=of s] {$1$};
        \node[state] (u) [right=of t] {$0$};
        \node[state] (v) [below=of r] {$5$};
        \node[state] (w) [right=of v] {$2$};
        \node[state] (x) [right=of w] {$2$};
        \node[state] (y) [right=of x] {$1$};
        \path
          (r) edge (s)
              edge (v)
          (s) edge (w)
          (t) edge (u)
              edge (w)
              edge (x)
          (u) edge (y)
          (w) edge (x)
          (x) edge (y)
          ;
      \end{tikzpicture}

  \end{enumerate}
\end{document}
