\documentclass[12pt,letterpaper]{article}
\usepackage{amsmath}
\usepackage{amsfonts}
\usepackage{amsthm}
\usepackage{amssymb}
\usepackage{cancel}
\usepackage[margin=1in]{geometry}
\usepackage{titling}
\usepackage{textcomp}
\usepackage{tikz}
\usepackage{tikz-qtree}
\usepackage{lplfitch}

\newcommand{\annotatebranch}[2]{\underset{#1}{#2}}
\newcommand{\closedbranch}[1]{\annotatebranch{\times}{#1}}
\newcommand{\openbranch}[1]{\annotatebranch{\circ}{#1}}

\setlength{\droptitle}{-10ex}

\preauthor{\begin{flushright}\large \lineskip 0.5em}
\postauthor{\par\end{flushright}}
\predate{\begin{flushright}\large}
\postdate{\par\end{flushright}}

\title{PHIL 112 Homework 4\vspace{-2ex}}
\author{Hardy Jones\\
        999397426\\
        Dr. Landry\vspace{-2ex}}
\date{Winter 2014}

\begin{document}
  \maketitle

  \begin{enumerate}
    \item Define
      \begin{enumerate}
        \item Theorem in \textit{PD}

          A sentence \textbf{P} of \textit{PL} is a theorem in \textit{PD} if and only if
          \textbf{P} is derivable in \textit{PD} from the empty set.

        \item Equivalence in \textit{PD}

          Sentences \textbf{P} and \textbf{Q} of \textit{PL} are equivalent in \textit{PD} if and only if
          \textbf{Q} is derivable in \textit{PD} from \{\textbf{P}\} and
          \textbf{P} is derivable in \textit{PD} from \{\textbf{Q}\}.
      \end{enumerate}

    \item Construct derivations that show each of the following:
      \begin{enumerate}
        \item
          $\{\neg (\forall x)(\exists y)Lxy, (\exists y)(\forall x)Lxy\}$ is inconsistent in \textit{PD}

          \fitchprf{\pline[1.]{\neg (\forall x)(\exists y)Lxy}[\textbf{Assum}] \\
                    \pline[2.]{(\exists y)(\forall x)Lxy}[\textbf{Assum}]}{
            \pline[3.]{(\exists x)\neg(\exists y)Lxy}[\textbf{QN 1}] \\
            \pline[4.]{(\exists x)(\forall y)\neg Lxy}[\textbf{QN 3}] \\
            \subproof{\pline[5.]{(\forall y)\neg Lay}[\textbf{Assum }$\exists$ \textbf{Elim}]}{
              \pline[6.]{(\forall y)\neg Lay}[\reit{5}]
            }
            \pline[7.]{(\forall y)\neg Lay}[\lexie{5}{6}] \\
            \subproof{\pline[8.]{(\forall x)Lxb}[\textbf{Assum }$\exists$ \textbf{Elim}]}{
              \pline[9.]{(\forall x)Lxb}[\reit{8}]
            }
            \pline[10.]{(\forall x)Lxb}[\lexie{8}{9}] \\
            \pline[11.]{\neg Lab}[\lalle{7}] \\
            \pline[12.]{Lab}[\lalle{10}]
          }

          From lines 11 and 12 we have $\neg Lab \land Lab$.

          Thus $\{\neg (\forall x)(\exists y)Lxy, (\exists y)(\forall x)Lxy\}$ is inconsistent in \textit{PD}.
        \item
          $\{((\exists x)Fx \lor (\exists x)Gx) \lif (\exists x)(Fx \lor Gx)\}$ is a theorem in \textit{PD}

          \fitchprf{}{
            \subproof{\pline[1.]{(\exi x)Fx \lor (\exi x)Gx}[\textbf{Assum}]}{
              \subproof{\pline[2.]{(\exi x)Fx}[\textbf{Assum}]}{
                \subproof{\pline[3.]{Fa}[\textbf{Assum}]}{
                  \pline[4.]{Fa \lor Ga}[\lori{3}] \\
                  \pline[5.]{(\exi x)(Fx \lor Gx)}[\lexii{4}]
                }
                \pline[6.]{(\exi x)(Fx \lor Gx)}[\lexie{2}{3--5}]
              }
              \subproof{\pline[7.]{(\exi x)Gx}[\textbf{Assum}]}{
                \subproof{\pline[8.]{Gb}[\textbf{Assum}]}{
                  \pline[9.]{Fb \lor Gb}[\lori{8}] \\
                  \pline[10.]{(\exi x)(Fx \lor Gx)}[\lexii{9}]
                }
                \pline[11.]{(\exi x)(Fx \lor Gx)}[\lexie{7}{8--10}]
              }
              \pline[12.]{(\exi x)(Fx \lor Gx)}[\lore{1}{2--6,7--11}]
            }
            \pline[13.]{((\exi x)Fx \lor (\exi x)Gx) \lif (\exi x)(Fx \lor Gx)}[\lifi{1--12}]
          }
      \end{enumerate}

    \pagebreak

    \item
      Symbolize Casino Slim's reasoning and
      construct a derivation in \textit{PD+}
      showing that the symbolized argument is valid in \textit{PD+}.

      \fitchprf{\pline[1.]{(\lall x)(\lall y)[(Fx \land Sy) \lif Bxy]}[\textbf{Assum}] \\
                \pline[2.]{(Fh \lor Ff) \land Ss}[\textbf{Assum}]}{
        \pline[3.]{Fh \lor Ff}[\lande{2}] \\
        \pline[4.]{Ss}[\lande{2}] \\
        \pline[5.]{(\lall y)[(Fx \land Sy) \lif Bhy]}[\lalle{1}] \\
        \pline[6.]{(Fh \land Ss) \lif Bhs}[\lalle{5}] \\
        \pline[7.]{\lnot(Fh \land Ss) \lor Bhs}[\textbf{Impl:} 5] \\
        \subproof{\pline[8.]{Fh \lor Ff}[\textbf{Assum}]}{
          \subproof{\pline[9.]{Fh}[\textbf{Assum}]}{
            \pline[10.]{(\exi x)Fx}[\lexii{9}]
          }
          \subproof{\pline[11.]{Ff}[\textbf{Assum}]}{
            \pline[12.]{(\exi x)Fx}[\lexii{11}]
          }
          \pline[13.]{(\exi x)Fx}[\lore{8}{9--10}{11--12}]
        }
        \pline[14.]{(\exi x)Fx}[\lexii{13}] \\
        \subproof{\pline[15.]{Fh}[\textbf{Assum}]}{
          \pline[16.]{Fh}[\reit{15}]
        }
        \pline[17.]{Fh}[\lexie{14}{15--16}] \\
        \pline[18.]{Fh \land Ss}[\landi{4,17}] \\
        \pline[19.]{Bhs}[\life{6}{18}] \\
        \pline[20.]{Bhs \lor Bfs}[\lori{19}] \\
        \pline[21.]{(\exi x)(Bhx \lor Bfx)}[\lexii{20}]
      }
  \end{enumerate}
\end{document}
