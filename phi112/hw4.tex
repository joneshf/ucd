\documentclass[12pt,letterpaper]{article}
\usepackage{amsmath}
\usepackage{amsfonts}
\usepackage{amsthm}
\usepackage{amssymb}
\usepackage{cancel}
\usepackage[margin=1in]{geometry}
\usepackage{titling}
\usepackage{textcomp}
\usepackage{tikz}
\usepackage{tikz-qtree}
\usepackage{lplfitch}

\newcommand{\annotatebranch}[2]{\underset{#1}{#2}}
\newcommand{\closedbranch}[1]{\annotatebranch{\times}{#1}}
\newcommand{\openbranch}[1]{\annotatebranch{\circ}{#1}}

\setlength{\droptitle}{-10ex}

\preauthor{\begin{flushright}\large \lineskip 0.5em}
\postauthor{\par\end{flushright}}
\predate{\begin{flushright}\large}
\postdate{\par\end{flushright}}

\title{PHIL 112 Homework 4\vspace{-2ex}}
\author{Hardy Jones\\
        999397426\\
        Dr. Landry\vspace{-2ex}}
\date{Winter 2014}

\begin{document}
  \maketitle

  \begin{enumerate}
    \item Define
      \begin{enumerate}
        \item Theorem in \textit{PD}

          A sentence \textbf{P} of \textit{PL} is a theorem in \textit{PD} if and only if
          \textbf{P} is derivable in \textit{PD} from the empty set.

        \item Equivalence in \textit{PD}

          Sentences \textbf{P} and \textbf{Q} of \textit{PL} are equivalent in \textit{PD} if and only if
          \textbf{Q} is derivable in \textit{PD} from \{\textbf{P}\} and
          \textbf{P} is derivable in \textit{PD} from \{\textbf{Q}\}.
      \end{enumerate}

    \item Construct derivations that show each of the following:
      \begin{enumerate}
        \item
          $\{\neg (\forall x)(\exists y)Lxy, (\exists y)(\forall x)Lxy\}$ is inconsistent in \textit{PD}

          \fitchprf{\pline[1.]{\neg (\forall x)(\exists y)Lxy}[\textbf{Assum}] \\
                    \pline[2.]{(\exists y)(\forall x)Lxy}[\textbf{Assum}]}{
            \pline[3.]{(\exists x)\neg(\exists y)Lxy}[\textbf{QN 1}] \\
            \pline[4.]{(\exists x)(\forall y)\neg Lxy}[\textbf{QN 3}] \\
            \subproof{\pline[5.]{(\forall y)\neg Lay}[\textbf{Assum }$/ \exists$ \textbf{Elim}]}{
              \pline[6.]{(\forall y)\neg Lay}[\reit{5}]
            }
            \pline[7.]{(\forall y)\neg Lay}[\lexie{5}{6}] \\
            \subproof{\pline[8.]{(\forall x)Lxb}[\textbf{Assum }$/ \exists$ \textbf{Elim}]}{
              \pline[9.]{(\forall x)Lxb}[\reit{8}]
            }
            \pline[10.]{(\forall x)Lxb}[\lexie{8}{9}] \\
            \pline[11.]{\neg Lab}[\lalle{7}] \\
            \pline[12.]{Lab}[\lalle{10}]
          }

          From lines 11 and 12 we have $\neg Lab \land Lab$.

          Thus $\{\neg (\forall x)(\exists y)Lxy, (\exists y)(\forall x)Lxy\}$ is inconsistent in \textit{PD}.
        \pagebreak
        \item
          $\{((\exists x)Fx \lor (\exists x)Gx) \lif (\exists x)(Fx \lor Gx)\}$ is a theorem in \textit{PD}

          \fitchprf{}{
            \subproof{\pline[1.]{(\exi x)Fx \lor (\exi x)Gx}[\textbf{Assum}]}{
              \subproof{\pline[2.]{(\exi x)Fx}[\textbf{Assum}]}{
                \subproof{\pline[3.]{Fa}[\textbf{Assum}]}{
                  \pline[4.]{Fa \lor Ga}[\lori{3}] \\
                  \pline[5.]{(\exi x)(Fx \lor Gx)}[\lexii{4}]
                }
                \pline[6.]{(\exi x)(Fx \lor Gx)}[\lexie{2}{3--5}]
              }
              \subproof{\pline[7.]{(\exi x)Gx}[\textbf{Assum}]}{
                \subproof{\pline[8.]{Gb}[\textbf{Assum}]}{
                  \pline[9.]{Fb \lor Gb}[\lori{8}] \\
                  \pline[10.]{(\exi x)(Fx \lor Gx)}[\lexii{9}]
                }
                \pline[11.]{(\exi x)(Fx \lor Gx)}[\lexie{7}{8--10}]
              }
              \pline[12.]{(\exi x)(Fx \lor Gx)}[\lore{1}{2--6,7--11}]
            }
            \pline[13.]{((\exi x)Fx \lor (\exi x)Gx) \lif (\exi x)(Fx \lor Gx)}[\lifi{1--12}]
          }
      \end{enumerate}

    \pagebreak

    \item
      Symbolize Casino Slim's reasoning and
      construct a derivation in \textit{PD+}
      showing that the symbolized argument is valid in \textit{PD+}.

      \fitchprf{\pline[1.]{(\lall x)(\lall y)[(Fx \land Sy) \lif Bxy]}[\textbf{Assum}] \\
                \pline[2.]{(Fh \lor Ff) \land Ss}[\textbf{Assum}]}{
        \pline[3.]{Fh \lor Ff}[\lande{2}] \\
        \pline[4.]{Ss}[\lande{2}] \\
        \pline[5.]{(\lall y)[(Fx \land Sy) \lif Bhy]}[\lalle{1}] \\
        \pline[6.]{(Fh \land Ss) \lif Bhs}[\lalle{5}] \\
        \pline[7.]{\lnot(Fh \land Ss) \lor Bhs}[\textbf{Impl:} 5] \\
        \subproof{\pline[8.]{Fh \lor Ff}[\textbf{Assum}]}{
          \subproof{\pline[9.]{Fh}[\textbf{Assum}]}{
            \pline[10.]{(\exi x)Fx}[\lexii{9}]
          }
          \subproof{\pline[11.]{Ff}[\textbf{Assum}]}{
            \pline[12.]{(\exi x)Fx}[\lexii{11}]
          }
          \pline[13.]{(\exi x)Fx}[\lore{8}{9--10}{11--12}]
        }
        \pline[14.]{(\exi x)Fx}[\lexii{13}] \\
        \subproof{\pline[15.]{Fh}[\textbf{Assum}]}{
          \pline[16.]{Fh}[\reit{15}]
        }
        \pline[17.]{Fh}[\lexie{14}{15--16}] \\
        \pline[18.]{Fh \land Ss}[\landi{4,17}] \\
        \pline[19.]{Bhs}[\life{6}{18}] \\
        \pline[20.]{Bhs \lor Bfs}[\lori{19}] \\
        \pline[21.]{(\exi x)(Bhx \lor Bfx)}[\lexii{20}]
      }

    \item
      Show that the following argument is valid.

      \begin{tabular}{l}
        $(\forall x)[(\exists y)(Byb \land Lxyb) \lif Fx]$ \\
        $(\exists x)(Cxb \land Lxab)$ \\
        \hline
        $(\forall x)(Cxb \lif \neg Fx) \lif \neg Bab$
      \end{tabular}

      \fitchprf{\pline[1.]{(\forall x)[(\exists y)(Byb \land Lxyb) \lif Fx]}[\textbf{Assum}] \\
                \pline[2.]{(\exists x)(Cxb \land Lxab)}[\textbf{Assum}]}{
        \subproof{\pline[3.]{(\forall x)(Cxb \lif \neg Fx)}[\textbf{Assum }$/ \lif$ \textbf{Intro}]}{
          \subproof{\pline[4.]{Ccb \land Lcab}[\textbf{Assum }$/ \exists$ \textbf{Elim}]}{
            \pline[5.]{Ccb}[$\land $\textbf{Elim 4}] \\
            \pline[6.]{Lcab}[$\land $\textbf{Elim 4}] \\
            \pline[7.]{Ccb \lif \neg Fc}[\lalle{3}] \\
            \pline[8.]{\neg Fc}[\textbf{MP} 5, 7] \\
            \pline[9.]{(\exists y)(Byb \land Lcyb) \lif Fc}[\lalle{1}] \\
            \pline[10.]{\neg (\exists y)(Byb \land Lcyb)}[\textbf{MT} 8, 9] \\
            \pline[11.]{(\forall y)\neg (Byb \land Lcyb)}[\textbf{QN} 10] \\
            \pline[12.]{\neg (Bab \land Lcab)}[\lalle{11}] \\
            \pline[13.]{\neg Bab \lor Lcab}[\textbf{DeM} 12] \\
            \pline[14.]{\neg Bab}[\textbf{DS} 6, 13]
          }
          \pline[15.]{\neg Bab}[\lexie{4}{5,14}]
        }
        \pline[16.]{(\forall x)(Cxb \lif \neg Fx) \lif \neg Bab}[$\lif$ \textbf{Intro} 3, 4-15]
      }

      Thus the argument is valid.

    \item
      Suppose that a set is inconsistent in PD.
      Is an argument that has the sentences in the set as premises valid in PD?

      Yes, the argument would still be valid.
      Validity only states that the conclusion must be derivable from the premises.
      Since a conclusion can still be derived, it can still be valid.

    \item
      Show in PDE that $\{a = b, b = c\} \vdash c = a$

      \fitchprf{\pline[1.]{a = b}[\textbf{Assum}] \\
                \pline[2.]{b = c}[\textbf{Assum}]}{
        \pline[3.]{a = c}[1, 2 $=$\textbf{Elim}] \\
        \pline[4.]{c = b}[1, 3 $=$\textbf{Elim}] \\
        \pline[5.]{c = a}[1, 4 $=$\textbf{Elim}]
      }

    \item
      Complete the following definition in PDE:

      $(\forall x)[Rf(x)g(x) \equiv Rg(x)f(x)]$

      \fitchprf{\pline[.]{(\forall x)(\forall y)(Ryx \lif Rxy)}[\textbf{Assum}]}{
        \subproof{\pline[.]{Rg(a)f(a)}[\textbf{Assum }$/ \equiv$ \textbf{Intro}]}{
          \pline[.]{(\forall y)(Ryf(a) \lif Rf(a)y)}[\lalle{1}] \\
          \pline[.]{Rg(a)f(a) \lif Rf(a)g(a)}[\lalle{1}] \\
          \pline[.]{Rf(a)g(a)}[\textbf{MP} 2, 4]
        }
        \subproof{\pline[.]{Rf(a)g(a)}[\textbf{Assum }$/ \equiv$ \textbf{Intro}]}{
          \pline[.]{(\forall y)(Ryg(a) \lif Rf(a)y)}[\lalle{1}] \\
          \pline[.]{Rf(a)g(a) \lif Rg(a)f(a)}[\lalle{1}] \\
          \pline[.]{Rg(a)f(a)}[\textbf{MP} 6, 8]
        }
        \pline[.]{Rf(a)g(a) \equiv Rg(a)f(a)}[$\equiv $ \textbf{Intro}] \\
        \pline[.]{(\forall x)[Rf(x)g(x) \equiv Rg(x)f(x)]}[\lalli {10}]
      }
  \end{enumerate}
\end{document}
