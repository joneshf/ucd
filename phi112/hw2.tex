\documentclass[12pt,letterpaper]{article}
\usepackage{amsmath}
\usepackage{amsfonts}
\usepackage{amsthm}
\usepackage{cancel}
\usepackage[margin=1in]{geometry}
\usepackage{titling}
\usepackage{textcomp}

\setlength{\droptitle}{-10ex}

\preauthor{\begin{flushright}\large \lineskip 0.5em}
\postauthor{\par\end{flushright}}
\predate{\begin{flushright}\large}
\postdate{\par\end{flushright}}

\title{PHIL 112 Homework 1\vspace{-2ex}}
\author{Hardy Jones\\
        999397426\\
        Dr. Landry\vspace{-2ex}}
\date{Winter 2014}

\begin{document}
  \maketitle

  \begin{enumerate}
    \item Define
      \begin{enumerate}
        \item Quantificational truth

          There are three values for quantificational truth.
          Sentences of Pl can be:
          quantificationally true, quantificationally false, or quantificationally indeterminate
          \begin{itemize}
            \item
              A sentence \textbf{P} of PL is quantificationally true if and only if
              \textbf{P} is true for all interpretations.
            \item
              A sentence \textbf{P} of PL is quantificationally false if and only if
              \textbf{P} is false for all interpretations.
            \item
              A sentence \textbf{P} of PL is quantificationally indeterminate if and only if
              \textbf{P} is neither quantificationally true nor quantificationally false.
          \end{itemize}
        \item Quantificational equivalence

          Two sentences \textbf{P} and \textbf{Q} of PL are quantificationally equivalent if and only if
          there is no interpretation on which \textbf{P} and \textbf{Q} have different quantificational truth values.
      \end{enumerate}
    \item Symbolize the following sentences, using the symbolization key given.
      UD: Everything
      Ixy: x is identitcal to y
      Mx: x is a math problem
      Lx: x is a logic problem
      Sxy: x is easier to solve than is y
      \begin{enumerate}
        \item Math problems are easier to solve than logic problems.
        \item Some logic problems are easier to solve than are others.
      \end{enumerate}
    \item Determine the truth values of the following with the given interpretation.
      UD: Set of people and planets
      Hx: x is a human being
      Lxy: x lives on y
      Px: x is a planet
      Sx: x is in the solar system
      e: earth

      \begin{enumerate}
        \item $(\exists x)(\forall y)[Hx \land (Py \supset Lxy)]$
        \item $(Pe \land Se) \land (\exists w)(Hw \land Lwe)$
        \item $(\exists x)(Hx \land Lxe) \equiv Se$
        \item $(\forall x)(\forall y)[(Hx \land Py) \supset \neg x \equiv y]$
      \end{enumerate}
    \item Determine the truth values for the following with the given interpretation.
      UD: Set of people
      Mx: x is a male
      Sx: x is a scientist
      Oxy: x is older than y
      a: Albert Einstein

      \begin{enumerate}
        \item $(\forall x)(Mx \land Sx) \supset \neg Sa$
        \item $(\forall x)[(Mx \land Sx) \supset \neg Sa]$
        \item $(\forall x)(\forall y)(Oxy \supset \neg Oyx)$
        \item $(\forall y)[\neg y \equiv [a \land (My \land Sy)]]$
      \end{enumerate}
    \item Construct an expansion of each of the following sentences for the set of constants $\{ a, s \}$
      \begin{enumerate}
        \item $(\forall x)(\exists y)Sxy \land D$
        \item $(\exists x)Fx \equiv \neg (\forall y)Gsy$
      \end{enumerate}
    \item
      Show that the following argument is not quantificationally valid.
      Use both the interpretation method and the truth-functional expansion method.

      \begin{tabular}{l}
        $(\forall y)(\exists x)(Py \supset Cyx)$ \\
        $(\exists x)Cxx$ \\
        \hline
        $(\exists x)\neg Cxx$
      \end{tabular}
    \item
      Are the sentences $(\forall y)Fy$ and $\neg (\exists y) \neg Fy$
      quantificationally equivalent?
  \end{enumerate}
\end{document}
