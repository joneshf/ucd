\documentclass[12pt,letterpaper]{article}
\usepackage{amsmath}
\usepackage{amsfonts}
\usepackage{amsthm}
\usepackage{amssymb}
\usepackage{cancel}
\usepackage[margin=1in]{geometry}
\usepackage{titling}
\usepackage{textcomp}

\setlength{\droptitle}{-10ex}

\preauthor{\begin{flushright}\large \lineskip 0.5em}
\postauthor{\par\end{flushright}}
\predate{\begin{flushright}\large}
\postdate{\par\end{flushright}}

\title{PHIL 112 Homework 1\vspace{-2ex}}
\author{Hardy Jones\\
        999397426\\
        Dr. Landry\vspace{-2ex}}
\date{Winter 2014}

\begin{document}
  \maketitle

  \begin{enumerate}
    \item Define
      \begin{enumerate}
        \item Quantificational truth

          There are three values for quantificational truth.
          Sentences of Pl can be:
          quantificationally true, quantificationally false, or quantificationally indeterminate
          \begin{itemize}
            \item
              A sentence \textbf{P} of PL is quantificationally true if and only if
              \textbf{P} is true for all interpretations.
            \item
              A sentence \textbf{P} of PL is quantificationally false if and only if
              \textbf{P} is false for all interpretations.
            \item
              A sentence \textbf{P} of PL is quantificationally indeterminate if and only if
              \textbf{P} is neither quantificationally true nor quantificationally false.
          \end{itemize}
        \item Quantificational equivalence

          Two sentences \textbf{P} and \textbf{Q} of PL are quantificationally equivalent if and only if
          there is no interpretation on which \textbf{P} and \textbf{Q} have different quantificational truth values.
      \end{enumerate}

    \item Symbolize the following sentences, using the symbolization key given.

      UD: Everything \\
      Ixy: x is identitcal to y \\
      Mx: x is a math problem \\
      Lx: x is a logic problem \\
      Sxy: x is easier to solve than is y
      \begin{enumerate}
        \item Math problems are easier to solve than logic problems.

          $(\forall x)(\forall y)[(Mx \land Ly) \supset Sxy]$

        \item Some logic problems are easier to solve than are others.

          $(\exists x)(\exists y)(Lx \land Ly \land Sxy)$
      \end{enumerate}
    \item Determine the truth values of the following with the given interpretation.

      UD: Set of people and planets \\
      Hx: x is a human being \\
      Lxy: x lives on y \\
      Px: x is a planet \\
      Sx: x is in the solar system \\
      e: earth

      \begin{enumerate}
        \item $(\exists x)(\forall y)[Hx \land (Py \supset Lxy)]$

          This is false
          for it states that some human lives on every planet, yet no humans live on Mercury.
        \item $(Pe \land Se) \land (\exists w)(Hw \land Lwe)$

          This is true
          for it states that Earth is a planet in the solar system and there is some human living on earth.
          You can verify this by being alive.
        \item $(\exists x)(Hx \land Lxe) \equiv Se$

          This is indeterminate.
          It states that there is some human living on earth if and only if the earth is in the solar system.
          There are some creation myths which believe the earth was created with the sun and humans, who have existed on it the whole time.
          Science tells us the earth is much older than the human species, yet it has been a part of the solar system long before the existence of humans, and would continue to be part of the solar system if our species became extinct.
          Since there is at least one interpretation where this is true and another where it is false,
          this sentence must be indeterminate.
        \item $(\forall x)(\forall y)[(Hx \land Py) \supset \neg x \equiv y]$

          This is true.
          It states that no human is a planet.
      \end{enumerate}
    \item Determine the truth values for the following with the given interpretation.

      UD: Set of people \\
      Mx: x is a male \\
      Sx: x is a scientist \\
      Oxy: x is older than y \\
      a: Albert Einstein

      \begin{enumerate}
        \item $(\forall x)(Mx \land Sx) \supset \neg Sa$

          This states that if all males are scientists,
          then Albert Einstein is not a scientist.

          This is true because the antecedent cannot be made true,
          since there will always be some male that is not a scientist.
        \item $(\forall x)[(Mx \land Sx) \supset \neg Sa]$

          This states that Albert Einstein is not a male scientist.

          This is false because we know that Albert Einstein was male and a scientist.
        \item $(\forall x)(\forall y)(Oxy \supset \neg Oyx)$

          This states that no person is older than another person that is older than themselves.
          This is true.
        \item $(\forall y)[\neg (y = a) \land (My \land Sy)]$

          This states that everyone who isn't Albert Einstein is a male scientist.

          This is clearly false because women are people also.
      \end{enumerate}
    \item Construct an expansion of each of the following sentences for the set of constants $\{ a, s \}$
      \begin{enumerate}
        \item $(\forall x)(\exists y)Sxy \land D$

          $[(\exists y)Say \land (\exists y)Ssy] \land D$ \\
          $[(Saa \lor Sas) \land (Ssa \lor Sss)] \land D$
        \item $(\exists x)Fx \equiv \neg (\forall y)Gsy$

          $(Fa \lor Fs) \equiv \neg (\forall y)Gsy$ \\
          $(Fa \lor Fs) \equiv \neg (Gsa \land Gss)$
      \end{enumerate}
    \item
      Show that the following argument is not quantificationally valid.
      Use both the interpretation method and the truth-functional expansion method.

      \begin{tabular}{l}
        $(\forall y)(\exists x)(Py \supset Cyx)$ \\
        $(\exists x)Cxx$ \\
        \hline
        $(\exists x)\neg Cxx$
      \end{tabular}

      We can use the interpretation of natural numbers.

      UD: $\mathbb{N}$
      Px: x is positive.
      Cxy: x is equal to y.

      Now since we're dealing with the naturals, every element of our set is positive.
      We also know that for every number we can find a number that is equal to it, namely itself.

      So we can see that both premises are true, however the conclusion is false.

      $\therefore$ this argument is not quantificationally valid.

      Using expansion it helps to rewrite the argument.

      $[(\forall y)(\exists x)(Py \supset Cyx) \land (\exists x)Cxx] \supset (\exists x)\neg Cxx$

      UD: $\{ 1, 2 \}$
      Px: x is positive.
      Cxy: x is equal to y.
      a: 1
      b: 2

      $[[(\exists x)(Pa \supset Cax) \land (\exists x)(Pb \supset Cbx)] \land (\exists x)Cxx] \supset (\exists x)\neg Cxx$ \\
      $[[[(Pa \supset Caa) \lor (Pa \supset Cab)] \land [(Pb \supset Cba) \lor (Pb \supset Cbb)]] \land (Caa \lor Cbb)] \supset \neg (Caa \lor Cbb)$ \\
      From this we can see that we necessarily have the antecedent as true, and the consequent as false.
      $\therefore$ This argument is quantificationally invalid by expansion.
    \item
      Are the sentences $(\forall y)Fy$ and $\neg (\exists y) \neg Fy$
      quantificationally equivalent?

      Yes these two terms are quantificationally equivalent.
      It is trivial to see once we rewrite one of the terms.

      \begin{align*}
        \neg (\exists y) \neg Fy &:: (\forall y) \neg \neg Fy \\
        &:: (\forall y) Fy
      \end{align*}

      $\therefore$ These two sentences are quantificationally equivalent.
  \end{enumerate}
\end{document}
