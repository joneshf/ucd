\documentclass[12pt,letterpaper]{article}
\usepackage{amsmath}
\usepackage{amsfonts}
\usepackage{amsthm}
\usepackage{amssymb}
\usepackage{cancel}
\usepackage[margin=1in]{geometry}
\usepackage{titling}
\usepackage{textcomp}
\usepackage{tikz}
\usepackage{tikz-qtree}

\newcommand{\annotatebranch}[2]{\underset{#1}{#2}}
\newcommand{\closedbranch}[1]{\annotatebranch{\times}{#1}}
\newcommand{\openbranch}[1]{\annotatebranch{\circ}{#1}}

\setlength{\droptitle}{-10ex}

\preauthor{\begin{flushright}\large \lineskip 0.5em}
\postauthor{\par\end{flushright}}
\predate{\begin{flushright}\large}
\postdate{\par\end{flushright}}

\title{PHIL 112 Homework 3\vspace{-2ex}}
\author{Hardy Jones\\
        999397426\\
        Dr. Landry\vspace{-2ex}}
\date{Winter 2014}

\begin{document}
  \maketitle

  \begin{enumerate}
    \item Explicate in terms of open and/or closed truth trees.
      \begin{enumerate}
        \item Quantificational validity

          An argument of \textbf{PL} is quantificationally valid if and only if
          the set consisting of the premises and the negation of the conclusion
          of the argument has a closed truth tree.

        \item Quantificational equivalence

          Two sentences \textbf{P} and \textbf{Q} of \textbf{PL}
          are quantificationally equivalent if and only if
          the set $\{\neg (\mathbf{P} \equiv \mathbf{Q})\}$ has a closed truth tree.
      \end{enumerate}
    \item
      Use the tree method to show whether:
      \begin{enumerate}
        \item is quantificationally true
        \item is quantificationally valid
        \item sentences are quantificationally equivalent
        \item quantificational entailment holds
      \end{enumerate}
      \begin{enumerate}
        \item
          $[Fa \supset (\forall x)Fx] \supset [(\exists x)Fx \supset (\forall x) Fx]$

          \begin{tikzpicture}[sibling distance=0.5cm]
            \begin{scope}[xshift=-1.5in]
              \tikzset{edge from parent/.style={edge from parent path={(\tikzparentnode) -- (\tikzchildnode)}}, every tree node/.style={text width=5em, align=left}}
              \Tree
                [. 1
                [. 2
                [. 3
                [. 4
                [. 5
                [. 6
                [. 7
                [. 8
                [. 9 ]]]]]]]]]
            \end{scope}
            \Tree
              [. {$\neg [[Fa \supset (\forall x)Fx] \supset [(\exists x)Fx \supset (\forall x) Fx]] \checkmark$}
              [. {$Fa \supset (\forall x)Fx \checkmark$}
              [. {$\neg [(\exists x)Fx \supset (\forall x) Fx] \checkmark$}
              [. {$(\exists x)Fx \checkmark$}
              [. {$\neg (\forall x) Fx \checkmark$}
              [. {$(\exists x) \neg Fx \checkmark$}
              [. {$Fb$}
              [. {$Fc$}
                [. {$\openbranch{\neg Fa}$} ]
                [. {$\openbranch{(\forall x)Fx}$} ]]]]]]]]]
            \begin{scope}[xshift=3.5in]
              \tikzset{edge from parent/.style={edge from parent path={(\tikzparentnode) -- (\tikzchildnode)}}, every tree node/.style={text width=5em, align=left}}
              \Tree
                [. SM
                [. {1 $\neg \supset$ D}
                [. {1 $\neg \supset$ D}
                [. {3 $\neg \supset$ D}
                [. {3 $\neg \supset$ D}
                [. {5 $\neg \forall$ D}
                [. {4 $\exists$ D}
                [. {6 $\exists$ D}
                [. {2 $\supset$ D} ]]]]]]]]]
            \end{scope}
          \end{tikzpicture}

          Since this tree is not closed, the sentence is not quantificationally true.

        \item
          \begin{tabular}{l}
            $(\forall x)[Nx \supset (\exists y)Rxy]$ \\
            $\neg (\exists x)Rxx \land Na$ \\
            \hline
            $(\exists y)Ray$
          \end{tabular}

          \begin{tikzpicture}[sibling distance=0.5cm]
            \begin{scope}[xshift=-1.5in]
              \tikzset{edge from parent/.style={edge from parent path={(\tikzparentnode) -- (\tikzchildnode)}}, every tree node/.style={text width=5em, align=left}}
              \Tree
                [. 1
                [. 2
                [. 3
                [. 4
                [. 5
                [. 6
                [. 7
                [. 8
                [. 9
                [. 10 ]]]]]]]]]]
            \end{scope}
            \Tree
              [. {$(\forall x)[Nx \supset (\exists y)Rxy]$}
              [. {$\neg (\exists x)Rxx \land Na \checkmark$}
              [. {$\neg (\exists y)Ray \checkmark$}
              [. {$\neg (\exists x)Rxx \checkmark$}
              [. {$Na$}
              [. {$(\forall y)\neg Ray$}
              [. {$Na \supset (\exists y)Ray \checkmark$}
              [. {$\closedbranch{\neg Na}$} ]
              [. {$(\exists y)Ray \checkmark$}
                [. {$Rab$}
                [. {$\closedbranch{\neg Rab}$} ]]]]]]]]]]
            \begin{scope}[xshift=3.5in]
              \tikzset{edge from parent/.style={edge from parent path={(\tikzparentnode) -- (\tikzchildnode)}}, every tree node/.style={text width=5em, align=left}}
              \Tree
                [. SM
                [. SM
                [. SM
                [. {2 $\land$ D}
                [. {2 $\land$ D}
                [. {3 $\neg \exists$ D}
                [. {1 $\forall$ D}
                [. {7 $\supset$ D}
                [. {8 $\exists$ D}
                [. {9 $\forall$ D} ]]]]]]]]]]
            \end{scope}
          \end{tikzpicture}

          Since this tree is closed, the argument is quantificationally valid.

          One interpretation is: \\
          UD: The natural numbers \\
          Nx: x is positive \\
          Rxy: x is less than y \\
          a: 1

        \pagebreak

        \item $[(\forall x)Fx \supset Ga] \equiv (\exists x)(Fx \supset Ga)$

          \begin{tikzpicture}[sibling distance=0.5cm]
            \begin{scope}[xshift=-2.5in]
              \tikzset{edge from parent/.style={edge from parent path={(\tikzparentnode) -- (\tikzchildnode)}}, every tree node/.style={text width=5em, align=left}}
              \Tree
                [. 1
                [. 2
                [. 3
                [. 4
                [. 5
                [. 6
                [. 7
                [. 8
                [. 9
                [. 10
                [. 11
                [. 12
                [. 13
                [. 14
                [. 15
                [. 16
                [. 17
                [. 18 ]]]]]]]]]]]]]]]]]]
            \end{scope}
            \Tree
              [. {$\neg [[(\forall x)Fx \supset Ga] \equiv (\exists x)(Fx \supset Ga)] \checkmark$}
              [. {$(\forall x)Fx \supset Ga$}
                [. {$\neg (\exists x)(Fx \supset Ga) \checkmark$}
                [. {$\ $}
                [. {$\ $}
                [. {$\ $}
                [. {$\ $}
                [. {$\ $}
                [. {$(\forall x)\neg(Fx \supset Ga)$}
                [. {$\neg(\forall x)Fx \checkmark$}
                [. {$\ $}
                [. {$\ $}
                [. {$\ $}
                [. {$(\exists x)\neg Fx \checkmark$}
                [. {$\neg Fb$}
                [. {$\neg(Fb \supset Ga) \checkmark$}
                [. {$Fb$}
                [. {$\closedbranch{\neg Ga}$} ]]]]]]]]]
                [. {$Ga$}
                  [. {$\neg(Fb \supset Ga) \checkmark$}
                  [. {$Fb$}
                  [. {$\closedbranch{\neg Ga}$} ]]]]]]]]]]]]
              [. {$\neg (\forall x)Fx \supset Ga \checkmark$}
                [. {$(\exists x)(Fx \supset Ga) \checkmark$}
                [. {$Fb \supset Ga \checkmark$}
                [. {$\neg \neg (\forall x)Fx \checkmark$}
                  [. {$(\forall x)Fx$}
                  [. {$Fb$}
                  [. {$\closedbranch{\neg Fb}$} ]
                  [. {$\openbranch{Ga}$} ]]]]
                  [. {$Ga$}
                  [. {$\ $}
                  [. {$\ $}
                  [. {$\openbranch{\neg Fb}$} ]
                  [. {$\openbranch{Ga}$} ]]]]]]]]
            \begin{scope}[xshift=3.0in]
              \tikzset{edge from parent/.style={edge from parent path={(\tikzparentnode) -- (\tikzchildnode)}}, every tree node/.style={text width=5em, align=left}}
              \Tree
                [. SM
                [. {1 $\neg \equiv$ D}
                [. {1 $\neg \equiv$ D}
                [. {3 $\exists$ D}
                [. {2 $\supset$ D}
                [. {5 $\neg \neg$ D}
                [. {6 $\forall$ D}
                [. {4 $\supset$ D}
                [. {3 $\neg \exists$ D}
                [. {2 $\supset$ D}
                [. {9 $\forall$ D}
                [. {11 $\neg \supset$ D}
                [. {11 $\neg \supset$ D}
                [. {10 $\neg \forall$ D}
                [. {14 $\exists$ D}
                [. {9 $\forall$ D}
                [. {11 $\neg \supset$ D}
                [. {11 $\neg \supset$ D} ]]]]]]]]]]]]]]]]]]
            \end{scope}
          \end{tikzpicture}

          Since there is at least one open branch, this tree is not closed.
          Thus, the two sentences are not quantificationally equivalent.

        \pagebreak

        \item $\{(\forall x)[(\exists y) Hg(x,y) \supset Bg(x,x)], Ha, a = g(a,b)\} \vDash (\exists y)Bg(y,y)$

          \begin{tikzpicture}[sibling distance=0.5cm]
            \begin{scope}[xshift=-1.5in]
              \tikzset{edge from parent/.style={edge from parent path={(\tikzparentnode) -- (\tikzchildnode)}}, every tree node/.style={text width=5em, align=left}}
              \Tree
                [. 1
                [. 2
                [. 3
                [. 4
                [. 5
                [. 6
                [. 7
                [. 8
                [. 9
                [. 10
                [. 11 ]]]]]]]]]]]
            \end{scope}
            \Tree
              [. {$(\forall x)[(\exists y) Hg(x,y) \supset Bg(x,x)]$}
              [. {$Ha$}
              [. {$a = g(a,b)$}
              [. {$\neg (\exists y)Bg(y,y) \checkmark$}
              [. {$(\forall y)\neg Bg(y,y)$}
              [. {$(\exists y) Hg(a,y) \supset Bg(a,a) \checkmark$}
              [. {$\neg(\exists y) Hg(a,y) \checkmark$}
                [. {$\ $}
                [. {$(\forall y)\neg Hg(a,y)$}
                [. {$\neg Hg(a,b)$}
                [. {$\closedbranch{\neg Ha}$} ]]]]]
              [. {$Bg(a,a)$}
                [. {$\closedbranch{\neg Bg(a,a)}$} ]]]]]]]]
            \begin{scope}[xshift=3.5in]
              \tikzset{edge from parent/.style={edge from parent path={(\tikzparentnode) -- (\tikzchildnode)}}, every tree node/.style={text width=5em, align=left}}
              \Tree
                [. SM
                [. SM
                [. SM
                [. SM
                [. {4 $\neg \exists$ D}
                [. {1 $\forall$ D}
                [. {6 $\supset$ D}
                [. {5 $\forall$ D}
                [. {7 $\neg \exists$ D}
                [. {9 $\forall$ D}
                [. {3 $=$ D} ]]]]]]]]]]]
            \end{scope}
          \end{tikzpicture}

          Since this tree is closed, the quantificational entailment holds.

          One interpretation is:

          UD: Natural numbers \\
          Hx: x is positive. \\
          g(x, y): Minimum of x and y. \\
          Bx: x is divisible by 1. \\
          a: 1 \\
          b: 2

      \end{enumerate}
    \item
      Why does the rule \textit{Existential Decomposition}
      require that the instantiating constant \textbf{a}
      be foreign to all preceding lines of the branch?

      By not requiring \textit{Existential Decomposition} to introduce foreign constants
      we have opened up the possibility that the same constant can be reused in a conflicting predicate.
      So, we require foreign constants with \textit{Existential Decomposition}
      in order to preserve truth, validity, equivalence, etc.
  \end{enumerate}
\end{document}
