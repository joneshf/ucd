\documentclass[12pt,letterpaper]{article}
\usepackage{amsmath}
\usepackage{amsfonts}
\usepackage{amsthm}
\usepackage{cancel}
\usepackage[margin=1in]{geometry}
\usepackage{titling}
\usepackage{syntax}
\usepackage{rail}
\usepackage{qtree}
\usepackage{fancyhdr}
\pagestyle{fancy}
\rhead{Jones, Hardy}

\setlength{\droptitle}{-10ex}

\preauthor{\begin{flushright}\large \lineskip 0.5em}
\postauthor{\par\end{flushright}}
\predate{\begin{flushright}\large}
\postdate{\par\end{flushright}}

\title{ECS 140A Homework 2\vspace{-2ex}}
\author{Jones, Hardy\\
        999397426\\
        Professor Olsson\vspace{-2ex}}
\date{Winter 2014}

\begin{document}
  \maketitle

  \begin{enumerate}
    \item Give the output using (a) static scoping and (b) dynamic scoping.

    \begin{enumerate}
      \item
        With static scoping, the program outputs

        \texttt{Q 1 1}

        \texttt{P 1 3}

        \texttt{Q 1 5}

        \texttt{P 1 3}

      \item
        With dynamic scoping, the program outputs

        \texttt{Q 1 1}

        \texttt{P 1 1}

        \texttt{Q 5 5}

        \texttt{P 5 5}

    \end{enumerate}

    \pagebreak

    \item Consider the following program.
      \begin{enumerate}
        \item Show the runtime stack, including static pointers,
          each time a block or procedure is entered or exited.

          \begin{tabular}{l l l}
            action & stack & static pointers \\
            \hline
            start: & (empty) & \\
            +A:    & A              & \\
            +P/D:  & A P            & P$\rightarrow$A \\
            +S:    & A P S          & P$\rightarrow$A S$\rightarrow$P \\
            +Q/C:  & A P S Q        & P$\rightarrow$A S$\rightarrow$P Q$\rightarrow$P \\
            +T:    & A P S Q T      & P$\rightarrow$A S$\rightarrow$P Q$\rightarrow$P T$\rightarrow$Q \\
            +R/B:  & A P S Q T R    & P$\rightarrow$A S$\rightarrow$P Q$\rightarrow$P T$\rightarrow$Q R$\rightarrow$A \\
            +U:    & A P S Q T R U  & P$\rightarrow$A S$\rightarrow$P Q$\rightarrow$P T$\rightarrow$Q R$\rightarrow$A U$\rightarrow$R \\
            -U:    & A P S Q T R    & P$\rightarrow$A S$\rightarrow$P Q$\rightarrow$P T$\rightarrow$Q R$\rightarrow$A \\
            -R/B:  & A P S Q T      & P$\rightarrow$A S$\rightarrow$P Q$\rightarrow$P T$\rightarrow$Q \\
            -T:    & A P S Q        & P$\rightarrow$A S$\rightarrow$P Q$\rightarrow$P \\
            -Q/C:  & A P S          & P$\rightarrow$A S$\rightarrow$P \\
            -S:    & A P            & P$\rightarrow$A \\
            -P/D:  & A              & \\
            +E:    & A E            & E$\rightarrow$A \\
            +R/F:  & A E R          & E$\rightarrow$A R$\rightarrow$A \\
            +U:    & A E R U        & E$\rightarrow$A R$\rightarrow$A U$\rightarrow$R \\
            -U:    & A E R          & E$\rightarrow$A R$\rightarrow$A \\
            -R/F:  & A E            & E$\rightarrow$A \\
            -E:    & A              & \\
            -A:    & (empty)        & \\
          \end{tabular}

      \pagebreak

        \item Show the runtime stack and the display each time a block or procedure is entered or exited.

          \begin{tabular}{l l l}
            action & stack & display (@ = pointer to or address of) \\
            \hline
            start: & (empty)        & (empty) \\
            +A:    & A              & @A \\
            +P/D:  & A P            & @A @P \\
            +S:    & A P S          & @A @P @S \\
            +Q/C:  & A P S Q        & @A @P @Q \\
            +T:    & A P S Q T      & @A @P @Q @T\\
            +R/B:  & A P S Q T R    & @A @R \\
            +U:    & A P S Q T R U  & @A @R @U \\
            -U:    & A P S Q T R    & @A @R \\
            -R/B:  & A P S Q T      & @A @P @Q @T \\
            -T:    & A P S Q        & @A @P @Q \\
            -Q/C:  & A P S          & @A @P @S \\
            -S:    & A P            & @A @P \\
            -P/D:  & A              & @A \\
            +E:    & A E            & @A @E \\
            +R/F:  & A E R          & @A @R \\
            +U:    & A E R U        & @A @R @U \\
            -U:    & A E R          & @A @R \\
            -R/F:  & A E            & @A @E \\
            -E:    & A              & @A \\
            -A:    & (empty)        & (empty) \\
          \end{tabular}
      \end{enumerate}

      \item Give the output from the following program for each of the following parameter passing combinations:

      \begin{enumerate}
        \item x, y, z by value.

          \texttt{1 2 3}

          \texttt{1 2 3}

          \texttt{2 1 1}

          \texttt{1 2 3}

        \item x, y, z by reference.

          \texttt{1 2 3}

          \texttt{2 1 1}

          \texttt{1 2 2}

          \texttt{1 1 1}

        \item x, y, z by name.

          \texttt{1 2 3}

          \texttt{2 2 1}

          \texttt{1 2 2}

          \texttt{1 1 1}

        \item x by reference, y by name, z by value.

          \texttt{1 2 3}

          \texttt{2 2 1}

          \texttt{1 2 2}

          \texttt{1 2 1}

        \item x, y by name, z by reference.

          \texttt{1 2 3}

          \texttt{2 2 1}

          \texttt{1 2 2}

          \texttt{1 1 1}
      \end{enumerate}

      \item
        Consider the following Ada-like program.

        Suppose OOPS is raised at A for each invocation of Q.

        \begin{enumerate}
          \item
            Assume the code in handler H1 and H2 do not execute a raise statement.
            For the invocation of Q at I1, which handler (H1, H2, or H3) will be executed? Explain.

            H1 will be executed as it is the closest exception handler to the point of exception.

          \item
            Assume the code in handler H1 and H2 do not execute a raise statement.
            For the invocation of Q through the invocation of R at I2,
            which handler (H1, H2, or H3) will be executed? Explain.

            H1 will be executed as it is the closest exception handler to the point of exception.

          \item
            Assume now that H1 executes a raise OOPS statement (but H2 doesn't).
            Explain what happens for each of the invocations of I1 and I2.

            For I1, when Q raises OOPS, H1 attempts to handle the exception.
            H1 then raises OOPS, so exception control goes up the call stack to H3.
            Assuming H3 doesn't raise OOPS as well, it handles the exception.

            For I2, when Q raises OOPS, H1 attempts to handle the exception.
            H1 then raises OOPS, so exception control goes up the call stack to H2.
            H2 doesn't raise OOPS, so it handles the exception.

          \item
            Assume now that H1 and H2 each executes a raise OOPS statement.
            Explain what happens for each of the invocations of I1 and I2.

            For I1, when Q raises OOPS, H1 attempts to handle the exception.
            H1 then raises OOPS, so exception control goes up the call stack to H3.
            Assuming H3 doesn't raise OOPS as well, it handles the exception.

            For I2, when Q raises OOPS, H1 attempts to handle the exception.
            H1 then raises OOPS, so exception control goes up the call stack to H2.
            H2 attempts to handle the exception.
            H2 then raises OOPS, so exception control goes up the call stack to H3.
            Assuming H3 doesn't raise OOPS as well, it handles the exception.
        \end{enumerate}
  \end{enumerate}
\end{document}
