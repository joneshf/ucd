\documentclass[12pt,letterpaper]{article}
\usepackage{amsmath}
\usepackage{amsfonts}
\usepackage{amsthm}
\usepackage{cancel}
\usepackage[margin=1in]{geometry}
\usepackage{titling}
\usepackage{syntax}
\usepackage{rail}
\usepackage{qtree}
\usepackage{fancyhdr}
\pagestyle{fancy}
\rhead{Jones, Hardy}

\setlength{\droptitle}{-10ex}

\preauthor{\begin{flushright}\large \lineskip 0.5em}
\postauthor{\par\end{flushright}}
\predate{\begin{flushright}\large}
\postdate{\par\end{flushright}}

\title{ECS 140A Homework 1\vspace{-2ex}}
\author{Jones, Hardy\\
        999397426\\
        Professor Olsson\vspace{-2ex}}
\date{Winter 2014}

\begin{document}
  \maketitle

  \begin{enumerate}
    \item Indicate if the string belongs to the class of object at the head of the column by placing ``yes'' or ``no'' in the approriate box.

      \begin{tabular}{| l || c | c | c | c |}
        \hline
        \multicolumn{1}{| c ||}{string} & $<$chairs$>$ & $<$set$>$ & $<$table$>$ & $<$room$>$ \\
        \hline
        oak                                 & no  & no  & yes & yes \\ \hline
        armchair bench oak                  & no  & no  & no  & yes \\ \hline
        dinette plant                       & no  & no  & no  & yes \\ \hline
        bench stool                         & yes & no  & no  & no  \\ \hline
        oak rug                             & no  & no  & no  & yes \\ \hline
        oak teak dinette                    & no  & no  & no  & no  \\ \hline
        teak teak                           & no  & no  & no  & no  \\ \hline
        teak bench oak oak stool            & no  & no  & no  & no  \\ \hline
        dinette plant                       & no  & no  & no  & yes \\ \hline
        rug plant                           & no  & no  & no  & no  \\ \hline
        couch                               & no  & no  & no  & no  \\ \hline
        armchair bench                      & yes & no  & no  & no  \\ \hline
        teak lamp                           & no  & no  & no  & no  \\ \hline
        bench stool teak bench stool        & no  & no  & no  & yes \\ \hline
        bench stool teak bench teak stool   & no  & no  & no  & no  \\ \hline
        teak bench plant                    & no  & no  & no  & yes \\ \hline
      \end{tabular}
    \item Rewrite the original grammar in EBNF notation (as given in the textbook).
      \begin{grammar}
        <room> ::= [<chairs>] <table>
                \alt <table> (couch | <stuff>)

        <table> ::= oak
                \alt teak
                \alt dinette
                \alt <table> <chairs>

        <chairs> ::= <chair> \{ <chair> \}

        <chair> ::= armchair
                \alt bench
                \alt stool

        <stuff> ::= rug
                \alt plant
      \end{grammar}

    \item Rewrite the above grammar using syntax graphs (aka syntax diagrams).
      \begin{rail}
        room : ( chairs * ) table | table ( 'couch' | stuff);
        table : 'oak' | 'teak' | 'dinette' | table chairs;
        chairs : ( chair + );
        chair : 'armchair' | 'bench' | 'stool';
        stuff : 'rug' | 'plant';
      \end{rail}

    \item
      Consider all strings that can be derived from $<$table$>$.
      Can each of those also be derived from $<$room$>$?
      Explain.

      Yes, every string that can be derived from $<$table$>$
      can be derived from $<$room$>$.
      Since a $<$room$>$ can be just a $<$table$>$,
      we can make a straight derivation to $<$table$>$.

    \item
      Suppose the production $<$table$>$ $<$chairs$>$ were added as an alternative in the rule for $<$room$>$.
      Would this change allow any additional strings to be considered a $<$room$>$?
      If yes, give one such new string.
      If no, explain.

      No, this would not allow additional strings in the language.
      Since $<$room$>$ can be a $<$table$>$, and a $<$table$>$ can be a $<$set$>$, and a set can be a $<$table$>$ $<$chairs$>$,
      we would not have gained anything by adding this alternative.

    \item
      Modify the original grammar so that in strings produced by $<$chairs$>$,
      any stools precede any armchairs or benches.
      Show only the changed rule(s) and any new rule(s),
      not the entire grammar.

      \begin{grammar}
        <chairs> ::=  ( <stoolie> | <armbench> ) \{ <armbench> \}

        <stoolie> ::= stool \{ <stoolie> \}

        <armbench> ::=  armchair
                   \alt bench
      \end{grammar}

    \item
      The original grammar does not give a unique parse tree for every input string.
      Give a string where the grammar is ambiguous and
      give two different parse trees for that string.
      What problems would ambiguity cause in a real programming language?

      A simple example is: \texttt{oak bench bench}.

      It can be parsed in the following ways.

      \Tree
        [.$room$
          [.$table$
            [.$set$ [
              [.$table$ \texttt{oak} ]
              [.$chairs$
                [.$chairs$
                  [.$chair$ [\texttt{bench} ] ] ]
                [.$chair$ [\texttt{bench} ] ] ] ] ] ] ]
      \Tree
        [.$room$
          [.$table$
            [.$set$
              [.$table$
                [.$set$
                  [.$table$ \texttt{oak} ]
                  [.$chairs$
                    [.$chair$ \texttt{bench} ] ] ] ]
              [.$chairs$
                [.$chair$ \texttt{bench} ] ] ] ] ]

      Ambiguity in a real programming language can lead to incorrect results.
      The simple case is arithmetic expressions.
      If the grammar is ambiguous with respect to arithmetic expressions,
      $1 + 2 \cdot 3$ could create the following incorrect parse tree resulting in $9$.

      \Tree
        [.$\cdot$
          [.+
            [1 2 ] ]
          [3 ] ]
  \end{enumerate}

\end{document}
