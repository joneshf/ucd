\documentclass[12pt,letterpaper]{article}
\usepackage{amsmath}
\usepackage{amsfonts}
\usepackage{amsthm}
\usepackage{mathtools}
\usepackage{cancel}
\usepackage[margin=1in]{geometry}
\usepackage{titling}
\usepackage{algorithm}
\usepackage{algpascal}
\usepackage{algpseudocode}

\setlength{\droptitle}{-10ex}

\preauthor{\begin{flushright}\large \lineskip 0.5em}
\postauthor{\par\end{flushright}}
\predate{\begin{flushright}\large}
\postdate{\par\end{flushright}}

\title{ECS 150 Homework 4\vspace{-2ex}}
\author{Hardy Jones\\
        999397426\\
        Professor Levitt\vspace{-2ex}}
\date{Winter 2015}

\begin{document}
  \maketitle

  \begin{enumerate}
    \item [1]
      Let's assume we have the following pages for the elements of each array:

      \begin{tabular}{r c r | r | r |}
        \multicolumn{3}{l|}{elements} & \texttt{A} & \texttt{B} \\
        \hline
        1   & - & 100  & 1          & 11 \\
        \hline
        101 & - & 200  & 2          & 12 \\
        \hline
        201 & - & 300  & 3          & 13 \\
        \hline
        301 & - & 400  & 4          & 14 \\
        \hline
        401 & - & 500  & 5          & 15 \\
        \hline
        501 & - & 600  & 6          & 16 \\
        \hline
        601 & - & 700  & 7          & 17 \\
        \hline
        701 & - & 800  & 8          & 18 \\
        \hline
        801 & - & 900  & 9          & 19 \\
        \hline
        901 & - & 1000 & 10         & 20 \\
        \hline
      \end{tabular}

      \begin{enumerate}
        \item The page reference string looks similar to

          \begin{tabular}{r r r r r r p{5cm}}
            1  ,& 11 ,& 10 ,& 1  ,& 11 ,& 10 ,& ...repeated 98 more times. \\
            2  ,& 12 ,& 9  ,& 2  ,& 12 ,& 9  ,& ...repeated 98 more times. \\
            3  ,& 13 ,& 8  ,& 3  ,& 13 ,& 8  ,& ...repeated 98 more times. \\
            4  ,& 14 ,& 7  ,& 4  ,& 14 ,& 7  ,& ...repeated 98 more times. \\
            5  ,& 15 ,& 6  ,& 5  ,& 15 ,& 6  ,& ...repeated 98 more times. \\
            6  ,& 16 ,& 5  ,& 6  ,& 16 ,& 5  ,& ...repeated 98 more times. \\
            7  ,& 17 ,& 4  ,& 7  ,& 17 ,& 4  ,& ...repeated 98 more times. \\
            8  ,& 18 ,& 3  ,& 8  ,& 18 ,& 3  ,& ...repeated 98 more times. \\
            9  ,& 19 ,& 2  ,& 9  ,& 19 ,& 2  ,& ...repeated 98 more times. \\
            10 ,& 20 ,& 1  ,& 10 ,& 20 ,& 1  ,& ...repeated 98 more times. \\
          \end{tabular}

        \item
          \begin{enumerate}
            \item
              Using FIFO, nearly every time we access a new page it is a page fault.
              The reason for so many page faults is every three sequential page accesses are different, with one exception.
              When we access 5, 15, 6, 6, 16, 5, there is one group of three where we access the same page twice.
              This is the only instance where we do not page fault.

              We can compute this value simply:
              in each row we perform three page accesses 100 times,
              we have 10 rows,
              minus one access that doesn't miss.

              So the number of page faults is $3 \cdot 100 \cdot 10 - 1 = 2999$.

            \item
              Using MIN, for each row, we can keep one page in memory,
              use the register to hold the addition temporarily,
              and swap between the two other pages.
              With the first row as an example, we have 1, 11, 10, 1, 11, 10.
              If we keep page 1 in memory, we would just swap out 11 and 10.
              So we miss the first access to page 1, then every access to pages 11 and 10.
              So each row has $1 + 2 \cdot 100 = 201$ misses.
              Again, we do not have a miss when we switch from row 5 to row 6.

              So the number of page faults is $(1 + 2 \cdot 100) \cdot 10 - 1 = 2000$.
          \end{enumerate}
      \end{enumerate}
    \item [2 \S 4.11]
      Since there are 32 bits in an address, and we have 20 of the bits taken already, the offset has 12 bits.
      So each page is $2^{12}$ bytes.

      We can calculate the number of pages as $2^9 \cdot 2^{11} = 2^{20}$.

    \item [3 \S 4.12]

    \item [4 \S 4.13]
      Since we know the address size, we only need to know one of two things:

      \begin{itemize}
        \item The sum $a + b + c$
        \item $d$
      \end{itemize}

      If we know $a + b + c$ then the page size is $2^{a + b + c}$.

      If we know $d$ then the page size is $2^{32 - d}$
    \item [5 \S 4.18]
      \begin{enumerate}
        \item NRU will replace page 0, as page 0 is the only page in class 0.
        \item FIFO will replace page 2, as page 2 is the oldest page.
        \item LRU will replace page 1, as page 1 has the oldest last reference.
        \item Second chance will replace page 0.

          It sees page 2 is the oldest but has its $R$ bit set,
          so page 2 is given a new loaded value of the current time
          and its $R$ bit cleared.
          The algorithm continues and finds page 0 as the next oldest.
          Since page 0 does not have its $R$ bit set, this page is removed.
      \end{enumerate}
    \item [6 \S 5.16]
      This has the advantage that for small files only one seek is necessary to read the data,
      rather than one seek for the i-node and another seek for the data.
    \item [7 \S 5.15]
      \begin{itemize}
        \item
          The linked list uses $D$ bits for each free block, so the size is $F \cdot D$ bits.
          The bitmap uses one bit per block, so the size is $B$ bits.

          So the linked list will use less space when $F \cdot D < B$
        \item
          For $D = 16$, we can manipulate the equation:

          \begin{align*}
            F \cdot 16 &< B \\
            \frac{F}{B} &< \frac{1}{16} \\
            \frac{F}{B} &< 0.0625
          \end{align*}

          So the linked list uses less space when there is less than 6.25\% disk space left.
      \end{itemize}
  \end{enumerate}
\end{document}
