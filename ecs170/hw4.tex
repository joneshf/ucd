\documentclass[12pt,letterpaper]{article}
\usepackage{amsmath}
\usepackage{amsfonts}
\usepackage{amsthm}
\usepackage{cancel}
\usepackage[margin=1in]{geometry}
\usepackage{titling}
\usepackage{graphicx}

\setlength{\droptitle}{-10ex}

\preauthor{\begin{flushright}\large \lineskip 0.5em}
\postauthor{\par\end{flushright}}
\predate{\begin{flushright}\large}
\postdate{\par\end{flushright}}

\title{ECS 170 Homework 3\vspace{-2ex}}
\author{Hardy Jones\\
        999397426\\
        Professor Davidson\vspace{-2ex}}
\date{Winter 2014}

\begin{document}
  \maketitle

  \begin{enumerate}
    \item
      Consider the figure
      \begin{enumerate}
        \item Which of the following (if any) are asserted by the network structure?
          \begin{enumerate}
            \item $P(B,I,M) = P(B)P(I)P(M)$

              This is not asserted because $B, I, M$ are not independent.
            \item $P(J|G) = P(J|G,I)$

              This is asserted because $J$ is conditionally independent of $I$ given $G$.
            \item $P(M|G,B,I) = P(M|G,B,I,J)$

              This is asserted because $J$ is conditionally indepented of $M$ given $G$.
          \end{enumerate}

        \item Calculate the value of $P(b,i,m,\neg g,j)$.
          \begin{align*}
            P(b,i,m,\neg g,j) &= P(b)P(i|b,m)P(m)P(\neg g|b,i,m)P(j|\neg g)\\
            &= (0.9)(0.9)(0.1)(0.9)(0.0)\\
            &= 0.0\\
          \end{align*}
      \end{enumerate}
    \item Consider the figure.
      \begin{enumerate}
        \item Give one example of conditional independence and another of unconditional independence.

        \textbf{JohnCalls} is conditionally independent of \textbf{Burglary} given \textbf{Alarm}.

        \textbf{Burglary} is unconditinally independent of \textbf{Earthquake}.
      \end{enumerate}
  \end{enumerate}
\end{document}
