\documentclass[12pt,letterpaper]{article}
\usepackage{amsmath}
\usepackage{amsfonts}
\usepackage{amsthm}
\usepackage{cancel}
\usepackage[margin=1in]{geometry}
\usepackage{titling}
\usepackage{graphicx}

\setlength{\droptitle}{-10ex}

\preauthor{\begin{flushright}\large \lineskip 0.5em}
\postauthor{\par\end{flushright}}
\predate{\begin{flushright}\large}
\postdate{\par\end{flushright}}

\title{ECS 170 Homework 7\vspace{-2ex}}
\author{Hardy Jones\\
        999397426\\
        Professor Davidson\vspace{-2ex}}
\date{Winter 2014}

\begin{document}
  \maketitle

  \begin{enumerate}
    \item
      Consider a cumulative discount reward with $\gamma = 0$ and $\gamma = 1$.
      What type of behavior would these reward functions encourage?

      The reward function with $\gamma = 0$ would reward short term goals.
      The reward function with $\gamma = 1$ would reward long term goals.

    \item
      Why would you prefer to use Q-learning to implement a black-jack player rather than the mini-max algorithm?

      BlackJack is a partially observable game.
      Minimax would have to search the entire state space for every possible card combination can could occur.
      This is realistically impossible.
      The only viable option is to create an evaluation function that will most likely not be optimal.

      Q-learning can learn the optimal choice to make given enough time to learn.

    \item
      One Q-learning algorithm described in class assumed that the $\delta$ and $r$ functions were deterministic.
      Can this algorithm be used for
        \begin{enumerate}
          \item Learning to control an inverted pendulum

            Yes, this is possible with deterministic functions.
          \item Play chess

            No, this is not possible even with deterministic $\delta$ and $r$ because the game is still inherently adversarial.
         \end{enumerate}

    \item
      Evaluate the table.
      \begin{tabular}{| c | c | c | c |}
      \hline
      Step & Start State & Subsequent State & Update to Q(s, a) \\
      \hline
      1  & B1 & A1 & Q(B1, Up) = \\
      \hline
      2  & A1 & A2 & Q(A1, Right) = \\
      \hline
      3  & A2 & A3 & Q(A2, Right) = \\
      \hline
      4  & A3 & B3 & Q(A3, Down) = \\
      \hline
      5  & B3 & B2 & Q(B3, Left) = \\
      \hline
      6  & B1 & A1 & Q(B1, Up) = \\
      \hline
      7  & A1 & A2 & Q(A1, Right) = \\
      \hline
      8  & A2 & A3 & Q(A2, Right) = \\
      \hline
      9  & A3 & B3 & Q(A3, Down) = \\
      \hline
      10 & B3 & B2 & Q(B3, Left) = \\
      \hline
      \end{tabular}
  \end{enumerate}
\end{document}
