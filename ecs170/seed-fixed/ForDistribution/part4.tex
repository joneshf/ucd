\documentclass[12pt,letterpaper]{article}
\usepackage{amsmath}
\usepackage{amsfonts}
\usepackage{amsthm}
\usepackage{cancel}
\usepackage[margin=1in]{geometry}
\usepackage{titling}
\usepackage{listings}
\usepackage{color}

\setlength{\droptitle}{-10ex}

\preauthor{\begin{flushright}\large \lineskip 0.5em}
\postauthor{\par\end{flushright}}
\predate{\begin{flushright}\large}
\postdate{\par\end{flushright}}

\title{ECS 170 Project 2 Part 4\vspace{-2ex}}
\author{Hardy Jones\\
        999397426\\
        Professor Davidson\vspace{-2ex}}
\date{Winter 2014}

\begin{document}
  \maketitle


  \definecolor{dkgreen}{rgb}{0,0.6,0}
  \definecolor{gray}{rgb}{0.5,0.5,0.5}
  \definecolor{mauve}{rgb}{0.58,0,0.82}

  \lstset{frame=tb,
    language=Java,
    aboveskip=3mm,
    belowskip=3mm,
    showstringspaces=false,
    columns=flexible,
    basicstyle={\small\ttfamily},
    numbers=none,
    numberstyle=\tiny\color{gray},
    keywordstyle=\color{blue},
    commentstyle=\color{dkgreen},
    stringstyle=\color{mauve},
    breaklines=true,
    breakatwhitespace=true
    tabsize=3
  }

  The successor function will first generate all of the possible states,
  then calculate the utility for each state,
  and sort them depending on who the current player is.

  Given a good utility function this should give states that end with a win or at least a draw position priority over states that end with a loss.
  This can be seen by assuming that MAX has the next state and all are terminal states.
  If the successor function generates the states in this way, it will generate any of the winning states first, followed by any ties, and finally any losses.
  MAX can then prune all states except the first win, or if there are no wins, the first draw, and if no draws, then the first loss.

  A similar argument holds for MIN if the states are ordered by lowest utility first.

\end{document}
