\documentclass[12pt,letterpaper]{article}
\usepackage{amsmath}
\usepackage{amsfonts}
\usepackage{amsthm}
\usepackage{cancel}
\usepackage[margin=1in]{geometry}
\usepackage{titling}
\usepackage{listings}
\usepackage{color}

\setlength{\droptitle}{-10ex}

\preauthor{\begin{flushright}\large \lineskip 0.5em}
\postauthor{\par\end{flushright}}
\predate{\begin{flushright}\large}
\postdate{\par\end{flushright}}

\title{ECS 170 Project 2 Part 1\vspace{-2ex}}
\author{Hardy Jones\\
        999397426\\
        Professor Davidson\vspace{-2ex}}
\date{Winter 2014}

\begin{document}
  \maketitle


  \definecolor{dkgreen}{rgb}{0,0.6,0}
  \definecolor{gray}{rgb}{0.5,0.5,0.5}
  \definecolor{mauve}{rgb}{0.58,0,0.82}

  \lstset{frame=tb,
    language=Java,
    aboveskip=3mm,
    belowskip=3mm,
    showstringspaces=false,
    columns=flexible,
    basicstyle={\small\ttfamily},
    numbers=none,
    numberstyle=\tiny\color{gray},
    keywordstyle=\color{blue},
    commentstyle=\color{dkgreen},
    stringstyle=\color{mauve},
    breaklines=true,
    breakatwhitespace=true
    tabsize=3
  }

  The evaluation function first checks to see if the state is an end state. If so it returns what is effectively $\infty$ for MAX and $-\infty$ for MIN.
  Assuming the state is not terminal, it takes into account all of the chips on the board.
  It looks first for any blocks, any three in a row, and any two in a row.
  Then it assigns a weighted value to each of these and sums them together.
  This should work because it will give more weight to a state that has more coins grouped together than a sparser board.
  This is ideal because it is easier to get four in a row, with a more densely packed board than a sparser one.

  The evaluation function as a numerical expression:
  \begin{align*}
    \text{utility} &= \infty \cdot win_{MAX} - \infty \cdot win_{MIN} \\
    &+ 500 \cdot \sum_{b \in blocks_{MAX}}b - 500 \cdot \sum_{b \in blocks_{MIN}}b \\
    &+ 75 \cdot \sum_{t \in threes_{MAX}}t - 75 \cdot \sum_{t \in threes_{MIN}}t \\
    &+ 25 \cdot \sum_{t \in twos_{MAX}}t - 25 \cdot \sum_{t \in twos_{MIN}}t
  \end{align*}
\end{document}
