\documentclass[12pt,letterpaper]{article}
\usepackage{amsmath}
\usepackage{amsfonts}
\usepackage{amsthm}
\usepackage{cancel}
\usepackage[margin=1in]{geometry}
\usepackage{titling}

\setlength{\droptitle}{-10ex}

\preauthor{\begin{flushright}\large \lineskip 0.5em}
\postauthor{\par\end{flushright}}
\predate{\begin{flushright}\large}
\postdate{\par\end{flushright}}

\title{ECS 170 Homework 2\vspace{-2ex}}
\author{Hardy Jones\\
        999397426\\
        Professor Davidson\vspace{-2ex}}
\date{Winter 2014}

\begin{document}
  \maketitle

  \begin{enumerate}
    \item
      Given a set of admissible heuristics $\mathcal{H} = \{h_1, h_2, ..., h_n\}$ one can define a new heuristic $h_{max}$ such that for any node $n$:
      \[
        h_{max}(n) = \underset{i}{\text{max}} \ h_i(n)
      \]
      \begin{enumerate}
        \item Show that $h_{max}$ is an admissible heuristic.

          \begin{proof}
            Assume $h_{max}$ is not an admissible heuristic.

            We know $h_{max}$ is some heuristic from the set $h_1, h_2, ..., h_n$.
            This means some heuristic in $\mathcal{H}$ is not admissible.

            This is a contradiction since $\mathcal{H}$ contains only admissible heuristics,
            thus our assumption was incorrect.

            Therefore, $h_{max}$ is an admissible heuristic.
          \end{proof}
        \item Show that $h_{max}$ dominates all other $h_i$

          \begin{proof}
            Assume $h_{max}$ does not dominate all other $h_i$.

            Then there must be some heuristic $h_j$ in $\mathcal{H}$ such that $h_{max}(n) < h_j(n)$.
            But we know that $\forall i, 1 \le i \le n, h_{max}(n) \ge h_i(n)$.

            This is a contradiction, thus our assumption was incorrect.

            Therefore, $h_{max}$ dominates all other $h_i$.
          \end{proof}
      \end{enumerate}
  \end{enumerate}
\end{document}
