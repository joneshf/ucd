\documentclass[12pt,letterpaper]{article}
\usepackage{amsmath}
\usepackage{amsfonts}
\usepackage{amsthm}
\usepackage{cancel}
\usepackage[margin=1in]{geometry}
\usepackage{titling}

\setlength{\droptitle}{-10ex}

\preauthor{\begin{flushright}\large \lineskip 0.5em}
\postauthor{\par\end{flushright}}
\predate{\begin{flushright}\large}
\postdate{\par\end{flushright}}

\title{ECS 170 Homework 2\vspace{-2ex}}
\author{Hardy Jones\\
        999397426\\
        Professor Davidson\vspace{-2ex}}
\date{Winter 2014}

\begin{document}
  \maketitle

  \begin{enumerate}
    \item
      Given a set of admissible heuristics $\mathcal{H} = \{h_1, h_2, ..., h_n\}$ one can define a new heuristic $h_{max}$ such that for any node $n$:
      \[
        h_{max}(n) = \underset{i}{\text{max}} \ h_i(n)
      \]
      \begin{enumerate}
        \item Show that $h_{max}$ is an admissible heuristic.

          \begin{proof}
            Assume $h_{max}$ is not an admissible heuristic.

            We know $h_{max}$ is some heuristic from the set $h_1, h_2, ..., h_n$.
            This means some heuristic in $\mathcal{H}$ is not admissible.

            This is a contradiction since $\mathcal{H}$ contains only admissible heuristics,
            thus our assumption was incorrect.

            Therefore, $h_{max}$ is an admissible heuristic.
          \end{proof}
        \item Show that $h_{max}$ dominates all other $h_i$

          \begin{proof}
            Assume $h_{max}$ does not dominate all other $h_i$.

            Then there must be some heuristic $h_j$ in $\mathcal{H}$ such that $h_{max}(n) < h_j(n)$.
            But we know that $\forall i, 1 \le i \le n, h_{max}(n) \ge h_i(n)$.

            This is a contradiction, thus our assumption was incorrect.

            Therefore, $h_{max}$ dominates all other $h_i$.
          \end{proof}
      \end{enumerate}

    \item
      Recall the ``number of misplaced tiles'' heuristic for the 8-puzzle problem.
      Show that this heuristic is admissible.

      \begin{proof}
        The rule for the 8-puzzle problem is:
        \begin{itemize}
          \item A tile can move from one position to another if the two positions are horizontally or vertically adjacent and the new position is blank.
        \end{itemize}

        If we decide to `relax' the rule for this problem we can get rid of the two restrictions that ``the two positions must be adjacent'' and ``the new position must be blank''.
        By removing these two restrictions, we get our ``number of misplaced tiles'' heuristic.

        We can show this is admissible by induction.

        In the base case, no tiles are out of place, so the heuristic gives the optimal cost of 0 (which is not an overestimation).

        For the inductive case, if $n$ tiles are out of place, then at least $n$ moves must be performed in order to reach the goal state.
        Again, this is not an overestimation.

        Thus, this heuristic is admissible.
      \end{proof}

    \item What are the practical benefits of knowing that a heuristic $h_1$ dominates another heuristic $h_2$ (assume we know both are admissible)?

      A dominating heuristic $h_1$ will be more efficient than $h_2$ for an algorithm like A*
      because $h_1$ will expand less nodes than $h_2$.

    \item
      If we consider the minimax with $\alpha$-$\beta$ pruning algoritm returns/calculates the Nash equilibrium.
      When the node $n$ is first expanded with values of $\alpha$ and $\beta$, what does this mean?

      The $\alpha$ value is the path that $MIN$ should take,
      and the $\beta$ value is the path that $MAX$ should take.

    \item
      What is the worst and best case time analysis of the minimax algorithm (without $\alpha$-$\beta$) in the big O notation, where $b$ is the branching factor, $m$ is the maximum depth of the tree, and $d$ is the depth of the least cost solution?

      Without any pruning, the best and worst case time is the same since minimax must enumerate the entire tree.
      Since it enumerates the tree in a depth first search, the time complexity is $O(b^m)$.
  \end{enumerate}
\end{document}
