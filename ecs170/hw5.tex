\documentclass[12pt,letterpaper]{article}
\usepackage{amsmath}
\usepackage{amsfonts}
\usepackage{amsthm}
\usepackage{cancel}
\usepackage[margin=1in]{geometry}
\usepackage{titling}
\usepackage{graphicx}

\setlength{\droptitle}{-10ex}

\preauthor{\begin{flushright}\large \lineskip 0.5em}
\postauthor{\par\end{flushright}}
\predate{\begin{flushright}\large}
\postdate{\par\end{flushright}}

\title{ECS 170 Homework 5\vspace{-2ex}}
\author{Hardy Jones\\
        999397426\\
        Professor Davidson\vspace{-2ex}}
\date{Winter 2014}

\begin{document}
  \maketitle

  \begin{enumerate}
    \item Give one advantage of using a belief network versus modeling the full joint distribution of a set of random variables.

    Bayesian networks have the distinct advantage of less computationally expensive to calculate than the full joint distribution.

    \item Describe the four types of relationships that can occur in belief networks.
      \begin{itemize}
        \item Direct cause

          Given two nodes $A$ and $B$ with an arrow from $A$ to $B$,
          we say that $A$ has a direct cause on $B$.
        \item Indirect cause

          Given three nodes $A$, $B$, and $C$
          with arrows from $A$ to $B$ and $B$ to $C$,
          we say that $A$ has an indirect cause on $C$.
        \item Common cause

          Given three nodes $A$, $B$, and $C$
          with arrows from $A$ to $B$ and $A$ to $C$,
          we say that $A$ is a common cause of $B$ and $C$.
        \item Common effect

          Given three nodes $A$, $B$, and $C$
          with arrows from $A$ to $C$ and $B$ to $C$,
          we say that $A$ and $B$ have a common effect of $C$.
      \end{itemize}

    \item Consider the following.
      \begin{enumerate}
        \item Which of the bayesian networks are correct? Explain.

        The correct networks are $(ii)$ and $(iii)$.

        $(i)$ suggests that $N$ is conditionally independent of $F_1$ given $M_1$.
        This is clearly not the case since if we count $n$ stars,
        there could be at least $n + 3$ actual stars if the focus were off.

        $(ii)$ is a direct implementation of the given problem.

        $(iii)$ is a reordering of the problem, so it is clearly correct.
        The measurement along with the number of stars can show if the focus is out.
        $M_1$ has a direct cause on $M_2$ because $M_2$ can never be more than 4 different from $M_1$.
        The two measurements provide a lower bound on the number of stars.

        \item Which is the most efficient network of the accurate network(s) and why?

        $(ii)$ is the more efficient because it has less edges.
        The full joint distribution of $(iii)$ will be much higher than $(ii)$.

        \item Assuming $N \in \{1,2,3\}$ and $M_1 \in \{0,1,2,3,4\}$,
        write out the conditional distribution for $P(M_1|N)$ in terms of $e$ and $f$.

        Using graph $(ii)$:

        \begin{tabular}{l l | c}
          $M_1$ & $N$ & $P(M_1|N)$ \\
          \hline
          0 & 1 & $e + f$ \\
          0 & 2 & $f$ \\
          0 & 3 & $f$ \\
          1 & 1 & $(\neg e) \cdot (\neg f)$ \\
          1 & 2 & $e \cdot (\neg f)$ \\
          1 & 3 & $0$ \\
          2 & 1 & $e$ \\
          2 & 2 & $(\neg e) \cdot (\neg f)$ \\
          2 & 3 & $e$ \\
          3 & 1 & $0$ \\
          3 & 2 & $e$ \\
          3 & 3 & $(\neg e) \cdot (\neg f)$ \\
          4 & 1 & $0$ \\
          4 & 2 & $0$ \\
          4 & 3 & $e$ \\
        \end{tabular}
      \end{enumerate}
  \end{enumerate}
\end{document}
