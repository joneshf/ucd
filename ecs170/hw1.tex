\documentclass[12pt,letterpaper]{article}
\usepackage{amsmath}
\usepackage{amsfonts}
\usepackage{amsthm}
\usepackage{cancel}
\usepackage[margin=1in]{geometry}
\usepackage{titling}

\setlength{\droptitle}{-10ex}

\preauthor{\begin{flushright}\large \lineskip 0.5em}
\postauthor{\par\end{flushright}}
\predate{\begin{flushright}\large}
\postdate{\par\end{flushright}}

\title{ECS 170 Homework 1\vspace{-2ex}}
\author{Hardy Jones\\
        999397426\\
        Professor Davidson\vspace{-2ex}}
\date{Winter 2014}

\begin{document}
  \maketitle

  \begin{enumerate}
    \item
      What is the difference among
      BFS, DFS, and uniform-cost search (Dijkstra's algorithm)
      with respect to their implementations in the generic tree search algorithm?

      \begin{enumerate}
        \item BFS uses a FIFO structure for the fringe.
        \item DFS uses a LIFO structure for the fringe.
        \item Uniform-cost search uses a priority FIFO structure for the fringe.
      \end{enumerate}
    \item
      Assuming there aren't any programming bugs,
      what is the most likely reason uniform cost search will sometimes return a greater cost path than depth first search?

      If the branching factor is large, most of the path costs are small, but a
      then UCS will explore many small-cost branches before exploring a any large-cost branches.
      If the small-cost branches find a path to the goal,
      UCS will choose this path as a solution,
      even though the overall cost would be greater than any large-cost branches
    \item
      Given this information (and assuming there aren't any programming bugs)
      what is the most likely reason A* would return a greater cost path
      than Dijkstra's Algorithm?

      A* can have a greater cost path than UCS
      if we choose a heuristic function which overestimates the cost to reach the goal.
      It's possible that there is some path on the optimal solution
      that would have an overestimation for the heuristic.
      A* would not expand this path,
      and instead attempt to find some other path.
      UCS on the other hand would just expand the path like any other.

    \item
      Consider the two statements
      (a) BFS is a special case of uniform cost search, and
      (b) uniform cost search is a special case of A*.
      Under what conditions are they true?

      \begin{enumerate}
        \item BFS is a special case of UCS when the cost function is constant.
        \item UCS is a special case of A* when the heuristic function is constant.
      \end{enumerate}

    \pagebreak

    \item
      Sudoku is a popular game in which the player tries to fill in all blank cells
      so that the resulting board contains 1 to 9 on each
      row, each column, and each 3 x 3 block.

      \begin{enumerate}
        \item
          Formulate it as a graph search problem.
          What are the state space, goal state, successor function, and the path costs?
          \begin{enumerate}
            \item
              The state space is any arrangement of the numbers 1-9 in each cell
              without repetition in the column, row, or 9x9 cell.
            \item
              The goal state is all 81 cells are filled in with
              the numbers 1-9 with no repetition in each column, row or 9x9 cell.
            \item
              The successor function chooses a cell and inputs a number from 1-9
              where the number chosen is not also in the same row, column, or 9x9 cell.
            \item
              The path cost is 1.
          \end{enumerate}
        \item
          Assume we use uninformed search.
          Which method would you prefer? Why?

          With uninformed search, it is preferable to use DFS.

          The branching factor can be huge.
          In the worst case, the branching factor $b$ is 9.
          The depth of the shallowest goal node depends on the initial state.
          With an entirely empty board the maximum depth $d$ is 81.
          This is the same as the the maximum length of any path in the state space,
          $m$ = 81

          In BFS this means we need to store around a maximum of $b^{d+1} = 9^82 = 1.77 x 10^78$ nodes in memory.
          In DFS this means we would only store around a maximum of $bm = 9 \cdot 81 = 729$ nodes in memory.

          The two algorithms have similar time complexity, so there is not much difference there.

          Finally, the main issue with using DFS is that it is possible for the search to never end.
          However, since we have a finite $d$ and $m$, we are guaranteed to find a solution.

          One is prohibitively expensive, while the other is an actual usable solution.
          For these reasons, DFS is recommended.
        \item
          Is a good heuristic possible in this case?
          If so, provide one. If not, why not?

          There are quite a few heuristics available for this problem.
          One such is choosing the cell with the least number of valid numbers,
          and inserting a number there.
          We end up with a greater probability of chosing the correct number to insert into a cell.
      \end{enumerate}

    \pagebreak

    \item
      Consider the classic farmer, fox, goose, and grain problem.
      It turns out that you can pose this as a graph search problem.
      \begin{enumerate}
        \item
          Describe a representation of the state space for this problem.
          What would the goal state look like in this representation.

          The state space would be each of the farmer, fox, goose, and grain
          in one of two locations: the west bank, and the east bank.

          The goal state would have each of the farmer, fox, goose, and grain
          on the east bank.
        \item
          What are the possible actions at a particular state?

          There are 8 possible actions:
          \begin{enumerate}
            \item Move just the farmer to the east bank.
            \item Move just the farmer to the west bank.
            \item Move the fox to the east bank.
            \item Move the fox to the west bank.
            \item Move the goose to the east bank.
            \item Move the goose to the west bank.
            \item Move the grain to the east bank.
            \item Move the grain to the west bank.
          \end{enumerate}

          Of course, each possible action involving moving something other than the farmer
          assumes that said item is on the appropriate bank.
        \item
          Our goal is to get everything to the other side in one piece,
          so what are the constraints on the state space that we need for the actions in part b?

          We have two constraints:
          \begin{enumerate}
            \item
              The fox and goose cannot be on one bank without the farmer also on that bank.
            \item
              The goose and grain cannot be on one bank without the farmer also on that bank.
          \end{enumerate}
        \item
          Suppose we want to use A* here.
          Describe a non-trivial admissible heuristic that we could use.

          One heuristic would be the number of items on the east bank.
          If a move would put another item on the east bank,
          it should be preferable to one that would remove an item from the east bank.
      \end{enumerate}
  \end{enumerate}
\end{document}
