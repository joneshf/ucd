\documentclass[12pt,letterpaper]{article}

\usepackage[margin=1in]{geometry}
\usepackage[round-mode=figures,round-precision=3,scientific-notation=false]{siunitx}
\usepackage[super]{nth}
\usepackage[title]{appendix}
\usepackage{amsfonts}
\usepackage{amsmath}
\usepackage{amsthm}
\usepackage{cancel}
\usepackage{caption}
\usepackage{color, colortbl}
\usepackage{dcolumn}
\usepackage{enumitem}
\usepackage{float}
\usepackage{fp}
\usepackage{mathtools}
\usepackage{pgfplots}
\usepackage{subcaption}
\usepackage{tikz}
\usepackage{titling}

\usepgfplotslibrary{statistics}

\definecolor{Gray}{gray}{0.8}

\pgfplotsset{compat=1.8}

\newcolumntype{d}{D{.}{.}{-1}}
\newcolumntype{g}{>{\columncolor{Gray}}c}

\newcommand*\biconditional[3]{
  This statement is #1 since both sides of the bi-conditional have #2 truth values.

  #3
}
\newcommand*\biconditionaltrue[2]{
  \biconditional{true}{the same}{#1 and #2.}
}
\newcommand*\biconditionalfalse[2]{
  \biconditional{false}{different}{#1 yet #2.}
}

\renewcommand{\labelenumi}{\S 1.\arabic*}
\renewcommand{\labelenumii}{\arabic*}
\renewcommand{\labelenumiii}{(\alph*)}

\setlength{\droptitle}{-10ex}

\preauthor{\begin{flushright}\large \lineskip 0.5em}
\postauthor{\par\end{flushright}}
\predate{\begin{flushright}\large}
\postdate{\par\end{flushright}}

\title{MAT 108 HW 2\vspace{-2ex}}
\author{Hardy Jones\\
        999397426\\
        Professor Bandyopadhyay\vspace{-2ex}}
\date{Spring 2015}

\begin{document}
  \maketitle

  \begin{enumerate}
    \setcounter{enumi}{1}
    \item
      \begin{enumerate}
        \setcounter{enumii}{5}
        \item
          \begin{enumerate}
            \item \biconditionaltrue{Triangles have three sides}{squares have four sides}
            \item \biconditionaltrue{7 + 5 = 12}{1 + 1 = 2}
            \setcounter{enumiii}{3}
            \item \biconditionaltrue{parallelograms have four sides}{27 is not prime.}
          \end{enumerate}
        \setcounter{enumii}{9}
        \item
          \begin{enumerate}
            \item
              ($f$ has a relative minimum at $x_0$ $\land$ $f$ is differentiable at $x_0$) $\implies$ $f'(x_0) = 0$
            \item
              $n$ is prime $\implies$ ($n = 2$ $\lor$ $n$ is odd)
            \item
              $R$ is irreflexive $\implies$ ($R$ is symmetric $\land$ $R$ is transitive)
            \item
              det $\mathbf{B} = 0$ $\implies$ ($\mathbf{B}$ is square $\land$ not invertible)
            \setcounter{enumiii}{5}
            \item
              ($2n < 4 \lor n > 4$) $\implies$ $2 < n - 6$
            \item
              $6 \geq n - 3$ $\implies$ ($n > 4 \lor n > 10$)
            \item
              $x$ is Cauchy $\implies$ $x$ is convergent.
          \end{enumerate}
        \setcounter{enumii}{14}
        \item
          \begin{enumerate}
            \item
              ($f$ has a relative minimum at $x_0$ $\land$ $f$ is differentiable at $x_0$) $\implies$ $f'(x_0) = 0$

              \begin{itemize}
                \item Converse:

                  $f'(x_0) = 0$ $\implies$ ($f$ has a relative minimum at $x_0$ $\land$ $f$ is differentiable at $x_0$)

                  This sentence is false since $f'(x_0) = 0$ can be true, but $f$ may have a relative maximum at $x_0$.

                \item Contrapositive:

                  ($\sim$ $f'(x_0) = 0$) $\implies$ ($\sim$ ($f$ has a relative minimum at $x_0$ $\land$ $f$ is differentiable at $x_0$))

                  $f'(x_0) \neq 0$ $\implies$ ($\sim$ $f$ has a relative minimum at $x_0$ $\lor$ $\sim$ $f$ is differentiable at $x_0$)

                  $f'(x_0) \neq 0$ $\implies$ ($f$ does not have a relative minimum at $x_0$ $\lor$ $f$ is not differentiable at $x_0$)

                  Since the original statement is true,
                  and the contrapositive has the same truth value,
                  this sentence is also true.

              \end{itemize}

            \item
              $n$ is prime $\implies$ ($n = 2$ $\lor$ $n$ is odd)

              \begin{itemize}
                \item Converse:

                  ($n = 2$ $\lor$ $n$ is odd) $\implies$ $n$ is prime

                  This sentence is false.
                  If $n = 9$ then $n$ is odd, but $n$ is not prime.
                \item Contrapositive:

                  ($\sim$ ($n = 2$ $\lor$ $n$ is odd)) $\implies$ ($\sim$ $n$ is prime)

                  ($\sim$ $n = 2$ $\lor$ $\sim$ $n$ is odd) $\implies$ $n$ is not prime

                  ($n \neq 2$ $\lor$ $n$ is not odd) $\implies$ $n$ is not prime

                  Since the original statement is true,
                  and the contrapositive has the same truth value,
                  this sentence is also true.

              \end{itemize}

            \setcounter{enumiii}{5}
            \item
              ($2n < 4 \lor n > 4$) $\implies$ $2 < n - 6$

              \begin{itemize}
                \item Converse:

                  $2 < n - 6$ $\implies$ ($2n < 4 \lor n > 4$)

                  This sentence is true.

                  We can see this by simplifying both sides a bit:

                  $8 < n$ $\implies$ ($n < 2 \lor 4 < n$)

                  If $n$ is greater than 8, it is also greater than 4.
                  So if the antecedent is true, the whole sentence is true.

                \item Contrapositive:

                  ($\sim$ $2 < n - 6$) $\implies$ ($\sim$ ($2n < 4 \lor n > 4$))

                  $2 \geq n - 6$ $\implies$ ($\sim$ $2n < 4$ $\land$ $\sim$ $n > 4$)

                  $2 \geq n - 6$ $\implies$ ($2n \geq 4$ $\land$ $n \leq 4$)

                  This sentence is false.

                  We can see this by simplifying both sides a bit:

                  $8 \geq n$ $\implies$ ($n \geq 2$ $\land$ $4 \geq n$)

                  If $n$ is 0, then the antecedent is true,
                  but 0 is not greater than or equal to 2 so the consequent is false.
              \end{itemize}

            \item
              $6 \geq n - 3$ $\implies$ ($n > 4 \lor n > 10$)

              \begin{itemize}
                \item Converse:

                  ($n > 4 \lor n > 10$) $\implies$ $6 \geq n - 3$

                  This sentence is false.

                  We can see this by simplifying both sides a bit:

                  ($n > 4 \lor n > 10$) $\implies$ $9 \geq n$

                  If $n$ is 100, then the antecedent is true,
                  but 100 is not less than or equal to 9 so the consequent is false.

                \item Contrapositive:

                  ($\sim$ ($n > 4 \lor n > 10$)) $\implies$ ($\sim$ $6 \geq n - 3$)

                  ($\sim$ $n > 4$ $\land$ $\sim$ $n > 10$) $\implies$ ($6 < n - 3$)

                  ($n \leq 4$ $\land$ $n \leq 10$) $\implies$ ($6 < n - 3$)

                  This sentence is false.

                  We can see this by simplifying both sides a bit:

                  ($n \leq 4$ $\land$ $n \leq 10$) $\implies$ ($9 < n$)

                  If $n$ is 0, then the antecedent is true,
                  but 0 is not greater than 9 so the consequent is false.
              \end{itemize}
          \end{enumerate}
        \item
          \begin{enumerate}
            \item
              We can use a truth table to enumerate all possibilities.

              \begin{tabular}{c | c | c | c | g}
                $P$ & $Q$ & $P \implies Q$ & $(P \implies Q) \implies Q$ & $[(P \implies Q) \implies Q] \implies P$ \\
                \hline
                T & T & T & T & T \\
                T & F & F & T & T \\
                F & T & T & T & F \\
                F & F & T & F & T \\
              \end{tabular}

              Since the truth value of final column is neither all true nor all false,
              this is neither a tautology nor a contradiction.
            \item
              We can use a truth table to enumerate all possibilities.

              \begin{tabular}{c | c | c | c | g}
                $P$ & $Q$ & $P \lor Q$ & $P \land (P \lor Q)$ & $P \iff P \land (P \lor Q)$ \\
                \hline
                T & T & T & T & T \\
                T & F & T & T & T \\
                F & T & T & F & T \\
                F & F & F & F & T \\
              \end{tabular}

              Since the truth value of final column is all true,
              this is a tautology.
            \item
              We can use a truth table to enumerate all possibilities.

              \begin{tabular}{c | c | c | c | c | g}
                $P$ & $Q$ & $P \implies Q$ & $\sim Q$ & $P \land \sim Q$ & $P \implies Q \iff P \land \sim Q$ \\
                \hline
                T & T & T & F & F  & F \\
                T & F & F & T & T  & F \\
                F & T & T & F & F  & F \\
                F & F & T & T & F  & F \\
              \end{tabular}

              Since the truth value of final column is all false,
              this is a contradiction.
          \end{enumerate}
      \end{enumerate}
    \item
      \begin{enumerate}
        \item
          \begin{enumerate}
            \setcounter{enumiii}{1}
            \item $(\forall x)(x \text{ is precious} \implies x \text{ is not beautiful})$
            \setcounter{enumiii}{3}
            \item $\sim (\exists x)(x \text{ is a right triangle} \land x \text{ is isosceles})$
            \item $(\forall x)(x \text{ is not isosceles} \implies x \text{ is a right triangle})$
            \setcounter{enumiii}{8}
            \item $(\forall x)(x \in \mathbb{Z} \implies (x > -4 \lor x < 6))$
            \setcounter{enumiii}{11}
            \item $(\forall x)(\forall y)(x \in \mathbb{Z} \land y \in \mathbb{Z} \land x < y \implies (\exists z)(x < z < y))$
          \end{enumerate}
        \setcounter{enumii}{4}
        \item
          \begin{itemize}
            \item
              All people dislike all taxes:

              $(\forall x)(x \text{ is a person} \implies (\forall y)(y \text{ is a tax} \implies x \text{ dislikes } y))$
            \item
              All people dislike some taxes:

              $(\forall x)(x \text{ is a person} \implies (\exists y)(y \text{ is a tax} \land x \text{ dislikes } y))$
            \item
              Some people dislike all taxes:

              $(\exists x)(x \text{ is a person} \land (\forall y)(y \text{ is a tax} \implies x \text{ dislikes } y))$
            \item
              Some people dislike some taxes:

              $(\exists x)(x \text{ is a person} \land (\exists y)(y \text{ is a tax} \land x \text{ dislikes } y))$
          \end{itemize}
        \item
          \begin{enumerate}
            \item
              This statement is true in all four universes $T, U, V, W$,
              using the elements $17, 6, 24, 2$ respectively.

              In $T$, both the antecedent and the consequent are true so the statement is true.
              In the other three, the antecedent is false so the statement is true.
            \item
              This statement is true only in universe $T$.
              The other three have no elements which satisfy both parts of the conjunct.
            \item
              This statement is true in universes $T, U, V$.

              In $T$, both the antecedent and the consequent are true so the statement is true.
              In $U$ and $V$, the antecedent is false so the statement is true.
              In $W$, one of the elements--namely 7--has a true antecedent but a false consequent so the statement is false.
            \item
              This statement is true only in universe $T$.
              The other three have no elements which satisfy both parts of the conjunct.
          \end{enumerate}
        \setcounter{enumii}{7}
        \item
          \begin{enumerate}
            \setcounter{enumiii}{2}
            \item
              This is false.
              We can see this by simplifying a bit.

              $(\exists x)(2x + 3 = 6x + 7)$

              $(\exists x)(2x = 6x + 4)$

              $(\exists x)(-4x = 4)$

              $(\exists x)(x = -1)$

              Since the universe is $\mathbb{N}$,
              it does not contain negative numbers.
              So, there are no elements which satisfy this sentence.
            \item
              We can see this by graphing the two equations.

              In Figure \ref{fig:8d} we see that the two lines intersect,
              so there exists some real number such that $3^x = x^2$.

              Thus, the sentence is true.

            \item
              We can see this by graphing the two equations.

              In Figure \ref{fig:8e} we see that the two do not intersect,
              so there is no real number such that $3^x = x$.

              Thus, the sentence is false.
            \item
              This sentence is true.

              We can see this by simplifying the equation a bit.

              \begin{align*}
                3(2 - x) &= 5 + 8(1 - x) \\
                6 - 3x &= 5 + 8 - 8x \\
                6 - 3x &= 13 - 8x \\
                5x &= 7 \\
                x &= \frac{7}{5} \\
              \end{align*}

              So there is exactly one value that satisfies the sentence, in specific $\frac{7}{5}$.

              Thus, the sentence is true.
            \setcounter{enumiii}{10}
            \item
              This sentence is false.

              If we let $x = -20$ then we have the value
              \[
                (-20)^3 + 17(-20)^2 + 6(-20) + 100 = -1220
              \]
              and this is less than 0.

              So the sentence is false.
            \item
              This sentence is true.

              For any two real numbers we can construct a new number between the two such that $w = \frac{x + y}{2}$.
          \end{enumerate}

          \begin{figure}
            \centering
            \begin{minipage}{.5\textwidth}
              \centering
              \begin{tikzpicture}
                \begin{axis}[ smooth
                            , axis x line=bottom
                            , axis y line=center
                            , axis on top=true
                            , xlabel=x
                            , xmax=5
                            , xmin=-5
                            , ylabel=y
                            , ymax=5
                            , ymin=0
                            ]
                  \addplot[very thick]{x^2}
                    node[below, pos=0.55, sloped]{$x^2$}
                  ;
                  \addplot[domain=-5:5, very thick]{3^x}
                    node[above, pos=0.03, sloped]{$3^x$}
                  ;
                \end{axis}
              \end{tikzpicture}
              \captionof{figure}{8 (d)}
              \label{fig:8d}
            \end{minipage}%
            \begin{minipage}{.5\textwidth}
              \centering
              \begin{tikzpicture}
                \begin{axis}[ smooth
                            , axis x line=middle
                            , axis y line=center
                            , axis on top=true
                            , xlabel=x
                            , xmax=5
                            , xmin=-5
                            , ylabel=y
                            , ymax=5
                            , ymin=-5
                            ]
                  \addplot[domain=-5:5, very thick]{x}
                    node[below, pos=0.65, sloped]{$x$}
                  ;
                  \addplot[domain=-5:5, very thick]{3^x}
                    node[above, pos=0.03, sloped]{$3^x$}
                  ;
                \end{axis}
              \end{tikzpicture}
              \captionof{figure}{8 (e)}
              \label{fig:8e}
            \end{minipage}
          \end{figure}

        \item
          \begin{enumerate}
            \item All naturals are greater than or equal to 1.
            \setcounter{enumiii}{2}
            \item All naturals that are prime and not equal to 2 are odd.
            \setcounter{enumiii}{6}
            \item For any odd natural $x$, $x^2$ is also odd.
          \end{enumerate}
        \item
          \begin{enumerate}
            \setcounter{enumiii}{8}
            \item
              This sentence is false.
              If we were given some unique $x$ then we would have to have:
              $x = 3^2 = 9$ and also $x = 4^2 = 16$, but this is false.
            \item
              This sentence is true since $x = y^2$ is a function.
            \item
              This sentence is false since the $x$ and $y$ aren't unique.
          \end{enumerate}
        \item
          \begin{enumerate}
            \setcounter{enumiii}{1}
            \item

              The converse of Theorem 1.3.2 (a) is:

              If $A(x)$ is an open sentence with variable $x$, then
              $(\exists x)A(x) \implies (\exists ! x)A(x)$

              We want to prove this statement false.

              \begin{proof}
                Let $A(x)$ be the open sentence $x > 1$ in the universe $\mathbb{N}$.

                Then we have the antecedent true,
                as every element in $\mathbb{N} \setminus \{1\}$ is greater than 1.

                However, the consequent is not true
                since $\mathbb{N} \setminus \{1\}$ contains more than one element.

                Thus this theorem is not true for all open sentences.
              \end{proof}
          \end{enumerate}
        \item
          \begin{enumerate}
            \item
              This sentence does not have to be true.
              We can have $a_n = b_n$, but $a_0 \neq b_0$ and
              then the two polynomials are not equal.
            \item
              This sentence does not have to be true.
              The reason is the same as the previous argument.
              All of the individual coefficients could be the same
              except for one and the polynomials would be different.
            \item
              This sentence does not have to be true.
              This sentence is a recapitulation of the previous sentence.
              So the answer is the same.
            \item
              This sentence must be true.
              There must be at least one coefficient that is not equal between the two polynomials.
          \end{enumerate}
      \end{enumerate}
  \end{enumerate}
\end{document}
