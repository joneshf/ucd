\documentclass[12pt,letterpaper]{article}

\usepackage[margin=1in]{geometry}
\usepackage[round-mode=figures,round-precision=3,scientific-notation=false]{siunitx}
\usepackage[super]{nth}
\usepackage[title]{appendix}
\usepackage{amsfonts}
\usepackage{amsmath}
\usepackage{amsthm}
\usepackage{cancel}
\usepackage{color, colortbl}
\usepackage{dcolumn}
\usepackage{enumitem}
\usepackage{fp}
\usepackage{mathtools}
\usepackage{pgfplots}
\usepackage{titling}

\usepgfplotslibrary{statistics}

\definecolor{Gray}{gray}{0.8}

\pgfplotsset{compat=1.8}

\newcolumntype{d}{D{.}{.}{-1}}
\newcolumntype{g}{>{\columncolor{Gray}}c}

\newcommand*\biconditional[3]{
  This statement is #1 since both sides of the bi-conditional have #2 truth values.

  #3
}
\newcommand*\biconditionaltrue[2]{
  \biconditional{true}{the same}{#1 and #2.}
}
\newcommand*\biconditionalfalse[2]{
  \biconditional{false}{different}{#1 yet #2.}
}

\renewcommand{\labelenumi}{\S 1.\arabic*}
\renewcommand{\labelenumii}{\arabic*}
\renewcommand{\labelenumiii}{(\alph*)}

\setlength{\droptitle}{-10ex}

\preauthor{\begin{flushright}\large \lineskip 0.5em}
\postauthor{\par\end{flushright}}
\predate{\begin{flushright}\large}
\postdate{\par\end{flushright}}

\title{MAT 108 HW 2\vspace{-2ex}}
\author{Hardy Jones\\
        999397426\\
        Professor Bandyopadhyay\vspace{-2ex}}
\date{Spring 2015}

\begin{document}
  \maketitle

  \begin{enumerate}
    \setcounter{enumi}{1}
    \item
      \begin{enumerate}
        \setcounter{enumii}{5}
        \item
          \begin{enumerate}
            \item \biconditionaltrue{Triangles have three sides}{squares have four sides}
            \item \biconditionaltrue{7 + 5 = 12}{1 + 1 = 2}
            \setcounter{enumiii}{3}
            \item \biconditionaltrue{parallelograms have four sides}{27 is not prime.}
          \end{enumerate}
        \setcounter{enumii}{9}
        \item
          \begin{enumerate}
            \item
              ($f$ has a relative minimum at $x_0$ $\land$ $f$ is differentiable at $x_0$) $\implies$ $f'(x_0) = 0$
            \item
              $n$ is prime $\implies$ ($n = 2$ $\lor$ $n$ is odd)
            \item
              $R$ is irreflexive $\implies$ ($R$ is symmetric $\land$ $R$ is transitive)
            \item
              det $\mathbf{B} = 0$ $\implies$ ($\mathbf{B}$ is square $\land$ not invertible)
            \setcounter{enumiii}{5}
            \item
              ($2n < 4 \lor n > 4$) $\implies$ $2 < n - 6$
            \item
              $6 \geq n - 3$ $\implies$ ($n > 4 \lor n > 10$)
            \item
              $x$ is Cauchy $\implies$ $x$ is convergent.
          \end{enumerate}
        \setcounter{enumii}{14}
        \item
          \begin{enumerate}
            \item
              ($f$ has a relative minimum at $x_0$ $\land$ $f$ is differentiable at $x_0$) $\implies$ $f'(x_0) = 0$

              \begin{itemize}
                \item Converse:

                  $f'(x_0) = 0$ $\implies$ ($f$ has a relative minimum at $x_0$ $\land$ $f$ is differentiable at $x_0$)

                  This sentence is false since $f'(x_0) = 0$ can be true, but $f$ may have a relative maximum at $x_0$.

                \item Contrapositive:

                  ($\sim$ $f'(x_0) = 0$) $\implies$ ($\sim$ ($f$ has a relative minimum at $x_0$ $\land$ $f$ is differentiable at $x_0$))

                  $f'(x_0) \ne 0$ $\implies$ ($\sim$ $f$ has a relative minimum at $x_0$ $\lor$ $\sim$ $f$ is differentiable at $x_0$)

                  $f'(x_0) \ne 0$ $\implies$ ($f$ does not have a relative minimum at $x_0$ $\lor$ $f$ is not differentiable at $x_0$)

                  Since the original statement is true,
                  and the contrapositive has the same truth value,
                  this sentence is also true.

              \end{itemize}

            \item
              $n$ is prime $\implies$ ($n = 2$ $\lor$ $n$ is odd)

              \begin{itemize}
                \item Converse:

                  ($n = 2$ $\lor$ $n$ is odd) $\implies$ $n$ is prime

                  This sentence is false.
                  If $n = 9$ then $n$ is odd, but $n$ is not prime.
                \item Contrapositive:

                  ($\sim$ ($n = 2$ $\lor$ $n$ is odd)) $\implies$ ($\sim$ $n$ is prime)

                  ($\sim$ $n = 2$ $\lor$ $\sim$ $n$ is odd) $\implies$ $n$ is not prime

                  ($n \ne 2$ $\lor$ $n$ is not odd) $\implies$ $n$ is not prime

                  Since the original statement is true,
                  and the contrapositive has the same truth value,
                  this sentence is also true.

              \end{itemize}

            \setcounter{enumiii}{5}
            \item
              ($2n < 4 \lor n > 4$) $\implies$ $2 < n - 6$

              \begin{itemize}
                \item Converse:

                  $2 < n - 6$ $\implies$ ($2n < 4 \lor n > 4$)

                  This sentence is true.

                  We can see this by simplifying both sides a bit:

                  $8 < n$ $\implies$ ($n < 2 \lor 4 < n$)

                  If $n$ is greater than 8, it is also greater than 4.
                  So if the antecedent is true, the whole sentence is true.

                \item Contrapositive:

                  ($\sim$ $2 < n - 6$) $\implies$ ($\sim$ ($2n < 4 \lor n > 4$))

                  $2 \geq n - 6$ $\implies$ ($\sim$ $2n < 4$ $\land$ $\sim$ $n > 4$)

                  $2 \geq n - 6$ $\implies$ ($2n \geq 4$ $\land$ $n \leq 4$)

                  This sentence is false.

                  We can see this by simplifying both sides a bit:

                  $8 \geq n$ $\implies$ ($n \geq 2$ $\land$ $4 \geq n$)

                  If $n$ is 0, then the antecedent is true,
                  but 0 is not greater than or equal to 2 so the consequent is false.
              \end{itemize}

            \item
              $6 \geq n - 3$ $\implies$ ($n > 4 \lor n > 10$)

              \begin{itemize}
                \item Converse:

                  ($n > 4 \lor n > 10$) $\implies$ $6 \geq n - 3$

                  This sentence is false.

                  We can see this by simplifying both sides a bit:

                  ($n > 4 \lor n > 10$) $\implies$ $9 \geq n$

                  If $n$ is 100, then the antecedent is true,
                  but 100 is not less than or equal to 9 so the consequent is false.

                \item Contrapositive:

                  ($\sim$ ($n > 4 \lor n > 10$)) $\implies$ ($\sim$ $6 \geq n - 3$)

                  ($\sim$ $n > 4$ $\land$ $\sim$ $n > 10$) $\implies$ ($6 < n - 3$)

                  ($n \leq 4$ $\land$ $n \leq 10$) $\implies$ ($6 < n - 3$)

                  This sentence is false.

                  We can see this by simplifying both sides a bit:

                  ($n \leq 4$ $\land$ $n \leq 10$) $\implies$ ($9 < n$)

                  If $n$ is 0, then the antecedent is true,
                  but 0 is not greater than 9 so the consequent is false.
              \end{itemize}
          \end{enumerate}
        \item
          \begin{enumerate}
            \item
              We can use a truth table to enumerate all possibilities.

              \begin{tabular}{c | c | c | c | g}
                $P$ & $Q$ & $P \implies Q$ & $(P \implies Q) \implies Q$ & $[(P \implies Q) \implies Q] \implies P$ \\
                \hline
                T & T & T & T & T \\
                T & F & F & T & T \\
                F & T & T & T & F \\
                F & F & T & F & T \\
              \end{tabular}

              Since the truth value of final column is neither all true nor all false,
              this is neither a tautology nor a contradiction.
            \item
              We can use a truth table to enumerate all possibilities.

              \begin{tabular}{c | c | c | c | g}
                $P$ & $Q$ & $P \lor Q$ & $P \land (P \lor Q)$ & $P \iff P \land (P \lor Q)$ \\
                \hline
                T & T & T & T & T \\
                T & F & T & T & T \\
                F & T & T & F & T \\
                F & F & F & F & T \\
              \end{tabular}

              Since the truth value of final column is all true,
              this is a tautology.
            \item
              We can use a truth table to enumerate all possibilities.

              \begin{tabular}{c | c | c | c | c | g}
                $P$ & $Q$ & $P \implies Q$ & $\sim Q$ & $P \land \sim Q$ & $P \implies Q \iff P \land \sim Q$ \\
                \hline
                T & T & T & F & F  & F \\
                T & F & F & T & T  & F \\
                F & T & T & F & F  & F \\
                F & F & T & T & F  & F \\
              \end{tabular}

              Since the truth value of final column is all false,
              this is a contradiction.
          \end{enumerate}
      \end{enumerate}
    \item
      \begin{enumerate}
        \item
          \begin{enumerate}
            \setcounter{enumiii}{1}
            \item
            \setcounter{enumiii}{3}
            \item
            \item
            \setcounter{enumiii}{8}
            \item
            \setcounter{enumiii}{11}
            \item
          \end{enumerate}
        \setcounter{enumii}{4}
        \item
        \item
        \setcounter{enumii}{7}
        \item
          \begin{enumerate}
            \setcounter{enumiii}{2}
            \item
            \item
            \item
            \item
            \setcounter{enumiii}{10}
            \item
            \item
          \end{enumerate}
        \item
          \begin{enumerate}
            \item
            \setcounter{enumiii}{2}
            \item
            \setcounter{enumiii}{6}
            \item
          \end{enumerate}
        \item
          \begin{enumerate}
            \setcounter{enumiii}{8}
            \item
            \item
            \item
          \end{enumerate}
        \item
          \begin{enumerate}
            \setcounter{enumiii}{1}
            \item
          \end{enumerate}
        \item
          \begin{enumerate}
            \item
            \item
            \item
            \item
          \end{enumerate}
      \end{enumerate}
  \end{enumerate}
\end{document}
