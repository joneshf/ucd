\documentclass[12pt,letterpaper]{article}
\usepackage{amsmath}
\usepackage{amsfonts}
\usepackage{amsthm}
\usepackage{mathtools}
\usepackage{cancel}
\usepackage[margin=1in]{geometry}
\usepackage{titling}
\usepackage{fp}
\usepackage{enumitem}
\usepackage[super]{nth}
\usepackage{dcolumn}
\usepackage[title]{appendix}
\usepackage{pgfplots}
\pgfplotsset{compat=1.8}
\usepgfplotslibrary{statistics}
\usepackage[round-mode=figures,round-precision=3,scientific-notation=false]{siunitx}
\usepackage{color, colortbl}
\definecolor{Gray}{gray}{0.8}
\newcolumntype{g}{>{\columncolor{Gray}}c}

\newcolumntype{d}{D{.}{.}{-1}}

\setlength{\droptitle}{-10ex}

\preauthor{\begin{flushright}\large \lineskip 0.5em}
\postauthor{\par\end{flushright}}
\predate{\begin{flushright}\large}
\postdate{\par\end{flushright}}

\title{MAT 108 HW 1\vspace{-2ex}}
\author{Hardy Jones\\
        999397426\\
        Professor Bandyopadhyay\vspace{-2ex}}
\date{Spring 2015}

\begin{document}
  \maketitle

  \begin{enumerate}[label=Problem \arabic*]
    \item
      \begin{enumerate}[label=\arabic*.]
        \item This sentence is a proposition. Since 18 is not a multiple of 12, this proposition is false.
        \item This sentence is a proposition. Since 3 is prime, this proposition is false.
        \item This sentence is not a proposition.
        \item This sentence is a proposition. Since $-\frac{1}{2}$ is rational and $3\pi$ is less than 10, this proposition is true.
        \item This sentence is not a proposition.
      \end{enumerate}
    \item
      \begin{enumerate}[label=\arabic*.]
        \item Since $B$ is a contradiction, this propositional phrase is always false. So this phrase is a contradiction.
        \item Since $B$ is a contradiction, the inner phrase is always false, and the negation of that will always be true. So this phrase is a tautology.
        \item Since $A$ is a tautology, this propositional phrase is always true. So this phrase is a tautology.
      \end{enumerate}
    \item
      \begin{enumerate}[label=\arabic*.]
        \item

          \begin{tabular}{c | c | c | c | g | c | c | g}
            $P$ & $Q$ & $R$ & $Q \vee R$ & $P \wedge (Q \vee R)$ & $P \wedge Q$ & $P \wedge R$ & $(P \wedge Q) \vee (P \wedge R)$ \\
            \hline
            T & T & T & T & T & T & T & T \\
            T & T & F & T & T & T & F & T \\
            T & F & T & T & T & F & T & T \\
            T & F & F & F & F & F & F & F \\
            F & T & T & T & F & F & F & F \\
            F & T & F & T & F & F & F & F \\
            F & F & T & T & F & F & F & F \\
            F & F & F & F & F & F & F & F \\
          \end{tabular}

          Since the two shaded columns have the same values at all respective locations,
          the two propositional phrases are equivalent.
        \item

          \begin{tabular}{c | c | c | c | g | c | g}
            $P$ & $Q$ & $R$ & $P \wedge Q$ & $(P \wedge Q) \vee R$ & $Q \wedge R$ & $P \vee (Q \wedge R)$ \\
            \hline
            T & T & T & T & T & T & T \\
            T & T & F & T & T & F & T \\
            T & F & T & F & T & F & T \\
            T & F & F & F & F & F & T \\
            F & T & T & F & T & T & T \\
            F & T & F & F & F & F & F \\
            F & F & T & F & T & F & F \\
            F & F & F & F & F & F & F \\
          \end{tabular}

          Since the two shaded columns do not have the same values at all respective locations,
          the two propositional phrases are not equivalent.
      \end{enumerate}
  \end{enumerate}

\end{document}
