\documentclass[12pt,letterpaper]{article}

\usepackage[margin=1in]{geometry}
\usepackage[round-mode=figures,round-precision=3,scientific-notation=false]{siunitx}
\usepackage[super]{nth}
\usepackage[title]{appendix}
\usepackage{amsfonts}
\usepackage{amsmath}
\usepackage{amsthm}
\usepackage{cancel}
\usepackage{caption}
\usepackage{color, colortbl}
\usepackage{dcolumn}
\usepackage{enumitem}
\usepackage{float}
\usepackage{fp}
\usepackage{mathtools}
\usepackage{pgfplots}
\usepackage{subcaption}
\usepackage{tabularx}
\usepackage{tikz}
\usepackage{titling}

\usepgfplotslibrary{statistics}

\definecolor{Gray}{gray}{0.8}

\pgfplotsset{compat=1.8}

\newcolumntype{d}{D{.}{.}{-1}}
\newcolumntype{g}{>{\columncolor{Gray}}c}

\newcommand*\biconditional[3]{
  This statement is #1 since both sides of the bi-conditional have #2 truth values.

  #3
}
\newcommand*\biconditionaltrue[2]{
  \biconditional{true}{the same}{#1 and #2.}
}
\newcommand*\biconditionalfalse[2]{
  \biconditional{false}{different}{#1 yet #2.}
}

\newcommand*\directproofsimple[6]{
  With integers $x$ and $y$, we want to show:

  If #1, then #2 is even.

  \begin{proof}
    Suppose #1.

    Then there exist some integers $p, q$ such that #3 and #4.

    Then we have #5$ = 2($#6$)$.

    Since #6 is an integer, we can rename as $r = $ #6.

    So #2 $ = 2r$, and is even.

    Thus if #1, then #2 is even
  \end{proof}
}

\renewcommand{\labelenumi}{\S 1.\arabic*}
\renewcommand{\labelenumii}{\arabic*}
\renewcommand{\labelenumiii}{(\alph*)}

\setlength{\droptitle}{-10ex}

\preauthor{\begin{flushright}\large \lineskip 0.5em}
\postauthor{\par\end{flushright}}
\predate{\begin{flushright}\large}
\postdate{\par\end{flushright}}

\title{MAT 108 HW 2\vspace{-2ex}}
\author{Hardy Jones\\
        999397426\\
        Professor Bandyopadhyay\vspace{-2ex}}
\date{Spring 2015}

\begin{document}
  \maketitle

  \begin{enumerate}
    \setcounter{enumi}{3}
    \item
      \begin{enumerate}
        \setcounter{enumii}{4}
        \item
          \begin{enumerate}
            \setcounter{enumiii}{4}
            \item
              \directproofsimple
                {$x$ and $y$ are odd}
                {$x + y$}
                {$x = 2p + 1$}
                {$y = 2q + 1$}
                {$x + y = (2p + 1) + (2q + 1) = 2p + 2q + 2$}
                {$p + q + 1$}
            \item
              \directproofsimple
                {$x$ and $y$ are odd}
                {$3x - 5y$}
                {$x = 2p + 1$}
                {$y = 2q + 1$}
                {$3x - 5y = 3(2p + 1) - 5(2q + 1) = 6p + 3 - 10q - 5 = 6p - 10q - 2$}
                {$3p - 5q - 1$}
            \setcounter{enumiii}{8}
            \item
              \directproofsimple
                {$x$ is even and $y$ is odd}
                {$xy$}
                {$x = 2p$}
                {$y = 2q + 1$}
                {$xy = (2p)(2q + 1) = 4pq + 2p$}
                {$2pq + p$}
          \end{enumerate}
        \item
          \begin{enumerate}
            \setcounter{enumiii}{3}
            \item
              With real numbers $a, b$ we want to prove $|a + b| \leq |a| + |b|$.
              \begin{proof}
                We prove this by cases.

                For cases 2 and 3,
                we choose without loss of generality $a \geq 0, b < 0$.
                The exact same argument holds for $a < 0, b \geq 0$.
                \begin{enumerate}[label=Case \arabic*]
                  \item $a \geq 0, b \geq 0$

                    Since $a \geq 0, b \geq 0, a + b \geq 0$.

                    So $|a + b| = a + b$.

                    Also, $|a| = a$ and $|b| = b$.

                    So $|a + b| = a + b = |a| + |b|$.

                    Thus, $|a + b| \leq |a| + |b|$.
                  \item $a \geq 0, b < 0, a + b \geq 0$

                    Since $a + b \geq 0, |a + b| = a + b$.

                    Also, $|a| = a$ and $|b| = -b$.

                    Since $b < 0 \implies 2b < 0 \implies b < -b \implies a + b < a + (-b)$,

                    we have $|a + b| = a + b < a + (-b) = |a| + |b|$.

                    Thus, $|a + b| \leq |a| + |b|$.
                  \item $a \geq 0, b < 0, a + b < 0$

                    Since $a + b < 0, |a + b| = -(a + b) = -a - b$.

                    Also, $|a| = a$ and $|b| = -b$.

                    Since $0 \leq a \implies 0 \leq 2a \implies -a \leq a \implies -a - b \leq a - b$,

                    we have $|a + b| = -a - b \leq a - b = a + (-b) = |a| + |b|$.

                    Thus, $|a + b| \leq |a| + |b|$.
                  \item $a < 0, b < 0$

                    Since $a < 0, b < 0, a + b < 0$.

                    So $|a + b| = - (a + b) = -a - b$.

                    Also, $|a| = -a$ and $|b| = -b$.

                    So $|a + b| = -a - b = -a + (-b) = |a| + |b|$.

                    Thus, $|a + b| \leq |a| + |b|$.
                \end{enumerate}

                Since these are all the possible cases,
                we have proven by exhaustion that

                $|a + b| \leq |a| + |b|$.
              \end{proof}
            \item
              With real numbers $a, b$ we want to prove if $|a| \leq b$, then $-b \leq a \leq b$.

              \begin{proof}
                We prove this by cases.

                \begin{enumerate}[label=Case \arabic*]
                  \item $a \geq 0$

                    Assume $|a| \leq b$.

                    Since $a \geq 0$, we have $|a| = a$.

                    Now, we know $|a| = a \leq b$.

                    Also $0 \leq |a| = a \leq b \implies 0 \leq b$.
                    And since both $a$ and $b$ are non-negative,
                    $a + b$ is also non-negative.

                    So we have $0 \leq a + b \implies -b \leq a$.

                    Then we have both $-b \leq a$ and $a \leq b$ or $-b \leq a \leq b$.

                    Thus if $|a| \leq b$, then $-b \leq a \leq b$.

                  \item $a < 0$

                    Assume $|a| \leq b$.

                    Since $a < 0$, we have $|a| = -a$.

                    Then, $|a| = -a \leq b \implies -a - b \leq 0 \implies -b \leq a$.

                    Now, by the definition of absolute value, $|a| \leq b \implies 0 \leq b$.

                    And $a < 0 \implies a < 0 \leq b \implies a < b \implies a \leq b$.

                    Then we have both $-b \leq a$ and $a \leq b$ or $-b \leq a \leq b$.

                    Thus if $|a| \leq b$, then $-b \leq a \leq b$.
                \end{enumerate}

                Since these are the only possible cases,
                we have proved by exhaustion that:

                If $|a| \leq b$, then $-b \leq a \leq b$.
              \end{proof}
            \item
              With real numbers $a, b$ we want to prove if $-b \leq a \leq b$, then $|a| \leq b$.

              \begin{proof}
                We prove this by cases.

                \begin{enumerate}[label=Case \arabic*]
                  \item $a \geq 0$

                    Assume $-b \leq a \leq b$.

                    Since $a \geq 0$, we have $|a| = a$.

                    We have $-b \leq a \leq b \implies a \leq b$.

                    Then, $a = |a| \leq b$.

                    Thus, if $-b \leq a \leq b$, then $|a| \leq b$.
                  \item $a < 0$

                    Assume $-b \leq a \leq b$.

                    Since $a < 0$, we have $|a| = -a$.

                    We have $-b \leq a \leq b \implies -b \leq a \implies -a - b \leq 0 \implies -a \leq b$.

                    Then $-a = |a| \leq b$.

                    Thus, if $-b \leq a \leq b$, then $|a| \leq b$.
                \end{enumerate}

                Since these are the only possible cases,
                we have proved by exhaustion that:

                If $-b \leq a \leq b$, then $|a| \leq b$.
              \end{proof}
          \end{enumerate}
        \setcounter{enumii}{8}
        \item
          \begin{enumerate}
            \setcounter{enumiii}{1}
            \item
              With integers $a, b, c$ we work backward to prove:

              if $a$ divides $b$ and $a$ divides $b + c$, then $a$ divides $3c$.

              If $a$ divides $3c$,
              then there exists some integer $p$ such that $3c = ap \implies 3b + 3c = 3b + ap \implies 3(b + c) = 3b + ap$.

              Now, if $a$ divides $b$, then there exists some integer $q$ such that $b = aq$.
              And also, if $a$ divides $b + c$, then there exists some integer $r$ such that $b + c = ar$

              So $3(b + c) = 3aq + ap \implies 3ar = 3aq + ap \implies 3r = 3q + p \implies 3r - 3q = p \implies 3(r - q) = p$.

              Looks like we can construct a proof if we can assume $p = 3(r - q)$.

              \begin{proof}
                Assume $a$ divides $b$ and $a$ divides $b + c$.

                These mean there exist integers $q, r$ such that $b = aq, b + c = ar$.

                Now construct another integer $p = 3(r - q)$.

                Then,
                \begin{align*}
                  3(r - q) &= p \\
                  3a(r - q) &= ap \\
                  3(ar - aq) &= ap \\
                  3(b + c - b) &= ap \\
                  3c &= ap \\
                \end{align*}

                So, we have that $a$ divides $3c$.

                Thus, if $a$ divides $b$ and $a$ divides $b + c$, then $a$ divides $3c$.
              \end{proof}
            \item
              With integers $a, b, c$ and the real number $x$,
              we work backward to prove:

              if $ab > 0$ and $bc < 0$, then $ax^2 + bx + c = 0$ has two real solutions.

              One way to prove this is to remember the quadratic formula:

              $x = \frac{-b \pm \sqrt{b^2 - 4ac}}{2a}$.

              This has exactly two real solutions when $b^2 - 4ac$ is positive.

              so we need $b^2 - 4ac > 0$.

              We can see:
              \begin{align*}
                b^2 - 4ac &> 0 \\
                b^2 &> 4ac \\
                \frac{1}{4}b^2 &> ac \\
                \frac{1}{4}b^2b^2 &> acb^2 \\
                \frac{1}{4}b^2b^2 &> ab^2c \\
                \frac{1}{4}b^2b^2 &> (ab)(bc) \\
                \frac{1}{4}b^2b^2 \geq 0 &> (ab)(bc) \\
              \end{align*}

              Since $\frac{1}{4}, b^2, b^2$ are all non-negative,
              we have the last inequality if $ab > 0$ and $bc < 0$.

              Looks like we have some way to proceed.

              \begin{proof}
                Assume $ab > 0$ and $bc < 0$.

                Then we know none of $a, b, c$ is 0.

                We have

                \begin{align*}
                  (ab)(bc) &< 0 \\
                  (ab)(bc) &< 0 \leq \frac{1}{4}b^2b^2 && \text{ since } b^2 \geq 0 \\
                  (ab)(bc) &< \frac{1}{4}b^2b^2 \\
                  ab^2c &< \frac{1}{4}b^2b^2 \\
                  ac &< \frac{1}{4}b^2 \\
                  4ac &< b^2 \\
                  0 &< b^2 - 4ac \\
                \end{align*}

                Since $b^2 - 4ac > 0$ we know that
                $x = \frac{-b \pm \sqrt{b^2 - 4ac}}{2a}$
                has exactly two real solutions.

                These solutions are:
                \[
                  x_1 = \frac{-b + \sqrt{b^2 - 4ac}}{2a}
                  ,
                  x_2 = \frac{-b - \sqrt{b^2 - 4ac}}{2a}
                \]

                Thus,
                if $ab > 0$ and $bc < 0$, then $ax^2 + bx + c = 0$ has two real solutions.
              \end{proof}
            \item
              With the real number $x$ we work backward to prove:

              if $x^3 + 2x^2 < 0$, then $2x + 5 < 11$.

              We find $2x + 5 < 11 \implies 2x < 6 \implies x < 3$.

              If we work a bit forward from the antecedent we see
              $x^3 + 2x^2 < 0 \implies x + 2 < 0 \implies x < -2$.

              Now, we should have enough to construct a proof.

              \begin{proof}
                Assume $x^3 + 2x^2 < 0$.

                Then we have
                $x^3 + 2x^2 < 0 \implies x + 2 < 0 \implies x < -2$.

                Now, if $x$ is less than $-2$, then $x$ is also less than 3.

                So we have $x < 3 \implies 2x < 6 \implies 2x + 5 < 11$.

                Thus, if $x^3 + 2x^2 < 0$, then $2x + 5 < 11$.
              \end{proof}
          \end{enumerate}
        \setcounter{enumii}{10}
        \item
          \begin{enumerate}
            \setcounter{enumiii}{1}
            \item
              The claim is solid.

              The proof has the correct idea, however, it is incorrect.

              When constructing the factors of $c$,
              a new integer should be chosen as otherwise,
              $b$ and $c$ are the same integer.

              While this is also a true claim,
              it does not prove what the original claim suggests.

              One way to fix the proof is as follows.

              \begin{proof}
                Suppose $a$ divides $b$ and $a$ divides $c$.
                Then for some integer $q, b = a q$,
                and for some integer $\mathbf{r}, c = ar$.

                Then $b + c = aq + ar = a(q + r)$.
                Since $q + r$ is also an integer, we rename $q + r = s$.
                So $b + c = as$, and $a$ divides $b + c$.
              \end{proof}

              So on the scale of \textbf{A}, \textbf{C}, \textbf{F}, this proof gets a grade of \textbf{C}.

            \setcounter{enumiii}{4}
            \item
              The claim is solid.

              The proof is also correct.

              So on the scale of \textbf{A}, \textbf{C}, \textbf{F}, this proof gets a grade of \textbf{A}.
          \end{enumerate}
      \end{enumerate}
    \item
      \begin{enumerate}
        \setcounter{enumii}{2}
        \item
          \begin{enumerate}
            \setcounter{enumiii}{2}
            \item
              We want to show by contraposition:

              if $x^2$ is not divisible by 4, then $x$ is odd.

              The contrapositive of this statement is:

              If $x$ is even, then $x^2$ is divisible by 4.

              \begin{proof}
                Assume $x$ is even.

                Then there exists some integer $p$ such that $x = 2p$.

                So $x^2 = (2p)^2 = 4p^2$.

                Since $p$ is an integer, $p^2$ is also an integer.
                So we can replace it with $p^2 = q$.

                Then $x^2 = 4q$, meaning that 4 divides $x^2$,
                or equivalently $x^2$ is divisible by 4.

                Therefore, $x^2$ is divisible by 4.

                Thus, if $x$ is even, then $x^2$ is divisible by 4.

                Therefore, if $x^2$ is not divisible by 4, then $x$ is odd.
              \end{proof}
            \item

              We want to show by contraposition:

              if $xy$ is even, then either $x$ or $y$ is even.

              The contrapositive of this statement is:

              if $x$ and $y$ are both odd, then $xy$ is odd.

              \begin{proof}
                Assume both $x$ and $y$ odd.

                Then there exists some integers $p, q$ such that $x = 2p + 1, y = 2q + 1$.

                So $xy = (2p + 1)(2q + 1) = 4pq + 2p + 2q + 1 = 2(2pq + p + q) + 1$.

                Since $2pq + p + q$ is an integer, we can replace it with $2pq + p + q = r$.

                Then $xy = 2r + 1$.

                Therefore, $xy$ is odd.

                Thus, if $x$ and $y$ are both odd, then $xy$ is odd.

                Therefore, if $xy$ is even, then either $x$ or $y$ is even.
              \end{proof}
          \end{enumerate}
        \setcounter{enumii}{4}
        \item
          Given a circle with center $(2, 4)$.
          \begin{enumerate}
            \item
              We want to prove that $(-1, 5), (5, 1)$ are not both on the circle.

              We prove by contradiction.

              \begin{proof}
                Assume $(-1, 5), (5, 1)$ are both on the circle.

                Then the distance from the center to each point must be the same.

                The distance from the center to $(-1, 5)$ is:

                \[
                  \sqrt{\left(2 - (-1)\right)^2 + \left(4 - 5\right)^2}
                  =
                  \sqrt{3^2 + \left(-1\right)^2}
                  =
                  \sqrt{9 + 1}
                  =
                  \sqrt{10}.
                \]

                The distance from the center to $(5, 1)$ is:

                \[
                  \sqrt{\left(2 - 5\right)^2 + \left(4 - 1\right)^2}
                  =
                  \sqrt{\left(-3\right)^2 + 3^2}
                  =
                  \sqrt{9 + 9}
                  =
                  \sqrt{18}.
                \]

                But $\sqrt{10} \neq \sqrt{18}$.

                Thus, the distance from the center to $(-1, 5)$ and $(5, 1)$ is not the same.

                So we have a contradiction.

                Then our assumption was incorrect.

                Thus, $(-1, 5), (5, 1)$ are not both on the circle.
              \end{proof}
            \item
              We want to prove:

              if the radius is less than 5,
              then the circle does not intersect the line $y = x - 6$.

              We prove by contradiction.

              \begin{proof}
                Assume the statement is false.

                That is,
                the radius is less than 5 and the circle intersects the line $y = x - 6$.

                If the radius is less than 5,
                then the distance from the center $(2, 4)$ to any point $(x, y)$ must be less than 5.

                That is:

                \[
                  \sqrt{(x - 2)^2 + (y - 4)^2} < 5
                \]

                Since we also have $y = x - 6$, we can substitute and simplify

                \begin{align*}
                  \sqrt{(x - 2)^2 + (y - 4)^2} &< 5 \\
                  \sqrt{(x - 2)^2 + ((x - 6) - 4)^2} &< 5 \\
                  \sqrt{(x - 2)^2 + (x - 10)^2} &< 5 \\
                  (x - 2)^2 + (x - 10)^2 &< 25 \\
                  (x^2 - 4x + 4) + (x^2 - 20x + 100) &< 25 \\
                  2x^2 - 24x + 104 &< 25 \\
                  2x^2 - 24x &< -79 \\
                  x^2 - 12x &< \frac{-79}{2} \\
                  x^2 - 12x + 36 &< \frac{-79}{2} + 36 \\
                  x^2 - 12x + 36 &< \frac{-79}{2} + \frac{72}{2} \\
                  x^2 - 12x + 36 &< \frac{-7}{2} \\
                  (x - 6)(x - 6) &< \frac{-7}{2} \\
                  (x - 6)^2 &< \frac{-7}{2} \\
                \end{align*}

                So we have $(x - 6)^2 < \frac{-7}{2} < 0$

                But we know that for any real number $x$, $x^2 \geq 0$.

                Since $x - 6$ is a real number, we have a contradiction,
                namely $(x - 6)^2 < 0$ and $(x - 6)^2 \geq 0$.

                Then our assumption must have been incorrect,
                and the original statement must in fact be true.

                Thus,
                if the radius is less than 5,
                then the circle does not intersect the line $y = x - 6$.
              \end{proof}
            \item
              We want to prove:

              if $(0, 3)$ is not inside the circle,
              then $(3, 1)$ is not inside the circle.

              \begin{proof}
                Assume $(0, 3)$ is not inside the circle.

                This means that the radius $r$, of the circle with center $(2, 4)$ is less than the distance to $(0, 3)$.

                We can create an equation for the circle by computing the radius as the distance from the center to any point $(x, y)$ on the circle:

                \[
                  \sqrt{(x - 2)^2 + (y - 4)^2} = r
                \]

                So we know that the distance from the center to $(0, 3)$ is greater than $r$,
                otherwise, $(0, 3)$ would be inside the circle.

                That is:

                \begin{align*}
                  \sqrt{(0 - 2)^2 + (3 - 4)^2} &> r \\
                  \sqrt{(- 2)^2 + (-1)^2} &> r \\
                  \sqrt{4 + 1} &> r \\
                  \sqrt{5} &> r \\
                \end{align*}

                Now, we can compute the distance from the center of the circle to the point $(3, 1)$.

                \begin{align*}
                  \sqrt{(3 - 2)^2 + (1 - 4)^2} &= \sqrt{1^2 + (-3)^2} \\
                  &= \sqrt{1 + 9} \\
                  &= \sqrt{10} \\
                \end{align*}

                Now, since $\sqrt{10} > \sqrt{5} > r$ we know that the point $(3, 1)$ is not inside the circle either.

                Thus
                if $(0, 3)$ is not inside the circle,
                then $(3, 1)$ is not inside the circle.
              \end{proof}
          \end{enumerate}
        \item
          \begin{enumerate}
            \item
              With positive integers $a, b$,
              we want to prove by contradiction:

              if $a$ divides $b$, then $a \leq b$.

              \begin{proof}
                Assume the statement
                \textit{if $a$ divides $b$, then $a \leq b$} is false.

                So we assume $a$ divides $b$ and $a > b$.

                Since we have $a$ divides $b$,
                then there exists some integer $p$ such that $b = ap$.

                Since both $a$ and $b$ are positive integers,
                $p$ must also be a positive integer,
                otherwise $b = ap$ would be less than or equal to 0.

                Therefore, $p > 0$.

                Since we assumed $a > b$,
                we have $b = ap > bp \implies b > bp \implies 1 > p \implies 0 \geq p$.

                Therefore, $p \leq 0$,

                Hence, $p > 0$ and $p \leq 0$ is a contradiction.

                Since we assumed a statement and used all true statements to arrive at this contradiction, our assumption was false.

                That is, the statement
                \textit{if $a$ divides $b$, then $a \leq b$} is true.

                Thus,
                if $a$ divides $b$, then $a \leq b$.
              \end{proof}
            \item
              With positive integers $a, b$,
              we want to prove by contradiction:

              if $ab$ is odd, then both $a$ and $b$ are odd.

              \begin{proof}
                Assume the statement
                \textit{if $ab$ is odd, then both $a$ and $b$ are odd} is false.

                So we assume $ab$ is odd and one of $a, b$ is odd while the other even.

                We use the result of the proof from $\S 1.4.5$ (i) above,
                namely,
                with integers $x, y$ if $x$ is even and $y$ is odd, then $xy$ is even.

                If we have $a$ odd and $b$ even, then $ab$ is even.
                If we have $b$ odd and $a$ even, then $ba = ab$ is even.

                In either case, we have a contradiction since we assumed $ab$ is odd.

                Since we assumed one statement, and used all true statements to arrive at this contradiction, our assumption was false.

                That is, the statement
                \textit{if $ab$ is odd, then both $a$ and $b$ are odd} is true.

                Thus,
                if $ab$ is odd, then both $a$ and $b$ are odd.
              \end{proof}
          \end{enumerate}
        \item
          \begin{enumerate}
            \setcounter{enumiii}{2}
            \item
              With positive integer $a$, we want to prove:
              $a$ is odd if and only if $a + 1$ is even.

              \begin{proof}

                Assume $a$ is odd.

                \begin{tabularx}{0.95\linewidth}{l@{}c@{}X}
                  $a$ is odd & $\iff$ & there exists some integer $p$ such that $a = 2p + 1$.               \\
                             & $\iff$ & $a + 1 = (2p + 1) + 1 = 2p + 2 = 2(p + 1)$                          \\
                             & $\iff$ & $a + 1 = 2(p + 1) = 2q$ where $p + 1 = q$ for integers $p, q$. \\
                             & $\iff$ & there exists some integer $q$ such that $a + 1 = 2q$.               \\
                             & $\iff$ & $a + 1$ is even.                                                    \\
                \end{tabularx}

                Since we started with $a$ is odd, and arrived at $a + 1$ is even
                through only biconditional statements we have proved:

                $a$ is odd if and only if $a + 1$ is even.
              \end{proof}
          \end{enumerate}
        \setcounter{enumii}{10}
        \item
          With real numbers $x, y, z$ such that $0 < x < y < z < 1$,

          we want to prove:

          at least two of the numbers are within $\frac{1}{2}$ of each other.

          The statement is saying that one of the following are true:

          Either $y - x < \frac{1}{2}$ or $z - y < \frac{1}{2}$.

          We prove by contradiction.
          \begin{proof}
            Suppose not.

            Assume it is not the case that

            at least two of the numbers are within $\frac{1}{2}$ of each other.

            That is,
            both $y - x \geq \frac{1}{2}$ and $z - y \geq \frac{1}{2}$.

            Then

            \begin{align*}
              (z - y) + (y - x) &\geq \frac{1}{2} + \frac{1}{2} = 1 \\
              z - y + y - x &\geq 1 \\
              z - x &\geq 1 \\
            \end{align*}

            But we also have
            $z < 1 \implies z < 1 < 1 + x \implies z < 1 + x \implies z - x < 1$.

            Since we have both $z - x \geq 1$ and $z - x < 1$, we have a contradiction.

            Then our assumption was false.

            Thus,
            at least two of the numbers are within $\frac{1}{2}$ of each other.
          \end{proof}
      \end{enumerate}
  \end{enumerate}
\end{document}
