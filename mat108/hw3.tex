\documentclass[12pt,letterpaper]{article}

\usepackage[margin=1in]{geometry}
\usepackage[round-mode=figures,round-precision=3,scientific-notation=false]{siunitx}
\usepackage[super]{nth}
\usepackage[title]{appendix}
\usepackage{amsfonts}
\usepackage{amsmath}
\usepackage{amsthm}
\usepackage{cancel}
\usepackage{caption}
\usepackage{color, colortbl}
\usepackage{dcolumn}
\usepackage{enumitem}
\usepackage{float}
\usepackage{fp}
\usepackage{mathtools}
\usepackage{pgfplots}
\usepackage{subcaption}
\usepackage{tikz}
\usepackage{titling}

\usepgfplotslibrary{statistics}

\definecolor{Gray}{gray}{0.8}

\pgfplotsset{compat=1.8}

\newcolumntype{d}{D{.}{.}{-1}}
\newcolumntype{g}{>{\columncolor{Gray}}c}

\newcommand*\biconditional[3]{
  This statement is #1 since both sides of the bi-conditional have #2 truth values.

  #3
}
\newcommand*\biconditionaltrue[2]{
  \biconditional{true}{the same}{#1 and #2.}
}
\newcommand*\biconditionalfalse[2]{
  \biconditional{false}{different}{#1 yet #2.}
}

\newcommand*\directproofsimple[6]{
  With integers $x$ and $y$, we want to show:

  If #1, then #2 is even.

  \begin{proof}
    Suppose #1.

    Then there exist some integers $p, q$ such that #3 and #4.

    Then we have #5$ = 2($#6$)$.

    Since #6 is an integer, we can rename as $r = $ #6.

    So #2 $ = 2r$, and is even.

    Thus if #1, then #2 is even
  \end{proof}
}

\renewcommand{\labelenumi}{\S 1.\arabic*}
\renewcommand{\labelenumii}{\arabic*}
\renewcommand{\labelenumiii}{(\alph*)}

\setlength{\droptitle}{-10ex}

\preauthor{\begin{flushright}\large \lineskip 0.5em}
\postauthor{\par\end{flushright}}
\predate{\begin{flushright}\large}
\postdate{\par\end{flushright}}

\title{MAT 108 HW 2\vspace{-2ex}}
\author{Hardy Jones\\
        999397426\\
        Professor Bandyopadhyay\vspace{-2ex}}
\date{Spring 2015}

\begin{document}
  \maketitle

  \begin{enumerate}
    \setcounter{enumi}{3}
    \item
      \begin{enumerate}
        \setcounter{enumii}{4}
        \item
          \begin{enumerate}
            \setcounter{enumiii}{4}
            \item
              \directproofsimple
                {$x$ and $y$ are odd}
                {$x + y$}
                {$x = 2p + 1$}
                {$y = 2q + 1$}
                {$x + y = (2p + 1) + (2q + 1) = 2p + 2q + 2$}
                {$p + q + 1$}
            \item
              \directproofsimple
                {$x$ and $y$ are odd}
                {$3x - 5y$}
                {$x = 2p + 1$}
                {$y = 2q + 1$}
                {$3x - 5y = 3(2p + 1) - 5(2q + 1) = 6p + 3 - 10q - 5 = 6p - 10q - 2$}
                {$3p - 5q - 1$}
            \setcounter{enumiii}{8}
            \item
              \directproofsimple
                {$x$ is even and $y$ is odd}
                {$xy$}
                {$x = 2p$}
                {$y = 2q + 1$}
                {$xy = (2p)(2q + 1) = 4pq + 2p$}
                {$2pq + p$}
          \end{enumerate}
        \item
          \begin{enumerate}
            \setcounter{enumiii}{3}
            \item
              With real numbers $a, b$ we want to prove $|a + b| \leq |a| + |b|$.
              \begin{proof}
                We prove this by cases.

                For cases 2 and 3,
                we choose without loss of generality $a \geq 0, b < 0$.
                The exact same argument holds for $a < 0, b \geq 0$.
                \begin{enumerate}[label=Case \arabic*]
                  \item $a \geq 0, b \geq 0$

                    Since $a \geq 0, b \geq 0, a + b \geq 0$.

                    So $|a + b| = a + b$.

                    Also, $|a| = a$ and $|b| = b$.

                    So $|a + b| = a + b = |a| + |b|$.

                    Thus, $|a + b| \leq |a| + |b|$.
                  \item $a \geq 0, b < 0, a + b \geq 0$

                    Since $a + b \geq 0, |a + b| = a + b$.

                    Also, $|a| = a$ and $|b| = -b$.

                    Since $b < 0 \implies 2b < 0 \implies b < -b \implies a + b < a + (-b)$,

                    we have $|a + b| = a + b < a + (-b) = |a| + |b|$.

                    Thus, $|a + b| \leq |a| + |b|$.
                  \item $a \geq 0, b < 0, a + b < 0$

                    Since $a + b < 0, |a + b| = -(a + b) = -a - b$.

                    Also, $|a| = a$ and $|b| = -b$.

                    Since $0 \leq a \implies 0 \leq 2a \implies -a \leq a \implies -a - b \leq a - b$,

                    we have $|a + b| = -a - b \leq a - b = a + (-b) = |a| + |b|$.

                    Thus, $|a + b| \leq |a| + |b|$.
                  \item $a < 0, b < 0$

                    Since $a < 0, b < 0, a + b < 0$.

                    So $|a + b| = - (a + b) = -a - b$.

                    Also, $|a| = -a$ and $|b| = -b$.

                    So $|a + b| = -a - b = -a + (-b) = |a| + |b|$.

                    Thus, $|a + b| \leq |a| + |b|$.
                \end{enumerate}

                Since these are all the possible cases,
                we have proven by exhaustion that

                $|a + b| \leq |a| + |b|$.
              \end{proof}
            \item
              With real numbers $a, b$ we want to prove if $|a| \leq b$, then $-b \leq a \leq b$.

              \begin{proof}
                We prove this by cases.

                \begin{enumerate}[label=Case \arabic*]
                  \item $a \geq 0$

                    Assume $|a| \leq b$.

                    Since $a \geq 0$, we have $|a| = a$.

                    Now, we know $|a| = a \leq b$.

                    Also $0 \leq |a| = a \leq b \implies 0 \leq b$.
                    And since both $a$ and $b$ are non-negative,
                    $a + b$ is also non-negative.

                    So we have $0 \leq a + b \implies -b \leq a$.

                    Then we have both $-b \leq a$ and $a \leq b$ or $-b \leq a \leq b$.

                    Thus if $|a| \leq b$, then $-b \leq a \leq b$.

                  \item $a < 0$

                    Assume $|a| \leq b$.

                    Since $a < 0$, we have $|a| = -a$.

                    Then, $|a| = -a \leq b \implies -a - b \leq 0 \implies -b \leq a$.

                    Now, by the definition of absolute value, $|a| \leq b \implies 0 \leq b$.

                    And $a < 0 \implies a < 0 \leq b \implies a < b \implies a \leq b$.

                    Then we have both $-b \leq a$ and $a \leq b$ or $-b \leq a \leq b$.

                    Thus if $|a| \leq b$, then $-b \leq a \leq b$.
                \end{enumerate}

                Since these are the only possible cases,
                we have proved by exhaustion that:

                If $|a| \leq b$, then $-b \leq a \leq b$.
              \end{proof}
            \item
              With real numbers $a, b$ we want to prove if $-b \leq a \leq b$, then $|a| \leq b$.

              \begin{proof}
                We prove this by cases.

                \begin{enumerate}[label=Case \arabic*]
                  \item $a \geq 0$

                    Assume $-b \leq a \leq b$.

                    Since $a \geq 0$, we have $|a| = a$.

                    We have $-b \leq a \leq b \implies a \leq b$.

                    Then, $a = |a| \leq b$.

                    Thus, if $-b \leq a \leq b$, then $|a| \leq b$.
                  \item $a < 0$

                    Assume $-b \leq a \leq b$.

                    Since $a < 0$, we have $|a| = -a$.

                    We have $-b \leq a \leq b \implies -b \leq a \implies -a - b \leq 0 \implies -a \leq b$.

                    Then $-a = |a| \leq b$.

                    Thus, if $-b \leq a \leq b$, then $|a| \leq b$.
                \end{enumerate}

                Since these are the only possible cases,
                we have proved by exhaustion that:

                If $-b \leq a \leq b$, then $|a| \leq b$.
              \end{proof}
          \end{enumerate}
        \setcounter{enumii}{8}
        \item
          \begin{enumerate}
            \setcounter{enumiii}{1}
            \item
              With integers $a, b, c$ we work backward to prove:

              if $a$ divides $b$ and $a$ divides $b + c$, then $a$ divides $3c$.

              If $a$ divides $3c$,
              then there exists some integer $p$ such that $3c = ap \implies 3b + 3c = 3b + ap \implies 3(b + c) = 3b + ap$.

              Now, if $a$ divides $b$, then there exists some integer $q$ such that $b = aq$.
              And also, if $a$ divides $b + c$, then there exists some integer $r$ such that $b + c = ar$

              So $3(b + c) = 3aq + ap \implies 3ar = 3aq + ap \implies 3r = 3q + p \implies 3r - 3q = p \implies 3(r - q) = p$.

              Looks like we can construct a proof if we can assume $p = 3(r - q)$.

              \begin{proof}
                Assume $a$ divides $b$ and $a$ divides $b + c$.

                These mean there exist integers $q, r$ such that $b = aq, b + c = ar$.

                Now construct another integer $p = 3(r - q)$.

                Then,
                \begin{align*}
                  3(r - q) &= p \\
                  3a(r - q) &= ap \\
                  3(ar - aq) &= ap \\
                  3(b + c - b) &= ap \\
                  3c &= ap \\
                \end{align*}

                So, we have that $a$ divides $3c$.

                Thus, if $a$ divides $b$ and $a$ divides $b + c$, then $a$ divides $3c$.
              \end{proof}
            \item
            \item
              With the real number $x$ we work backward to prove:

              if $x^3 + 2x^2 < 0$, then $2x + 5 < 11$.

              We find $2x + 5 < 11 \implies 2x < 6 \implies x < 3$.

              If we work a bit forward from the antecedent we see
              $x^3 + 2x^2 < 0 \implies x + 2 < 0 \implies x < -2$.

              Now, we should have enough to construct a proof.

              \begin{proof}
                Assume $x^3 + 2x^2 < 0$.

                Then we have
                $x^3 + 2x^2 < 0 \implies x + 2 < 0 \implies x < -2$.

                Now, if $x$ is less than $-2$, then $x$ is also less than 3.

                So we have $x < 3 \implies 2x < 6 \implies 2x + 5 < 11$.

                Thus, if $x^3 + 2x^2 < 0$, then $2x + 5 < 11$.
              \end{proof}
          \end{enumerate}
        \setcounter{enumii}{10}
        \item
          \begin{enumerate}
            \setcounter{enumiii}{1}
            \item
              The claim is solid.

              The proof has the correct idea, however, it is incorrect.

              When constructing the factors of $c$,
              a new integer should be chosen as otherwise,
              $b$ and $c$ are the same integer.

              While this is also a true claim,
              it does not prove what the original claim suggests.

              One way to fix the proof is as follows.

              \begin{proof}
                Suppose $a$ divides $b$ and $a$ divides $c$.
                Then for some integer $q, b = a q$,
                and for some integer $\mathbf{r}, c = ar$.

                Then $b + c = aq + ar = a(q + r)$.
                Since $q + r$ is also an integer, we rename $q + r = s$.
                So $b + c = as$, and $a$ divides $b + c$.
              \end{proof}

              So on the scale of \textbf{A}, \textbf{C}, \textbf{F}, this proof gets a grade of \textbf{C}.

            \setcounter{enumiii}{4}
            \item
              The claim is solid.

              The proof is also correct.

              So on the scale of \textbf{A}, \textbf{C}, \textbf{F}, this proof gets a grade of \textbf{A}.
          \end{enumerate}
      \end{enumerate}
    \item
      \begin{enumerate}
        \setcounter{enumii}{2}
        \item
          \begin{enumerate}
            \setcounter{enumiii}{2}
            \item
              We want to show by contraposition:

              if $x^2$ is not divisible by 4, then $x$ is odd.

              The contrapositive of this statement is:

              If $x$ is even, then $x^2$ is divisible by 4.

              \begin{proof}
                Assume $x$ is even.

                Then there exists some integer $p$ such that $x = 2p$.

                So $x^2 = (2p)^2 = 4p^2$.

                Since $p$ is an integer, $p^2$ is also an integer.
                So we can replace it with $p^2 = q$.

                Then $x^2 = 4q$, meaning that 4 divides $x^2$,
                or equivalently $x^2$ is divisible by 4.

                Therefore, $x^2$ is divisible by 4.

                Thus, if $x$ is even, then $x^2$ is divisible by 4.

                Therefore, if $x^2$ is not divisible by 4, then $x$ is odd.
              \end{proof}
            \item

              We want to show by contraposition:

              if $xy$ is even, then either $x$ or $y$ is even.

              The contrapositive of this statement is:

              if $x$ and $y$ are both odd, then $xy$ is odd.

              \begin{proof}
                Assume both $x$ and $y$ odd.

                Then there exists some integers $p, q$ such that $x = 2p + 1, y = 2q + 1$.

                So $xy = (2p + 1)(2q + 1) = 4pq + 2p + 2q + 1 = 2(2pq + p + q) + 1$.

                Since $2pq + p + q$ is an integer, we can replace it with $2pq + p + q = r$.

                Then $xy = 2r + 1$.

                Therefore, $xy$ is odd.

                Thus, if $x$ and $y$ are both odd, then $xy$ is odd.

                Therefore, if $xy$ is even, then either $x$ or $y$ is even.
              \end{proof}
          \end{enumerate}
        \setcounter{enumii}{4}
        \item
        \item
          \begin{enumerate}
            \item
            \item
          \end{enumerate}
        \item
          \begin{enumerate}
            \setcounter{enumiii}{2}
            \item
          \end{enumerate}
        \setcounter{enumii}{10}
        \item
      \end{enumerate}
  \end{enumerate}
\end{document}
