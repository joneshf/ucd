\documentclass[12pt,letterpaper]{article}

\usepackage[margin=1in]{geometry}
\usepackage[round-mode=figures,round-precision=3,scientific-notation=false]{siunitx}
\usepackage[super]{nth}
\usepackage[title]{appendix}
\usepackage{amsfonts}
\usepackage{amsmath}
\usepackage{amssymb}
\usepackage{amsthm}
\usepackage{calculator}
\usepackage{cancel}
\usepackage{caption}
\usepackage{color, colortbl}
\usepackage{dcolumn}
\usepackage{enumitem}
\usepackage{float}
\usepackage{fp}
\usepackage{ifthen}
\usepackage{mathtools}
\usepackage{pgfplots}
\usepackage{subcaption}
\usepackage{tabularx}
\usepackage{tikz}
\usepackage{titling}

\usepgfplotslibrary{statistics}

\definecolor{Gray}{gray}{0.8}

\pgfplotsset{compat=1.8}

\newcolumntype{d}{D{.}{.}{-1}}
\newcolumntype{g}{>{\columncolor{Gray}}c}

\DeclarePairedDelimiter\ceil{\lceil}{\rceil}
\DeclarePairedDelimiter\floor{\lfloor}{\rfloor}

\newcommand*\biconditional[3]{
  This statement is #1 since both sides of the bi-conditional have #2 truth values.

  #3
}
\newcommand*\biconditionaltrue[2]{
  \biconditional{true}{the same}{#1 and #2.}
}
\newcommand*\biconditionalfalse[2]{
  \biconditional{false}{different}{#1 yet #2.}
}

\newcommand*\directproofsimple[6]{
  With integers $x$ and $y$, we want to show:

  If #1, then #2 is even.

  \begin{proof}
    Suppose #1.

    Then there exist some integers $p, q$ such that #3 and #4.

    Then we have #5$ = 2($#6$)$.

    Since #6 is an integer, we can rename as $r = $ #6.

    So #2 $ = 2r$, and is even.

    Thus if #1, then #2 is even
  \end{proof}
}

\newcommand*\crossproduct[2]{%
  \{%
    \foreach \x [count=\i] in {#1} {%
      \foreach \y [count=\j] in {#2} {%
        \ifthenelse{\equal{\i}{\j} \AND \equal{\i}{1}}{}{, }%
        (\x, \y)%
      }%
    }%
  \}%
}

\newcommand*\st[2]{%
  \MULTIPLY{4}{#1}{\fours}%
  \MULTIPLY{5}{#2}{\fivet}%
  \ADD{\fours}{\fivet}{\foursplusfivet}%
  \[4\left(#1\right) + 5\left(#2\right) = \fours + \fivet = \foursplusfivet\]%
}

\newcommand*\PCI{%
  Principle of Complete Induction%
}

\newcommand*\reflexive[4]{%
  Choose any $#3 \in \mathbb{#1}$.

  #4

  So $#2$ is reflexive on $#3$

  Since the choice of $#3$ was arbitrary,
  this result holds for all $#3 \in \mathbb{#1}$.

  Thus, $#2$ is reflexive on $\mathbb{#1}$
}

\newcommand*\congruentboth[3]{%
  \MULTIPLY{#2}{#3}{\congruentbothkl}
  \MULTIPLY{-1}{\congruentbothkl}{\congruentbothnegkl}
  \ADD{\congruentbothkl}{#1}{\congruentbothpos}
  \ADD{\congruentbothnegkl}{#1}{\congruentbothneg}

  We want to find an $a$ such that $a - #1 = #2k$ and $a - #1 = #3l$.

  We can solve this algebraically by substitution.

  \begin{align*}
    a - #1 &= #2k \\
    \left(#3l + #1\right) - #1 &= #2k && \text{since } a = #3l + #1 \\
    #3l &= #2k \\
    l &= \frac{#2}{#3}k \\
  \end{align*}

  Since $k, l \in \mathbb{Z}$ we know that $k$ must be a multiple of #3.

  Choose $k = #3$.
  then we have $l = \frac{#2}{#3}#3 = #2$,
  and we end up with
  \begin{align*}
    a - #1 &= #3l \\
    &= #3(#2) \\
    &= \congruentbothkl \\
    a &= \congruentbothpos \\
  \end{align*}
  Choose $k = -#3$.
  then we have $l = \frac{#2}{#3}\left(-#3\right) = -#2$,
  and we end up with
  \begin{align*}
    a - #1 &= #3l \\
    &= #3(-#2) \\
    &= \congruentbothnegkl \\
    a &= \congruentbothneg \\
  \end{align*}

  So a positive integer that is congruent to #1 (mod #2) and congruent to #1 (mod #3) is $\congruentbothpos$.

  And a negative integer that is congruent to #1 (mod #2) and congruent to #1 (mod #3) is $\congruentbothneg$.
}

\renewcommand{\labelenumi}{\S 3.\arabic*}
\renewcommand{\labelenumii}{\arabic*}
\renewcommand{\labelenumiii}{(\alph*)}

\setlength{\droptitle}{-10ex}

\preauthor{\begin{flushright}\large \lineskip 0.5em}
\postauthor{\par\end{flushright}}
\predate{\begin{flushright}\large}
\postdate{\par\end{flushright}}

\title{MAT 108 HW 7\vspace{-2ex}}
\author{Hardy Jones\\
        999397426\\
        Professor Bandyopadhyay\vspace{-2ex}}
\date{Spring 2015}

\begin{document}
  \maketitle

  \begin{enumerate}
    \setcounter{enumi}{1}
    \item
      \begin{enumerate}
        \setcounter{enumii}{4}
        \item
          \begin{enumerate}
            \setcounter{enumiii}{1}
            \item
              \begin{proof}
                We need to show reflexivity, symmetry, and transitivity.

                \begin{itemize}
                  \item
                    \reflexive{N}{R}{m}{
                      Then $m$ has a digit $p \in \{0, 1, \dots, 9\}$ in its tens place.
                      Since $m$ has the same digit $p$ in its tens place as $m$,
                      we have $m R m$.
                    }
                  \item
                    Choose any $m, n \in \mathbb{N}$ with $m R n$.

                    Then $m$ has the same digit in the tens place as $n$.
                    Then $n$ has the same digit in the tens place as $m$.
                    Then $n R m$.
                    So $R$ is symmetric on $m, n$.

                    Since the choice of $m, n$ was arbitrary,
                    this result holds for all $m, n \in \mathbb{N}$.

                    Thus, $R$ is symmetric on $\mathbb{N}$
                  \item
                    Choose any $m, n, p \in \mathbb{N}$ with $m R n$ and $n R p$.

                    Then $m$ has the same digit in the tens place as $n$.
                    Call this digit $q$.
                    Then $n$ has the same digit in the tens place as $p$.
                    Call this digit $r$.

                    Since $q$ and $r$ are the tens digit of $n$, we have $q = r$
                    So we know that the tens place of $p$ is $q$.

                    This means that $m$ has the same digit in the tens place as $p$.
                    So $m R p$.
                    And $R$ is transitive on $m, n, p$.

                    Since the choice of $m, n, p$ was arbitrary,
                    this result holds for all $m, n, p \in \mathbb{N}$.

                    Thus, $R$ is transitive on $\mathbb{N}$
                \end{itemize}
              \end{proof}

              An element of $106 / R$ less than $50$ is $3$,
              since the tens place of 106 is 0, the tens place of 3 is 0, and $3 < 50$.

              An element of $106 / R$ between $150$ and $300$ is $203$,
              since the tens place of 106 is 0, the tens place of 203 is 0, and $150 < 203 < 300$.

              An element of $106 / R$ greater than $1000$ is $2003$,
              since the tens place of 106 is 0, the tens place of 2003 is 0, and $1000 < 2003$.

              Three elements of $635 / R$ are $30, 31$, and $32$,
              since the tens place of 635 is 3, the tens place of 30 is 3, the tens place of 31 is 3, and the tens place of 32 is 3.
            \item
              \begin{proof}
                We need to show reflexivity, symmetry, and transitivity.

                \begin{itemize}
                  \item
                    Choose any $m \in \mathbb{R}$.

                    Then

                    So $V$ is reflexive on $m$

                    Since the choice of $m$ was arbitrary,
                    this result holds for all $m \in \mathbb{R}$.

                    Thus, $V$ is reflexive on $\mathbb{R}$
                  \item
                    Choose any $m, n \in \mathbb{R}$ with $m V n$.

                    Then $m$ has the same digit in the tens place as $n$.
                    Then $n$ has the same digit in the tens place as $m$.
                    Then $n V m$.
                    So $V$ is symmetric on $m, n$.

                    Since the choice of $m, n$ was arbitrary,
                    this result holds for all $m, n \in \mathbb{R}$.

                    Thus, $V$ is symmetric on $\mathbb{R}$
                  \item
                    Choose any $m, n, p \in \mathbb{R}$ with $m V n$ and $n V p$.

                    Then $m$ has the same digit in the tens place as $n$.
                    Call this digit $q$.
                    Then $n$ has the same digit in the tens place as $p$.
                    Call this digit $r$.

                    Since $q$ and $r$ are the tens digit of $n$, we have $q = r$
                    So we know that the tens place of $p$ is $q$.

                    This means that $m$ has the same digit in the tens place as $p$.
                    So $m V p$.
                    And $V$ is transitive on $m, n, p$.

                    Since the choice of $m, n, p$ was arbitrary,
                    this result holds for all $m, n, p \in \mathbb{R}$.

                    Thus, $V$ is transitive on $\mathbb{R}$
                \end{itemize}
              \end{proof}

              An element of $106 / R$ less than $50$ is $3$,
              since the tens place of 106 is 0, the tens place of 3 is 0, and $3 < 50$.

              An element of $106 / R$ between $150$ and $300$ is $203$,
              since the tens place of 106 is 0, the tens place of 203 is 0, and $150 < 203 < 300$.

              An element of $106 / R$ greater than $1000$ is $2003$,
              since the tens place of 106 is 0, the tens place of 2003 is 0, and $1000 < 2003$.

              Three elements of $635 / R$ are $30, 31$, and $32$,
              since the tens place of 635 is 3, the tens place of 30 is 3, the tens place of 31 is 3, and the tens place of 32 is 3.
            \item
            \item
            \setcounter{enumiii}{7}
            \item
          \end{enumerate}
        \item
        \setcounter{enumii}{7}
        \item
          \begin{enumerate}
            \setcounter{enumiii}{3}
            \item
              The equivalence classes for the relation congruence modulo 7 are:
              \begin{alignat*}{6}
                \overline{0} &= \{\dots, -14, & -7, & 0, & 7,  & 14, \dots\} \\
                \overline{1} &= \{\dots, -13, & -6, & 1, & 8,  & 15, \dots\} \\
                \overline{2} &= \{\dots, -12, & -5, & 2, & 9,  & 16, \dots\} \\
                \overline{3} &= \{\dots, -11, & -4, & 3, & 10, & 17, \dots\} \\
                \overline{4} &= \{\dots, -10, & -3, & 4, & 11, & 18, \dots\} \\
                \overline{5} &= \{\dots, -9,  & -2, & 5, & 12, & 19, \dots\} \\
                \overline{6} &= \{\dots, -8,  & -1, & 6, & 13, & 20, \dots\} \\
              \end{alignat*}
          \end{enumerate}
        \item
          We use the equivalent definition that $a \equiv b (mod m)$
          is the same as $a - b = k m$ for some $k \in \mathbb{Z}$.
          \begin{enumerate}
            \setcounter{enumiii}{3}
            \item
              \congruentboth{3}{4}{5}
            \item
              \congruentboth{1}{3}{7}
          \end{enumerate}
        \setcounter{enumii}{10}
        \item
          \begin{proof}
            We can show that $S$ is not an equivalence relation by showing that $S$ is not transitive.

            Choose $x = 1, y = 2, z = 4$.
            Then $x, y, z, w \in \mathbb{N}$ and $1 + 2 = 3 = 3(1), 2 + 4 = 6 = 3(2)$,
            so $x S y$ and $y S z$.

            Now, if $S$ were transitive, we'd have $x S z$.
            However, $1 + 4 = 5$ and 3 does not divide 5.

            So $S$ is not transitive.

            Thus, $S$ is not an equivalence relation.
          \end{proof}
      \end{enumerate}
    \item
      \begin{enumerate}
        \setcounter{enumii}{2}
        \item
          \begin{enumerate}
            \item
              The relation $R$ creates equivalence classes where each $x$ in each class has the same fractional part.

              The following are examples of these equivalence classes:
              \[
                \mathbb{Z}, \{\dots, -1.5, 0.5, 1.5, \dots\}, \{\dots, e - 3, e - 2, e - 1, e, e + 1, \dots\}
              \]

              Then we have the partition is the family of sets of the form:
              \[
                A_r = \{x \in \mathbb{R} : x - \lfloor x \rfloor = r\}, \text{ where } r \in [0, 1)
              \]
            \setcounter{enumiii}{3}
            \item
              For any $x, y \in \mathbb{R}$, $x$ and $y$ are in the same equivalence class iff $x = \pm y$.

              The following are examples of these equivalence classes:

              \[
                \{0\}, \{-1, 1\}, \left\{-\frac{3}{7}, \frac{3}{7}\right\}, \{-e, e\}
              \]

              Then we have the partition as the family of sets of the form:
              \[
                A_r = \{r, -r\}, \text{ where } r \in \mathbb{R}
              \]
          \end{enumerate}
        \item
        \setcounter{enumii}{5}
        \item
          \begin{enumerate}
            \item
            \setcounter{enumiii}{4}
            \item
          \end{enumerate}
      \end{enumerate}
    \item [\S 4.1]
      \begin{enumerate}
        \item
          \begin{enumerate}
            \setcounter{enumiii}{4}
            \item
            \item
          \end{enumerate}
        \setcounter{enumii}{2}
        \item
          \begin{enumerate}
            \setcounter{enumiii}{1}
            \item
            \setcounter{enumiii}{4}
            \item
            \item
          \end{enumerate}
        \item
          \begin{enumerate}
            \setcounter{enumiii}{4}
            \item
            \item
          \end{enumerate}
        \item
          \begin{enumerate}
            \setcounter{enumiii}{1}
            \item
            \item
          \end{enumerate}
        \setcounter{enumii}{7}
        \item
          \begin{enumerate}
            \setcounter{enumiii}{1}
            \item
            \item
          \end{enumerate}
        \item
          \begin{enumerate}
            \setcounter{enumiii}{1}
            \item
            \setcounter{enumiii}{3}
            \item
          \end{enumerate}
        \item
          \begin{enumerate}
            \setcounter{enumiii}{1}
            \item
            \setcounter{enumiii}{3}
            \item
          \end{enumerate}
        \item
          \begin{enumerate}
            \setcounter{enumiii}{2}
            \item
            \item
          \end{enumerate}
        \setcounter{enumii}{16}
        \item
          \begin{enumerate}
            \item
            \item
          \end{enumerate}
      \end{enumerate}
  \end{enumerate}
\end{document}
