\documentclass[12pt,letterpaper]{article}

\usepackage[margin=1in]{geometry}
\usepackage[round-mode=figures,round-precision=3,scientific-notation=false]{siunitx}
\usepackage[super]{nth}
\usepackage[title]{appendix}
\usepackage{amsfonts}
\usepackage{amsmath}
\usepackage{amssymb}
\usepackage{amsthm}
\usepackage{calculator}
\usepackage{cancel}
\usepackage{caption}
\usepackage{color, colortbl}
\usepackage{dcolumn}
\usepackage{enumitem}
\usepackage[mathscr]{euscript}
\usepackage{float}
\usepackage{fp}
\usepackage{ifthen}
\usepackage{mathtools}
\usepackage{pgfplots}
\usepackage{subcaption}
\usepackage{tabularx}
\usepackage{tikz}
\usepackage{titling}

\usepgfplotslibrary{statistics}

\definecolor{Gray}{gray}{0.8}

\pgfplotsset{compat=1.8}

\newcolumntype{d}{D{.}{.}{-1}}
\newcolumntype{g}{>{\columncolor{Gray}}c}

\DeclarePairedDelimiter\ceil{\lceil}{\rceil}
\DeclarePairedDelimiter\floor{\lfloor}{\rfloor}

\newcommand*\biconditional[3]{
  This statement is #1 since both sides of the bi-conditional have #2 truth values.

  #3
}
\newcommand*\biconditionaltrue[2]{
  \biconditional{true}{the same}{#1 and #2.}
}
\newcommand*\biconditionalfalse[2]{
  \biconditional{false}{different}{#1 yet #2.}
}

\newcommand*\directproofsimple[6]{
  With integers $x$ and $y$, we want to show:

  If #1, then #2 is even.

  \begin{proof}
    Suppose #1.

    Then there exist some integers $p, q$ such that #3 and #4.

    Then we have #5$ = 2($#6$)$.

    Since #6 is an integer, we can rename as $r = $ #6.

    So #2 $ = 2r$, and is even.

    Thus if #1, then #2 is even
  \end{proof}
}

\newcommand*\crossproduct[2]{%
  \{%
    \foreach \x [count=\i] in {#1} {%
      \foreach \y [count=\j] in {#2} {%
        \ifthenelse{\equal{\i}{\j} \AND \equal{\i}{1}}{}{, }%
        (\x, \y)%
      }%
    }%
  \}%
}

\newcommand*\st[2]{%
  \MULTIPLY{4}{#1}{\fours}%
  \MULTIPLY{5}{#2}{\fivet}%
  \ADD{\fours}{\fivet}{\foursplusfivet}%
  \[4\left(#1\right) + 5\left(#2\right) = \fours + \fivet = \foursplusfivet\]%
}

\newcommand*\PCI{%
  Principle of Complete Induction%
}

\newcommand*\reflexive[4]{%
  Choose any $#3 \in #1$.

  #4

  So $#2$ is reflexive on $#3$

  Since the choice of $#3$ was arbitrary,
  this result holds for all $#3 \in #1$.

  Thus, $#2$ is reflexive on $#1$
}

\newcommand*\symmetric[5]{%
  Choose any $#3, #4 \in #1$ with $#3 #1 #4$.

  #5

  So $#1$ is symmetric on $#3, #4$.

  Since the choice of $#3, #4$ was arbitrary,
  this result holds for all $#3, #4 \in #1$.

  Thus, $#1$ is symmetric on $#1$
}

\newcommand*\transitive[6]{%
  Choose any $#3, #4, #5 \in #1$ with $#3 #2 #4$ and $#4 #2 #5$.

  #6

  So $#2$ is transitive on $#3, #4, #5$.

  Since the choice of $#3, #4, #5$ was arbitrary, \\
  this result holds for all $#3, #4, #5 \in #1$.

  Thus, $#2$ is transitive on $#1$
}

\newcommand*\equivrel[8]{%
  \begin{itemize}
    \item \reflexive{#1}{#2}{#3}{#6}
    \item \symmetric{#1}{#2}{#3}{#4}{#7}
    \item \transitive{#1}{#2}{#3}{#4}{#5}{#8}
  \end{itemize}
}

\newcommand*\congruentboth[3]{%
  \MULTIPLY{#2}{#3}{\congruentbothkl}
  \MULTIPLY{-1}{\congruentbothkl}{\congruentbothnegkl}
  \ADD{\congruentbothkl}{#1}{\congruentbothpos}
  \ADD{\congruentbothnegkl}{#1}{\congruentbothneg}

  We want to find an $a$ such that $a - #1 = #2k$ and $a - #1 = #3l$.

  We can solve this algebraically by substitution.

  \begin{align*}
    a - #1 &= #2k \\
    \left(#3l + #1\right) - #1 &= #2k && \text{since } a = #3l + #1 \\
    #3l &= #2k \\
    l &= \frac{#2}{#3}k \\
  \end{align*}

  Since $k, l \in \mathbb{Z}$ we know that $k$ must be a multiple of #3.

  Choose $k = #3$.
  then we have $l = \frac{#2}{#3}#3 = #2$,
  and we end up with
  \begin{align*}
    a - #1 &= #3l \\
    &= #3(#2) \\
    &= \congruentbothkl \\
    a &= \congruentbothpos \\
  \end{align*}
  Choose $k = -#3$.
  then we have $l = \frac{#2}{#3}\left(-#3\right) = -#2$,
  and we end up with
  \begin{align*}
    a - #1 &= #3l \\
    &= #3(-#2) \\
    &= \congruentbothnegkl \\
    a &= \congruentbothneg \\
  \end{align*}

  So a positive integer that is congruent to #1 (mod #2) and congruent to #1 (mod #3) is $\congruentbothpos$.

  And a negative integer that is congruent to #1 (mod #2) and congruent to #1 (mod #3) is $\congruentbothneg$.
}

\newcommand*\threexy[4]{%
  $x = #1, y = #2$

  $3x + y = 3(#1) + #2 = #3 + #2 = #4$
}

\newcommand*\threexyprime[2]{%
  \MULTIPLY{3}{#1}{\threexythreex}
  \ADD{\threexythreex}{#2}{\threexythreexy}

  \threexy{#1}{#2}{\threexythreex}{\threexythreexy}

  $\threexythreexy$ is prime, so $(#1, #2) \in R$.
}

\newcommand*\threexynotprime[2]{%
  \MULTIPLY{3}{#1}{\threexythreex}
  \ADD{\threexythreex}{#2}{\threexythreexy}

  \threexy{#1}{#2}{\threexythreex}{\threexythreexy}

  $\threexythreexy$ is not prime, so $(#1, #2) \notin R$.
}

\renewcommand{\labelenumi}{\S 3.\arabic*}
\renewcommand{\labelenumii}{\arabic*}
\renewcommand{\labelenumiii}{(\alph*)}

\setlength{\droptitle}{-10ex}

\preauthor{\begin{flushright}\large \lineskip 0.5em}
\postauthor{\par\end{flushright}}
\predate{\begin{flushright}\large}
\postdate{\par\end{flushright}}

\title{MAT 108 HW 7\vspace{-2ex}}
\author{Hardy Jones\\
        999397426\\
        Professor Bandyopadhyay\vspace{-2ex}}
\date{Spring 2015}

\begin{document}
  \maketitle

  \begin{enumerate}
    \setcounter{enumi}{1}
    \item
      \begin{enumerate}
        \setcounter{enumii}{4}
        \item
          \begin{enumerate}
            \setcounter{enumiii}{1}
            \item
              \begin{proof}
                We need to show reflexivity, symmetry, and transitivity.

                \equivrel{\mathbb{N}}{R}{m}{n}{p}{
                  Then $m$ has a digit $p \in \{0, 1, \dots, 9\}$ in its tens place.
                  Since $m$ has the same digit $p$ in its tens place as $m$,
                  we have $m R m$.
                }{
                  Then $m$ has the same digit in the tens place as $n$.
                  Then $n$ has the same digit in the tens place as $m$.
                  Then $n R m$.
                }{
                  Then $m$ has the same digit in the tens place as $n$.
                  Call this digit $q$.
                  Then $n$ has the same digit in the tens place as $p$.
                  Call this digit $r$.\\
                  Since $q$ and $r$ are the tens digit of $n$, we have $q = r$
                  So we know that the tens place of $p$ is $q$. \\
                  This means that $m$ has the same digit in the tens place as $p$.
                  So $m R p$.
                }
              \end{proof}

              An element of $106 / R$ less than $50$ is $3$,
              since the tens place of 106 is 0, the tens place of 3 is 0, and $3 < 50$.

              An element of $106 / R$ between $150$ and $300$ is $203$,
              since the tens place of 106 is 0, the tens place of 203 is 0, and $150 < 203 < 300$.

              An element of $106 / R$ greater than $1000$ is $2003$,
              since the tens place of 106 is 0, the tens place of 2003 is 0, and $1000 < 2003$.

              Three elements of $635 / R$ are $30, 31$, and $32$,
              since the tens place of 635 is 3, the tens place of 30 is 3, the tens place of 31 is 3, and the tens place of 32 is 3.
            \item
              \begin{proof}
                We need to show reflexivity, symmetry, and transitivity.

                \equivrel{\mathbb{R}}{V}{x}{y}{z}{
                  Then $x = x$, so $x V x$.
                }{
                  Then either $x = y$ or $x y = 1$. \\
                  This also means $y = x$ or $y x = 1$.
                  So we have $y V x$.
                }{
                  Then either $x = y$ or $xy = 1$ and either $y = z$ or $yz = 1$.
                  \begin{itemize}
                    \item If $x = y$ and $y = z$, then $x = z$ so $x V z$.
                    \item If $x = y$ and $yz = 1$, then $xz = yz = 1$ so $x V z$.
                    \item If $xy = 1$ and $y = z$, then $xz = xy = 1$ so $x V z$.
                    \item If $xy = 1$ and $yz = 1$, then $\frac{1}{z} = y = \frac{1}{x} \implies \frac{1}{z} = \frac{1}{x} \implies x = z$ so $x V z$.
                  \end{itemize}
                }
              \end{proof}

              The equivalence class of $3 = \{3, \frac{1}{3}\}$.

              The equivalence class of $-\frac{2}{3} = \{-\frac{2}{3}, -\frac{3}{2}\}$.

              The equivalence class of $0 = \{0\}$.
            \item
              \begin{proof}
                \equivrel{\mathbb{N}}{R}{a}{b}{c}{
                  Then the prime factorization of $a$ has exactly as many 2s as the prime factorization of $a$. \\
                  So $a R a$.
                }{
                  Then the prime factorization of $a$ has exactly as many 2s as the prime factorization of $b$. \\
                  This means the prime factorization of $b$ has exactly as many 2s as the prime factorization of $a$. \\
                  So $b R a$.
                }{
                  Then the prime factorization of $a$ has exactly as many 2s as the prime factorization of $b$.
                  Call this number of 2s $p$. \\
                  And the prime factorization of $b$ has exactly as many 2s as the prime factorization of $c$.
                  Call this number of 2s $q$. \\
                  Then $p = q$.
                  This means the prime factorization of $a$ has exactly as many 2s as the prime factorization of $c$. \\
                  So $a R c$.
                }
              \end{proof}

              Three elements in $1 / R$ are $2 = 2^{1}, 6 = 2^{1}(3), 10 = 2^{1}(5)$.

              Three elements in $4 / R$ are $16 = 2^{4}, 48 = 2^{4}(3), 80 = 2^{4}(5)$.

              Three elements in $72 / R$ are

              $4722366482869645213696 = 2^{72}$,

              $14167099448608935641088 = 2^{72}(3)$,

              $23611832414348226068480 = 2^{72}(5)$.
            \item
              \begin{proof}
                \equivrel{\mathbb{R} \times \mathbb{R}}{T}{(a,b)}{(c,d)}{(e,f)}{
                  Then $a^2 + b^2 = a^2 + b^2$. So $(a, b) T (a, b)$.
                }{
                  Then $a^2 + b^2 = c^2 + d^2 \implies c^2 + d^2 = a^2 + b^2$.
                  So $(c, d) T (a, b)$.
                }{
                  Then $a^2 + b^2 = c^2 + d^2$ and $c^2 + d^2 = e^2 + f^2$. \\
                  But this means $a^2 + b^2 = c^2 + d^2 = e^2 + f^2$. \\
                  So $a^2 + b^2 = e^2 + f^2$. \\
                  And $(a, b) T (e, f)$.
                }
              \end{proof}

              \begin{figure}
                \centering
                \begin{minipage}{.5\textwidth}
                  \centering
                  \begin{tikzpicture}
                    \draw [->] (-3,0) -- (3,0) node [above left]  {$x$};
                    \draw [->] (0,-3) -- (0,3) node [below right] {$y$};
                    \draw (0,0) circle[radius=2] ;
                    \draw (2,-3pt) -- (2,3pt)   node [above] {$\sqrt{5}$};
                    \draw (-2,-3pt) -- (-2,3pt) node [above] {$-\sqrt{5}$};
                    \draw (-3pt,2) -- (3pt,2)   node [right] {$\sqrt{5}$};
                    \draw (-3pt,-2) -- (3pt,-2) node [right] {$-\sqrt{5}$};
                  \end{tikzpicture}
                  \captionof{figure}{$(1, 2) / \left(\mathbb{R} \times \mathbb{R}\right)$}
                  \label{fig:rra}
                \end{minipage}%
                \begin{minipage}{.5\textwidth}
                  \centering
                  \begin{tikzpicture}
                    \draw [->] (-3,0) -- (3,0) node [above left]  {$x$};
                    \draw [->] (0,-3) -- (0,3) node [below right] {$y$};
                    \draw (0,0) circle[radius=2] ;
                    \draw (2,-3pt) -- (2,3pt)   node [above] {$4$};
                    \draw (-2,-3pt) -- (-2,3pt) node [above] {$-4$};
                    \draw (-3pt,2) -- (3pt,2)   node [right] {$4$};
                    \draw (-3pt,-2) -- (3pt,-2) node [right] {$-4$};
                  \end{tikzpicture}
                  \captionof{figure}{$(4, 0) / \left(\mathbb{R} \times \mathbb{R}\right)$}
                  \label{fig:rrb}
                \end{minipage}
              \end{figure}

              See Figures \ref{fig:rra}, \ref{fig:rrb}.

            \setcounter{enumiii}{7}
            \item
              \begin{proof}
                Call the set of all differentiable functions $D$.
                \equivrel{D}{R}{f}{g}{h}{
                  Then $f$ has the same first derivative as $f$. \\
                  So $f R f$.
                }{
                  Then $f$ has the same first derivative as $g$.
                  This means that $g$ has the same first derivative as $f$. \\
                  So $g R f$.
                }{
                  Then $f$ has the same first derivative as $g$.
                  Call this derivative $p$.
                  Then $g$ has the same first derivative as $h$.
                  Call this derivative $q$. \\
                  Then $f' = g' = p = q = g' = h'$. \\
                  So $f' = h'$ and $f R h$.
                }
              \end{proof}

              Three elements in $x^2 / R$ are: $x^2, x^2 + 1, x^2 + 2$.

              Three elements in $(4x^2 + 10x) / R$ are: $4x^2 + 10x, 4x^2 + 10x + 1, 4x^2 + 10x + 2$.

              $x^3 / R$ is the set of all functions whose first derivative is $3x^2$.

              $7 / R$ is the set of all functions whose first derivative is $0$.
          \end{enumerate}
        \item
          \begin{proof}
            \equivrel{\mathbb{Q}}{R}{\frac{a}{b}}{\frac{c}{d}}{\frac{e}{f}}{
              Then $ab = ba$ so $\frac{a}{b} R \frac{a}{b}$.
            }{
              Then $ad = bc$.
              But also $bc = ad \implies cb = ad \implies cb = da$. \\
              So $\frac{c}{d} R \frac{a}{b}$.
            }{
              Then $ad = bc$.
              Then $cf = de$. \\
              So $c = \frac{de}{f}$ and $ad = b\frac{de}{f} \implies a = \frac{be}{f} \implies af = be$. \\
              So $\frac{a}{b} R \frac{e}{f}$
            }
          \end{proof}

          The ordered pairs in $\frac{2}{3} / R$ are pairs $(p, q) \in \mathbb{Z} \times \mathbb{Z}$ such that $2q = 3p$.
        \setcounter{enumii}{7}
        \item
          \begin{enumerate}
            \setcounter{enumiii}{3}
            \item
              The equivalence classes for the relation congruence modulo 7 are:
              \begin{alignat*}{6}
                \overline{0} &= \{\dots, -14, & -7, & 0, & 7,  & 14, \dots\} \\
                \overline{1} &= \{\dots, -13, & -6, & 1, & 8,  & 15, \dots\} \\
                \overline{2} &= \{\dots, -12, & -5, & 2, & 9,  & 16, \dots\} \\
                \overline{3} &= \{\dots, -11, & -4, & 3, & 10, & 17, \dots\} \\
                \overline{4} &= \{\dots, -10, & -3, & 4, & 11, & 18, \dots\} \\
                \overline{5} &= \{\dots, -9,  & -2, & 5, & 12, & 19, \dots\} \\
                \overline{6} &= \{\dots, -8,  & -1, & 6, & 13, & 20, \dots\} \\
              \end{alignat*}
          \end{enumerate}
        \item
          We use the equivalent definition that $a \equiv b (mod m)$
          is the same as $a - b = k m$ for some $k \in \mathbb{Z}$.
          \begin{enumerate}
            \setcounter{enumiii}{3}
            \item
              \congruentboth{3}{4}{5}
            \item
              \congruentboth{1}{3}{7}
          \end{enumerate}
        \setcounter{enumii}{10}
        \item
          \begin{proof}
            We can show that $S$ is not an equivalence relation by showing that $S$ is not transitive.

            Choose $x = 1, y = 2, z = 4$.
            Then $x, y, z, w \in \mathbb{N}$ and $1 + 2 = 3 = 3(1), 2 + 4 = 6 = 3(2)$,
            so $x S y$ and $y S z$.

            Now, if $S$ were transitive, we'd have $x S z$.
            However, $1 + 4 = 5$ and 3 does not divide 5.

            So $S$ is not transitive.

            Thus, $S$ is not an equivalence relation.
          \end{proof}
      \end{enumerate}
    \item
      \begin{enumerate}
        \setcounter{enumii}{2}
        \item
          \begin{enumerate}
            \item
              The relation $R$ creates equivalence classes where each $x$ in each class has the same fractional part.

              The following are examples of these equivalence classes:
              \[
                \mathbb{Z}, \{\dots, -1.5, 0.5, 1.5, \dots\}, \{\dots, e - 3, e - 2, e - 1, e, e + 1, \dots\}
              \]

              Then we have the partition is the family of sets of the form:
              \[
                A_r = \{x \in \mathbb{R} : x - \lfloor x \rfloor = r\}, \text{ where } r \in [0, 1)
              \]
            \setcounter{enumiii}{3}
            \item
              For any $x, y \in \mathbb{R}$, $x$ and $y$ are in the same equivalence class iff $x = \pm y$.

              The following are examples of these equivalence classes:

              \[
                \{0\}, \{-1, 1\}, \left\{-\frac{3}{7}, \frac{3}{7}\right\}, \{-e, e\}
              \]

              Then we have the partition as the family of sets of the form:
              \[
                A_r = \{r, -r\}, \text{ where } r \in \mathbb{R}
              \]
          \end{enumerate}
        \item
          Let's first enumerate all of the relations.

          \[
            \{(i, i), (i, -i), (-i, i), (-i, -i), (1, 1), (1, -1), (-1, 1), (-1, -1)\}
          \]

          From this, we can see the equivalence classes:

          $i / C = -i / C = \{i, -i\}$,

          $1 / C = -1 / C = \{1, -1\}$

          So the partition $\mathscr{P} = \{\{i, -i\}, \{1, -1\}\}$
        \setcounter{enumii}{5}
        \item
          \begin{enumerate}
            \item
              For $m, n \in \mathbb{N}$, $m R n$ iff $10^p < m \leq n < 10^{p + 1}$ for $p \in \{0, 1, 2, \dots\}$.
            \setcounter{enumiii}{4}
            \item
              For $m, n \in \mathbb{Z}$, $m R n$ iff either both $m, n < 3$ or both $m, n \geq 3$.
          \end{enumerate}
      \end{enumerate}
    \item [\S 4.1]
      \begin{enumerate}
        \item
          \begin{enumerate}
            \setcounter{enumiii}{4}
            \item
              If we look at some of the elements of the relation we see:

              \[\{(1, 1), (1, 2), (1, 3), \dots\}\]

              Since 1 is mapped to many values, this relation is not a function.
            \item
              If we look at some of the elements of the relation we see:

              \[\{(0, 0), (1, 1), (4, 2), (4, -2), (9, 3), (9, -3), \dots\}\]

              Since 4 is mapped to many values, this relation is not a function.
          \end{enumerate}
        \setcounter{enumii}{2}
        \item
          \begin{enumerate}
            \setcounter{enumiii}{1}
            \item
              The domain of the mapping is $\mathbb{R}$.

              The range of the mapping is $\{x \in \mathbb{R} : x \geq 5\}$.

              Another possible codomain is $\mathbb{R}$.
            \setcounter{enumiii}{4}
            \item
              Let's expand the definition of $\mathcal{X}_\mathbb{N}(x)$.

              \[
                \mathcal{X}_\mathbb{N}(x)
                =
                \begin{cases}
                  1 & \text{ if } x \in \mathbb{N} \\
                  0 & \text{ if } x \notin \mathbb{N} \\
                \end{cases}
              \]

              So we see,
              the domain of the mapping is $\mathbb{N}$.

              The range of the mapping is $\{0, 1\}$.

              Another possible codomain is $\mathbb{R}$.
            \item
              This mapping is the $\cosh(x)$.

              The domain of the mapping is $\mathbb{R}$.

              The range of the mapping is $[0, \infty)$.

              Another possible codomain is $\mathbb{R}$.
          \end{enumerate}
        \item
          \begin{enumerate}
            \begin{figure}
              \centering
              \begin{minipage}{.5\textwidth}
                \begin{tikzpicture}
                  \begin{axis}[xlabel=$x$, ylabel=$y$, xmax=6, ymax=6, xmin=0, ymin=0, grid=major]
                    \addplot[domain=3:5]{sqrt(5 - x) + sqrt(x - 3)}
                    ;
                  \end{axis}
                \end{tikzpicture}
                \captionof{figure}{$f(x) = \sqrt{5 - x} + \sqrt{x - 3}$}
                \label{fig:a}
              \end{minipage}%
            \end{figure}

            \setcounter{enumiii}{4}
            \item
              $f(x)$ is only defined when $x \leq 5$ and $x \geq 3$.

              So the domain of $f$ is $[3, 5]$.

              Looking at Figure \ref{fig:a},
              the range of $f$ is $[\sqrt{2}, 2]$
            \item
              $f(x)$ is only defined when $x \geq -2$ and $x \leq -2$.

              So the domain of $f$ is $\{-2\}$.

              The range of $f$ is $\{0\}$
          \end{enumerate}
        \item
          \begin{enumerate}
            \setcounter{enumiii}{1}
            \item
              \begin{proof}
                We enumerate all possibilities:

                \begin{itemize}
                  \item \threexynotprime{1}{1}
                  \item \threexyprime{1}{2}
                  \item \threexynotprime{1}{3}
                  \item \threexyprime{2}{1}
                  \item \threexynotprime{2}{2}
                  \item \threexynotprime{2}{3}
                  \item \threexynotprime{3}{1}
                  \item \threexyprime{3}{2}
                  \item \threexynotprime{3}{3}
                \end{itemize}

                So we have:
                \[
                  R = \{(1, 2), (2, 1), (3, 2)\}
                \]

                Now, dom($R$) = $A$.

                If we choose $(1, y), (1, z) \in R$, then $y = z = 2$.
                If we choose $(2, y), (2, z) \in R$, then $y = z = 1$.
                If we choose $(3, y), (3, z) \in R$, then $y = z = 2$.

                From this we have shown that $R$ is a function with domain $A$.
              \end{proof}
            \item
              \begin{proof}
                Choose any $x \in \mathbb{Z}$.

                Then $x \cdot x - 2 = - y \implies x^2 + y = 2$,
                so for any $x \in \mathbb{Z}$, we have $(x, 2 - x^2) \in R$.
                Then the domain of $R$ is $\mathbb{Z}$.

                Now, choose $(x, y), (x, z) \in R$.

                We have $x^2 + y = 2 \implies y = 2 - x^2 = z$,
                so $y = z$.

                Thus we have shown that $R$ is a function on $\mathbb{Z}$.
              \end{proof}
          \end{enumerate}
        \setcounter{enumii}{7}
        \item
          \begin{enumerate}
            \setcounter{enumiii}{1}
            \item $\{x \in U : \mathcal{X}_A(x) = 0\} = A^C$
            \item $\{x \in U : \mathcal{X}_A(x) = 2\} = \varnothing$,
              since Rng($\mathcal{X}_A(x)$) = $\{0, 1\}$
          \end{enumerate}
        \item
          \begin{enumerate}
            \setcounter{enumiii}{1}
            \item $x_n = \frac{(-1)^n}{n}$
            \setcounter{enumiii}{3}
            \item $x_n = n (\text{mod } 3)$
          \end{enumerate}
        \item
          \begin{enumerate}
            \setcounter{enumiii}{1}
            \item
              $f(6) = \overline{6} = \overline{0} = \{\dots, -12, -6, 0, 6, 12, \dots\}$
            \setcounter{enumiii}{3}
            \item
              All pre-images of $\overline{1} = \{\dots, -11, -5, 1, 5, 11, \dots\}$
          \end{enumerate}
        \item
          \begin{enumerate}
            \setcounter{enumiii}{2}
            \item $f$ is a function.
            \item
              We have $f : \mathbb{Z}_4 \to \mathbb{Z}_6$ with $f\left(\overline{x}\right) = [2x + 1]$.

              The equivalence classes of $\mathbb{Z}_4$ are $\overline{0}, \overline{1}, \overline{2}, \overline{3}$.

              The equivalence classes of $\mathbb{Z}_6$ are $[0], [1], [2], [3], [4], [5]$.

              So $f\left(\overline{0}\right) = [1], f\left(\overline{1}\right) = [3], f\left(\overline{2}\right) = [5], f\left(\overline{3}\right) = [7] = [1]$

              Then we have $f\left(\overline{4}\right) = f\left(\overline{0}\right) = [1]$,
              but $f\left(\overline{4}\right) = [9] = [3]$, and $[1] \neq [3]$.

              So $f$ is not a function because it is not well defined.
          \end{enumerate}
        \setcounter{enumii}{16}
        \item
          \begin{enumerate}
            \item
              In order for the relation from $A$ to $B$ to be a function we need a few things.

              We need the relation to have the domain $A$.

              So we need that the relation must have exactly $m$ relations.
              We also need that the relation must have each $a \in A$ in the relation exactly once.
              I.e. $\{(a_1, b_1), (a_1, b_2), \dots\}$ is not a function.

              We can enumerate all possibilities:

              \begin{align*}
                \{
                  &\{ (a_1, b_1), (a_2, b_1), \dots, (a_m, b_1) \}, \\
                  &\{ (a_1, b_2), (a_2, b_1), \dots, (a_m, b_1) \}, \\
                  & \quad \quad \quad \quad \quad \quad \vdots \\
                  &\{ (a_1, b_n), (a_2, b_1), \dots, (a_m, b_1) \}, \\
                  &\{ (a_1, b_1), (a_2, b_2), \dots, (a_m, b_1) \}, \\
                  &\{ (a_1, b_2), (a_2, b_2), \dots, (a_m, b_1) \}, \\
                  & \quad \quad \quad \quad \quad \quad \vdots \\
                  &\{ (a_1, b_1), (a_2, b_n), \dots, (a_m, b_n) \}, \\
                  &\{ (a_1, b_2), (a_2, b_n), \dots, (a_m, b_n) \}, \\
                  & \quad \quad \quad \quad \quad \quad \vdots \\
                  &\{ (a_1, b_n), (a_2, b_n), \dots, (a_m, b_n) \}, \\
                \}
              \end{align*}

              After careful counting, we see that there are exactly $mn$ relations that are functions.
            \item
              For each $a_i \in A$, there are exactly $n$ mappings to some $b_j \in B$. Since there are $m$ possible $a_i$, there are exactly $mn$ relations that are functions.
          \end{enumerate}
      \end{enumerate}
  \end{enumerate}
\end{document}
