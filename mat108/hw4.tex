\documentclass[12pt,letterpaper]{article}

\usepackage[margin=1in]{geometry}
\usepackage[round-mode=figures,round-precision=3,scientific-notation=false]{siunitx}
\usepackage[super]{nth}
\usepackage[title]{appendix}
\usepackage{amsfonts}
\usepackage{amsmath}
\usepackage{amssymb}
\usepackage{amsthm}
\usepackage{cancel}
\usepackage{caption}
\usepackage{color, colortbl}
\usepackage{dcolumn}
\usepackage{enumitem}
\usepackage{float}
\usepackage{fp}
\usepackage{mathtools}
\usepackage{pgfplots}
\usepackage{subcaption}
\usepackage{tabularx}
\usepackage{tikz}
\usepackage{titling}

\usepgfplotslibrary{statistics}

\definecolor{Gray}{gray}{0.8}

\pgfplotsset{compat=1.8}

\newcolumntype{d}{D{.}{.}{-1}}
\newcolumntype{g}{>{\columncolor{Gray}}c}

\newcommand*\biconditional[3]{
  This statement is #1 since both sides of the bi-conditional have #2 truth values.

  #3
}
\newcommand*\biconditionaltrue[2]{
  \biconditional{true}{the same}{#1 and #2.}
}
\newcommand*\biconditionalfalse[2]{
  \biconditional{false}{different}{#1 yet #2.}
}

\newcommand*\directproofsimple[6]{
  With integers $x$ and $y$, we want to show:

  If #1, then #2 is even.

  \begin{proof}
    Suppose #1.

    Then there exist some integers $p, q$ such that #3 and #4.

    Then we have #5$ = 2($#6$)$.

    Since #6 is an integer, we can rename as $r = $ #6.

    So #2 $ = 2r$, and is even.

    Thus if #1, then #2 is even
  \end{proof}
}

\renewcommand{\labelenumi}{\S 1.\arabic*}
\renewcommand{\labelenumii}{\arabic*}
\renewcommand{\labelenumiii}{(\alph*)}

\setlength{\droptitle}{-10ex}

\preauthor{\begin{flushright}\large \lineskip 0.5em}
\postauthor{\par\end{flushright}}
\predate{\begin{flushright}\large}
\postdate{\par\end{flushright}}

\title{MAT 108 HW 4\vspace{-2ex}}
\author{Hardy Jones\\
        999397426\\
        Professor Bandyopadhyay\vspace{-2ex}}
\date{Spring 2015}

\begin{document}
  \maketitle

  \begin{enumerate}
    \setcounter{enumi}{5}
    \item
      \begin{enumerate}
        \setcounter{enumii}{5}
        \item
          \begin{enumerate}
            \setcounter{enumiii}{8}
            \item
              We want to show that there exists some $K \in \mathbb{N}$,
              such that for all $r \in \mathbb{R}$ if $r > K$, then $\frac{1}{r^2} < 0.01$.

              It helps to rewrite the consequent a bit first:

              \[
                \frac{1}{r^2} < 0.01 \implies \frac{1}{r^2} < \frac{1}{100} \implies 100 < r^2
              \]

              Since we're only concerned with $r$ greater than some natural number,
              we only need to concern ourself with $r > 0$.

              So we can simplify the consequent a bit more:

              \[
                100 < r^2 \implies 10 < r
              \]

              Now it's fairly obvious that we need $K \geq 10$.

              \begin{proof}
                Choose $K = 10$.

                $K$ is a natural number.

                Now, for any arbitrary real number $r$ that is greater than $K$,
                we have:

                \[
                  10 < r \implies 100 < r^2 \implies \frac{1}{r^2} < \frac{1}{100} \implies \frac{1}{r^2} < 0.01
                \]

                Thus we have shown,

                there exists some natural number $K$,
                such that for any real number $r$,
                if $r$ is greater than $K$, then $r$ is less than $0.01$.
              \end{proof}
            \item
              We want to show that there exist integers $L, G$,
              such that for any real number $x$,
              if $L < x < G$, then $40 > 10 - 2x > 12$.

              It helps to manipulate the consequent a bit:

              \[
                40 > 10 - 2x > 12 \implies 30 > -2x > 2 \implies -15 < x < -1
              \]

              So if we have $L = -15, G = -1$, then we can work backwards to the proof.

              \begin{proof}
                Choose $L = -15, G = -1$.

                Then $L, G \in \mathbb{Z}$.

                Now, for any real number $x$.

                If $L < x < G$, then we have:

                \[
                  -15 < x < -1 \implies 30 > -2x > 2  \implies 40 > 10 - 2x > 12
                \]

                So we have shown that there exist integers $L, G$,
                such that for any real number $x$,
                if $L < x < G$, then $40 > 10 - 2x > 12$.
              \end{proof}
            \setcounter{enumiii}{13}
            \item

              We want to show for any positive real numbers $x, y$ with $x < y$,
              there exists some natural $M$,
              such that if $n$ is a natural and $n > M$, then $\frac{1}{n} < y - x$.

              Let's manipulate the consequent a bit.

              \[
                \frac{1}{n} < y - x \implies x + \frac{1}{n} < y
              \]

              So if we can show that $x + \frac{1}{n}$ is less than $y$, we can prove this.

              We have that $M < n$, so $\frac{1}{n} < \frac{1}{M}$.

              So, if we choose our $M$ carefully, we can prove this.

              \begin{proof}
                Given any two positive real numbers $x, y$ with $x < y$,
                choose a natural $M$ such that $x + \frac{1}{M} < y$.

                Now, for any natural $n > M$ we have:

                \[
                  M < n \implies \frac{1}{n} < \frac{1}{M}
                \]

                Then we have:

                \[
                  x + \frac{1}{M} < y \implies x + \frac{1}{n} < x + \frac{1}{M} < y \implies x + \frac{1}{n} < y \implies \frac{1}{n} < y - x
                \]

                So we have shown that for any positive real numbers $x, y$ with $x < y$,
                there exists some natural $M$,
                such that if $n$ is a natural and $n > M$, then $\frac{1}{n} < y - x$.
              \end{proof}
          \end{enumerate}
      \end{enumerate}
  \end{enumerate}
  \renewcommand{\labelenumi}{\S 2.\arabic*}
  \begin{enumerate}
    \item
      \begin{enumerate}
        \setcounter{enumii}{13}
        \item
          \begin{enumerate}
            \setcounter{enumiii}{1}
            \item $\mathcal{P}(X) = \{\varnothing, \{S\}, \{\{S\}\}, \{S, \{S\}\}\}$
            \setcounter{enumiii}{3}
            \item
              \begin{align*}
                \mathcal{P}(X) =
                &\{ \varnothing \\
                &, \{1\}, \{\{\varnothing\}\}, \{\{2, \{3\}\}\} \\
                &, \{1, \{\varnothing\}\}, \{1, \{2, \{3\}\}\}, \{\{\varnothing\}, \{2, \{3\}\}\} \\
                &, \{1, \{\varnothing\}, \{2, \{3\}\}\} \\
                &\}
              \end{align*}
          \end{enumerate}
        \item
          \begin{enumerate}
            \setcounter{enumiii}{1}
            \item True.
            \item False.
            \item True.
          \end{enumerate}
        \setcounter{enumii}{16}
        \item
          \begin{enumerate}
            \item True.
            \item True.
            \setcounter{enumiii}{3}
            \item True.
            \setcounter{enumiii}{5}
            \item True.
          \end{enumerate}
        \item

          We want to prove, for any sets $A, B$, $A = B$ if and only if $\mathcal{P}(A) = \mathcal{P}(B)$.

          We need to show both directions

          \begin{proof}
            \begin{itemize}
              \item $(\Longrightarrow)$

              Assume $A = B$.

              Then, $\mathcal{P}(A) = \{X | X \subseteq A\} = \{X | X \subseteq B\} = \mathcal{P}(B)$.

              Thus, if $A = B$, then $\mathcal{P}(A) = \mathcal{P}(B)$.

              \item $(\Longleftarrow)$

              Assume $\mathcal{P}(A) = \mathcal{P}(B)$.

              Then we have $\mathcal{P}(A) \subseteq \mathcal{P}(B)$, and $\mathcal{P}(B) \subseteq \mathcal{P}(A)$

              Now, since $A \subseteq A, A \in \mathcal{P}(A)$.

              And since $\mathcal{P}(A) \subseteq \mathcal{P}(B), A \in \mathcal{P}(B)$.

              This means $A \subseteq B$.

              Additionally,
              since $B \subseteq B, B \in \mathcal{P}(B)$.

              And since $\mathcal{P}(B) \subseteq \mathcal{P}(A), B \in \mathcal{P}(A)$.

              This means $B \subseteq A$.

              So we have $A \subseteq B$ and $B \subseteq A$.

              Then $A = B$.

              Thus, if $\mathcal{P}(A) = \mathcal{P}(B)$, then $A = B$.
            \end{itemize}

            Since we have shown both directions,
            we have shown for any sets $A, B$, $A = B$ if and only if $\mathcal{P}(A) = \mathcal{P}(B)$.
          \end{proof}
      \end{enumerate}
    \item
      \begin{enumerate}
        \setcounter{enumii}{2}
        \item
          \begin{enumerate}
            \setcounter{enumiii}{4}
            \item
              Since $\mathbb{Z}^+ \cap \mathbb{Z}^- = \varnothing$, $\mathbb{Z}^+ - \mathbb{Z}^- = \mathbb{Z}+$
            \item
              Since $E \cap D = \varnothing$, and $E \cup D = \mathbb{Z}$, and $\mathbb{Z}$ is the universe, $E^C = D$.
            \setcounter{enumiii}{7}
            \item
              Since $\left(E \cap \mathbb{Z}^-\right)^C = E^C \cup \left(\mathbb{Z}^-\right)^C = D \cup \mathbb{Z}^+ \cup \{0\}$.

              In words this is all of the integers except the negative evens.

              Or $\{-1, -3, -5, \dots\} \cup \{0, 1, 2, \dots\}$
            \item
              $\varnothing^C = \mathbb{Z}$, which is the universe.
          \end{enumerate}
        \setcounter{enumii}{4}
        \item
          The only disjoint pairs of sets are $C, D$.
      \end{enumerate}
  \end{enumerate}
\end{document}
