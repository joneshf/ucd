\documentclass[12pt,letterpaper]{article}

\usepackage[margin=1in]{geometry}
\usepackage[round-mode=figures,round-precision=3,scientific-notation=false]{siunitx}
\usepackage[super]{nth}
\usepackage[title]{appendix}
\usepackage{amsfonts}
\usepackage{amsmath}
\usepackage{amssymb}
\usepackage{amsthm}
\usepackage{cancel}
\usepackage{caption}
\usepackage{color, colortbl}
\usepackage{dcolumn}
\usepackage{enumitem}
\usepackage{float}
\usepackage{fp}
\usepackage{ifthen}
\usepackage{mathtools}
\usepackage{pgfplots}
\usepackage{subcaption}
\usepackage{tabularx}
\usepackage{tikz}
\usepackage{titling}

\usepgfplotslibrary{statistics}

\definecolor{Gray}{gray}{0.8}

\pgfplotsset{compat=1.8}

\newcolumntype{d}{D{.}{.}{-1}}
\newcolumntype{g}{>{\columncolor{Gray}}c}

\newcommand*\biconditional[3]{
  This statement is #1 since both sides of the bi-conditional have #2 truth values.

  #3
}
\newcommand*\biconditionaltrue[2]{
  \biconditional{true}{the same}{#1 and #2.}
}
\newcommand*\biconditionalfalse[2]{
  \biconditional{false}{different}{#1 yet #2.}
}

\newcommand*\directproofsimple[6]{
  With integers $x$ and $y$, we want to show:

  If #1, then #2 is even.

  \begin{proof}
    Suppose #1.

    Then there exist some integers $p, q$ such that #3 and #4.

    Then we have #5$ = 2($#6$)$.

    Since #6 is an integer, we can rename as $r = $ #6.

    So #2 $ = 2r$, and is even.

    Thus if #1, then #2 is even
  \end{proof}
}

\newcommand*\crossproduct[2]{%
  \{%
    \foreach \x [count=\i] in {#1} {%
      \foreach \y [count=\j] in {#2} {%
        \ifthenelse{\equal{\i}{\j} \AND \equal{\i}{1}}{}{, }%
        (\x, \y)%
      }%
    }%
  \}%
}

\renewcommand{\labelenumi}{\S 2.\arabic*}
\renewcommand{\labelenumii}{\arabic*}
\renewcommand{\labelenumiii}{(\alph*)}

\setlength{\droptitle}{-10ex}

\preauthor{\begin{flushright}\large \lineskip 0.5em}
\postauthor{\par\end{flushright}}
\predate{\begin{flushright}\large}
\postdate{\par\end{flushright}}

\title{MAT 108 HW 5\vspace{-2ex}}
\author{Hardy Jones\\
        999397426\\
        Professor Bandyopadhyay\vspace{-2ex}}
\date{Spring 2015}

\begin{document}
  \maketitle

  \begin{enumerate}
    \setcounter{enumi}{1}
    \item
      \begin{enumerate}
        \setcounter{enumii}{12}
        \item
          \begin{enumerate}
            \setcounter{enumiii}{2}
            \item
              $A \times B$

              \crossproduct
                {$\varnothing$, $\{\varnothing\}$, ${\{\varnothing, \{\varnothing\}\}}$}
                {${(\varnothing, \{\varnothing\})}$, $\{\varnothing\}$, ${(\{\varnothing\}, \varnothing)}$}

              $B \times A$

              \crossproduct
                {${(\varnothing, \{\varnothing\})}$, $\{\varnothing\}$, ${(\{\varnothing\}, \varnothing)}$}
                {$\varnothing$, $\{\varnothing\}$, ${\{\varnothing, \{\varnothing\}\}}$}
            \item
              $A \times B$

              \crossproduct
                {{(2, 4)}, {(3, 1)}}
                {{(4, 1)}, {(2, 3)}}

              $B \times A$

              \crossproduct
                {{(4, 1)}, {(2, 3)}}
                {{(2, 4)}, {(3, 1)}}
          \end{enumerate}
        \setcounter{enumii}{16}
        \item
          We're asked to show that $(a, b) = (x, y)$ iff $a = x$ and $b = y$.

          \begin{proof}
            \begin{align*}
              (a, b) = (x, y)
              &\iff \{\{a\}, \{a, b\}\} = \{\{x\}, \{x, y\}\} \\
              &\iff \left(\{\{a\}, \{a, b\}\} \subseteq \{\{x\}, \{x, y\}\}\right) \\
              &\quad\quad \land \left(\{\{x\}, \{x, y\}\} \subseteq \{\{a\}, \{a, b\}\}\right)
              \\
              &\iff \left(\{a\} \in \{\{x\}, \{x, y\}\}\right) \\
              &\quad\quad \land \left(\{a, b\} \in \{\{x\}, \{x, y\}\}\right) \\
              &\quad\quad \land \left(\{x\} \in \{\{a\}, \{a, b\}\}\right) \\
              &\quad\quad \land \left(\{x, y\} \in \{\{a\}, \{a, b\}\}\right)
              \\
              &\iff \left(\{a\} = \{x\}\right) \land \left(\{a, b\} = \{x, y\}\right)
              \\
              &\iff \left(a = x\right) \land \left(b = y\right)
              \\
            \end{align*}

            Since we have connected both sides with a series of bi-conditional statements,
            we have proven that:

            $(a, b) = (x, y)$ iff $a = x$ and $b = y$.
          \end{proof}
        \item
          \begin{enumerate}
            \item
              \begin{proof}
                \[
                  A \Delta B = \left(A - B\right) \cup \left(B - A\right) = \left(B - A\right) \cup \left(A - B\right) = B \Delta A
                \]
              \end{proof}
            \item
              This proof is a bit longer than the others.

              \begin{proof}
                \begin{align*}
                  A \Delta B
                  &= \left(A - B\right) \cup \left(B - A\right) \\
                  &= \{x | \left(x \in A \land x \notin B\right) \lor \left(x \in B \land x \notin A\right)\} \\
                  &= \{x | \left[\left(x \in A \land x \notin B\right) \lor x \in B\right] \land \left[\left(x \in A \land x \notin B\right) \lor x \notin A\right]\} \\
                  &= \{x | \left(x \in A \lor x \in B\right) \land \left(x \notin B \lor x \in B\right) \\
                  &\quad \quad \land \left(x \in A \lor x \notin A\right) \land \left(x \notin B \lor x \notin A\right)\} \\
                  &= \{x | \left(x \in A \lor x \in B\right) \land \left(x \notin B \lor x \notin A\right)\} \\
                  &= \{x | \left(x \in A \lor x \in B\right) \land \left(x \notin A \lor x \notin B\right)\} \\
                  &= \{x | \left(x \in A \lor x \in B\right) \land \sim\left(x \in A \land x \in B\right)\} \\
                  &= \{x | \left(x \in A \cup B\right) \land \sim\left(x \in A \cap B\right)\} \\
                  &= \{x | \left(x \in A \cup B\right) \land \left(x \notin A \cap B\right)\} \\
                  &= \left(A \cup B\right) - \left(A \cap B\right) \\
                \end{align*}
              \end{proof}
            \item
              \begin{proof}
                \[
                  A \Delta A = \left(A - A\right) \cup \left(A - A\right) = \varnothing \cup \varnothing = \varnothing
                \]
              \end{proof}
            \item
              \begin{proof}
                \[
                  A \Delta \varnothing = \left(A - \varnothing\right) \cup \left(\varnothing - A\right) = A \cup \varnothing = A
                \]
              \end{proof}
          \end{enumerate}
      \end{enumerate}
    \item
      \begin{enumerate}
        \item
          \begin{enumerate}
            \setcounter{enumiii}{5}
            \item
            \setcounter{enumiii}{7}
            \item
            \setcounter{enumiii}{9}
            \item
          \end{enumerate}
        \setcounter{enumii}{11}
        \item
        \setcounter{enumii}{14}
        \item
          \begin{enumerate}
            \setcounter{enumiii}{4}
            \item
            \item
          \end{enumerate}
        \item
        \item
          \begin{enumerate}
            \setcounter{enumiii}{2}
            \item
            \item
          \end{enumerate}
      \end{enumerate}
    \item
      \begin{enumerate}
        \setcounter{enumii}{5}
        \item
          \begin{enumerate}
            \setcounter{enumiii}{8}
            \item
            \setcounter{enumiii}{10}
            \item
          \end{enumerate}
        \item
          \begin{enumerate}
            \setcounter{enumiii}{11}
            \item
            \item
          \end{enumerate}
        \item
          \begin{enumerate}
            \setcounter{enumiii}{7}
            \item
          \end{enumerate}
        \setcounter{enumii}{11}
        \item
          \begin{enumerate}
            \setcounter{enumiii}{1}
            \item
          \end{enumerate}
      \end{enumerate}
  \end{enumerate}
\end{document}
