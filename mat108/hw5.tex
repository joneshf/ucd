\documentclass[12pt,letterpaper]{article}

\usepackage[margin=1in]{geometry}
\usepackage[round-mode=figures,round-precision=3,scientific-notation=false]{siunitx}
\usepackage[super]{nth}
\usepackage[title]{appendix}
\usepackage{amsfonts}
\usepackage{amsmath}
\usepackage{amssymb}
\usepackage{amsthm}
\usepackage{cancel}
\usepackage{caption}
\usepackage{color, colortbl}
\usepackage{dcolumn}
\usepackage{enumitem}
\usepackage{float}
\usepackage{fp}
\usepackage{ifthen}
\usepackage{mathtools}
\usepackage{pgfplots}
\usepackage{subcaption}
\usepackage{tabularx}
\usepackage{tikz}
\usepackage{titling}

\usepgfplotslibrary{statistics}

\definecolor{Gray}{gray}{0.8}

\pgfplotsset{compat=1.8}

\newcolumntype{d}{D{.}{.}{-1}}
\newcolumntype{g}{>{\columncolor{Gray}}c}

\newcommand*\biconditional[3]{
  This statement is #1 since both sides of the bi-conditional have #2 truth values.

  #3
}
\newcommand*\biconditionaltrue[2]{
  \biconditional{true}{the same}{#1 and #2.}
}
\newcommand*\biconditionalfalse[2]{
  \biconditional{false}{different}{#1 yet #2.}
}

\newcommand*\directproofsimple[6]{
  With integers $x$ and $y$, we want to show:

  If #1, then #2 is even.

  \begin{proof}
    Suppose #1.

    Then there exist some integers $p, q$ such that #3 and #4.

    Then we have #5$ = 2($#6$)$.

    Since #6 is an integer, we can rename as $r = $ #6.

    So #2 $ = 2r$, and is even.

    Thus if #1, then #2 is even
  \end{proof}
}

\newcommand*\crossproduct[2]{%
  \{%
    \foreach \x [count=\i] in {#1} {%
      \foreach \y [count=\j] in {#2} {%
        \ifthenelse{\equal{\i}{\j} \AND \equal{\i}{1}}{}{, }%
        (\x, \y)%
      }%
    }%
  \}%
}

\renewcommand{\labelenumi}{\S 2.\arabic*}
\renewcommand{\labelenumii}{\arabic*}
\renewcommand{\labelenumiii}{(\alph*)}

\setlength{\droptitle}{-10ex}

\preauthor{\begin{flushright}\large \lineskip 0.5em}
\postauthor{\par\end{flushright}}
\predate{\begin{flushright}\large}
\postdate{\par\end{flushright}}

\title{MAT 108 HW 5\vspace{-2ex}}
\author{Hardy Jones\\
        999397426\\
        Professor Bandyopadhyay\vspace{-2ex}}
\date{Spring 2015}

\begin{document}
  \maketitle

  \begin{enumerate}
    \setcounter{enumi}{1}
    \item
      \begin{enumerate}
        \setcounter{enumii}{12}
        \item
          \begin{enumerate}
            \setcounter{enumiii}{2}
            \item
              $A \times B$

              \crossproduct
                {$\varnothing$, $\{\varnothing\}$, ${\{\varnothing, \{\varnothing\}\}}$}
                {${(\varnothing, \{\varnothing\})}$, $\{\varnothing\}$, ${(\{\varnothing\}, \varnothing)}$}

              $B \times A$

              \crossproduct
                {${(\varnothing, \{\varnothing\})}$, $\{\varnothing\}$, ${(\{\varnothing\}, \varnothing)}$}
                {$\varnothing$, $\{\varnothing\}$, ${\{\varnothing, \{\varnothing\}\}}$}
            \item
              $A \times B$

              \crossproduct
                {{(2, 4)}, {(3, 1)}}
                {{(4, 1)}, {(2, 3)}}

              $B \times A$

              \crossproduct
                {{(4, 1)}, {(2, 3)}}
                {{(2, 4)}, {(3, 1)}}
          \end{enumerate}
        \setcounter{enumii}{16}
        \item
          We're asked to show that $(a, b) = (x, y)$ iff $a = x$ and $b = y$.

          \begin{proof}
            \begin{align*}
              (a, b) = (x, y)
              &\iff \{\{a\}, \{a, b\}\} = \{\{x\}, \{x, y\}\} \\
              &\iff \left(\{\{a\}, \{a, b\}\} \subseteq \{\{x\}, \{x, y\}\}\right) \\
              &\quad\quad \land \left(\{\{x\}, \{x, y\}\} \subseteq \{\{a\}, \{a, b\}\}\right)
              \\
              &\iff \left(\{a\} \in \{\{x\}, \{x, y\}\}\right) \\
              &\quad\quad \land \left(\{a, b\} \in \{\{x\}, \{x, y\}\}\right) \\
              &\quad\quad \land \left(\{x\} \in \{\{a\}, \{a, b\}\}\right) \\
              &\quad\quad \land \left(\{x, y\} \in \{\{a\}, \{a, b\}\}\right)
              \\
              &\iff \left(\{a\} = \{x\}\right) \land \left(\{a, b\} = \{x, y\}\right)
              \\
              &\iff \left(a = x\right) \land \left(b = y\right)
              \\
            \end{align*}

            Since we have connected both sides with a series of bi-conditional statements,
            we have proven that:

            $(a, b) = (x, y)$ iff $a = x$ and $b = y$.
          \end{proof}
        \item
          \begin{enumerate}
            \item
              \begin{proof}
                \[
                  A \Delta B = \left(A - B\right) \cup \left(B - A\right) = \left(B - A\right) \cup \left(A - B\right) = B \Delta A
                \]
              \end{proof}
            \item
              This proof is a bit longer than the others.

              \begin{proof}
                \begin{align*}
                  A \Delta B
                  &= \left(A - B\right) \cup \left(B - A\right) \\
                  &= \{x | \left(x \in A \land x \notin B\right) \lor \left(x \in B \land x \notin A\right)\} \\
                  &= \{x | \left[\left(x \in A \land x \notin B\right) \lor x \in B\right] \land \left[\left(x \in A \land x \notin B\right) \lor x \notin A\right]\} \\
                  &= \{x | \left(x \in A \lor x \in B\right) \land \left(x \notin B \lor x \in B\right) \\
                  &\quad \quad \land \left(x \in A \lor x \notin A\right) \land \left(x \notin B \lor x \notin A\right)\} \\
                  &= \{x | \left(x \in A \lor x \in B\right) \land \left(x \notin B \lor x \notin A\right)\} \\
                  &= \{x | \left(x \in A \lor x \in B\right) \land \left(x \notin A \lor x \notin B\right)\} \\
                  &= \{x | \left(x \in A \lor x \in B\right) \land \sim\left(x \in A \land x \in B\right)\} \\
                  &= \{x | \left(x \in A \cup B\right) \land \sim\left(x \in A \cap B\right)\} \\
                  &= \{x | \left(x \in A \cup B\right) \land \left(x \notin A \cap B\right)\} \\
                  &= \left(A \cup B\right) - \left(A \cap B\right) \\
                \end{align*}
              \end{proof}
            \item
              \begin{proof}
                \[
                  A \Delta A = \left(A - A\right) \cup \left(A - A\right) = \varnothing \cup \varnothing = \varnothing
                \]
              \end{proof}
            \item
              \begin{proof}
                \[
                  A \Delta \varnothing = \left(A - \varnothing\right) \cup \left(\varnothing - A\right) = A \cup \varnothing = A
                \]
              \end{proof}
          \end{enumerate}
      \end{enumerate}
    \item
      \begin{enumerate}
        \item
          \begin{enumerate}
            \setcounter{enumiii}{5}
            \item
              \[
                \bigcup_{i = 1}^{10} A_i = \{1, 2, \dots, 19\}
              ,
                \bigcap_{i = 1}^{10} A_i = \varnothing
              \]
            \setcounter{enumiii}{7}
            \item
              \[
                \bigcup_{r \in (0, \infty)} A_r = [-\pi, \infty)
              ,
                \bigcap_{r \in (0, \infty)} A_r = [-\pi, 0)
              \]
            \setcounter{enumiii}{9}
            \item
              \[
                \bigcup_{i = 1}^{\infty} M_i = \mathbb{Z}
              ,
                \bigcap_{i = 1}^{\infty} M_i = \{0\}
              \]
          \end{enumerate}
        \setcounter{enumii}{11}
        \item
          Let $A_n = (0, \frac{1}{n})$.

          Then for any $m, n \in \mathbb{N}$

          \[
            M_m \cap M_n
            =
            \begin{cases}
              \left(0, \frac{1}{m}\right), & \mbox{if } m < n \\
              \left(0, \frac{1}{n}\right), & \mbox{otherwise} \\
            \end{cases}
          \]

          But, $\bigcap_{i = 1}^{\infty} M_i = \varnothing$
        \setcounter{enumii}{14}
        \item
          \begin{enumerate}
            \setcounter{enumiii}{4}
            \item
              \begin{proof}
                Choose an arbitrary $x \in \bigcup\limits_{i = 1}^k A_i$.

                Then there exists some $l \in \mathbb{N}$ such that $l \leq k$ and $x \in A_l$.

                Now, since $l \leq k, l \leq m$, so $A_l \subseteq \bigcup\limits_{i = 1}^m A_i$,
                and $x \in \bigcup\limits_{i = 1}^m A_i$.

                Since the choice of $x$ was arbitrary, this works for all $x \in \bigcup\limits_{i = 1}^k A_i$.

                Then every $x$ contained in $\bigcup\limits_{i = 1}^k A_i$ is also in $\bigcup\limits_{i = 1}^m A_i$.

                Thus $\bigcup\limits_{i = 1}^k A_i \subseteq \bigcup\limits_{i = 1}^m A_i$
              \end{proof}
            \item
              \begin{proof}
                Choose an arbitrary $x \in \bigcap\limits_{i = 1}^m A_i$.

                Then for all $l \in \{1, 2, \dots, k, k + 1, \dots, m\}, x \in A_l$.

                This implies that for all $l \in \{1, 2, \dots, k\}, x \in A_l$.

                Which means that $x \in \bigcap\limits_{i = 1}^k A_i$.

                Since the choice of $x$ was arbitrary, this works for all $x \in \bigcap\limits_{i = 1}^m A_i$.

                Then every $x$ contained in $\bigcap\limits_{i = 1}^m A_i$ is also in $\bigcap\limits_{i = 1}^k A_i$.

                Thus, $\bigcap\limits_{i = 1}^m A_i \subseteq \bigcap\limits_{i = 1}^k A_i$.
              \end{proof}
          \end{enumerate}
        \item
          \begin{enumerate}
            \item
              \begin{proof}
                We need to show both sides for any $k \in \mathbb{N}$.

                First choose some arbitrary $k \in \mathbb{N}$.
                \begin{itemize}
                  \item $(\subseteq)$

                    Choose some $x \in \bigcap\limits_{i = 1}^k A_i$.

                    Then for all $l \in \{1, 2, \dots, k\}, x \in A_l$.

                    This means that $x \in A_k$.

                    Since the choice of $x$ was arbitrary, this works for all $x \in \bigcap\limits_{i = 1}^k A_i$.

                    Then every $x$ contained in $\bigcap\limits_{i = 1}^k A_i$ is also in $A_k$.

                    Thus, $\bigcap\limits_{i = 1}^k A_i \subseteq A_k$.

                  \item $(\supseteq)$

                    Choose some $x \in A_k$.

                    Since $\mathcal{A}$ is a decreasing nested family of sets,
                    for any $i \in \mathbb{N} \leq k, A_k \subseteq A_i$.

                    Now, since $x$ is an element of $A_k$, $x$ is an element of all supersets of $A_k$.

                    That is to say that $x \in A_{k-1} \land x \in A_{k-2} \land \dots \land x \in A_1$.

                    So $x \in \bigcap\limits_{i = 1}^k A_i$.

                    Since the choice of $x$ was arbitrary, this works for all $x \in A_k$.

                    Then every $x$ contained in $A_k$ is also in $\bigcap\limits_{i = 1}^k A_i$.

                    Thus, $A_k \subseteq \bigcap\limits_{i = 1}^k A_i$.
                \end{itemize}

                Since we have shown both $\bigcap\limits_{i = 1}^k A_i \subseteq A_k$,
                and $A_k \subseteq \bigcap\limits_{i = 1}^k A_i$, for any $k \in \mathbb{N}$.

                We have shown that for all $k \in \mathbb{N}, \bigcap\limits_{i = 1}^k A_i = A_k$.
              \end{proof}
            \item
              \begin{proof}
                We need to show both sides.

                \begin{itemize}
                  \item $(\subseteq)$

                    Choose some $x \in \bigcup\limits_{i = 1}^{\infty} A_i$.

                    Then there exists some $l \in \mathbb{N}$ such that $x \in A_l$.

                    Now, any $n \in \mathbb{N}$ is greater than or equal to $1$.

                    So $A_l \subseteq A_1$, since $1 \leq l$, and $\mathcal{A}$ is a decreasing nested family of sets.

                    Then $A_l \subseteq A_1, x \in A_1$.

                    Since the choice of $x$ was arbitrary, this works for all $x \in \bigcup\limits_{i = 1}^{\infty} A_i$.

                    Then every $x$ contained in $\bigcup\limits_{i = 1}^{\infty} A_i$ is also in $A_1$.

                    Thus, $\bigcup\limits_{i = 1}^{\infty} A_i \subseteq A_1$.
                  \item $(\supseteq)$

                    \begin{align*}
                      A_1 &\subseteq A_1 \\
                      &\subseteq A_1 \cup A_2 \\
                      &\subseteq A_1 \cup A_2 \cup A_3 \\
                      &\ \ \vdots \\
                      &\subseteq \bigcup\limits_{i = 1}^{\infty} A_i \\
                    \end{align*}
                \end{itemize}

                Since we have shown both sides.

                We have $\bigcup\limits_{i = 1}^{\infty} A_i = A_1$
              \end{proof}
          \end{enumerate}
        \item
          \begin{enumerate}
            \setcounter{enumiii}{2}
            \item Let $A_i = \{0, 1\}$, then $\mathcal{A} = \{\{0, 1\}\}$, and $\bigcap\limits_{i = 1}^{\infty}A_i = \{0, 1\}$
            \item Let $A_i = \varnothing$, then $\mathcal{A} = \{\varnothing\}$, and $\bigcap\limits_{i = 1}^{\infty}A_i = \varnothing$
          \end{enumerate}
      \end{enumerate}
    \item
      \begin{enumerate}
        \setcounter{enumii}{5}
        \item
          \begin{enumerate}
            \setcounter{enumiii}{8}
            \item
              \begin{proof}
                We show by PMI.
                $\sum\limits_{i = 1}^n \frac{1}{\left(2i - 1\right)\left(2i + 1\right)} = \frac{n}{2n + 1}$
                \begin{itemize}
                  \item Base Case.

                    Let $n = 1$.

                    \[
                      \sum_{i = 1}^1 \frac{1}{\left(2i - 1\right)\left(2i + 1\right)} = \frac{1}{\left(2\left(1\right) - 1\right)\left(2\left(1\right) + 1\right)} = \frac{1}{3} = \frac{1}{2\left(1\right) + 1}
                    \]
                  \item Inductive Case.

                    Assume for some $n \in \mathbb{N}, \sum\limits_{i = 1}^n \frac{1}{\left(2i - 1\right)\left(2i + 1\right)} = \frac{n}{2n + 1}$.

                    Then

                    \begin{align*}
                      \sum_{i = 1}^{n + 1} \frac{1}{\left(2i - 1\right)\left(2i + 1\right)}
                      &= \sum_{i = 1}^n \frac{1}{\left(2i - 1\right)\left(2i + 1\right)} \\
                      &\quad + \frac{1}{\left(2\left(n + 1\right) - 1\right)\left(2\left(n + 1\right) + 1\right)}\\
                      &= \frac{n}{2n + 1} + \frac{1}{\left(2n + 2 - 1\right)\left(2n + 2 + 1\right)}\\
                      &= \frac{n}{2n + 1} + \frac{1}{\left(2n + 1\right)\left(2n + 3\right)}\\
                      &= \frac{n(2n + 3) + 1}{\left(2n + 1\right)\left(2n + 3\right)}\\
                      &= \frac{2n^2 + 3n + 1}{\left(2n + 1\right)\left(2n + 3\right)}\\
                      &= \frac{\left(2n + 1\right)\left(n + 1\right)}{\left(2n + 1\right)\left(2n + 3\right)}\\
                      &= \frac{n + 1}{2n + 3}\\
                      &= \frac{n + 1}{2\left(n + 1\right) + 1} \\
                    \end{align*}

                  \item
                    From the Base case and the inductive case,
                    we use the PMI to state $\sum\limits_{i = 1}^n \frac{1}{\left(2i - 1\right)\left(2i + 1\right)} = \frac{n}{2n + 1}, \forall n \in \mathbb{N}$
                \end{itemize}
              \end{proof}
            \setcounter{enumiii}{10}
            \item
              \begin{proof}
                We show by PMI.
                $\prod\limits_{i = 1}^{n}(2i - 1) = \frac{\left(2n\right)!}{n! 2^n}$
                \begin{itemize}
                  \item Base Case.

                    Let $n = 1$.

                    \[
                      \prod\limits_{i = 1}^1(2i - 1) = 2(1) - 1 = 2 - 1 = 1 = \frac{2}{2} = \frac{2(1)}{(1)2} = \frac{\left(2\left(1\right)\right)!}{1! 2^1}
                    \]
                  \item Inductive Case.

                    Assume for some $n \in \mathbb{N}, \prod\limits_{i = 1}^{n}(2i - 1) = \frac{\left(2n\right)!}{n! 2^n}$.

                    Then

                    \begin{align*}
                      \prod\limits_{i = 1}^{n + 1}(2i - 1)
                      &= \prod\limits_{i = 1}^{n}(2i - 1) \cdot \left(2(n + 1) - 1\right) \\
                      &= \frac{\left(2n\right)!}{n! 2^n} \cdot \left(2(n + 1) - 1\right) \\
                      &= \frac{\left(2n\right)!}{n! 2^n} \cdot \left(2n + 2 - 1\right) \\
                      &= \frac{\left(2n\right)!}{n! 2^n} \cdot \left(2n + 1\right) \\
                      &= \frac{\left(2n + 1\right)!}{n! 2^n} \\
                      &= \frac{\left(2n + 1\right)!}{n! 2^n} \cdot \frac{2n + 2}{2n + 2} \\
                      &= \frac{\left(2n + 2\right)!}{n! 2^n \left(2n + 2\right)} \\
                      &= \frac{\left(2n + 2\right)!}{n! 2^n \left(2\left(n + 1\right)\right)} \\
                      &= \frac{\left(2n + 2\right)!}{n! 2^{n + 1} \left(n + 1\right)} \\
                      &= \frac{\left(2n + 2\right)!}{\left(n + 1\right)! 2^{n + 1}} \\
                      &= \frac{\left(2\left(n + 1\right)\right)!}{\left(n + 1\right)! 2^{n + 1}} \\
                    \end{align*}

                  \item
                    From the Base case and the inductive case,
                    we use the PMI to state $\prod\limits_{i = 1}^{n}(2i - 1) = \frac{\left(2n\right)!}{n! 2^n}, \forall n \in \mathbb{N}$
                \end{itemize}
              \end{proof}
          \end{enumerate}
        \item
          \begin{enumerate}
            \setcounter{enumiii}{11}
            \item
              \begin{proof}
                We show by PMI.
                $\forall x > 0 \in \mathbb{R}, \left(1 + x\right)^n \geq 1 + nx$
                \begin{itemize}
                  \item Base Case.

                    Let $n = 1$.

                    \[
                      \left(1 + x\right)^1 = 1 + x  = 1 + (1)x  \geq 1 + (1)x
                    \]
                  \item Inductive Case.

                    Assume for some $n \in \mathbb{N}, \forall x > 0 \in \mathbb{R}, \left(1 + x\right)^n \geq 1 + nx$.

                    Then

                    \begin{align*}
                      \left(1 + x\right)^{n + 1}
                      &= \left(1 + x\right)^n\left(1 + x\right) \\
                      &\geq \left(1 + nx\right)\left(1 + x\right) \\
                      &= 1 + x + nx + nx^2 \\
                      &= 1 + nx + x + nx^2 \\
                      &= 1 + (n + 1)x + nx^2 \\
                      &\geq 1 + (n + 1)x \\
                    \end{align*}

                  \item
                    From the Base case and the inductive case,
                    we use the PMI to state $\forall x > 0 \in \mathbb{R}, \left(1 + x\right)^n \geq 1 + nx, \forall n \in \mathbb{N}$
                \end{itemize}
              \end{proof}
            \item
              \begin{proof}
                We show by PMI.
                $\frac{n^3}{3} + \frac{n^5}{5} + \frac{7n}{15} \in \mathbb{N}$
                \begin{itemize}
                  \item Base Case.

                    Let $n = 1$.

                    \[
                      \frac{1^3}{3} + \frac{1^5}{5} + \frac{7\left(1\right)}{15} = \frac{1}{3} + \frac{1}{5} + \frac{7}{15}  = \frac{5}{15} + \frac{3}{15} + \frac{7}{15}  = \frac{15}{15}  = 1
                    \]

                    And $1 \in \mathbb{N}$
                  \item Inductive Case.

                    Assume for some $n \in \mathbb{N}, \frac{n^3}{3} + \frac{n^5}{5} + \frac{7n}{15} \in \mathbb{N}$.

                    Then

                    \begin{align*}
                      \frac{\left(n + 1\right)^3}{3} + \frac{\left(n + 1\right)^5}{5} + \frac{7\left(n + 1\right)}{15}
                      &= \frac{n^3 + 3n^2 + 3n + 1}{3} \\
                      &\quad+ \frac{n^5 + 5n^4 + 10n^3 + 10n^2 + 5n + 1}{5} \\
                      &\quad+ \frac{7n + 7}{15} \\
                      &= \frac{n^3}{3} + \frac{3n^2 + 3n}{3} + \frac{1}{3} \\
                      &\quad+ \frac{n^5}{5} + \frac{5n^4 + 10n^3 + 10n^2 + 5n}{5} + \frac{1}{5} \\
                      &\quad+ \frac{7n}{15} + \frac{7}{15} \\
                      &= \left(\frac{n^3}{3} + \frac{n^5}{5} + \frac{7n}{15}\right) + \left(\frac{1}{3} + \frac{1}{5} + \frac{7}{15}\right) \\
                      &\quad+ n^2 + n + n^4 + 2n^3 + 2n^2 + n \\
                    \end{align*}

                    Now, since we assumed $\frac{n^3}{3} + \frac{n^5}{5} + \frac{7n}{15} \in \mathbb{N}$,
                    and $\frac{1}{3} + \frac{1}{5} + \frac{7}{15} = 1 \in \mathbb{N}$, and $n^2 + n + n^4 + 2n^3 + 2n^2 + n \in \mathbb{N}$,

                    we have $\frac{n^3}{3} + \frac{n^5}{5} + \frac{7n}{15} + 1 + n^2 + n + n^4 + 2n^3 + 2n^2 + n \in \mathbb{N}$.

                    Thus $\frac{\left(n + 1\right)^3}{3} + \frac{\left(n + 1\right)^5}{5} + \frac{7\left(n + 1\right)}{15} \in \mathbb{N}$

                  \item
                    From the Base case and the inductive case,
                    we use the PMI to state $\frac{n^3}{3} + \frac{n^5}{5} + \frac{7n}{15} \in \mathbb{N}, \forall n \in \mathbb{N}$
                \end{itemize}
              \end{proof}
          \end{enumerate}
        \item
          \begin{enumerate}
            \setcounter{enumiii}{7}
            \item
              \begin{proof}
                We show by the Generalized PMI.
                $\sqrt{n} < \frac{1}{\sqrt{1}} + \frac{1}{\sqrt{2}} + \dots + \frac{1}{\sqrt{n}}$, for $n \geq 2$
                \begin{itemize}
                  \item Base Case.

                    Let $n = 2$.

                    \begin{align*}
                      1 &< \sqrt{2} \\
                      2 &< \sqrt{2} + 1 \\
                      \sqrt{2}\left(\sqrt{2}\right) &< \sqrt{2}\left(1 + \frac{1}{\sqrt{2}}\right) \\
                      \sqrt{2} &< 1 + \frac{1}{\sqrt{2}} \\
                      \sqrt{2} &< \frac{1}{1} + \frac{1}{\sqrt{2}} \\
                      \sqrt{2} &< \frac{1}{\sqrt{1}} + \frac{1}{\sqrt{2}} \\
                    \end{align*}
                  \item Inductive Case.

                    Assume for some $n \geq 2 \in \mathbb{N}, \sqrt{n} < \frac{1}{\sqrt{1}} + \frac{1}{\sqrt{2}} + \dots + \frac{1}{\sqrt{n}}$.

                    Then

                    \begin{align*}
                      \sqrt{n} &< \frac{1}{\sqrt{1}} + \frac{1}{\sqrt{2}} + \dots + \frac{1}{\sqrt{n}} \\
                      \sqrt{n}\left(\sqrt{n}\right) &< \sqrt{n}\left(\frac{1}{\sqrt{1}} + \frac{1}{\sqrt{2}} + \dots + \frac{1}{\sqrt{n}}\right) \\
                      n &< \sqrt{n}\left(\frac{1}{\sqrt{1}} + \frac{1}{\sqrt{2}} + \dots + \frac{1}{\sqrt{n}}\right) \\
                      n &< \sqrt{n + 1}\left(\frac{1}{\sqrt{1}} + \frac{1}{\sqrt{2}} + \dots + \frac{1}{\sqrt{n}}\right) \\
                      n + 1 &< \sqrt{n + 1}\left(\frac{1}{\sqrt{1}} + \frac{1}{\sqrt{2}} + \dots + \frac{1}{\sqrt{n}}\right) + \frac{\sqrt{n + 1}}{\sqrt{n + 1}} \\
                      n + 1 &< \sqrt{n + 1}\left(\frac{1}{\sqrt{1}} + \frac{1}{\sqrt{2}} + \dots + \frac{1}{\sqrt{n}} + \frac{1}{\sqrt{n + 1}}\right) \\
                      \sqrt{n + 1}\left(\sqrt{n + 1}\right) &< \sqrt{n + 1}\left(\frac{1}{\sqrt{1}} + \frac{1}{\sqrt{2}} + \dots + \frac{1}{\sqrt{n}} + \frac{1}{\sqrt{n + 1}}\right) \\
                      \sqrt{n + 1} &< \frac{1}{\sqrt{1}} + \frac{1}{\sqrt{2}} + \dots + \frac{1}{\sqrt{n}} + \frac{1}{\sqrt{n + 1}} \\
                    \end{align*}

                  \item
                    From the Base case and the inductive case,
                    we use the Generalized PMI to state:

                    $\sqrt{n} < \frac{1}{\sqrt{1}} + \frac{1}{\sqrt{2}} + \dots + \frac{1}{\sqrt{n}}$, for $n \geq 2$.
                \end{itemize}
              \end{proof}
          \end{enumerate}
        \setcounter{enumii}{11}
        \item
          \begin{enumerate}
            \setcounter{enumiii}{1}
            \item
              \begin{proof}
                We show by PMI.
                Every $n$-player tournament has a top player.
                \begin{itemize}
                  \item Base Case.

                    Let $n = 1$.
                    Then this tournament has a top player vacuously.
                  \item Inductive Case.

                    Assume for some $n \in \mathbb{N}$,
                    the $n$-player tournament has a top player $x$.

                    Now if we add a new player, $y$,
                    then this tournament is now an $n + 1$-player tournament.
                    $y$ will play all other $n$ players,
                    and three outcomes are possible.

                    \begin{enumerate}
                      \item
                        If $y$ beats $x$,
                        then $y$ also beats a player that beats all other players.
                        So $y$ is also a top player.
                      \item
                        If $y$ does not beat $x$, but beats a player $z$ that beats $x$,
                        then for every other player $w$, $y$ beats a player that beats $w$.
                        So $y$ is also a top player.
                      \item
                        If $y$ does not beat $x$, nor does $y$ beat a player $z$ that beats $x$,
                        then $y$ is not a top player.
                        However, $x$ still remains a top player.
                    \end{enumerate}

                    In any of the outcomes, there is always a top player.

                  \item
                    From the Base case and the inductive case,
                    we use the PMI to state $\forall n \in \mathbb{N}$,
                    every $n$-player tournament has a top player.
                \end{itemize}
              \end{proof}
          \end{enumerate}
      \end{enumerate}
  \end{enumerate}
\end{document}
