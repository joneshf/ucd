\documentclass[12pt,letterpaper]{article}
\usepackage{amsmath}
\usepackage{amsfonts}
\usepackage{amsthm}
\usepackage{mathtools}
\usepackage{cancel}
\usepackage[margin=1in]{geometry}
\usepackage{titling}
\usepackage{fp}
\usepackage{enumitem}
\usepackage[super]{nth}
\usepackage{pgfplots}
\pgfplotsset{compat=1.8}
\usepgfplotslibrary{statistics}

\setlength{\droptitle}{-10ex}

\preauthor{\begin{flushright}\large \lineskip 0.5em}
\postauthor{\par\end{flushright}}
\predate{\begin{flushright}\large}
\postdate{\par\end{flushright}}

\title{STA 032 Homework 2\vspace{-2ex}}
\author{Hardy Jones\\
        999397426\\
        Professor Melcon\vspace{-2ex}}
\date{Winter 2015}

\begin{document}
  \maketitle

  \begin{enumerate}
    \item [$\S$ 2.1]
      \begin{enumerate}
        \item [3]
          \begin{enumerate}[label=(\arabic*)]
            \item
              The possible outcomes are:

              \begin{tabular}{| c | c | c | c |}
                \hline
                Q1 & Q2 & Q3 & Q4 \\
                \hline
                T & T & T & T \\
                T & T & T & F \\
                T & T & F & T \\
                T & T & F & F \\
                T & F & T & T \\
                T & F & T & F \\
                T & F & F & T \\
                T & F & F & F \\
                F & T & T & T \\
                F & T & T & F \\
                F & T & F & T \\
                F & T & F & F \\
                F & F & T & T \\
                F & F & T & F \\
                F & F & F & T \\
                F & F & F & F \\
                \hline
              \end{tabular}
            \item
              The answers are all the same only twice.
              Once when the answers are all True,
              and once when the answers are all False.

              So the probability of this event is $\frac{2}{16} = \frac{1}{8} = 0.125 = 12.5\%$
            \item
              Exactly one answer is True occurs four times.

              So the probability of this event is $\frac{4}{16} = \frac{1}{4} = 0.25 = 25\%$
            \item
              At most one answer is True occurs five times.

              So the probability of this event is $\frac{5}{16} = 0.3125 = 31.25\%$
          \end{enumerate}
        \item [5]
          \begin{enumerate}[label=(\arabic*)]
            \item
              The possible outcomes are:
              $\{1, 2, 31, 32, 41, 42, 341, 342, 431, 432\}$
            \item
              If we interview exactly one candidate $A = \{1, 2\}$
            \item
              If we interview exactly three candidates $B = \{341, 342, 431, 432\}$
            \item
              If we interview candidate 3 $C = \{31, 32, 341, 342, 431, 432\}$
            \item
              If we do not interview candidate 2 $D = \{1, 31, 41, 341, 431\}$
            \item
              If we interview candidate 4 $E = \{41, 42, 341, 342, 431, 432\}$.

              $A$ and $E$ are mutually exclusive as $A \cap E = \{\}$

              $B$ and $E$ are not mutually exclusive as $B \cap E = \{341, 342, 431, 432\}$

              $C$ and $E$ are not mutually exclusive as $C \cap E = \{341, 342, 431, 432\}$

              $D$ and $E$ are not mutually exclusive as $D \cap E = \{41, 341, 431\}$
          \end{enumerate}
        \item [6]
          \begin{enumerate}[label=(\arabic*)]
            \item
              The equally likely outcomes are $\{\{1, 2\}, \{1, 3\}, \{1, 4\}, \{2, 3\}, \{2, 4\}, \{3, 4\}\}$
            \item
              There are six equally likely outcomes.
              Candidate 1 and candidate 2 have to be in the same outcome.
              This only occurs once.

              So the probability that both candidates are qualified is
              $\frac{1}{6} = 0.1\overline{6} = 16.\overline{6}\%$
            \item
              There are six equally likely outcomes.
              One of (but not both) candidate 1 and candidate 2 have to be in the outcome
              This occurs four times.

              So the probability that exactly one candidate is qualified is
              $\frac{4}{6} = 0.\overline{6} = 66.\overline{6}\%$
          \end{enumerate}
        \item [15]
          We have our probabilities.

          $P\left(R\right) = 0.85, P\left(M\right) = 0.78, P\left(R \cap M\right) = 0.65$

          These imply

          $P\left(R^C\right) = 0.15, P\left(M^C\right) = 0.22, P\left(R \cup M\right) = P\left(R\right) + P\left(M\right) - P\left(R \cap M\right) = 0.98$
          \begin{enumerate}[label=(\arabic*)]
            \item
              We want to find $P\left(M \cap R^C\right)$

              \[
                P\left(M \cap R^C\right) = P\left(M\right) - P\left(M \cap R\right) = 0.78 - 0.65 = 0.13
              \]

              So the probability that a student is proficient in mathematics but not reading is $13\%$.
            \item
              We want to find $P\left(R \cap M^C\right)$

              \[
                P\left(R \cap M^C\right) = P\left(R\right) - P\left(R \cap M\right) = 0.85 - 0.65 = 0.2
              \]

              So the probability that a student is proficient in reading but not mathematics is $20\%$.
            \item
              We want to find $P\left(\left(R \cup M\right)^C\right)$

              \[
                P\left(\left(R \cup M\right)^C\right) = 1 - P\left(R \cup M\right) = 1 - 0.98 = 0.02
              \]

              So the probability that a student is proficient in neither reading nor mathematics is $2\%$.
          \end{enumerate}
      \end{enumerate}

    \item [$\S$ 2.2]
      \begin{enumerate}
        \item [6]
          Since we know nothing of the requirements for the positions,
          we assume each faculty member has an equally likely chance for each position,
          and also that each faculty member can hold only one position.

          We want to find

          \[
            \binom{10}{3} = \frac{10!}{3!(10 - 3)!} = \frac{10!}{3! \cdot 7!} = \frac{10 \cdot 9 \cdot 8}{3!} = \frac{10 \cdot 9 \cdot 8}{3 \cdot 2} = 10 \cdot 3 \cdot 4 = 120
          \]

          So there are 120 ways we can choose these positions.
        \item [7]
          Each True-False question has two possibilities, and there are 10 of those questions.
          Each multiple choice question has four possibilities, and there are five of those questions.

          This means there are $2^{10} \cdot 4^5 = 2^{10} \cdot (2^2)^5 = 2^{10} \cdot 2^{10} = 2^{20} = 1,048,576$ ways a student can fill out the test.
        \item [10]
          Assuming each employee is equally likely to work each shift,
          we want to find

          \begin{align*}
            \binom{15}{6}\binom{9}{5}\binom{4}{4} &= \frac{15!}{6!(15-6)!} \frac{9!}{5!(9-5)!} \frac{4!}{4!(4-4)!} \\
            &= \frac{15!}{6! \cdot 9!} \frac{9!}{5! \cdot 4!} \frac{4!}{4! \cdot 0!} \\
            &= \frac{15! \cdot 9! \cdot 4!}{6! \cdot 9! \cdot 5! \cdot 4! \cdot 4! \cdot 0!} \\
            &= \frac{15!}{6! \cdot 5! \cdot 4!} \\
            &= \frac{15 \cdot 14 \cdot 13 \cdot 12 \cdot 11 \cdot 10 \cdot 9 \cdot 8 \cdot 7}{5 \cdot 4 \cdot 3 \cdot 2 \cdot 4 \cdot 3 \cdot 2} \\
            &= \frac{14 \cdot 13 \cdot 11 \cdot 10 \cdot 9 \cdot 7}{2} \\
            &= 14 \cdot 13 \cdot 11 \cdot 9 \cdot 7 \cdot 5 \\
            &= 630630 \\
          \end{align*}

          So there are $630,630$ possible ways to make the shift assignment.
      \end{enumerate}

    \item [$\S$ 2.3]

      \begin{enumerate}
        \item [10]
          We have our probabilities

          $P\left(A \cap N^C\right) = 0.2, P\left(N \cap A^C\right) = 0.7, P\left(A \cap N\right) = 0.1$
          \begin{enumerate}[label=(\arabic*)]
            \item
              We want to find

              \[
                P\left(A\right) = P\left(A \cap N^C\right) + P\left(A \cap N\right) = 0.2 + 0.1 = 0.3
              \]

              So the probability that the allocation sector is damaged is $30\%$.
            \item
              We want to find

              \[
                P\left(N\right) = P\left(N \cap A^C\right) + P\left(A \cap N\right) = 0.7 + 0.1 = 0.8
              \]

              So the probability that a nonessential sector is damaged is $80\%$.
            \item
              We want to find

              \[
                P\left(N | A\right) = \frac{P\left(N \cap A\right)}{P\left(A\right)} = \frac{0.1}{0.3} = 0.\overline{3}
              \]

              So the probability that a nonessential sector is damaged given the drive has a damaged allocation sector is $33.\overline{3}\%$.
            \item
              We want to find

              \[
                P\left(A | N\right) = \frac{P\left(A \cap N\right)}{P\left(N\right)} = \frac{0.1}{0.8} = 0.125
              \]

              So the probability that an allocation sector is damaged given the drive has a damaged nonessential sector is $12.5\%$.
            \item
              We want to find

              \[
                P\left(N^C | A\right) = \frac{P\left(N^C \cap A\right)}{P\left(A\right)} = \frac{0.2}{0.3} = 0.\overline{6}
              \]

              So the probability that no nonessential sector is damaged given the drive has a damaged allocation sector is $66.\overline{6}\%$.
            \item
              We want to find

              \[
                P\left(A^C | N\right) = \frac{P\left(A^C \cap N\right)}{P\left(N\right)} = \frac{0.7}{0.8} = 0.875
              \]

              So the probability that no allocation sector is damaged given the drive has a damaged nonessential sector is $87.5\%$.
          \end{enumerate}
        \item [17]
          \begin{enumerate}[label=(\arabic*)]
            \item
              We want to find $P\left(G1\right)$.

              There are $56 + 24 = 80$ samples with gene 1 dominant.

              So the probability that a randomly sampled individual has gene 1 dominant is $\frac{80}{100} = 80\%$
            \item
              We want to find $P\left(G2\right)$.

              There are $56 + 14 = 70$ samples with gene 2 dominant.

              So the probability that a randomly sampled individual has gene 2 dominant is $\frac{70}{100} = 70\%$
            \item
              We need to know

              \[
                P\left(G1 \cap G2\right) = \frac{56}{100} = 0.56
              \]

              With $P\left(G1 \cap G2\right)$ we can find

              \[
                P\left(G2 | G1\right) = \frac{P\left(G2 \cap G1\right)}{P\left(G1\right)} = \frac{0.56}{0.8} = 0.7
              \]

              So the probability that gene 2 id dominant given that gene 1 is dominant is $70\%$.
            \item
              Since $P\left(G2 | G1\right) = P\left(G2\right)$ we know that these two events are independent of each other.

              So these genes are in linkage equilibrium.
          \end{enumerate}
        \item [38]
          \begin{proof}
            We need to prove each independence.
            \begin{itemize}
              \item
                $A^C$ is independent of $B$
                \begin{align*}
                  P\left(A^C \cap B\right) &= P\left(B\right) - P\left(A \cap B\right) \\
                  &= P\left(B\right) - P\left(B \cap A\right) \\
                  &= P\left(B\right) - P\left(B\right)P\left(A\right) \\
                  &= P\left(B\right) - P\left(B\right)\left(1 - P\left(A^C\right)\right) \\
                  &= P\left(B\right) - P\left(B\right) + P\left(B\right)P\left(A^C\right) \\
                  &= P\left(B\right)P\left(A^C\right) \\
                  &= P\left(A^C\right)P\left(B\right) \\
                \end{align*}
                Since $P\left(A^C \cap B\right) = P\left(A^C\right)P\left(B\right)$, $A^C$ is independent of $B$
              \item
                $A$ is independent of $B^C$
                \begin{align*}
                  P\left(A \cap B^C\right) &= P\left(A\right) - P\left(A \cap B\right) \\
                  &= P\left(A\right) - P\left(A \cap B\right) \\
                  &= P\left(A\right) - P\left(A\right)P\left(B\right) \\
                  &= P\left(A\right) - P\left(A\right)\left(1 - P\left(B^C\right)\right) \\
                  &= P\left(A\right) - P\left(A\right) + P\left(A\right)P\left(B^C\right) \\
                  &= P\left(A\right)P\left(B^C\right) \\
                \end{align*}
                Since $P\left(A \cap B^C\right) = P\left(A\right)P\left(B^C\right)$, $A$ is independent of $B^C$
              \item
                $A^C$ is independent of $B^C$
                \begin{align*}
                  P\left(A^C \cap B^C\right) &= P\left(\left(A \cup B\right)^C\right) \\
                  &= 1 - P\left(A \cup B\right) \\
                  &= 1 - \left(P\left(A\right) + P\left(B\right) - P\left(A \cap B\right)\right) \\
                  &= 1 - \left(P\left(A\right) + P\left(B\right) - P\left(A\right)P\left(B\right)\right) \\
                  &= 1 - \left(P\left(A\right) + \left(1 - P\left(A\right)\right)P\left(B\right)\right) \\
                  &= 1 - \left(P\left(A\right) + P\left(A^C\right)P\left(B\right)\right) \\
                  &= \left(1 - P\left(A\right)\right) - P\left(A^C\right)P\left(B\right) \\
                  &= P\left(A^C\right) - P\left(A^C\right)P\left(B\right) \\
                  &= P\left(A^C\right)\left(1 - P\left(B\right)\right) \\
                  &= P\left(A^C\right)P\left(B^C\right) \\
                \end{align*}
                Since $P\left(A^C \cap B^C\right) = P\left(A^C\right)P\left(B^C\right)$, $A^C$ is independent of $B^C$
            \end{itemize}

            From these three, we have shown that
            if $A$ and $B$ are independent events,
            then $A^C$ and $B$, $A$ and $B^C$, and $A^C$ and $B^C$ are also independent events.
          \end{proof}
      \end{enumerate}
  \end{enumerate}
\end{document}
