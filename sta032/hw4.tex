\documentclass[12pt,letterpaper]{article}
\usepackage{amsmath}
\usepackage{amsfonts}
\usepackage{amsthm}
\usepackage{mathtools}
\usepackage{cancel}
\usepackage[margin=1in]{geometry}
\usepackage{titling}
\usepackage{fp}
\usepackage{enumitem}
\usepackage[super]{nth}
\usepackage{dcolumn}
\usepackage{siunitx}
\usepackage{pgfplots}
\pgfplotsset{compat=1.8}
\usepgfplotslibrary{statistics}

\newcolumntype{d}{D{.}{.}{-1}}

\newcommand*\dif{\mathop{}\!\mathrm{d}}

\setlength{\droptitle}{-10ex}

\preauthor{\begin{flushright}\large \lineskip 0.5em}
\postauthor{\par\end{flushright}}
\predate{\begin{flushright}\large}
\postdate{\par\end{flushright}}

\title{STA 032 Homework 4\vspace{-2ex}}
\author{Hardy Jones\\
        999397426\\
        Professor Melcon\vspace{-2ex}}
\date{Winter 2015}

\begin{document}
  \maketitle

  \begin{enumerate}
    \item [$\S$ 4.1]
      \begin{enumerate}
        \item [2]
          \begin{enumerate}[label=(\arabic*)]
            \item $p_x = P(X = 1) = 0.20$
            \item $p_y = P(Y = 1) = 0.45$
            \item $p_z = P(X = 1 \cup Y = 1) = P(X = 1) + P(Y = 1) = 0.20 + 0.45 = 0.65$
            \item
              No, this is not possible.
              Each set is only one color, so $X$ and $Y$ are mutually exclusive.
            \item
              Yes, $p_z = p_x + p_y$.
            \item
              Yes.

              If a red set is chosen,

              then $X = 1$ and $Y = 0$ so $Z = 1 + 0 = 1 = X + Y$.

              If a white set is chosen,

              then $X = 0$ and $Y = 1$ so $Z = 0 + 1 = 1 = X + Y$.

              If a blue set is chosen,

              then $X = 0$ and $Y = 0$ so $Z = 0 + 0 = 0 = X + Y$.

              These are the only possible choices,
              so by enumeration, $Z = X + Y$.
          \end{enumerate}
      \end{enumerate}
    \item [$\S$ 4.2]
      \begin{enumerate}
        \item [8]
          We have $X \sim Bin(20, 0.2)$
          \begin{enumerate}[label=(\arabic*)]
            \item
              We want to find $P(X = 4)$.
              We can use Table A.1 and compute $f(4) - f(3) = 0.630 - 0.411 = 0.219$.

              So the probability that exactly four contracts have overruns is 0.219.
            \item
              We want to find $P(X < 3)$.

              Again, we use Table A.1 and find $f(2) = 0.206$

              So the probability that fewer than three contracts have overruns is 0.206.
            \item
              We want to find $P(X = 0)$.

              Again, we use Table A.1 and find $f(0) = 0.012$

              So the probability that none of the contracts have overruns is 0.012.
            \item
              $\mu_X = 20(0.2) = 4$.

              So the mean number of overruns is 4.
            \item
              $\sigma_X = \sqrt{20(0.2)(1 - 0.2)} = \sqrt{4(0.8)} = \sqrt{3.2} \approx 1.789$

              So the standard deviation of the number of overruns is 1.79.
          \end{enumerate}
        \item [11]
          We have $A \sim Bin(100, 04.12)$ and $B \sim Bin(200, 0.05)$
          \begin{enumerate}[label=(\arabic*)]
            \item

              $\hat{p}_A = \frac{12}{100} = 0.12$
              $\sigma_A = \sqrt{\frac{0.12(1 - 0.12)}{100}} \approx 0.03249$

              So the estimated proportion of defective parts is 0.12
              and uncertainty in this estimate is approximately 0.0325.
            \item

              $\hat{p}_B = \frac{10}{200} = 0.05$
              $\sigma_B = \sqrt{\frac{0.05(1 - 0.05)}{200}} \approx 0.01541$

              So the estimated proportion of defective parts is 0.05
              and uncertainty in this estimate is approximately 0.0154.
            \item

              The estimated difference is $|\hat{p}_A - \hat{p}_B| = |0.12 - 0.05| = |0.07| = 0.07$.
              The uncertainty in this difference is $\sqrt{\sigma_A^2 + \sigma_B^2} = \sqrt{0.03249^2 + 0.01541^2} \approx 0.0360$.
          \end{enumerate}
        \item [20]
          We have $X \sim Bin(8, 0.8)$
          \begin{enumerate}[label=(\arabic*)]
            \item
              We want to find $P(X \le 1)$.

              We use Table A.1 and find $f(1) = 0.000$.

              So the probability that no more than one policy holder in the sample has a smoke detector is 0.000.
            \item
              Yes, having exactly one policy holder in a sample size of 8 would be next to impossible.
            \item
              No, although the chances are small,
              it is still possible that the claim is true and
              the sample happened to choose mostly policy holders without a smoke detector.
            \item

              Again we turn to Table A.1 and find $f(6) = 0.497$.
            \item

              No, 6 in 8 has a probability of about 0.5.
              So it's a coin flip as to whether or not the sample would have six policy holders with smoke detectors.
          \end{enumerate}
      \end{enumerate}
    \item [$\S$ 4.3]
      \begin{enumerate}
        \item [7]
          \begin{enumerate}[label=(\arabic*)]
            \item
              We have $X \sim Poisson(4)$.

              So we solve $P(X = 5) = e^{-4}\frac{4^5}{5!} \approx 0.15629$

              Thus, the probability that 5 messages are given a minute is 0.156.
            \item
              We have $X \sim Poisson(4 \cdot 1.5) = Poisson(6)$.

              So we solve $P(X = 9) = e^{-6}\frac{6^9}{9!} \approx 0.068838$

              Thus, the probability that 9 messages are given in 1.5 minutes is 0.0688.
            \item
              We have $X \sim Poisson(4 \cdot 0.5) = Poisson(2)$.

              So we solve
              \begin{align*}
                P(X < 3) &= P(X = 0) + P(X = 1) + P(X = 2) \\
                &= e^{-2}\frac{2^0}{0!} + e^{-2}\frac{2^1}{1!} + e^{-2}\frac{2^2}{2!} \\
                &\approx 0.13533 + 0.27067 + 0.27067 \\
                &\approx 0.67667
              \end{align*}

              Thus, the probability that fewer than 3 messages are given in 30 seconds is 0.677.
          \end{enumerate}
        \item [8]
          \begin{enumerate}[label=(\arabic*)]
            \item
              We have $X \sim Poisson(4)$.

              So we solve $P(X = 3) = e^{-4}\frac{4^3}{3!} \approx 0.19536$

              Thus, the probabiltiy that 3 cars arrive in a given second is 0.195.
            \item
              We have $X \sim Poisson(4 \cdot 3) = Poisson(12)$.

              So we solve $P(X = 8) = e^{-12}\frac{12^8}{8!} \approx 0.065523$

              Thus, the probabiltiy that 8 cars arrive in a three seconds is 0.0655.
            \item
              We have $X \sim Poisson(4 \cdot 2) = Poisson(8)$.

              So we solve
              \begin{align*}
                P(X > 3) &= 1 - P(X \le 3) \\
                &= 1 - \left( P(X = 0) + P(X = 1) + P(X = 2) + P(X = 3) \right) \\
                &= 1 - \left( e^{-8}\frac{8^0}{0!} + e^{-8}\frac{8^1}{1!} + e^{-8}\frac{8^2}{2!} + e^{-8}\frac{8^3}{3!} \right) \\
                &\approx 1 - \left( 0.00033546 + 0.0026837 + 0.010735 + 0.028626 \right) \\
                &\approx 0.95761
              \end{align*}

              Thus, the probability that more than 3 cars arrive in 2 seconds is 0.958.
          \end{enumerate}
        \item [17]
          \begin{enumerate}[label=(\arabic*)]
            \item
              Since we have two samples,
              we compute $\lambda_M = \frac{14 + 11}{2} = \frac{25}{2} = 12.5$

              So we estimate Mom's cookies have a mean of 12.5 chips per cookie.
            \item
              Since we have two samples,
              we compute $\lambda_G = \frac{6 + 8}{2} = \frac{14}{2} = 7$

              So we estimate Grandma's cookies have a mean of 7 chips per cookie.
            \item
              We solve $\sigma_{\lambda_M} = \sqrt{\frac{12.5}{2}} = \sqrt{6.25} = 2.5$.

              So the uncertainty in our estimate of Mom's mean is 2.5 chips per cookie.
            \item
              We solve $\sigma_{\lambda_M} = \sqrt{\frac{7}{2}} = \sqrt{3.5} \approx 1.8708286933869707$.

              So the uncertainty in our estimate of Grandma's mean is approximately 1.87 chips per cookie.
            \item
              We solve $\lambda_{M - G} = \lambda_M - \lambda_G = 12.5 - 7 = 5.5$

              and $\sigma_{M - G} = \sqrt{\sigma_M^2 + \sigma_G^2} = \sqrt{6.25 + 3.5} = \sqrt{9.75} \approx 3.1225$.

              So on average,
              we estimate that Mom's cookies have 5.5 more chips than Grandma's
              with an uncertainty of 3.12 chips per cookie.
          \end{enumerate}
      \end{enumerate}
    \item [$\S$ 4.4]
      \begin{enumerate}
        \item [4]
          We have $X \sim Geom(0.4)$
          \begin{enumerate}[label=(\arabic*)]
            \item
              $P(X = 3) = 0.4(1 - 0.4)^{3 - 1} = 0.4(0.6)^2 = 0.144$.

              So the probability that a car goes three days without encountering a red light at the intersection is 0.144.
            \item
              \begin{align*}
                P(X \le 3) &= P(X = 0) + P(X = 1) + P(X = 2) + P(X = 3) \\
                &= 0 + 0.4(1 - 0.4)^{1 - 1} + 0.4(1 - 0.4)^{2 - 1} + 0.4(1 - 0.4)^{3 - 1} \\
                &= 0 + 0.4(0.6)^0+ 0.4(0.6)^1 + 0.4(0.6)^2 \\
                &= 0 + 0.4 + 0.24 + 0.144 \\
                &= 0.784
              \end{align*}

              So the probability that a car goes fewer than four days without encountering a red light at the intersection is 0.784.
            \item
              $\mu_X = \frac{1}{0.4} = 2.5$.

              So the mean number of days a car goes without encountering a red light at the intersection is 2.5.
            \item
              $\sigma_X^2 = \frac{1 - 0.4}{0.4^2} = \frac{0.6}{0.4^2} = 3.75$.

              So the variance is 3.75.
          \end{enumerate}
        \item [8]
          We have $X \sim Geom(0.01)$
          \begin{enumerate}[label=(\arabic*)]
            \item
              $\mu_X = \frac{1}{0.01} = 100$.

              So the mean number of packages that will be filled before the process is stopped is 100.
            \item
              $\sigma_X^2 = \frac{1 - 0.01}{0.01^2} = \frac{0.99}{0.01^2} = 9900$.

              So the variance is 9900.
            \item
              Now, we have $Y \sim NB(4, 0.01)$.

              $\mu_Y = 4 \cdot \mu_X = 4 \cdot 100 = 400$

              $\sigma_Y = 4 \cdot \mu_X = 4 \cdot 9900 = 39600$

              So the mean number of packages that will be filled before the process is stopped is 400,
              with a variance of 39600.
          \end{enumerate}
      \end{enumerate}
  \end{enumerate}
\end{document}
