\documentclass[12pt,letterpaper]{article}
\usepackage{amsmath}
\usepackage{amsfonts}
\usepackage{amsthm}
\usepackage{mathtools}
\usepackage{cancel}
\usepackage[margin=1in]{geometry}
\usepackage{titling}
\usepackage{fp}
\usepackage{enumitem}
\usepackage[super]{nth}
\usepackage{dcolumn}
\usepackage{siunitx}
\usepackage{pgfplots}
\pgfplotsset{compat=1.8}
\usepgfplotslibrary{statistics}

\newcolumntype{d}{D{.}{.}{-1}}

\newcommand*\dif{\mathop{}\!\mathrm{d}}

\setlength{\droptitle}{-10ex}

\preauthor{\begin{flushright}\large \lineskip 0.5em}
\postauthor{\par\end{flushright}}
\predate{\begin{flushright}\large}
\postdate{\par\end{flushright}}

\title{STA 032 Homework 4\vspace{-2ex}}
\author{Hardy Jones\\
        999397426\\
        Professor Melcon\vspace{-2ex}}
\date{Winter 2015}

\begin{document}
  \maketitle

  \begin{enumerate}
    \item [$\S$ 4.1]
      \begin{enumerate}
        \item [2]
          \begin{enumerate}[label=(\arabic*)]
            \item $p_x = P(X = 1) = 0.20$
            \item $p_y = P(Y = 1) = 0.45$
            \item $p_z = P(X = 1 \cup Y = 1) = P(X = 1) + P(Y = 1) = 0.20 + 0.45 = 0.65$
            \item
              No, this is not possible.
              Each set is only one color, so $X$ and $Y$ are mutually exclusive.
            \item
              Yes, $p_z = p_x + p_y$.
            \item
              Yes.

              If a red set is chosen,

              then $X = 1$ and $Y = 0$ so $Z = 1 + 0 = 1 = X + Y$.

              If a white set is chosen,

              then $X = 0$ and $Y = 1$ so $Z = 0 + 1 = 1 = X + Y$.

              If a blue set is chosen,

              then $X = 0$ and $Y = 0$ so $Z = 0 + 0 = 0 = X + Y$.

              These are the only possible choices,
              so by enumeration, $Z = X + Y$.
          \end{enumerate}
      \end{enumerate}
    \item [$\S$ 4.2]
      \begin{enumerate}
        \item [8]
          We have $X \sim Bin(20, 0.2)$
          \begin{enumerate}[label=(\arabic*)]
            \item
              We want to find $P(X = 4)$.
              We can use Table A.1 and compute $f(4) - f(3) = 0.630 - 0.411 = 0.219$.

              So the probability that exactly four contracts have overruns is 0.219.
            \item
              We want to find $P(X < 3)$.

              Again, we use Table A.1 and find $f(2) = 0.206$

              So the probability that fewer than three contracts have overruns is 0.206.
            \item
              We want to find $P(X = 0)$.

              Again, we use Table A.1 and find $f(0) = 0.012$

              So the probability that none of the contracts have overruns is 0.012.
            \item
              $\mu_X = 20(0.2) = 4$.

              So the mean number of overruns is 4.
            \item
              $\sigma_X = \sqrt{20(0.2)(1 - 0.2)} = \sqrt{4(0.8)} = \sqrt{3.2} \approx 1.789$

              So the standard deviation of the number of overruns is 1.79.
          \end{enumerate}
        \item [11]
          We have $A \sim Bin(100, 04.12)$ and $B \sim Bin(200, 0.05)$
          \begin{enumerate}[label=(\arabic*)]
            \item

              $\hat{p}_A = \frac{12}{100} = 0.12$
              $\sigma_A = \sqrt{\frac{0.12(1 - 0.12)}{100}} \approx 0.03249$

              So the estimated proportion of defective parts is 0.12
              and uncertainty in this estimate is approximately 0.0325.
            \item

              $\hat{p}_B = \frac{10}{200} = 0.05$
              $\sigma_B = \sqrt{\frac{0.05(1 - 0.05)}{200}} \approx 0.01541$

              So the estimated proportion of defective parts is 0.05
              and uncertainty in this estimate is approximately 0.0154.
            \item

              The estimated difference is $|\hat{p}_A - \hat{p}_B| = |0.12 - 0.05| = |0.07| = 0.07$.
              The uncertainty in this difference is $\sqrt{\sigma_A^2 + \sigma_B^2} = \sqrt{0.03249^2 + 0.01541^2} \approx 0.0360$.
          \end{enumerate}
        \item [20]
          We have $X \sim Bin(8, 0.8)$
          \begin{enumerate}[label=(\arabic*)]
            \item
              We want to find $P(X \le 1)$.

              We use Table A.1 and find $f(1) = 0.000$.

              So the probability that no more than one policy holder in the sample has a smoke detector is 0.000.
            \item
              Yes, having exactly one policy holder in a sample size of 8 would be next to impossible.
            \item
              No, although the chances are small,
              it is still possible that the claim is true and
              the sample happened to choose mostly policy holders without a smoke detector.
            \item

              Again we turn to Table A.1 and find $f(6) = 0.497$.
            \item

              No, 6 in 8 has a probability of about 0.5.
              So it's a coin flip as to whether or not the sample would have six policy holders with smoke detectors.
          \end{enumerate}
      \end{enumerate}
    \item [$\S$ 4.3]
      \begin{enumerate}
        \item [7]
          \begin{enumerate}[label=(\arabic*)]
            \item
            \item
            \item
          \end{enumerate}
        \item [8]
          \begin{enumerate}[label=(\arabic*)]
            \item
            \item
            \item
          \end{enumerate}
        \item [17]
          \begin{enumerate}[label=(\arabic*)]
            \item
            \item
            \item
            \item
            \item
          \end{enumerate}
      \end{enumerate}
    \item [$\S$ 4.4]
      \begin{enumerate}
        \item [4]
          \begin{enumerate}[label=(\arabic*)]
            \item
            \item
            \item
            \item
          \end{enumerate}
        \item [8]
          \begin{enumerate}[label=(\arabic*)]
            \item
            \item
            \item
          \end{enumerate}
      \end{enumerate}
  \end{enumerate}
\end{document}
