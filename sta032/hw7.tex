\documentclass[12pt,letterpaper]{article}
\usepackage{amsmath}
\usepackage{amsfonts}
\usepackage{amsthm}
\usepackage{mathtools}
\usepackage{cancel}
\usepackage[margin=1in]{geometry}
\usepackage{titling}
\usepackage{fp}
\usepackage{enumitem}
\usepackage[super]{nth}
\usepackage{dcolumn}
\usepackage{siunitx}
\usepackage{pgfplots}
\pgfplotsset{compat=1.8}
\usepgfplotslibrary{statistics}

\newcolumntype{d}{D{.}{.}{-1}}

\newcommand*\dif{\mathop{}\!\mathrm{d}}
\newcommand\numberthis{\refstepcounter{equation}\tag{\theequation}}

\setlength{\droptitle}{-10ex}

\preauthor{\begin{flushright}\large \lineskip 0.5em}
\postauthor{\par\end{flushright}}
\predate{\begin{flushright}\large}
\postdate{\par\end{flushright}}

\title{STA 032 Homework 7\vspace{-2ex}}
\author{Hardy Jones\\
        999397426\\
        Professor Melcon\vspace{-2ex}}
\date{Winter 2015}

\begin{document}
  \maketitle

  \begin{enumerate}
    \item [$\S$ 5.2]
      \begin{enumerate}
        \item [1]
          We have $n = 70$ independent Bernoulli trials with $X = 28$ successes.
          \begin{enumerate}[label=(\arabic*)]
            \item
              $\frac{28}{70} = 0.4 = 40\%$ of the sampled automobiles had emission levels that exceeded the standard.
            \item
              Taking $p = 0.4$ we have a $X \sim Bin(70, 0.4)$.

              We take $\tilde{n} = 70 + 4 = 74$,
              $\tilde{p} = \frac{28 + 2}{74} = 0.\overline{405}$.

              A 95\% confidence interval can now be found.

              \begin{align*}
                \tilde{p} \pm z_{\frac{\alpha}{2}} \sqrt{\frac{\tilde{p}(1 - \tilde{p})}{\tilde{n}}}
                &\approx 0.405 \pm z_{0.025} \sqrt{\frac{0.405(1 - 0.405)}{74}} \\
                &\approx 0.405 \pm 1.96 \sqrt{0.00326}\\
                &\approx 0.405 \pm 0.112
              \end{align*}

              So the interval is (0.293, 0.517).
            \item
              Using the values calculated before,
              a 98\% confidence interval can be found.

              \begin{align*}
                \tilde{p} \pm z_{\frac{\alpha}{2}} \sqrt{\frac{\tilde{p}(1 - \tilde{p})}{\tilde{n}}}
                &\approx 0.405 \pm z_{0.01} \sqrt{\frac{0.405(1 - 0.405)}{74}} \\
                &\approx 0.405 \pm 2.33 \sqrt{0.00326}\\
                &\approx 0.405 \pm 0.133
              \end{align*}

              So the interval is (0.272, 0.538).
            \item
              Since we're going to be solving more than one of these similar questions,
              let's find a closed form to calculate this easily.

              For each problem we need to know $\alpha, \tilde{p}$ and the range $r$.

              We want to solve the following equation for $n$:

              \begin{align*}
                r &= z_{\frac{\alpha}{2}} \sqrt{\frac{\tilde{p}(1 - \tilde{p})}{\tilde{n}}} \\
                &= z_{\frac{\alpha}{2}} \sqrt{\frac{\tilde{p}(1 - \tilde{p})}{n + 4}} \\
                \frac{r}{z_{\frac{\alpha}{2}}} &= \sqrt{\frac{\tilde{p}(1 - \tilde{p})}{n + 4}} \\
                \left(\frac{r}{z_{\frac{\alpha}{2}}}\right)^2 &= \frac{\tilde{p}(1 - \tilde{p})}{n + 4} \\
                n + 4 &= \frac{\tilde{p}(1 - \tilde{p})}{\left(\frac{r}{z_{\frac{\alpha}{2}}}\right)^2} \\
                n &= \frac{\tilde{p}(1 - \tilde{p})}{\left(\frac{r}{z_{\frac{\alpha}{2}}}\right)^2} - 4 \\
                n &= z_{\frac{\alpha}{2}}^2 \frac{\tilde{p}(1 - \tilde{p})}{r^2} - 4 \numberthis \label{eq:within} \\
              \end{align*}

              Now with equation \ref{eq:within} we can solve with plug and chug.

              \[
                n = 1.96^2 \frac{0.405(1 - 0.405)}{0.10^2} - 4 = 88.573
              \]

              So 89 samples are needed for the proportion to exceed the standard to within $\pm$ 0.10 with 95\% confidence.
            \item
              Using equation \ref{eq:within} we can solve with plug and chug.

              \[
                n = 2.33^2 \frac{0.405(1 - 0.405)}{0.10^2} - 4 = 126.823
              \]

              So 127 samples are needed for the proportion to exceed the standard to within $\pm$ 0.10 with 98\% confidence.
            \item
              We would do well to also find a closed form for this question.

              We need the $\tilde{n}, \tilde{p}$ and the upper bound $u$.

              Then we can solve the following equation for $z_\alpha$:

              \begin{align*}
                u &= \tilde{p} + z_\alpha \sqrt{\frac{\tilde{p}(1 - \tilde{p})}{\tilde{n}}} \\
                u - \tilde{p} &= z_\alpha \sqrt{\frac{\tilde{p}(1 - \tilde{p})}{\tilde{n}}} \\
                \frac{u - \tilde{p}}{\sqrt{\frac{\tilde{p}(1 - \tilde{p})}{\tilde{n}}}} &= z_\alpha \\
                z_\alpha &= \frac{u - \tilde{p}}{\sqrt{\frac{\tilde{p}(1 - \tilde{p})}{\tilde{n}}}} \\
                z_\alpha &= (u - \tilde{p})\sqrt{\frac{\tilde{n}}{\tilde{p}(1 - \tilde{p})}} \numberthis \label{eq:upper} \\
              \end{align*}

              Now, using equation \ref{eq:upper} we can solve for an upper bound of 0.50:

              \begin{align*}
                z_\alpha &= (0.50 - 0.405)\sqrt{\frac{74}{0.405(1 - 0.405)}} \\
                &= 1.66 \\
              \end{align*}

              So the z score corresponds to $0.9515 = 95.15\%$.

              Thus we can say with 95.15\% confidence that less than half of the vehicles in the state exceed the standard.
          \end{enumerate}
        \item [4]
          We have $n = 444$ independent Bernoulli trials with $X = 170$ successes.

          $\frac{170}{444} = 0.382\overline{882}$ of the sampled smokers used the patch.

          So we have $X \sim Bin(444, 0.383)$.

          We also calculate $\tilde{n} = 444 + 4 = 448$ and
          $\tilde{p} = \frac{170 + 2}{448} = 0.384$
          \begin{enumerate}[label=(\arabic*)]
            \item
              A 95\% confidence interval can be found:

              \begin{align*}
                \tilde{p} \pm z_{\frac{\alpha}{2}} \sqrt{\frac{\tilde{p}(1 - \tilde{p})}{\tilde{n}}}
                &= 0.384 \pm 1.96 \sqrt{\frac{0.384(1 - 0.384)}{448}} \\
                &= 0.384 \pm 0.0450 \\
              \end{align*}

              So the interval is (0.339, 0.429).
            \item
              A 98\% confidence interval can be found:

              \begin{align*}
                \tilde{p} \pm z_{\frac{\alpha}{2}} \sqrt{\frac{\tilde{p}(1 - \tilde{p})}{\tilde{n}}}
                &= 0.384 \pm 2.58 \sqrt{\frac{0.384(1 - 0.384)}{448}} \\
                &= 0.384 \pm 0.0592 \\
              \end{align*}

              So the interval is (0.325, 0.443).
            \item
              Using equation \ref{eq:upper} we can solve for an upper bound of 0.40:

              \begin{align*}
                z_\alpha &= (0.40 - 0.384)\sqrt{\frac{448}{0.384(1 - 0.384)}} \\
                &= 0.70 \\
              \end{align*}

              So the z score corresponds to $0.7580 = 75.80\%$.

              Thus we can say with 75.80\% confidence that proportion is less than 0.40.
            \item
              Using equation \ref{eq:within} we can solve with plug and chug.

              \[
                n = 1.96^2 \frac{0.384(1 - 0.384)}{0.03^2} - 4 = 1005.675
              \]

              So 1006 samples are needed for a 95\% confidence to specify the proportion within $\pm$ 0.03.
            \item
              Using equation \ref{eq:within} we can solve with plug and chug.

              \[
                n = 2.58^2 \frac{0.384(1 - 0.384)}{0.03^2} - 4 = 1749.479
              \]

              So 1750 samples are needed for a 99\% confidence to specify the proportion within $\pm$ 0.03.
          \end{enumerate}
      \end{enumerate}
    \item [$\S$ 5.3]
      \begin{enumerate}
        \item [10]
          We have $n = 15, \overline{X} = 13, s = 2$.

          We compute
          \[
            13 \pm 2.977 \frac{2}{\sqrt{15}} = (11.463, 14.537)
          \]

          So we can say with 99\% confidence
          that the mean track length is in the interval (11.463\si{\um}, 14.537\si{\um}).
        \item [11]
          We have $n = 6, \overline{X} = 2.03, s = 0.090$.

          We compute
          \[
            2.03 \pm 2.015 \frac{0.090}{\sqrt{6}} = (1.956, 2.104)
          \]

          So we can say with 90\% confidence
          that the mean deflection caused by a 160\si{\kN} load
          is in the interval (1.956\si{\mm}, 2.104\si{\mm}).
      \end{enumerate}
    \item [$\S$ 5.4]
      \begin{enumerate}
        \item [3]
        \item [4]
      \end{enumerate}
    \item [$\S$ 5.5]
      \begin{enumerate}
        \item [4]
        \item [5]
      \end{enumerate}
    \item [$\S$ 5.6]
      \begin{enumerate}
        \item [8]
        \item [9]
      \end{enumerate}
  \end{enumerate}
\end{document}
