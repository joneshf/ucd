\documentclass[12pt,letterpaper]{article}
\usepackage{amsmath}
\usepackage{amsfonts}
\usepackage{amsthm}
\usepackage{mathtools}
\usepackage{cancel}
\usepackage[margin=1in]{geometry}
\usepackage{titling}
\usepackage{fp}
\usepackage{enumitem}
\usepackage[super]{nth}
\usepackage{dcolumn}
\usepackage{siunitx}
\usepackage{pgfplots}
\pgfplotsset{compat=1.8}
\usepgfplotslibrary{statistics}

\newcolumntype{d}{D{.}{.}{-1}}

\newcommand*\dif{\mathop{}\!\mathrm{d}}

\setlength{\droptitle}{-10ex}

\preauthor{\begin{flushright}\large \lineskip 0.5em}
\postauthor{\par\end{flushright}}
\predate{\begin{flushright}\large}
\postdate{\par\end{flushright}}

\title{STA 032 Homework 5\vspace{-2ex}}
\author{Hardy Jones\\
        999397426\\
        Professor Melcon\vspace{-2ex}}
\date{Winter 2015}

\begin{document}
  \maketitle

  \begin{enumerate}
    \item [$\S$ 4.5]
      \begin{enumerate}
        \item [8]
          We have $F \sim N(4.1, 0.36)$
          \begin{enumerate}[label=(\arabic*)]
            \item
              We can find
              \begin{align*}
                P(3.7 < F < 4.4) &= P(F < 4.4) - P(F > 3.7) \\
                &= P\left(Z < \frac{4.4 - 4.1}{0.6}\right) -  P\left(Z < \frac{3.7 - 4.1}{0.6}\right) \\
                &= P\left(Z < \frac{0.3}{0.6}\right) -  P\left(Z < \frac{- 0.4}{0.6}\right) \\
                &= P\left(Z < \frac{1}{2}\right) -  P\left(Z < \frac{-2}{3}\right) \\
                &\approx 0.6915 -  0.2546 \\
                &\approx 0.4369 \\
              \end{align*}

              So approximately 43.69\% of the female cats are between 3.7 kg and 4.4 kg.
            \item
              We just need to find
              \[
                P(Z > 0.5) = 1 - P(Z < 0.5) \approx 1 - 0.6915 \approx 0.3085
              \]

              So approximately 30.85\% of the female cats are heavier than 0.5 standard deviations above the mean.
            \item
              We need to first find the closest area to 0.8.
              The closest area is 0.8023.
              This area corresponds to a z-score of 0.85.

              Now we just work backwards

              \begin{align*}
                0.85 &= \frac{x - 4.1}{0.6} \\
                0.85(0.6) + 4.1 &= x \\
                4.61 &= x \\
              \end{align*}

              So the female cat weight on the \nth{80} percentile is 4.61kg.
            \item
              We can find
              \begin{align*}
                P(F > 4.5) &= P\left(Z > \frac{4.5 - 4.1}{0.6}\right) \approx P(Z > 0.66) \approx P(Z < -0.66) \\
                &\approx 0.2546
              \end{align*}

              So the probability that a female cat chosen at random weighs more than 4.5kg is approximately 25.46\%.
            \item
              We assume that the sample is large enough that each choice is independent.

              We want to find
              \[
                P(F > 4.5) P(F < 4.5)^5 \approx 0.2546 (1 - 0.2546)^5 \approx 0.2546 (0.7475)^5 \approx 0.05858
              \]

              So the probability that exactly one of six females cats chosen at random weighs more than 4.5 kg is 5.86\%
          \end{enumerate}
        \item [20]
          We have $X \sim N(200, 100)$
          \begin{enumerate}[label=(\arabic*)]
            \item
              \[
                P(X \le 160) = P\left(Z < \frac{160 - 200}{10}\right) = P\left(Z < \frac{-40}{10}\right) = P(Z < -4)
              \]

              Since the table only goes to -3.69 all we can say is that $P(Z < -4) < 0.0001$
              So the probability of the strength being less than or equal to 160N is less than 0.01\%.
            \item
              Yes, a strength of 160N would be unusually small.

              Intuitively this makes sense, 160N is four standard deviations away from the the mean.
              99.7\% of the strengths are within three standard deviations,
              so it is natural to expect 160N to be unusually small.
            \item
              Yes, this would be enough evidence.

              The process is producing adhesive with strength far below the mean.
              If a measurement is 160N, something is messing the process up.
          \end{enumerate}
        \item [21]
          We have $R_1 \sim N(100, 25)$ and $R_2 \sim N(120, 100)$.

          It will help to compute $R_2 - R_1 \sim N(120 - 100, 100 + 25) = N(20, 125)$
          \begin{enumerate}[label=(\arabic*)]
            \item
              We need to solve $P(R_2 > R_1) = P(R_2 - R_1 > 0)$

              \begin{align*}
                P(R_2 - R_1 > 0) &= P\left(Z > \frac{0 - 20}{\sqrt{125}}\right) \\
                &\approx P\left(Z > -1.79\right) \\
                &\approx P\left(Z < 1.79\right) \\
                &\approx 0.9633 \\
              \end{align*}

              So the probability that $R_2$ exceeds $R_1$ is approximately 96.33\%.
            \item
              We need to solve $P(R_2 > R_1 + 30) = P(R_2 - R_1 > 30)$

              \begin{align*}
                P(R_2 - R_1 > 30) &= P\left(Z > \frac{30 - 20}{\sqrt{125}}\right) \\
                &\approx P\left(Z > 0.89\right) \\
                &\approx P\left(Z < -0.89\right) \\
                &\approx 0.1867 \\
              \end{align*}

              So the probability that $R_2$ exceeds $R_1$ by 30 $\Omega$ is approximately 18.67\%.
          \end{enumerate}
        \item [26]
          \begin{enumerate}
            \item $k = 1$

              \begin{align*}
                P(|X - \mu| \ge \sigma) &= P(X - \mu < -\sigma \cup X - \mu > \sigma) \\
                &= P(X - \mu < -\sigma) + P(X - \mu > \sigma) \\
                &= P(X - \mu < -\sigma) + P(X - \mu > \sigma) \\
                &= P(X < \mu - \sigma) + P(X > \mu + \sigma) \\
                &= P\left(Z < \frac{\mu - \sigma - \mu}{\sigma}\right) + P\left(Z > \frac{\mu + \sigma - \mu}{\sigma}\right) \\
                &= P(Z < -1) + P(Z > 1) \\
                &= P(Z < -1) + P(Z < -1) \\
                &= 2P(Z < -1) \\
                &= 2(0.1587) \\
                &= 0.3174 \\
              \end{align*}

              \[
                \frac{1}{1^2} = 1
              \]

            \item $k = 2$

              \begin{align*}
                P(|X - \mu| \ge 2 \sigma) &= P(X - \mu < -2 \sigma \cup X - \mu > 2 \sigma) \\
                &= P(X - \mu < -2 \sigma) + P(X - \mu > 2 \sigma) \\
                &= P(X - \mu < -2 \sigma) + P(X - \mu > 2 \sigma) \\
                &= P(X < \mu - 2 \sigma) + P(X > \mu + 2 \sigma) \\
                &= P\left(Z < \frac{\mu - 2 \sigma - \mu}{\sigma}\right) + P\left(Z > \frac{\mu + 2 \sigma - \mu}{\sigma}\right) \\
                &= P(Z < -2) + P(Z > 2) \\
                &= P(Z < -2) + P(Z < -2) \\
                &= 2P(Z < -2) \\
                &= 2(0.0228) \\
                &= 0.0456 \\
              \end{align*}

              \[
                \frac{1}{2^2} = 0.25
              \]

            \item $k = 3$

              \begin{align*}
                P(|X - \mu| \ge 3 \sigma) &= P(X - \mu < -3 \sigma \cup X - \mu > 3 \sigma) \\
                &= P(X - \mu < -3 \sigma) + P(X - \mu > 3 \sigma) \\
                &= P(X - \mu < -3 \sigma) + P(X - \mu > 3 \sigma) \\
                &= P(X < \mu - 3 \sigma) + P(X > \mu + 3 \sigma) \\
                &= P\left(Z < \frac{\mu - 3 \sigma - \mu}{\sigma}\right) + P\left(Z > \frac{\mu + 3 \sigma - \mu}{\sigma}\right) \\
                &= P(Z < -3) + P(Z > 3) \\
                &= P(Z < -3) + P(Z < -3) \\
                &= 2P(Z < -3) \\
                &= 2(0.0013) \\
                &= 0.0026 \\
              \end{align*}

              \[
                \frac{1}{3^2} = 0.\overline{1}
              \]
          \end{enumerate}

          So the actual probabilities are much smaller than Chebyshev's bound.
      \end{enumerate}

    \item [$\S$ 4.7]
      \begin{enumerate}
        \item [2]
          We have $Req \sim Exp(\lambda)$ and $\mu_{Req} = 0.5$
          \begin{enumerate}[label=(\arabic*)]
            \item
              \[
                \lambda = \frac{1}{\mu_{Req}} = 2
              \]
            \item
              We want to find
              \begin{align*}
                P(Req < x_{50}) &= 0.5 \\
                1 - e^{-2x_{50}} &= \\
                0.5 &= e^{-2x_{50}} \\
                \ln 2 &= 2x_{50} \\
                \frac{\ln 2}{2} &= x_{50} \\
                0.3466 &\approx
              \end{align*}

              So the median time between requests is approximately 0.3466 seconds.
            \item
              The standard deviation is the same as the mean, 0.5 seconds.
            \item
              We want to find
              \begin{align*}
                P(Req < x_{80}) &= 0.8 \\
                1 - e^{-2x_{80}} &= \\
                0.2 &= e^{-2x_{80}} \\
                1.6094 &\approx 2x_{80} \\
                \frac{1.6094}{2} &\approx x_{80} \\
                0.8047 &\approx
              \end{align*}

              So the \nth{80} percentile time between requests is approximately 0.8047 seconds.
            \item
              We want to find
              \[
                P(Req > 1) = e^{-2(1)} \approx 0.1353
              \]

              So the probability of more than one second elapsing between requests is 13.53\%.
            \item
              Since exponential distributions have no concept of memory,
              we have the same result as the previous question.

              If two seconds have elapsed since the last request,
              the probability of another second elapsing before the next request is 13.53\%.
          \end{enumerate}
        \item [4]
          We have $D \sim Exp(\lambda)$ and $\mu_D = 12$.

          Thus, $\lambda = \frac{1}{12} = 0.08\overline{3}$
          \begin{enumerate}[label=(\arabic*)]
            \item
              We want to find

              \[
                P(D > 15) = e^{-\frac{15}{12}} \approx 0.2865
              \]

              So the probability that the distance between two flaws is greater than 15m is approximately 28.65\%.
            \item
              We want to find

              \[
                P(8 < D < 20) = P(D < 20) - P(D > 8) = 1 - e^{-\frac{20}{12}} - e^{-\frac{8}{12}} \approx 0.2977
              \]

              So the probability that the distance between two flaws is between 8m and 20m is approximately 29.77\%.
            \item
              We want to find
              \begin{align*}
                P(D < x_{50}) &= 0.5 \\
                1 - e^{-\frac{x_{50}}{12}} &= \\
                0.5 &= e^{-\frac{x_{50}}{12}} \\
                \ln 2 &= \frac{x_{50}}{12} \\
                12 \ln 2 &= x_{50} \\
                8.3178 &\approx
              \end{align*}

              So the median distance between flaws is approximately 8.3178m.
            \item
              The standard deviation is the same as the mean 12m.
            \item
              We want to find
              \begin{align*}
                P(Req < x_{65}) &= 0.65 \\
                1 - e^{-\frac{x_{65}}{12}} &= \\
                0.35 &= e^{-\frac{x_{65}}{12}} \\
                1.0498 &\approx \frac{x_{65}}{12} \\
                12.5979 &\approx x_{65} \\
              \end{align*}

              So the \nth{65} percentile distance between flaws is approximately 12.5979m.
          \end{enumerate}
      \end{enumerate}

    \item [$\S$ 4.8]
      \begin{enumerate}
        \item [1]
          We have $W \sim U(0, 15)$
          \begin{enumerate}[label=(\arabic*)]
            \item
              The mean wait time is $\frac{0 + 15}{2} = 7.5$ minutes.
            \item
              The standard deviation is $\frac{15 - 0}{\sqrt{12}} \approx 4.3301$ minutes.
            \item
              We want to find

              \[
                P(5 < W < 11) = \frac{11 - 5}{15 - 0} = 0.4
              \]

              So the probability that the wait time is between 5 and 11 minutes is 40\%.
            \item
              We want to find
              \begin{align*}
                P(W < 5)^4 P(W > 5)^6 &= \left(\frac{5 - 0}{15 - 0}\right)^4 \left(\frac{15 - 5}{15 - 0}\right)^6 = \left(\frac{1}{81}\right) \left(\frac{64}{729}\right) \\
                &\approx 0.00108382
              \end{align*}

              So the probability that the wait time is less than 5 minutes on exactly 4 of 10 days is approximately 0.1084\%.
          \end{enumerate}
        \item [2]
          \begin{enumerate}[label=(\arabic*)]
            \item
            \item
            \item
            \item
          \end{enumerate}
      \end{enumerate}

  \end{enumerate}
\end{document}
