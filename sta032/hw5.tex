\documentclass[12pt,letterpaper]{article}
\usepackage{amsmath}
\usepackage{amsfonts}
\usepackage{amsthm}
\usepackage{mathtools}
\usepackage{cancel}
\usepackage[margin=1in]{geometry}
\usepackage{titling}
\usepackage{fp}
\usepackage{enumitem}
\usepackage[super]{nth}
\usepackage{dcolumn}
\usepackage{siunitx}
\usepackage{pgfplots}
\pgfplotsset{compat=1.8}
\usepgfplotslibrary{statistics}

\newcolumntype{d}{D{.}{.}{-1}}

\newcommand*\dif{\mathop{}\!\mathrm{d}}

\setlength{\droptitle}{-10ex}

\preauthor{\begin{flushright}\large \lineskip 0.5em}
\postauthor{\par\end{flushright}}
\predate{\begin{flushright}\large}
\postdate{\par\end{flushright}}

\title{STA 032 Homework 5\vspace{-2ex}}
\author{Hardy Jones\\
        999397426\\
        Professor Melcon\vspace{-2ex}}
\date{Winter 2015}

\begin{document}
  \maketitle

  \begin{enumerate}
    \item [$\S$ 4.5]
      \begin{enumerate}
        \item [8]
          We have $F \sim N(4.1, 0.36)$
          \begin{enumerate}[label=(\arabic*)]
            \item
              We can find
              \begin{align*}
                P(3.7 < F < 4.4) &= P(F < 4.4) - P(F > 3.7) \\
                &= P\left(Z < \frac{4.4 - 4.1}{0.6}\right) -  P\left(Z < \frac{3.7 - 4.1}{0.6}\right) \\
                &= P\left(Z < \frac{0.3}{0.6}\right) -  P\left(Z < \frac{- 0.4}{0.6}\right) \\
                &= P\left(Z < \frac{1}{2}\right) -  P\left(Z < \frac{-2}{3}\right) \\
                &\approx 0.6915 -  0.2546 \\
                &\approx 0.4369 \\
              \end{align*}

              So approximately 43.69\% of the female cats are between 3.7 kg and 4.4 kg.
            \item
              We just need to find
              \[
                P(Z > 0.5) = 1 - P(Z < 0.5) \approx 1 - 0.6915 \approx 0.3085
              \]

              So approximately 30.85\% of the female cats are heavier than 0.5 standard deviations above the mean.
            \item
              We need to first find the closest area to 0.8.
              The closest area is 0.8023.
              This area corresponds to a z-score of 0.85.

              Now we just work backwards

              \begin{align*}
                0.85 &= \frac{x - 4.1}{0.6} \\
                0.85(0.6) + 4.1 &= x \\
                4.61 &= x \\
              \end{align*}

              So the female cat weight on the \nth{80} percentile is 4.61kg.
            \item
              We can find
              \[
                P(F > 4.5) = P\left(Z > \frac{4.5 - 4.1}{0.6}\right) \approx P(Z > 0.66) \approx P(Z < -0.66) \approx 0.2546
              \]

              So the probability that a female cat chosen at random weighs more than 4.5kg is approximately 25.46\%.
            \item
              We assume that the sample is large enough that each choice is independent.

              We want to find
              \[
                P(F > 4.5) P(F < 4.5)^5 \approx 0.2546 (1 - 0.2546)^5 \approx 0.2546 (0.7475)^5 \approx 0.05858
              \]

              So the probability that exactly one of six females cats chosen at random weighs more than 4.5 kg is 5.86\%
          \end{enumerate}
        \item [20]
          \begin{enumerate}[label=(\arabic*)]
            \item
            \item
            \item
          \end{enumerate}
        \item [21]
          \begin{enumerate}[label=(\arabic*)]
            \item
            \item
          \end{enumerate}
        \item [26]
      \end{enumerate}

    \item [$\S$ 4.7]
      \begin{enumerate}
        \item [2]
          \begin{enumerate}[label=(\arabic*)]
            \item
            \item
            \item
            \item
            \item
            \item
          \end{enumerate}
        \item [4]
          \begin{enumerate}[label=(\arabic*)]
            \item
            \item
            \item
            \item
            \item
          \end{enumerate}
      \end{enumerate}

    \item [$\S$ 4.8]
      \begin{enumerate}
        \item [1]
          \begin{enumerate}[label=(\arabic*)]
            \item
            \item
            \item
            \item
          \end{enumerate}
        \item [2]
          \begin{enumerate}[label=(\arabic*)]
            \item
            \item
            \item
            \item
          \end{enumerate}
      \end{enumerate}

  \end{enumerate}
\end{document}
