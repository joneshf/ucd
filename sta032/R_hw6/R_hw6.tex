\documentclass[12pt,letterpaper]{article}
\usepackage{amsmath}
\usepackage{amsfonts}
\usepackage{amsthm}
\usepackage{mathtools}
\usepackage{cancel}
\usepackage[margin=1in]{geometry}
\usepackage{titling}
\usepackage{fp}
\usepackage{enumitem}
\usepackage[super]{nth}
\usepackage{dcolumn}
\usepackage{minted}
\usepackage[title]{appendix}
\usepackage{pgfplots}
\pgfplotsset{compat=1.8}
\usepgfplotslibrary{statistics}
\usepackage[round-mode=figures,round-precision=3,scientific-notation=false]{siunitx}

\newcolumntype{d}{D{.}{.}{-1}}

\setlength{\droptitle}{-10ex}

\preauthor{\begin{flushright}\large \lineskip 0.5em}
\postauthor{\par\end{flushright}}
\predate{\begin{flushright}\large}
\postdate{\par\end{flushright}}

\title{STA 032 R Homework 6\vspace{-2ex}}
\author{Hardy Jones\\
        999397426\\
        Professor Melcon\vspace{-2ex}}
\date{Winter 2015}

\begin{document}
  \maketitle

  \newmintedfile[rData]{r}{ fontsize=\footnotesize
                          , frame=single
                          }

  \begin{enumerate}
    \item
      Using the scheme given in the assignment,

      $
        \begin{pmatrix}
          \overline{X} - Z_{\frac{\alpha}{2}} \frac{s}{\sqrt{n}}
          & \overline{X} + Z_{\frac{\alpha}{2}} \frac{s}{\sqrt{n}} \\
          \overline{X} - t_{n-1;\frac{\alpha}{2}} \frac{s}{\sqrt{n}}
          & \overline{X} + t_{n-1;\frac{\alpha}{2}} \frac{s}{\sqrt{n}} \\
        \end{pmatrix}
      $
      \begin{enumerate}
        \item
          $
            \begin{pmatrix}
              \num{3.720728} & \num{7.519790} \\
              \num{3.425003} & \num{7.815515} \\
            \end{pmatrix}
          $
        \item
          $
            \begin{pmatrix}
              \num{2.339011} & \num{8.127328} \\
              \num{1.888439} & \num{8.577900} \\
            \end{pmatrix}
          $
        \item
          $
            \begin{pmatrix}
              \num{3.474872} & \num{5.791677} \\
              \num{3.428046} & \num{5.838504} \\
            \end{pmatrix}
          $
        \item
          $
            \begin{pmatrix}
              \num{2.699661} & \num{4.756874} \\
              \num{2.658082} & \num{4.798454} \\
            \end{pmatrix}
          $
      \end{enumerate}
    \item
      \begin{enumerate}
        \item
          The proportion of \texttt{N} confidence intervals that cover the true mean for $Z_{\frac{\alpha}{2}}$ is \num{0.93013}.

          The proportion of \texttt{N} confidence intervals that cover the true mean for $t_{n - 1; \frac{\alpha}{2}}$ is \num{0.95069}.
        \item
          The proportion of \texttt{N} confidence intervals that cover the true mean for $Z_{\frac{\alpha}{2}}$ is \num{0.89294}.

          The proportion of \texttt{N} confidence intervals that cover the true mean for $t_{n - 1; \frac{\alpha}{2}}$ is \num{0.91292}.
        \item
          The proportion of \texttt{N} confidence intervals that cover the true mean for $Z_{\frac{\alpha}{2}}$ is \num{0.94434}.

          The proportion of \texttt{N} confidence intervals that cover the true mean for $t_{n - 1; \frac{\alpha}{2}}$ is \num{0.94995}.
        \item
          The proportion of \texttt{N} confidence intervals that cover the true mean for $Z_{\frac{\alpha}{2}}$ is \num{0.93079}.

          The proportion of \texttt{N} confidence intervals that cover the true mean for $t_{n - 1; \frac{\alpha}{2}}$ is \num{0.93649}.
        \item
          The $t_{n - 1; \frac{\alpha}{2}}$ coverage is better than the $Z_{\frac{\alpha}{2}}$ coverage.
          This is because the sample size of each simulation was smaller than 30.
        \item
          The $t_{n - 1; \frac{\alpha}{2}}$ coverage is better than the $Z_{\frac{\alpha}{2}}$ coverage.
          This again is because the sample size of each simulation was smaller than 30.
        \item
          The coverages reported in part (a) were higher than the coverages in part (b).
          This is probably due to the fact that the population in part (a) was normally distibuted,
          while the population in part (b) was exponentially distributed.
        \item
          The coverages reported in part (c) were again slightly higher than the coverages in part (d).
          However, the difference is less drastic between the two coverages.
          The Central Limit Theorem is starting to take effect,
          so it matters less that the population in part (d) was exponentially distributed.
      \end{enumerate}
  \end{enumerate}

  \begin{appendices}
    \section{R code}

        \subsection*{Problem 1}
            \rData{prob1.R}
            \subsubsection*{(a)}
                \rData{prob1a.R}
            \subsubsection*{(b)}
                \rData{prob1b.R}

        \subsection*{Problem 2}
            \rData{prob2.R}
            \subsubsection*{(a)}
                \rData{prob2a.R}
            \subsubsection*{(b)}
                \rData{prob2b.R}

        \subsection*{Problem 3}
            \rData{prob3.R}
            \subsubsection*{(a)}
                \rData{prob3a.R}
            \subsubsection*{(b)}
                \rData{prob3b.R}

\end{appendices}


\end{document}
