\documentclass[12pt,letterpaper]{article}
\usepackage{amsmath}
\usepackage{amsfonts}
\usepackage{amsthm}
\usepackage{mathtools}
\usepackage{cancel}
\usepackage[margin=1in]{geometry}
\usepackage{titling}
\usepackage{fp}
\usepackage{enumitem}

% From https://code.google.com/p/linear-algebra/source/browse/linalgjh.sty#80
% Using brackets instead of parens.

%-------------bmat
% For matrices with arguments.
% Usage: \begin{bmat}{c|c|c} 1 &2 &3 \end{bmat}
\newenvironment{bmat}[1]{
  \left[\begin{array}{@{}#1@{}}
}{\end{array}\right]
}

\setlength{\droptitle}{-10ex}

\preauthor{\begin{flushright}\large \lineskip 0.5em}
\postauthor{\par\end{flushright}}
\predate{\begin{flushright}\large}
\postdate{\par\end{flushright}}

\title{STA 032 Homework 1\vspace{-2ex}}
\author{Hardy Jones\\
        999397426\\
        Professor Melcon\vspace{-2ex}}
\date{Fall 2014}

\begin{document}
  \maketitle

  \begin{enumerate}
    \item [$\S$ 1.1]
      \begin{enumerate}
        \item [3]
          \begin{enumerate}
            \item False
            \item True
          \end{enumerate}
        \item [8]
          \begin{enumerate}
            \item This interview is an observational study.
            \item
              No, the conclusion is not well-justified given the information presented.
              It does not state how many were in each category.
              If there was only one person in the low exercise category,
              then that category is underrepresented.
          \end{enumerate}
      \end{enumerate}

    \item [$\S$ 1.2]
      \begin{enumerate}
        \item [4]
          No, the sample median can differ from one of the values in the sample.
          For instance, the sample median of the following data,
          $D = \{1, 9\}$
          is $\frac{1 + 9}{2} = \frac{10}{2} = 5 \notin D$.
        \item [6]
          Yes, it is possible for the standard deviation to be more than the mean for positive data.
          For instance:

          \begin{tabular}{c | c | c}
            $x$        & $\overline{x}$        & $\sigma$ \\
            \hline
            $\{1, 9\}$ & $\frac{1 + 9}{2} = 5$ & $\sqrt{\frac{(1-5)^2 + (9 - 5)^2}{1}} = \sqrt{(-4)^2 + 4^2} = \sqrt{16 + 16} = \sqrt{32}$
          \end{tabular}

          And $\sqrt{32} > 5$.
        \item [10]
          \begin{enumerate}[label=\arabic* )]
            \item
              \begin{align*}
                \overline{x} &= \frac{0(27) + 1(22) + 2(30) + 3(12) + 4(7) + 5(2)}{27 + 22 + 30 + 12 + 7 + 2} \\
                &= \frac{22 + 60 + 36 + 28 + 10}{100} \\
                &= \frac{156}{100} \\
                &= 1.56
              \end{align*}

              So of the sample, the mean number of children had was $1.56$.
          \end{enumerate}
        \item [14]
      \end{enumerate}
  \end{enumerate}
\end{document}
