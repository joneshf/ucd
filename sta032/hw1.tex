\documentclass[12pt,letterpaper]{article}
\usepackage{amsmath}
\usepackage{amsfonts}
\usepackage{amsthm}
\usepackage{mathtools}
\usepackage{cancel}
\usepackage[margin=1in]{geometry}
\usepackage{titling}
\usepackage{fp}
\usepackage{enumitem}
\usepackage[super]{nth}

% From https://code.google.com/p/linear-algebra/source/browse/linalgjh.sty#80
% Using brackets instead of parens.

%-------------bmat
% For matrices with arguments.
% Usage: \begin{bmat}{c|c|c} 1 &2 &3 \end{bmat}
\newenvironment{bmat}[1]{
  \left[\begin{array}{@{}#1@{}}
}{\end{array}\right]
}

\setlength{\droptitle}{-10ex}

\preauthor{\begin{flushright}\large \lineskip 0.5em}
\postauthor{\par\end{flushright}}
\predate{\begin{flushright}\large}
\postdate{\par\end{flushright}}

\title{STA 032 Homework 1\vspace{-2ex}}
\author{Hardy Jones\\
        999397426\\
        Professor Melcon\vspace{-2ex}}
\date{Fall 2014}

\begin{document}
  \maketitle

  \begin{enumerate}
    \item [$\S$ 1.1]
      \begin{enumerate}
        \item [3]
          \begin{enumerate}
            \item False
            \item True
          \end{enumerate}
        \item [8]
          \begin{enumerate}
            \item This interview is an observational study.
            \item
              No, the conclusion is not well-justified given the information presented.
              It does not state how many were in each category.
              If there was only one person in the low exercise category,
              then that category is underrepresented.
          \end{enumerate}
      \end{enumerate}

    \item [$\S$ 1.2]
      \begin{enumerate}
        \item [4]
          No, the sample median can differ from one of the values in the sample.
          For instance, the sample median of the following data,
          $D = \{1, 9\}$
          is $\frac{1 + 9}{2} = \frac{10}{2} = 5 \notin D$.
        \item [6]
          Yes, it is possible for the standard deviation to be more than the mean for positive data.
          For instance:

          \begin{tabular}{c | c | c}
            $x$        & $\overline{x}$        & $\sigma$ \\
            \hline
            $\{1, 9\}$ & $\frac{1 + 9}{2} = 5$ & $\sqrt{\frac{(1-5)^2 + (9 - 5)^2}{1}} = \sqrt{(-4)^2 + 4^2} = \sqrt{16 + 16} = \sqrt{32}$
          \end{tabular}

          And $\sqrt{32} > 5$.
        \item [10]
          \begin{enumerate}[label=\arabic* )]
            \item
              \begin{align*}
                \overline{x} &= \frac{0(27) + 1(22) + 2(30) + 3(12) + 4(7) + 5(2)}{27 + 22 + 30 + 12 + 7 + 2} \\
                &= \frac{22 + 60 + 36 + 28 + 10}{100} \\
                &= \frac{156}{100} \\
                &= 1.56
              \end{align*}

              So of the sample, the mean number of children had was $1.56$.

            \item
              \begin{align*}
                \sigma &= \sqrt{\frac{27(-1.56)^2 + 22(-0.56)^2 + 30(0.44)^2 + 12(1.44)^2 + 7(2.44)^2 + 2(3.44)^2}{99}} \\
                &=\sqrt{\frac{27(2.4336) + 22(0.3136) + 30(0.1936) + 12(2.0736) + 7(5.9536) + 2(11.8336)}{99}} \\
                &=\sqrt{\frac{65.7072 + 6.8992 + 5.8080 + 24.8832 + 41.6752 + 23.6672}{99}} \\
                &=\sqrt{\frac{168.64}{99}} \\
                &=\sqrt{1.70\overline{34}} \\
                &\approx 1.31 \\
              \end{align*}

              So of the sample, the standard deviation is $\approx 1.31$

            \item
              Since there are 100 samples,
              we want to find the average of the \nth{50} and \nth{51} samples in ascending order.
              There are 27 samples of value 0 and 22 samples of value 1.
              Since there are 30 samples of value 2,
              the \nth{50} and \nth{51} samples are both 2.

              So the median value is $\frac{2 + 2}{2} = 2$

            \item

              Since there are 100 samples,
              the first quartile is the \nth{25} sample in ascending order.
              As there are 27 samples of value 0,
              the first quartile is 0.

            \item
              The mean in $1.56$,
              so the number of women that had more children than the mean
              is the total of samples with values 2, 3, 4, and 5 = $30 + 12 + 7 + 2 = 51$.

              So 51 of the 100 women had more children than the mean.

            \item
              The standard deviation is $\approx 1.31$,
              so we want all samples where the value is greater than the mean plus the standard deviation.
              In other words we want all samples with value $> 1.56 + 1.31 = 2.87$.
              We want to total the number of samples with values 3, 4, and 5 = $12 + 7 + 2 = 21$

              So 21 of the 100 women had more than one standard deviation's worth more children.

            \item
              WAT!!
          \end{enumerate}
        \item [14]
          \begin{enumerate}[label=\arabic* )]
            \item
              We can calculate the total from the information given,
              and use that total to calculate the change.

              \begin{align*}
                \text{total} &= \$70,000 \cdot 10 = \$700,000 \\
                \text{new total} &= \frac{\text{total} - \$100,000 + \$1,000,000}{10} \\
                &= \frac{\text{total} + \$900,000}{10} \\
                &= \frac{\$700,000 + \$900,000}{10} \\
                &= \frac{\$16,000,000}{10} \\
                &= \$1,600,000 \\
              \end{align*}

              The new mean is \$1,600,000.

            \item
              Since only the largest datum was changed,
              the value of the median does not change.

              The median is still \$55,000.
          \end{enumerate}
      \end{enumerate}
  \end{enumerate}
\end{document}
