\documentclass[12pt,letterpaper]{article}
\usepackage{amsmath}
\usepackage{amsfonts}
\usepackage{amsthm}
\usepackage{mathtools}
\usepackage{cancel}
\usepackage[margin=1in]{geometry}
\usepackage{titling}
\usepackage{fp}
\usepackage{enumitem}
\usepackage[super]{nth}
\usepackage{pgfplots}
\pgfplotsset{compat=1.8}
\usepgfplotslibrary{statistics}

\setlength{\droptitle}{-10ex}

\preauthor{\begin{flushright}\large \lineskip 0.5em}
\postauthor{\par\end{flushright}}
\predate{\begin{flushright}\large}
\postdate{\par\end{flushright}}

\title{STA 032 Homework 1\vspace{-2ex}}
\author{Hardy Jones\\
        999397426\\
        Professor Melcon\vspace{-2ex}}
\date{Winter 2015}

\begin{document}
  \maketitle

  \begin{enumerate}
    \item [$\S$ 1.1]
      \begin{enumerate}
        \item [3]
          \begin{enumerate}[label=(\arabic*)]
            \item False
            \item True
          \end{enumerate}
        \item [8]
          \begin{enumerate}[label=(\arabic*)]
            \item This interview is an observational study.
            \item
              No, the conclusion is not well-justified given the information presented.
              It does not state how many were in each category.
              If there was only one person in the low exercise category,
              then that category is underrepresented.
          \end{enumerate}
      \end{enumerate}

    \item [$\S$ 1.2]
      \begin{enumerate}
        \item [4]
          No, the sample median can differ from one of the values in the sample.
          For instance, the sample median of the following data,
          $D = \{1, 9\}$
          is $\frac{1 + 9}{2} = \frac{10}{2} = 5 \notin D$.
        \item [6]
          Yes, it is possible for the standard deviation to be more than the mean for positive data.
          For instance:

          \begin{tabular}{c | c | c}
            $x$        & $\overline{x}$        & $\sigma$ \\
            \hline
            $\{1, 9\}$ & $\frac{1 + 9}{2} = 5$ & $\sqrt{\frac{(1-5)^2 + (9 - 5)^2}{1}} = \sqrt{(-4)^2 + 4^2} = \sqrt{16 + 16} = \sqrt{32}$
          \end{tabular}

          And $\sqrt{32} > 5$.
        \item [10]
          \begin{enumerate}[label=(\arabic*)]
            \item
              \begin{align*}
                \overline{x} &= \frac{0(27) + 1(22) + 2(30) + 3(12) + 4(7) + 5(2)}{27 + 22 + 30 + 12 + 7 + 2} \\
                &= \frac{22 + 60 + 36 + 28 + 10}{100} \\
                &= \frac{156}{100} \\
                &= 1.56
              \end{align*}

              So of the sample, the mean number of children had was $1.56$.

            \item
              \begin{align*}
                \sigma &= \sqrt{\frac{27(-1.56)^2 + 22(-0.56)^2 + 30(0.44)^2 + 12(1.44)^2 + 7(2.44)^2 + 2(3.44)^2}{99}} \\
                &=\sqrt{\frac{27(2.4336) + 22(0.3136) + 30(0.1936) + 12(2.0736) + 7(5.9536) + 2(11.8336)}{99}} \\
                &=\sqrt{\frac{65.7072 + 6.8992 + 5.8080 + 24.8832 + 41.6752 + 23.6672}{99}} \\
                &=\sqrt{\frac{168.64}{99}} \\
                &=\sqrt{1.70\overline{34}} \\
                &\approx 1.31 \\
              \end{align*}

              So of the sample, the standard deviation is $\approx 1.31$

            \item
              Since there are 100 samples,
              we want to find the average of the \nth{50} and \nth{51} samples in ascending order.
              There are 27 samples of value 0 and 22 samples of value 1.
              Since there are 30 samples of value 2,
              the \nth{50} and \nth{51} samples are both 2.

              So the median value is $\frac{2 + 2}{2} = 2$

            \item

              Since there are 100 samples,
              the first quartile is the \nth{25} sample in ascending order.
              As there are 27 samples of value 0,
              the first quartile is 0.

            \item
              The mean in $1.56$,
              so the number of women that had more children than the mean
              is the total of samples with values 2, 3, 4, and 5 = $30 + 12 + 7 + 2 = 51$.

              So 51 of the 100 women had more children than the mean.

            \item
              The standard deviation is $\approx 1.31$,
              so we want all samples where the value is greater than the mean plus the standard deviation.
              In other words we want all samples with value $> 1.56 + 1.31 = 2.87$.
              We want to total the number of samples with values 3, 4, and 5 = $12 + 7 + 2 = 21$

              So 21 of the 100 women had more than one standard deviation's worth more children.

            \item
              The mean is $1.56$, the standard deviation is $\approx 1.31$,
              so we want the number of women that had between $\overline{x} - \sigma$ and $\overline{x} + \sigma$ children.

              These values are $\overline{x} - \sigma = 1.56 - 1.31 = 0.25$ and
              $\overline{x} + \sigma = 1.56 + 1.31 = 2.87$.

              So we want the number of samples with values 1 or 2 = $22 + 30 = 52$.

              So 52 of the 100 women had a number of children within one standard deviation of the mean.
          \end{enumerate}
        \item [14]
          \begin{enumerate}[label=(\arabic*)]
            \item
              We can calculate the total from the information given,
              and use that total to calculate the change.

              \begin{align*}
                \text{total} &= \$70,000 \cdot 10 = \$700,000 \\
                \overline{x_0} &= \frac{\text{total} - \$100,000 + \$1,000,000}{10} \\
                &= \frac{\text{total} + \$900,000}{10} \\
                &= \frac{\$700,000 + \$900,000}{10} \\
                &= \frac{\$1,600,000}{10} \\
                &= \$160,000 \\
              \end{align*}

              The new mean is \$160,000.

            \item
              Since only the largest datum was changed,
              the value of the median does not change.

              The median is still \$55,000.

            \item
              We can calculate the variance from the given information,
              and use that variance to calculate the change.

              \[
                \sigma^2 = \frac{1}{n - 1} \left[\left(\sum_{i} x_i^2\right) - n \overline{x}^2\right]
              \]

              We want to initially find this term: $\sum_{i} x_i^2$, as the rest we already know.

              \begin{align*}
                \sigma^2 &= \frac{1}{n - 1} \left[\left(\sum_{i} x_i^2\right) - n \overline{x}^2\right] \\
                (n - 1) \sigma^2 &= \left(\sum_{i} x_i^2\right) - n \overline{x}^2 \\
                (n - 1) \sigma^2 + n \overline{x}^2 &= \sum_{i} x_i^2 \\
                9 (20,000)^2 + 10 (70,000)^2 &= \sum_{i} x_i^2 \\
                9 (400,000,000) + 10 (4,900,000,000) &= \sum_{i} x_i^2 \\
                3,600,000,000 + 49,000,000,000 &= \sum_{i} x_i^2 \\
                52,600,000,000 &= \sum_{i} x_i^2 \\
              \end{align*}

              Now that we have this value,
              we can perform the adjustment.

              \begin{align*}
                \text{new sum} &= \sum_i x_i^2 - 100,000^2 + 1,000,000^2 \\
                &= 52,600,000,000 - 10,000,000,000 + 1,000,000,000,000 \\
                &= 1,042,600,000,000
              \end{align*}

              Finally, we calculate our new standard deviation $\sigma_0$

              \begin{align*}
                \sigma_0 &= \sqrt{\frac{1}{9} \left(\text{new sum} - 10 \overline{x_0}^2\right)} \\
                &= \sqrt{\frac{1}{9} \left(1,042,600,000,000 - 10 (160,000)^2\right)} \\
                &= \sqrt{\frac{1}{9} \left(1,042,600,000,000 - 256,000,000,000 \right)} \\
                &= \sqrt{87,400,000,000} \\
                &\approx 295,634.91 \\
              \end{align*}

              So, the new standard deviation $\sigma_0 \approx \$295,634.91$

          \end{enumerate}
      \end{enumerate}

    \item [$\S$ 1.3]

      \begin{enumerate}
        \item [11]
          \begin{enumerate}[label=(\arabic*)]
            \item

              It will help to sort this data initially.
              \[
                x = \{41, 42, 42, 44, 44, 45, 45, 46, 49, 49, 51, 51, 53, 56, 57, 59, 59, 65, 67, 71, 77, 100\}
              \]

              We have $n = 22$ samples.

              In order to construct a box plot,
              we need the median, \nth{1} quartile, \nth{3} quartile and IQR.

              \begin{itemize}
                \item
                  $\text{median} = \frac{x_{11} + x_{12}}{2} = \frac{51 + 51}{2} = 51$
                \item
                  $\text{\nth{1} quartile} = \frac{x_{5} + x_{6}}{2} = \frac{44 + 45}{2} = 44.5$
                \item
                  $\text{\nth{3} quartile} = \frac{x_{16} + x_{17}}{2} = \frac{59 + 59}{2} = 59$
                \item
                  $\text{IQR} = \text{\nth{3} quartile} - \text{\nth{1} quartile} = 59 - 44.5 = 14.5$
              \end{itemize}
              From the IQR, we calculate $1.5 \text{IQR} = 21.75$.
              Now, we can label the outliers as anything below $\text{\nth{1} quartile} - 1.5\text{IQR} = 44.5 - 21.75 = 22.75$,
              and anything above $\text{\nth{3} quartile} + 1.5\text{IQR} = 59 + 22.75 = 81.75$.

              This gives the following plot:

              \begin{tikzpicture}
                \begin{axis}
                  [ xtick={1}
                  , xticklabels={Absent}
                  , boxplot/draw direction=y
                  ]
                  \addplot+[boxplot]
                    table[row sep=\\,y index=0] {
                      data\\
                      41 \\ 42 \\ 42 \\ 44 \\ 44 \\ 45 \\ 45 \\ 46 \\ 49 \\ 49 \\ 51 \\ 51 \\ 53 \\ 56 \\ 57 \\ 59 \\ 59 \\ 65 \\ 67 \\ 71 \\ 77 \\ 100 \\
                    };
                \end{axis}
              \end{tikzpicture}
            \item

              Yes. Since $100 > \text{\nth{3} quartile} + 1.5\text{IQR} = 81.75$,
              the number of absences on January 28 was an outlier.
          \end{enumerate}
        \item [12]
          The mean cannot be determined from a box plot.
          All of the other statistics can be determined from a box plot.
        \item [15]
          \begin{enumerate}[label=(\arabic*)]
            \item
              The IQR for A is $6.02 - 1.42 = 4.60$.

              The IQR for B is $9.13 - 5.27 = 3.86$
            \item
              Since the $\text{minimum} > \text{\nth{1} quartile} - 1.5 \text{IQR} = 1.42 - 1.5(4.6) = -5.48$,
              and the $\text{maximum} < \text{\nth{3} quartile} + 1.5 \text{IQR} = 6.02 + 1.5(4.6) = 12.92$,
              we have enough information to construct the box plot:

              \begin{tikzpicture}
                \begin{axis}
                  [ xtick={1}
                  , xticklabels={A}
                  , boxplot/draw direction=y
                  ]
                  \addplot+[boxplot prepared={
                    lower whisker=0.066,
                    lower quartile=1.42,
                    median=2.60,
                    upper quartile=6.02,
                    upper whisker=10.08,
                  }]
                  coordinates {};
                \end{axis}
              \end{tikzpicture}

            \item

              Since the $\text{minimum} < \text{\nth{1} quartile} - 1.5 \text{IQR} = 5.27 - 1.5(3.86) = -0.52$,
              We do not know how many outliers actually exist in order to properly construct the box plot.
          \end{enumerate}
        \item [16]
          Since histogram $a$ is skewed to the right, and box plot $(4)$ has a long upper whisker, we pair these together.
          Since histogram $d$ is nearly symmetric, and box plot $(3)$ has near symmetry, we pair these together.
          Since histogram $b$ has mostly large values and what looks like a small outlier, we pair it with box plot $(2)$.
          Then, we pair the last two together.

          Our final pairing is: a - 4, b - 2, c - 1, d - 3
      \end{enumerate}
  \end{enumerate}
\end{document}
