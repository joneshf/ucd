\documentclass[12pt,letterpaper]{article}
\usepackage{amsmath}
\usepackage{amsfonts}
\usepackage{amsthm}
\usepackage{mathtools}
\usepackage{cancel}
\usepackage[margin=1in]{geometry}
\usepackage{titling}
\usepackage{fp}
\usepackage{enumitem}
\usepackage[super]{nth}
\usepackage{dcolumn}
\usepackage{pgfplots}
\pgfplotsset{compat=1.8}
\usepgfplotslibrary{statistics}

\newcolumntype{d}{D{.}{.}{-1}}

\newcommand*\dif{\mathop{}\!\mathrm{d}}

\setlength{\droptitle}{-10ex}

\preauthor{\begin{flushright}\large \lineskip 0.5em}
\postauthor{\par\end{flushright}}
\predate{\begin{flushright}\large}
\postdate{\par\end{flushright}}

\title{STA 032 Homework 3\vspace{-2ex}}
\author{Hardy Jones\\
        999397426\\
        Professor Melcon\vspace{-2ex}}
\date{Winter 2015}

\begin{document}
  \maketitle

  \begin{enumerate}
    \item [$\S$ 2.4]
      \begin{enumerate}
        \item [5]
          \begin{enumerate}[label=(\arabic*)]
            \item
              \begin{tabular}{| c | d |}
                \hline
                $x$ & \multicolumn{1}{c |}{$f(x)$} \\
                \hline
                1 & 0.7  \\
                2 & 0.15 \\
                3 & 0.1  \\
                4 & 0.03 \\
                5 & 0.02 \\
                \hline
              \end{tabular}
            \item
              \[
                P(X \le 2) = P(X = 1) + P(X = 2) = 0.7 + 0.15 = 0.85
              \]
            \item
              \[
                P(X > 3) = P(X = 4) + P(X = 5) = 0.03 + 0.02 = 0.05
              \]
            \item
              \begin{align*}
                \mu_X &= x_1f(x_1) + x_2f(x_2) + x_3f(x_3) + x_4f(x_4) + x_5f(x_5) \\
                &= 1(0.7) + 2(0.15) + 3(0.1) + 4(0.03) + 5(0.02) \\
                &= 0.7 + 0.3 + 0.3 + 0.12 + 0.1 \\
                &= 1.52 \\
              \end{align*}
            \item
              \begin{align*}
                \sigma_X &= \sqrt{x_1^2f(x_1) + x_2^2f(x_2) + x_3^2f(x_3) + x_4^2f(x_4) + x_5^2f(x_5) - \mu_X^2} \\
                &= \sqrt{1^2(0.7) + 2^2(0.15) + 3^2(0.1) + 4^2(0.03) + 5^2(0.02) - 1.52^2} \\
                &= \sqrt{1(0.7) + 4(0.15) + 9(0.1) + 16(0.03) + 25(0.02) - 2.3104} \\
                &= \sqrt{0.8696} \\
                &\approx 0.93 \\
              \end{align*}
          \end{enumerate}
        \item [8]
          \begin{enumerate}[label=(\arabic*)]
            \item
              We want to find
              \[
                P(X \le 2) = F(2) = 0.83
              \]
            \item
              We want to find
              \[
                P(X > 3) = F(4) - F(3) = 1.00 - 0.95 = 0.05
              \]
            \item
              We want to find
              \[
                P(X = 1) = F(1) - F(0) = 0.72 - 0.41 = 0.31
              \]
            \item
              We want to find
              \[
                P(X = 0) = F(0) = 0.41
              \]
            \item
              If we look at the probability of each number of error,
              we can answer this.
              \begin{align*}
                P(X = 0) &= 0.41 \\
                P(X = 1) &= 0.31 \\
                P(X = 2) &= F(2) - F(1) = 0.83 - 0.72 = 0.11 \\
                P(X = 3) &= F(3) - F(2) = 0.95 - 0.83 = 0.12 \\
                P(X = 4) &= 0.05 \\
              \end{align*}
              Since $P(X = 0)$ has the largest probability,
              it is most probable that 0 errors will be detected.
          \end{enumerate}
        \item [15]
          \begin{enumerate}[label=(\arabic*)]
            \item
              \begin{align*}
                \mu_t &= \int_{-\infty}^{\infty} \! t f(t) \, \dif t \\
                &= 0.1 \int_{0}^{\infty} \! t e^{-0.1 t} \, \dif t
              \end{align*}

              Using the tabular method for integration by parts

              \begin{tabular}{| c | c |}
                \hline
                $u$ & $\dif v$ \\
                \hline
                $t$ & $e^{-0.1 t}$    \\
                $1$ & $-10e^{-0.1 t}$ \\
                $0$ & $100e^{-0.1 t}$ \\
                \hline
              \end{tabular}

              So we have
              \begin{align*}
                \mu_t &= 0.1 \left(t (-10e^{-0.1 t}) - 1 (100e^{-0.1 t}) \right) \Big|_0^{\infty} \\
                &= 0.1 \left(-(10t + 100) (e^{-0.1 t}) \right) \Big|_0^{\infty} \\
                &= -\frac{t + 10}{e^{0.1 t}} \Big|_0^{\infty} \\
                &= 0 - (-10) \\
                &= 10
              \end{align*}
            \item
              \begin{align*}
                \sigma_t &= \sqrt{\int_0^\infty \! t^2 f(t) \, \dif t - \mu_t^2} \\
                &= \sqrt{0.1 \int_0^\infty \! t^2 e^{-0.1 t} \, \dif t - 10^2} \\
                &= \sqrt{0.1 \int_0^\infty \! t^2 e^{-0.1 t} \, \dif t - 100} \\
              \end{align*}
              Using the tabular method for integration by parts

              \begin{tabular}{| c | c |}
                \hline
                $u$ & $\dif v$ \\
                \hline
                $t^2$ & $e^{-0.1 t}$    \\
                $2t$  & $-10e^{-0.1 t}$ \\
                $2$   & $100e^{-0.1 t}$ \\
                $0$   & $-1000e^{-0.1 t}$ \\
                \hline
              \end{tabular}

              So we have
              \begin{align*}
                \sigma_t &= \sqrt{0.1 \left( t^2(-10e^{-0.1 t}) - 2t (100e^{-0.1 t}) + 2 (-1000e^{-0.1 t}) \right) \Big|_0^\infty - 100} \\
                &= \sqrt{0.1 \left( (-10t^2 - 200t - 2000) e^{-0.1 t} \right) \Big|_0^\infty - 100} \\
                &= \sqrt{\frac{-t^2 - 20t - 200}{e^{0.1 t}} \Big|_0^\infty - 100} \\
                &= \sqrt{(0 - (-200)) - 100} \\
                &= \sqrt{100} \\
                &= 10 \\
              \end{align*}
            \item
              \begin{align*}
                F(x) &= \int_0^x \! f(t) \, \dif t \\
                &= 0.1 \int_0^x \! e^{-0.1 t} \, \dif t \\
                &= - e^{-0.1 t} \Big|_0^x \\
                &= - \frac{1}{e^{0.1 t}} \Big|_0^x \\
                &= - \frac{1}{e^{0.1 x}} - (- 1) \\
                &= 1 - \frac{1}{e^{0.1 x}} \\
              \end{align*}
            \item
              We want to find $P(X < 12) = F(11)$,
              since we are dealing with discrete random variables.

              \[
                F(11) = 1 - \frac{1}{e^{0.1 (11)}} \approx 0.6671
              \]

              So the probability that the lifetime will be less than 12 months is approximately $66.71\%$.
          \end{enumerate}
        \item [24]
          \begin{enumerate}[label=(\arabic*)]
            \item
            \item
            \item
            \item
            \item
            \item
            \item
          \end{enumerate}
      \end{enumerate}
    \item [$\S$ 2.5]
      \begin{enumerate}
        \item [8]
          \begin{enumerate}[label=(\arabic*)]
            \item
            \item
            \item
            \item
            \item
          \end{enumerate}
        \item [10]
          \begin{enumerate}[label=(\arabic*)]
            \item
            \item
          \end{enumerate}
      \end{enumerate}
  \end{enumerate}
\end{document}
