\begin{appendices}
    \section{R code}
        We start by loading the data into R.

        \rData{preamble.R}

        \subsection*{(a)}

            We can simply call the \textit{summary} function to create the summary.

            \rData{a.R}

        \subsection*{(b)}

            We can get the mean and standard deviation with the functions \textit{mean} and \textit{sd}

            \rData{b.R}

        \subsection*{(c)}

            The \textit{table} function summarizes the data as categorical.
            In this case, it categorizes \textit{data\$Semester} into \textit{Fall} and \textit{Spring}

            \rData{c.R}

        \subsection*{(d)}

            The \textit{nrow} function counts the number of rows in the data set.

            \rData{d.R}

        \subsection*{(e)}

            \rData{e.R}

        \subsection*{(f)}

            \rData{f.R}

        \subsection*{(g)}

            We can use the \textit{quantile} function to compute these values.

            \rData{g.R}

        \subsection*{(h)}

            We can use the \textit{quantile} function to compute these values.

            \rData{h.R}

        \subsection*{(i)}

            We can use the \textit{quantile} function to compute these values.

            \rData{i.R}
\end{appendices}
