\documentclass[12pt,letterpaper]{article}
\usepackage{amsmath}
\usepackage{amsfonts}
\usepackage{amsthm}
\usepackage{mathtools}
\usepackage{cancel}
\usepackage[margin=1in]{geometry}
\usepackage{titling}
\usepackage{fp}
\usepackage{enumitem}
\usepackage[super]{nth}
\usepackage{dcolumn}
\usepackage[round-mode=figures,round-precision=3,scientific-notation=false]{siunitx}
\usepackage{pgf}
\usepackage{pgfplots}
\usepackage{xparse}
\pgfplotsset{compat=1.8}
\usepgfplotslibrary{statistics}

\newcolumntype{d}{D{.}{.}{-1}}

\newcommand*\dif{\mathop{}\!\mathrm{d}}
\newcommand\numberthis{\refstepcounter{equation}\tag{\theequation}}
\newcommand\hypot[7]{
  We have $n = #1, \overline{X} = #2$ and $s = #3$.
  \begin{enumerate}
    \item
      We have our hypotheses:
      \begin{align*}
        H_0 &: \mu #5 #4 \\
        H_A &: \mu #6 #4 \\
      \end{align*}
    \item
      The test statistic is:
      \pgfmathsetmacro{\result}{(#2 - #4) / (#3 / sqrt(#1))}
      \[
        Z_s = \frac{\overline{X} - \mu_0}{\frac{s}{\sqrt{n}}} = \frac{#2 - #4}{\frac{#3}{\sqrt{#1}}} = \num{\result}
      \]
    \item
      The corresponding p-value is: $\num{#7}$
    \item

      \pgfmathsetmacro{\bool}{ifthenelse(#7 < 0.05, 1, 0)}
      \ifdim \bool pt = 1 pt
        Since the p-value is so small, we reject the null hypothesis.
      \else
        Since the p-value is large, we assume the null hypothesis is true.
      \fi
  \end{enumerate}
}

\DeclarePairedDelimiter\floor{\lfloor}{\rfloor}

\setlength{\droptitle}{-10ex}

\preauthor{\begin{flushright}\large \lineskip 0.5em}
\postauthor{\par\end{flushright}}
\predate{\begin{flushright}\large}
\postdate{\par\end{flushright}}

\title{STA 032 Homework 8\vspace{-2ex}}
\author{Hardy Jones\\
        999397426\\
        Professor Melcon\vspace{-2ex}}
\date{Winter 2015}

\begin{document}
  \maketitle

  \begin{enumerate}
    \item [$\S$ 6.1]
      \begin{enumerate}
        \item [5]
          \hypot{80}{4.5}{2.7}{5.4}{\ge}{<}{0.0014}
          \begin{enumerate}[label=(\arabic*)]
            \item
              See part iii above.
            \item
              I am convinced the mean number of sick days is less than 5.4 after allowing telecommuting.
              The probability that the sample was from a population where the mean number of sick days was greater than 5.4 is exceedingly small.
              It is much more likely that the sample was taken from a population with a mean number of sick days less than 5.4.
          \end{enumerate}
        \item [8]
          \hypot{100}{25}{60}{0}{=}{\ne}{1}
          \begin{enumerate}[label=(\arabic*)]
            \item
              See part iii above.
            \item
              I am convinced that the laser is properly calibrated.
              Since the probability that the sample was from a population with 0 mean error is effectively 1,
              it makes sense that the laser is properly calibrated.
          \end{enumerate}
      \end{enumerate}
    \item [$\S$ 6.2]
      \begin{enumerate}
        \item [9]
          \begin{enumerate}[label=(\arabic*)]
            \item
              The null hypothesis should be:
              \[
                H_0: \mu \le 8 \textrm{ years}
              \]
              If $H_0$ is rejected,
              then it is known that the batteries will have a mean lifetime of more than 8 years and
              so can be installed in pacemakers.

              If $H_0$ is not rejected,
              then the batteries might have a mean lifetime less than 8 years and
              so cannot be installed in pacemakers.
            \item
              The null hypothesis should be:
              \[
                H_0: \mu \le \num{60000} \textrm{ miles}
              \]
              If $H_0$ is rejected,
              then it is known that the tires will have a mean lifetime more than \num{60000} miles and
              so the new material can be used to make tires.

              If $H_0$ is not rejected,
              then the tires might have a mean lifetime less than \num{60000} miles and
              so the new material cannot be used to make tires.
            \item
              The null hypothesis should be:
              \[
                H_0: \mu = \num{10} \textrm{ \si{\milli\liter\per\second}}
              \]
              If $H_0$ is rejected,
              then it is known that the flow rate will be \num{10} \si{\milli\liter\per\second} and
              so the flowmeter should be recalibrated.

              If $H_0$ is not rejected,
              then the flow rate might not be \num{10} \si{\milli\liter\per\second} and
              so the flowmeter should not be recalibrated.
          \end{enumerate}
        \item [14]
          i. Greater than 0.05.

          We can compute the necessary values from the information given.

          With the interval from 1.2 to 2.0,
          we can compute $\overline{X} = \frac{1.2 + 2.0}{2} = 1.6$.

          We also know that
          \begin{align*}
            1.2 &= 1.6 - 1.96 \frac{s}{\sqrt{n}} \\
            1.96 \frac{s}{\sqrt{n}} &= 0.4 \\
            \frac{s}{\sqrt{n}} &= \frac{0.4}{1.96} \\
            \frac{s}{\sqrt{n}} &= \num{0.20408163265306123} \\
          \end{align*}

          So, we can compute a z-score
          \[
            Z = \frac{\overline{X} - \mu_0}{\frac{s}{\sqrt{n}}} = \frac{1.6 - 1.4}{\num{0.20408163265306123}} = 0.98
          \]

          The corresponding p-value is 0.8365, which much greater than 0.05.
      \end{enumerate}
    \item [$\S$ 6.3]
      \begin{enumerate}
        \item [7]
          We want the hypotheses:

          \begin{align*}
            H_0 &: \mu \le 0.7 \\
            H_A &: \mu > 0.7 \\
          \end{align*}

          So $n = 150$ and $p = 0.7$.

          Thus, $\hat{p} \sim N(150, \frac{0.7 (1 - 0.7)}{150}) = N(150, \num{0.00140})$.

          This lets us compute $\sigma_{\hat{p}} = \sqrt{\num{0.00140}} = \num{0.0374}$,
          and an observed $\hat{p} = \frac{110}{150} = \num{0.7333333}$.

          So we have a z-score of $Z = \frac{\num{0.7333} - 0.7}{\num{0.0374}} = 0.8913$.

          The corresponding p-value is $1 - 0.8133 = 0.1867$.

          Since this value is much higher than 0.05,
          we do not reject the null hypothesis.
          We cannot conclude that more than 70\% of the households in the city have high-speed internet access.
        \item [8]
          We want the hypotheses:

          \begin{align*}
            H_0 &: \mu \ge 0.08 \\
            H_A &: \mu < 0.08 \\
          \end{align*}

          So $n = 300$ and $p = 0.08$.

          Thus, $\hat{p} \sim N(300, \frac{0.08 (1 - 0.08)}{300}) = N(300, \num{0.0002453})$.

          This lets us compute $\sigma_{\hat{p}} = \sqrt{\num{0.0002453}} = \num{0.015663}$,
          and an observed $\hat{p} = \frac{12}{300} = \num{0.04}$.

          So we have a z-score of $Z = \frac{\num{0.04} - 0.08}{\num{0.015663}} = \num{-2.553789184702803}$.

          The corresponding p-value is $0.0054$.

          Since this value is very small, we reject the null hypothesis.
          We can conclude that less than 8\% of the produced parts are defective.
      \end{enumerate}
    \item [$\S$ 6.4]
      \begin{enumerate}
        \item [3]
          \begin{enumerate}[label=(\arabic*)]
            \item
            \item
            \item
          \end{enumerate}
        \item [4]
          \begin{enumerate}[label=(\arabic*)]
            \item
            \item
            \item
          \end{enumerate}
      \end{enumerate}
    \item [$\S$ 6.5]
      \begin{enumerate}
        \item [7]
          \begin{enumerate}[label=(\arabic*)]
            \item
            \item
          \end{enumerate}
      \end{enumerate}
    \item [$\S$ 6.6]
      \begin{enumerate}
        \item [12]
      \end{enumerate}
    \item [$\S$ 6.7]
      \begin{enumerate}
        \item [13]
      \end{enumerate}
    \item [$\S$ 6.12]
      \begin{enumerate}
        \item [4]
          \begin{enumerate}[label=(\arabic*)]
            \item
            \item
            \item
            \item
          \end{enumerate}
        \item [5]
          \begin{enumerate}[label=(\arabic*)]
            \item
            \item
            \item
            \item
          \end{enumerate}
      \end{enumerate}
  \end{enumerate}
\end{document}
