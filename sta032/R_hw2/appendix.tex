\begin{appendices}
    \section{R code}

        \subsection*{Problem 1}
            We start by loading the representation of the deck of cards.

            \rData{preamble.R}

            Next we create some helper functions to keep this code comprehensible.

            \rData{1.R}

            \subsubsection*{(a)}

                Now, we can very simply create a predicate to test whether we have a pair, and run the simulation.

                \rData{1a.R}

            \subsubsection*{(b)}

                We can do similarly for hearts.

                \rData{1b.R}

            \subsubsection*{(c)}

                And finally, for the complex test of a pair with one card a diamond and the other card a heart.

                \rData{1c.R}

        \subsection*{Problem 2}
            We need a data structure for coins.
            Again we break down functions so they are simple.

            We also have a simulator and a probability reporter.

            \rData{2.R}

            \subsubsection*{(a)}
                This problem is a direct check of whether the coin was blue or not.

                \rData{2a.R}
            \subsubsection*{(b)}
                This problem is a direct check of whether the coin was heads or not.

                \rData{2b.R}
            \subsubsection*{(c)}
                Things start to get a bit trickier at this point.

                \rData{2c.R}
            \subsubsection*{(d)}
                \rData{2d.R}
            \subsubsection*{(e)}
                \rData{2e.R}
            \subsubsection*{(f)}
                \rData{2f.R}
            \subsubsection*{(g)}
                \rData{2g.R}

        % \subsection*{Problem 3}
\end{appendices}
