\documentclass[12pt,letterpaper]{article}
\usepackage{amsmath}
\usepackage{amsfonts}
\usepackage{amsthm}
\usepackage[margin=1in]{geometry}
\usepackage{titling}

\setlength{\droptitle}{-10ex}

\preauthor{\begin{flushright}\large \lineskip 0.5em}
\postauthor{\par\end{flushright}}
\predate{\begin{flushright}\large}
\postdate{\par\end{flushright}}

\title{MAT 25 Homework 4\vspace{-2ex}}
\author{Hardy Jones\\
        999397426\\
        Professor Bae\vspace{-2ex}}
\date{Fall 2013}

\begin{document}

  \maketitle

  \begin{enumerate}
  	\item 1.4.10
      Show that the set of all finite subsets of $\mathbb{N}$ is a countable set.
      \begin{proof}
        In order to prove this, we simply need to construct an isomorphism from $\mathbb{N}$ to the set of all finite subsets of $\mathbb{N}$, hereafter referred to as $\mathcal{S}$.

        Let's take a look at some elements of $\mathcal{S}$. We have:
        \[S = \left\{\{\}, \{1\}, \{2\}, \{1,2\}, \{3\}, \{1,3\}, \{2,3\}, \{1,2,3\}, ...\right\}\]

        What we see is that we can arbitrarily number these sets:

        \begin{matrix}
          \mathbb{N}: & 1 & 2 & 3 & 4 & 5 & 6 & 7 & 8 & ... \\
          & \updownarrow & \updownarrow & \updownarrow & \updownarrow & \updownarrow & \updownarrow & \updownarrow & \updownarrow & ... \\
          \mathcal{S}: & \{\} & \{1\} & \{2\} & \{1,2\} & \{3\} & \{1,3\} & \{2,3\} & \{1,2,3\} & ...
        \end{matrix}

        And thus we have a mapping that is 1-1 and onto between $\mathbb{N}$ and $\mathcal{S}$.

        So $\mathcal{S}$ has the same cardinality as $\mathbb{N}$,
        from which it follows that $\mathcal{S}$ is countable.
      \end{proof}

    \item 1.4.12
      A real number $x \in R$ is called algebraic
      if there exist integers $a_0, a_1, a_2, . . . , a_n \in N$, not all zero, such that
      $a_nx^n + a_{n−1}x^n−1 + . . . + a_1x + a_0 = 0$.
      Said another way, a real number is algebraic if it is the root of a polynomial with integer
      coefficients.
      Real numbers that are not algebraic are called transcendental numbers.

      \begin{enumerate}
        \item Show that $\sqrt{2}$, $\sqrt[3]{2}$, and $\sqrt{2} + \sqrt{3}$ are algebraic numbers.
          \begin{proof}
            \begin{enumerate}
              \item
                If we let $x = \sqrt{2}$, then we can manipulate this equation.
                \begin{align*}
                  x &= \sqrt{2} \\
                  x^2 &= 2 \\
                  x^2 - 2 &= 0
                \end{align*}

                So we have $a_0 = -2, a_2 = 1, a_1 = a_3 = a_4 = ... = a_n = 0$
                Thus, $\sqrt{2}$ is an algebraic number.

              \item
                Following similar logic.
                If we let $x = \sqrt[3]{2}$, then we can manipulate this equation.
                \begin{align*}
                  x &= \sqrt[3]{2} \\
                  x^3 &= 2 \\
                  x^3 - 2 &= 0
                \end{align*}

                So we have $a_0 = -2, a_3 = 1, a_1 = a_3 = a_4 = ... = a_n = 0$
                Thus, $\sqrt[3]{2}$ is an algebraic number.

              \item
                This is a bit more complex.
                \begin{align*}
                  x &= \sqrt{2} + \sqrt{3} \\
                  x^2 &= 2 + 3 + 2\sqrt{6} \\
                  x^2 - 5 &= 2\sqrt{6} \\
                  x^4 - 10x^2 + 25 &= 24 \\
                  x^4 - 10x^2 + 1 &= 0
                \end{align*}

                So we have $a_0 = 1, a_2 = -10, a_4 = 1, a_1 = a_3 = a_5 = a_6 = ... = a_n = 0$
                Thus, $\sqrt{2} + \sqrt{3}$ is an algebraic number.
            \end{enumerate}
          \end{proof}
        \item Fix $n \in N$,
          and let $\mathcal{A}_n$ be the algebraic numbers obtained as roots of
          polynomials with integer coefficients that have degree $n$.
          Using the fact that every polynomial has a finite number of roots,
          show that $\mathcal{A}_n$ is countable.

          \begin{proof}
            Since every polynomial has a unique representation,
            we can say that each polynomial can be thought of as a set of integers.

            Let's call this set $\mathcal{P} = \{a_0, a_1, ..., a_n | n \in \mathbb{N}\}$.

            We've already shown that the set of all finite subsets of $\mathbb{N}$ is countable.
            And, since $\mathcal{P}$ is actually the set of all finite subsets of $\mathbb{N}$,
            $\mathcal{P}$ is countable.

            Since each polynomial in $\mathcal{P}$ has a finite number of roots,
            only a finite number of algebraic numbers correspond to each polynomial.
            We can construct countable subsets of $\mathcal{A}_n$ comprised of these roots.

            Now, since each polynomial has a corresponding countable subset in $\mathcal{A}_n$,
            we can take the union of all these subsets in $\mathcal{A}_n$.
            According to Theorem 1.4.13.i,
            we have the union of all these subsets in $\mathcal{A}_n$ is countable.

            And of course, the union of all these subsets in $\mathcal{A}_n$ is
            just $\mathcal{A}_n$. So $\mathcal{A}_n$ is countable.
          \end{proof}

        \item Now, argue that the set of all algebraic numbers is countable.
          What may we conclude about the set of transcendental numbers?

          \begin{proof}
            From above we have that $\mathcal{A}_n$ is countable.
            If we look at all algebraic numbers, we can construct this set as
            \[\mathcal{A} = \bigcup_{n=1}^{\infty}{\mathcal{A}_n}\]

            From Theorem 1.4.13.ii we know that this set is countable.
            So the algebraic numbers are countable.

            And since transcendental numbers are all other real numbers,
            it follows that the transcendentals make up the rest of the reals.
            So $\mathcal{T} = \mathbb{R} \backslash \mathcal{A}$,
            similar to the relation between the rationals and irrationals.
            Following this relation,
            since the reals are uncountable, and $\mathcal{A}$ is countable,
            we must have that $\mathcal{T}$ is uncountable.
          \end{proof}
      \end{enumerate}

    \item Let $f : \mathbb{N} \times \mathbb{N} \rightarrow \mathbb{N}$
      be a function defined as
      \[f(m, n) = (2m - 1) \cdot 2^{n - 1}\]

      \begin{enumerate}
        \item Show that $f$ is 1-1.

          \begin{proof}
            We want to show
            \[\forall a, b, c, d \in \mathbb{N}, f(a,b) = f(c,d) \implies a = c \text{ and } b = d\]

            \begin{align*}
              f(a,b) &= f(c,d) \\
              (2a - 1) \cdot 2^{b - 1} &= (2c - 1) \cdot 2^{d - 1} \\
              (2a - 1) \cdot 2^b \cdot \frac{1}{2} &= (2c - 1) \cdot 2^d \cdot \frac{1}{2} \\
              (2a - 1) \cdot 2^b &= (2c - 1) \cdot 2^d \\
              2a - 1 &= (2c - 1) \cdot 2^{d-b}
            \end{align*}

            Now, the LHS is an odd number.
            Looking at the RHS, $2c - 1$ is an odd number,
            and based on The Trichotomy Law we have three cases to consider:

            \begin{enumerate}
              \item $b < d$
                In this case, $2^{d-b}$ is an even number,
                so the RHS would have to be an even number.
                But the LHS (an odd number) cannot be an even number also.
                So, this case is impossible.
              \item $b > d$
                In this case, $2^{d-b}$ is a fraction,
                so the RHS would also be a fraction.
                But LHS is a positive integer.
                So, this case is also impossible.
              \item $b = d$
                This must be the case, and we needn't show anything to prove it.
                What this means however is that $2^{d-b}$ is $1$.
            \end{enumerate}

            \begin{align*}
              2a - 1 &= 2c - 1 \\
              2a &= 2c \\
              a &= c
            \end{align*}

            Now, we have $a = c$ and $b = d$.

            So we have shown that \[\forall a, b, c, d \in \mathbb{N}, f(a,b) = f(c,d) \implies a = c \text{ and } b = d\]
            Thus $f$ is 1-1.
          \end{proof}

        \item Using (a), show that $\mathbb{Q}^+ = \left\{q \in \mathbb{Q} : q > 0 \right\}$

          \begin{proof}
            We can use $f$ to construct a map:
            \[f : \mathbb{Q}^+ \rightarrow \mathbb{N}\]
            If we take the components of each rational:
            \[\mathbb{Q}^+ = \left\{r, s \in \mathbb{N}, q \in \mathbb{Q} : \frac{r}{s} = q > 0 \right\}\]

            Now we use $f$ to map $\mathbb{Q}^+$ to $\mathbb{N}$.
            Since $f$ is 1-1, this implies that $\mathbb{Q}^+ \sim \mathbb{N}$,
            or that $\mathbb{Q}^+$ is countable.
          \end{proof}

        \item Show that $\mathbb{Q}$ is countable.

          \begin{proof}
            We can use the result of (b) to help us here.

            We simply need another function that is 1-1 with $f$.

            Let's try the function $g : \mathbb{Q} \rightarrow \mathbb{Q}^+$,

            \[
              g(q) = \left\{
                \begin{array}{l l}
                  q & \quad \text{if } q > 0\\
                  \frac{-q + 1}{2} & \quad \text{if } q <= 0
                \end{array} \right.
            \]

            It should be easy to see that this is injective.

            That is:

            \[\forall r, s \in \mathbb{Q}, g(r) = g(s) \implies r = s\]

            Let's show it quickly, we need to handle two cases:

            \begin{enumerate}
              \item $r > 0, s > 0$
                \begin{align*}
                  g(r) &= g(s) \\
                  r &= s
                \end{align*}

              \item $r < 0, s < 0$
                \begin{align*}
                  g(r) &= g(s) \\
                  \frac{-r + 1}{2} &= \frac{-s + 1}{2} \\
                  -r + 1 &= -s + 1 \\
                  -r &= -s \\
                  r &= s
                \end{align*}

            \end{enumerate}

            So $g$ is 1-1.
            And since $g$ is 1-1, we can construct a mapping
            $f \circ g : \mathbb{Q} \rightarrow \mathbb{Q}^+ \rightarrow \mathbb{N}$.

            So $\mathbb{Q}$ is countable.
          \end{proof}
      \end{enumerate}
  \end{enumerate}

\end{document}
