\documentclass[12pt,letterpaper]{article}
\usepackage{amsmath}
\usepackage{amsfonts}
\usepackage{amsthm}
\usepackage[margin=1in]{geometry}
\usepackage{titling}

\setlength{\droptitle}{-10ex}

\preauthor{\begin{flushright}\large \lineskip 0.5em}
\postauthor{\par\end{flushright}}
\predate{\begin{flushright}\large}
\postdate{\par\end{flushright}}

\title{MAT 25 Homework 4\vspace{-2ex}}
\author{Hardy Jones\\
        999397426\\
        Professor Bae\vspace{-2ex}}
\date{Fall 2013}

\begin{document}

  \maketitle

  \begin{enumerate}
  	\item 1.4.10
      Show that the set of all finite subsets of $\mathbb{N}$ is a countable set.
      \begin{proof}
        In order to prove this, we simply need to construct an isomorphism from $\mathbb{N}$ to the set of all finite subsets of $\mathbb{N}$, hereafter referred to as $S$.

        Let's take a look at some elements of $S$. We have:
        \[S = \left\{\{\}, \{1\}, \{2\}, \{1,2\}, \{3\}, \{1,3\}, \{2,3\}, \{1,2,3\}, ...\right\}\]

        What we see is that we can arbitrarily number these sets:

        \begin{matrix}
          \mathbb{N}: & 1 & 2 & 3 & 4 & 5 & 6 & 7 & 8 & ... \\
          & \updownarrow & \updownarrow & \updownarrow & \updownarrow & \updownarrow & \updownarrow & \updownarrow & \updownarrow & ... \\
          S: & \{\} & \{1\} & \{2\} & \{1,2\} & \{3\} & \{1,3\} & \{2,3\} & \{1,2,3\} & ...
        \end{matrix}

        And thus we have a mapping that is 1-1 and onto between $\mathbb{N}$ and $S$.
        So $S$ has the same cardinality of $\mathbb{N}$ from which it follows that $S$ is countable.
      \end{proof}
  \end{enumerate}

\end{document}
