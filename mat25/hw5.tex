\documentclass[12pt,letterpaper]{article}
\usepackage{amsmath}
\usepackage{amsfonts}
\usepackage{amsthm}
\usepackage[margin=1in]{geometry}
\usepackage{titling}

\setlength{\droptitle}{-10ex}

\preauthor{\begin{flushright}\large \lineskip 0.5em}
\postauthor{\par\end{flushright}}
\predate{\begin{flushright}\large}
\postdate{\par\end{flushright}}

\title{MAT 25 Homework 5\vspace{-2ex}}
\author{Hardy Jones\\
        999397426\\
        Professor Bae\vspace{-2ex}}
\date{Fall 2013}

\begin{document}

  \maketitle

  \begin{enumerate}
    \item 2.2.1
      \begin{enumerate}
        \item
          $\lim \frac{1}{6n^2 + 1} = 0$

          We need to show
          \begin{align*}
            \frac{1}{6n^2 + 1} &< \epsilon \\
            \frac{1}{\epsilon} &< 6n^2 + 1 \\
            \frac{1}{\epsilon} - 1 &< 6n^2 \\
            \frac{1 - \epsilon}{\epsilon} &< 6n^2 \\
            \frac{1 - \epsilon}{6\epsilon} &< n^2 \\
            \sqrt{\frac{1 - \epsilon}{6\epsilon}} &< n
          \end{align*}

          Let $\epsilon > 0$.
          Choose $N \in \mathbb{N} | N > \sqrt{\frac{1 - \epsilon}{6\epsilon}}$.

          Let $n \ge N$.
          So, $n \ge N > \sqrt{\frac{1 - \epsilon}{6\epsilon}} \implies \frac{1}{6n^2 + 1} < \epsilon$

          Thus $|a_n - 0| < \epsilon$.

        \item
          $\lim \frac{3n + 1}{2n + 5} = \frac{3}{2}$

          We need to show
          \begin{align*}
            \frac{3n + 1}{2n + 5} &< \epsilon \\
            3n + 1 &< 2n\epsilon + 5\epsilon \\
            1 - 5\epsilon &< (2\epsilon - 3)n \\
            \frac{1 - 5\epsilon}{2\epsilon - 3} &< n \\
          \end{align*}

          Let $\epsilon > 0$.
          Choose $N \in \mathbb{N} | N > \frac{1 - 5\epsilon}{2\epsilon - 3}$.

          Let $n \ge N$.
          So, $n \ge N > \frac{1 - 5\epsilon}{2\epsilon - 3} \implies \frac{3n + 1}{2n + 5} < \epsilon$.

          Thus $|a_n - \frac{3}{2}| < \epsilon$.

        \item
          $\lim \frac{2}{\sqrt{n + 3}} = 0$

          We need to show
          \begin{align*}
            \frac{2}{\sqrt{n + 3}} &< \epsilon \\
            \frac{2}{\epsilon} &< \sqrt{n + 3} \\
            \frac{4}{\epsilon^2} &< n + 3 \\
            \frac{4}{\epsilon^2} - 3 &< n
          \end{align*}

          Let $\epsilon > 0$.
          Choose $N \in \mathbb{N} | N > \frac{4}{\epsilon^2} - 3$.

          Let $n \ge N$.
          So, $n \ge N > \frac{4}{\epsilon^2} - 3 \implies \frac{2}{\sqrt{n + 3}} < \epsilon$.

          Thus $|a_n - 0| < \epsilon$.
      \end{enumerate}

    \item 2.2.5
      \begin{enumerate}
        \item
          $a_n = \left\lfloor\frac{1}{n}\right\rfloor$

          It is easy to see that after the first element in the sequence,
          all values are $0$.

          $\lim a_n = 0$

          \begin{proof}
            Let $\epsilon > 0$.
            Choose $N > 1$.

            Let $n \ge N$.
            So, $n \ge N > 1 \implies \left\lfloor\frac{1}{n}\right\rfloor = 0 < \epsilon$.

            Thus, $|a_n - 0| < \epsilon$.
          \end{proof}

        \item
          $a_n = \left \lfloor \frac{10 + n}{2n} \right \rfloor$

          Again we see that after some elements all values are $0$.

          $\lim a_n = 0$

          \begin{proof}
            Let $\epsilon > 0$.
            Choose $N > 10$.

            Let $n \ge N$.
            So, $n \ge N > 10 \implies \left \lfloor \frac{10 + n}{2n} \right \rfloor = 0 < \epsilon$.

            Thus, $|a_n - 0| < \epsilon$.
          \end{proof}
      \end{enumerate}

    \item 2.2.7
      \begin{enumerate}
        \item A sequence ($a_n$) diverges to $\infty$ if,
          for every positive number $\epsilon$,
          there exists an $N \in \mathbb{N}$ such that
          whenever $n \ge N$ it follows that $|a_n| > \epsilon$

          $\lim \sqrt{n} = \infty$

          \begin{proof}
            Let $\epsilon > 0$.

            We want to show
            \begin{align*}
              \sqrt{n} &> \epsilon \\
              n &> \epsilon^2
            \end{align*}

            Choose $N > \epsilon^2$.

            Let $n \ge N$.
            So, $n \ge N > \epsilon^2 \implies \sqrt{n} > \epsilon$.

            Thus, $|a_n| > \epsilon$.
          \end{proof}

        \item
          It states that this particular sequence does not diverge to $\infty$.
          The reason being,
          if you choose some $\epsilon > 0$, and any $N \in \mathbb{N}$,
          then $\exists n \ge N |$ either $n = 0$ or $n+1 = 0$.
          So, $|a_n| \not\ge \epsilon, \forall n$.
      \end{enumerate}

    \item 2.3.4
      Using the Algebraic Limit Theorem,
      if $\lim a_n = l_1$ and $\lim a_n = l_2$,
      then
      \begin{align*}
        \lim (a_n - a_n) &= 0 \\
        l_1 - l_2 &= 0 \\
        l_1 = l_2 \\
      \end{align*}
  \end{enumerate}

\end{document}
