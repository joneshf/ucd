\documentclass[12pt,letterpaper]{article}
\usepackage{amsmath}
\usepackage{amsfonts}
\usepackage{amsthm}
\usepackage[margin=1in]{geometry}
\usepackage{titling}

\setlength{\droptitle}{-10ex}

\preauthor{\begin{flushright}\large \lineskip 0.5em}
\postauthor{\par\end{flushright}}
\predate{\begin{flushright}\large}
\postdate{\par\end{flushright}}

\title{MAT 25 Homework 2\vspace{-2ex}}
\author{Hardy Jones\\
        999397426\\
        Professor Bae\vspace{-2ex}}
\date{Fall 2013}

\begin{document}

  \maketitle

  \begin{enumerate}
    \item 1.2.3 (c) \\
      Show that $(A \cup B)^c = A^c \cap B^c$
      by demonstrating inclusion both ways.

      \begin{proof}
        We need to show  two things:
        $(A \cup B)^c \subseteq A^c \cap B^c$
        and
        $A^c \cap B^c \subseteq (A \cup B)^c$

        \begin{enumerate}
          \item
            By definition of the compliment,
            \[(A \cup B)^c = \{x : x \notin (A \cup B)\}\]
            This means, given some $x$ in the set:
            $x \notin A$ and $x \notin B$.

            Or using the definition of complement:
            $x \in A^c$ and $x \in B^c$.

            From the definition of intersection we get:
            $x \in A^c \cap B^c$

            Since $x$ was an arbitrary choice,
            this result holds for all $x$ in the set.

            Or more succinctly,
            $\forall x \in (A \cup B)^c, x \in A^c \cap B^c$.

            By the definition of inclusion, we can say:
            \[(A \cup B)^c \subseteq A^c \cap B^c\]

          \item
            \[A^c \cap B^c = \{x : x \in A^c \text{and} x \in B^c\}\]

            This means, given some $x$ in the set, we can say:
            $x \notin A$ and $x \notin B$.

            If $x$ is not in either $A$ or $B$,
            then it cannot be in the union of those two sets.
            That is: $x \notin A \cup B$

            Using the definition of complement we can say:
            $x \in (A \cup B)^c$.

            Again, since $x$ was arbitrary,
            the result holds for all elements of the set.

            Or more succinctly,
            $\forall x \in A^c \cap B^c, x \in (A \cup B)^c$.

            By the definition of inclusion, we can say:
            \[A^c \cap B^c \subseteq (A \cup B)^c\]
        \end{enumerate}

        From a and b we have both sides of inclusion,
        so by the definition of set equality:
        \[(A \cup B)^c = A^c \cap B^c\]
      \end{proof}

    \item 1.2.7 \\
      Given $f : D \to \mathbb{R}$ and a subset $B \subseteq \mathbb{R}$
      let $f^{-1}(B) = \{x \in D : f(x) \in B\}$
      \begin{enumerate}
        \item Let $f(x) = x^2$.
        if $A = [0,4]$ and $B = [-1,1]$,
        find $f^{-1}(A)$ and $f^{-1}(B)$.

        Does $f^{-1}(A \cap B) = f^{-1}(A) \cap f^{-1}(B)$?

        Does $f^{-1}(A \cup B) = f^{-1}(A) \cup f^{-1}(B)$?

        $f^{-1}(A) = \{x \in [-2,2]\}$
        $f^{-1}(B) = \{x \in [-1,1]\}$

        \begin{align*}
          f^{-1}(A \cap B) &= f^{-1}([0,4] \cap [-1,1]) \\
          &= f^{-1}([0,4] \cap [-1,1]) \\
          &= f^{-1}([0,1]) \\
          &= \{x \in [-1,1]\} \\
          &= \{x \in [-2,2] \} \cap \{x \in [-1,1]\} \\
          &= f^{-1}(A) \cap f^{-1}(B)
        \end{align*}

        So this part is true.

        \begin{align*}
          f^{-1}(A \cup B) &= f^{-1}([0,4] \cup [-1,1]) \\
          &= f^{-1}([0,4] \cup [-1,1]) \\
          &= f^{-1}([-1,4]) \\
          &= \{x \in [-2,2]\} \\
          &= \{x \in [-2,2] \} \cup \{x \in [-1,1]\} \\
          &= f^{-1}(A) \cup f^{-1}(B)
        \end{align*}

        So this part is true.

      \end{enumerate}
  \end{enumerate}

\end{document}
