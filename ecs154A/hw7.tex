\documentclass[12pt,letterpaper]{article}
\usepackage{amsmath}
\usepackage{amsfonts}
\usepackage{amsthm}
\usepackage{cancel}
\usepackage{circuitikz}
\usepackage[margin=1in]{geometry}
\usepackage{titling}
\usepackage{tikz}
\usepackage{siunitx}
\usetikzlibrary{calc}
\usetikzlibrary{positioning}
\usetikzlibrary{matrix}

\pgfdeclarelayer{background}
\pgfsetlayers{background,main}


\setlength{\droptitle}{-10ex}

\preauthor{\begin{flushright}\large \lineskip 0.5em}
\postauthor{\par\end{flushright}}
\predate{\begin{flushright}\large}
\postdate{\par\end{flushright}}

\title{ECS 154A Homework 7\vspace{-2ex}}
\author{Hardy Jones\\
        999397426\\
        Professor Nitta\vspace{-2ex}}
\date{Fall 2013}

\begin{document}
  \maketitle

  \begin{enumerate}
    \item
      The advantage of write back cache is that main memory does not have to be written to immediately.
      The main memory can be lazily written to when there are free cycles.

      Two disadvantages are that it is more complex.
      If there is a cache miss and the location to store the new data is ready to be written, it will be written at that time.
      Then memory will be read into the cache.
      This makes for two memory access instead of just one.

    \item
      L1 cache should use virtual addresses.
      Since the cache is so close to the processor, speed is the main concern.
      A translation from virtual addresses to physical addresses would slow down L1 access.
      So, it's best to keep L1 addressed virtually.

    \item
      If we take average access time to be:
      \[Average Access Time = Hit Rate \cdot Upper Access Time + Miss Rate \cdot Lower Access Time\]
      We should be able to compute this relatively easily.
      We need some numbers first.
      \begin{align*}
        L1 \ Access Time &= 1 \ cycle \\
        L1 \ Hit Rate &= 0.90 \\
        L1 \ Miss Rate &= 0.10 \\
        L2 \ Access Time &= 10 \ cycle \\
        L2 \ Hit Rate &= 0.95 \\
        L2 \ Miss Rate &= 0.05 \\
        Memory Access Time &= 1000 \ cycle \\
      \end{align*}

      So we want to compute:
      \begin{align*}
        Average Access Time &= L1 \ Hit Rate \cdot L1 \ Access Time \\
        &+ L1 \ Miss Rate (L2 \ Hit Rate \cdot L2 \ Access Time \\
        &+ L2 \ Miss Rate (Memory Access Time))
      \end{align*}

      Now we just plug and chug the numbers.

      \begin{align*}
        Average Access Time &= 0.90 \cdot 1 \ cycle + 0.10 (0.95 \cdot 10 \ cycle + 0.05 (1000 \ cycle)) \\
        &= 0.90 \cdot 1 \ cycle + 0.10 (0.95 \cdot 10 \ cycle + 50 \ cycle) \\
        &= 0.90 \cdot 1 \ cycle + 0.10 (9.5 \ cycle + 50 \ cycle) \\
        &= 0.90 \cdot 1 \ cycle + 0.10 (59.5 \ cycle) \\
        &= 0.90 \cdot 1 \ cycle + 5.95 \ cycle \\
        &= 0.90 \ cycle + 5.95 \ cycle \\
        &= 6.85 \ cycle
      \end{align*}

      So the average access time is 6.85 cycles.
    \item
      ...
  \end{enumerate}
\end{document}
