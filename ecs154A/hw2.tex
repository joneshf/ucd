\documentclass[12pt,letterpaper]{article}
\usepackage{amsmath}
\usepackage{amsfonts}
\usepackage{amsthm}
\usepackage{cancel}
\usepackage{circuitikz}
\usepackage[margin=1in]{geometry}
\usepackage{titling}
\usepackage{tikz}
\usetikzlibrary{calc}
\usetikzlibrary{positioning}
\usetikzlibrary{matrix}

\pgfdeclarelayer{background}
\pgfsetlayers{background,main}


\setlength{\droptitle}{-10ex}

\preauthor{\begin{flushright}\large \lineskip 0.5em}
\postauthor{\par\end{flushright}}
\predate{\begin{flushright}\large}
\postdate{\par\end{flushright}}

\title{ECS 154A Homework 2\vspace{-2ex}}
\author{Hardy Jones\\
        999397426\\
        Professor Nitta\vspace{-2ex}}
\date{Fall 2013}

\begin{document}
  \maketitle

  \begin{enumerate}
    \item Find four four-variable Karnaugh maps with cheaper PoS than SoP.

    \begin{tikzpicture}
      \matrix (karnaugh1) [matrix of math nodes] {
        0 & 1 & 1 & 0 \\
        0 & 1 & 1 & 0 \\
        1 & 1 & 1 & 1 \\
        1 & 1 & 1 & 1 \\
      } ;

      \foreach \i/\bits in {1/00,2/01,3/11,4/10} {
        \node [left  = 2mm of karnaugh1-\i-1] {$\bits$} ;
        \node [above = 2mm of karnaugh1-1-\i] {$\bits$} ;
      }

      \node [left  = .6cm of karnaugh1.west]  (YZ) {$YZ$} ;
      \node [above = .5cm of karnaugh1.north] (WX) {$WX$} ;

      \draw    ($(WX.north       -| karnaugh1.west)  + (-.75mm,0)$)
            -- ($(karnaugh1.south -| karnaugh1.west)  + (-.75mm,0)$)
               ($(YZ.west        |- karnaugh1.north) + (0,+.40mm)$)
            -- ($(karnaugh1.east  |- karnaugh1.north) + (0,+.40mm)$) ;

      \begin{pgfonlayer}{background}
        \begin{scope}[opacity=.5]
          \fill [red]
                (karnaugh1-1-1.north west) rectangle (karnaugh1-2-1.south east)
                (karnaugh1-1-4.north west) rectangle (karnaugh1-2-4.south east) ;
          \fill [blue]
                (karnaugh1-1-2.north west) rectangle (karnaugh1-4-3.south east) ;
          \fill [green]
                (karnaugh1-3-1.north west) rectangle (karnaugh1-4-4.south east) ;
        \end{scope}
      \end{pgfonlayer}
    \end{tikzpicture}

    \begin{tikzpicture}
      \matrix (karnaugh2) [matrix of math nodes] {
        1 & 1 & 1 & 1 \\
        0 & 0 & 0 & 0 \\
        1 & 1 & 1 & 0 \\
        1 & 1 & 1 & 1 \\
      } ;

      \foreach \i/\bits in {1/00,2/01,3/11,4/10} {
        \node [left  = 2mm of karnaugh2-\i-1] {$\bits$} ;
        \node [above = 2mm of karnaugh2-1-\i] {$\bits$} ;
      }

      \node [left  = .6cm of karnaugh2.west]  (YZ) {$YZ$} ;
      \node [above = .5cm of karnaugh2.north] (WX) {$WX$} ;

      \draw    ($(WX.north       -| karnaugh2.west)  + (-.75mm,0)$)
            -- ($(karnaugh2.south -| karnaugh2.west)  + (-.75mm,0)$)
               ($(YZ.west        |- karnaugh2.north) + (0,+.40mm)$)
            -- ($(karnaugh2.east  |- karnaugh2.north) + (0,+.40mm)$) ;

      \begin{pgfonlayer}{background}
        \begin{scope}[opacity=.5]
          \fill [red]
                (karnaugh2-1-1.north west) rectangle (karnaugh2-1-4.south east)
                (karnaugh2-4-1.north west) rectangle (karnaugh2-4-4.south east) ;
          \fill [blue]
                (karnaugh2-3-2.north west) rectangle (karnaugh2-3-3.south east) ;
          \fill [orange]
                (karnaugh2-3-1.north west) rectangle (karnaugh2-4-2.south east) ;
          \fill [green]
                (karnaugh2-2-1.north west) rectangle (karnaugh2-2-4.south east) ;
          \fill [yellow]
                (karnaugh2-2-4.north west) rectangle (karnaugh2-3-4.south east) ;
        \end{scope}
      \end{pgfonlayer}
    \end{tikzpicture}

    \begin{tikzpicture}
      \matrix (karnaugh2) [matrix of math nodes] {
        1 & 0 & 1 & 1 \\
        1 & 0 & 1 & 1 \\
        1 & 1 & 1 & 1 \\
        1 & 1 & 1 & 1 \\
      } ;

      \foreach \i/\bits in {1/00,2/01,3/11,4/10} {
        \node [left  = 2mm of karnaugh2-\i-1] {$\bits$} ;
        \node [above = 2mm of karnaugh2-1-\i] {$\bits$} ;
      }

      \node [left  = .6cm of karnaugh2.west]  (YZ) {$YZ$} ;
      \node [above = .5cm of karnaugh2.north] (WX) {$WX$} ;

      \draw    ($(WX.north       -| karnaugh2.west)  + (-.75mm,0)$)
            -- ($(karnaugh2.south -| karnaugh2.west)  + (-.75mm,0)$)
               ($(YZ.west        |- karnaugh2.north) + (0,+.40mm)$)
            -- ($(karnaugh2.east  |- karnaugh2.north) + (0,+.40mm)$) ;

      \begin{pgfonlayer}{background}
        \begin{scope}[opacity=.5]
          \fill [red]
                (karnaugh2-1-1.north west) rectangle (karnaugh2-4-1.south east)
                (karnaugh2-1-4.north west) rectangle (karnaugh2-4-4.south east) ;
          \fill [blue]
                (karnaugh2-1-3.north west) rectangle (karnaugh2-4-4.south east) ;
          \fill [orange]
                (karnaugh2-3-1.north west) rectangle (karnaugh2-4-4.south east) ;
          \fill [green]
                (karnaugh2-1-2.north west) rectangle (karnaugh2-2-2.south east) ;
        \end{scope}
      \end{pgfonlayer}
    \end{tikzpicture}

    \begin{tikzpicture}
      \matrix (karnaugh2) [matrix of math nodes] {
        1 & 0 & 0 & 1 \\
        1 & 1 & 0 & 0 \\
        1 & 0 & 0 & 0 \\
        1 & 0 & 0 & 1 \\
      } ;

      \foreach \i/\bits in {1/00,2/01,3/11,4/10} {
        \node [left  = 2mm of karnaugh2-\i-1] {$\bits$} ;
        \node [above = 2mm of karnaugh2-1-\i] {$\bits$} ;
      }

      \node [left  = .6cm of karnaugh2.west]  (YZ) {$YZ$} ;
      \node [above = .5cm of karnaugh2.north] (WX) {$WX$} ;

      \draw    ($(WX.north       -| karnaugh2.west)  + (-.75mm,0)$)
            -- ($(karnaugh2.south -| karnaugh2.west)  + (-.75mm,0)$)
               ($(YZ.west        |- karnaugh2.north) + (0,+.40mm)$)
            -- ($(karnaugh2.east  |- karnaugh2.north) + (0,+.40mm)$) ;

      \begin{pgfonlayer}{background}
        \begin{scope}[opacity=.5]
          \fill [red]
                (karnaugh2-1-1.north west) rectangle (karnaugh2-4-1.south east) ;
          \fill [yellow]
                (karnaugh2-2-1.north west) rectangle (karnaugh2-2-2.south east) ;
          \fill [blue]
                (karnaugh2-1-1.north west) rectangle (karnaugh2-1-1.south east)
                (karnaugh2-1-4.north west) rectangle (karnaugh2-1-4.south east)
                (karnaugh2-4-1.north west) rectangle (karnaugh2-4-1.south east)
                (karnaugh2-4-4.north west) rectangle (karnaugh2-4-4.south east) ;
          \fill [orange]
                (karnaugh2-1-2.north west) rectangle (karnaugh2-1-3.south east)
                (karnaugh2-4-2.north west) rectangle (karnaugh2-4-3.south east) ;
          \fill [green]
                (karnaugh2-3-2.north west) rectangle (karnaugh2-4-3.south east) ;
          \fill [gray]
                (karnaugh2-2-3.north west) rectangle (karnaugh2-3-4.south east) ;
        \end{scope}
      \end{pgfonlayer}
    \end{tikzpicture}

    \item
      Can 1,500,000,000 be represented exactly in single-point floating precision?
      What is the next number that can be represented?

      Yes, 1,500,000,000 can be represented exactly in single-point floating precision.
      The next highest number that can be accurately represented is 1,500,000,128

    \item
      Convert each to signed magnitude, one's complement and two's complement.

      \begin{enumerate}
        \item 010100100111 \\
          Signed mag: 1319 \\
          Ones' comp: 101011011000 \\
          Twos' comp: 101011011001

        \item 101101101011 \\
          Signed mag: -2875 \\
          Ones' comp: 010010010100 \\
          Twos' comp: 010010010101

        \item 010010010111 \\
          Signed mag: 1175 \\
          Ones' comp: 101101101000 \\
          Twos' comp: 101101101001

        \item 101010010000 \\
          Signed mag: -656 \\
          Ones' comp: 010101101111 \\
          Twos' comp: 010101110000

      \end{enumerate}
    \item
      Implement the truth table with 2-1 and 4-1 mux's.

      \begin{tikzpicture}
        % First 4-1 Mux
        \draw (0,10) -- (2,10);
        \draw (0,9) -- (2,9);
        \draw (0,8) -- (2,8);
        \draw (0,7) -- (2,7);
        % selectors
        \draw (2.25,5.75) -- (2.25,6.25);
        \draw (2.75,5.75) -- (2.75,6.75);

        \draw (2,6) -- (2,11) -- (3,10) -- (3,7) -- (2,6);

        % Second 4-1 Mux
        % input
        \draw (0,4) -- (2,4);
        \draw (0,3) -- (2,3);
        \draw (0,2) -- (2,2);
        \draw (0,1) -- (2,1);
        % selectors
        \draw (2.25,-0.25) -- (2.25,0.25);
        \draw (2.75,-0.25) -- (2.75,0.75);

        \draw (2,0) -- (2,5) -- (3,4) -- (3,1) -- (2,0);

        % 2-1 Mux
        \draw (5,4) -- (5,7) -- (6,6) -- (6,5) -- (5,4);

        % Connect Mux's
        \draw (3,8.5) -| (4,6) -- (5,6);
        \draw (3,2.5) -| (4,5) -- (5,5);

        % Function output
        \draw (6,5.5) -- (6.5,5.5);
      \end{tikzpicture}
  \end{enumerate}
\end{document}
