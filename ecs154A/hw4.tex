\documentclass[12pt,letterpaper]{article}
\usepackage{amsmath}
\usepackage{amsfonts}
\usepackage{amsthm}
\usepackage{cancel}
\usepackage{circuitikz}
\usepackage[margin=1in]{geometry}
\usepackage{titling}
\usepackage{tikz}
\usetikzlibrary{calc}
\usetikzlibrary{positioning}
\usetikzlibrary{matrix}

\pgfdeclarelayer{background}
\pgfsetlayers{background,main}


\setlength{\droptitle}{-10ex}

\preauthor{\begin{flushright}\large \lineskip 0.5em}
\postauthor{\par\end{flushright}}
\predate{\begin{flushright}\large}
\postdate{\par\end{flushright}}

\title{ECS 154A Homework 4\vspace{-2ex}}
\author{Hardy Jones\\
        999397426\\
        Professor Nitta\vspace{-2ex}}
\date{Fall 2013}

\begin{document}
  \maketitle

  \begin{enumerate}
    \item Implement the following D flipflop as a JK flipflop.

      D

    \begin{tikzpicture}
      \matrix (karnaugh1) [matrix of math nodes] {
        1 & 0 & 0 & 1 \\
        1 & 1 & 1 & 0 \\
        1 & 1 & 1 & 1 \\
        0 & 1 & 0 & 0 \\
      } ;

      \foreach \i/\bits in {1/00,2/01,3/11,4/10} {
        \node [left  = 2mm of karnaugh1-\i-1] {$\bits$} ;
        \node [above = 2mm of karnaugh1-1-\i] {$\bits$} ;
      }

      \node [left  = .6cm of karnaugh1.west]  (YZ) {$YZ$} ;
      \node [above = .5cm of karnaugh1.north] (QX) {$QX$} ;

      \draw    ($(QX.north       -| karnaugh1.west)  + (-.75mm,0)$)
            -- ($(karnaugh1.south -| karnaugh1.west)  + (-.75mm,0)$)
               ($(YZ.west        |- karnaugh1.north) + (0,+.40mm)$)
            -- ($(karnaugh1.east  |- karnaugh1.north) + (0,+.40mm)$) ;

      % \begin{pgfonlayer}{background}
      %   \begin{scope}[opacity=.5]
      %     \fill [red]
      %           (karnaugh1-1-1.north west) rectangle (karnaugh1-2-1.south east)
      %           (karnaugh1-1-4.north west) rectangle (karnaugh1-2-4.south east) ;
      %     \fill [blue]
      %           (karnaugh1-1-2.north west) rectangle (karnaugh1-4-3.south east) ;
      %     \fill [green]
      %           (karnaugh1-3-1.north west) rectangle (karnaugh1-4-4.south east) ;
      %   \end{scope}
      % \end{pgfonlayer}
    \end{tikzpicture}

  \end{enumerate}
\end{document}
