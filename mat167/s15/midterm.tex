\documentclass[12pt,letterpaper]{article}

\usepackage[margin=1in]{geometry}
\usepackage[round-mode=figures,round-precision=3,scientific-notation=false]{siunitx}
\usepackage[super]{nth}
\usepackage[title]{appendix}
\usepackage{amsfonts}
\usepackage{amsmath}
\usepackage{amsthm}
\usepackage{cancel}
\usepackage{color, colortbl}
\usepackage{dcolumn}
\usepackage{enumitem}
\usepackage{fp}
\usepackage{mathtools}
\usepackage{pgfplots}
\usepackage{systeme}
\usepackage{tikz}
\usepackage{titling}

\usetikzlibrary{arrows, automata}

\usepgfplotslibrary{statistics}

\pgfplotsset{compat=1.8}

\definecolor{Gray}{gray}{0.8}

\newcolumntype{d}{D{.}{.}{-1}}
\newcolumntype{g}{>{\columncolor{Gray}}c}

\DeclarePairedDelimiter\ceil{\lceil}{\rceil}
\DeclarePairedDelimiter\floor{\lfloor}{\rfloor}

\newcommand*\circled[1]{
  \tikz[baseline=(char.base)]{
    \node[shape=circle,draw,inner sep=2pt] (char) {#1};
  }
}

\makeatletter
\renewcommand*\env@matrix[1][*\c@MaxMatrixCols c]{%
  \hskip -\arraycolsep
  \let\@ifnextchar\new@ifnextchar
  \array{#1}}
\makeatother

\renewcommand{\labelenumii}{(\arabic*)}

\setlength{\droptitle}{-10ex}

\preauthor{\begin{flushright}\large \lineskip 0.5em}
\postauthor{\par\end{flushright}}
\predate{\begin{flushright}\large}
\postdate{\par\end{flushright}}

\title{MAT 167 Midterm\vspace{-2ex}}
\author{Hardy Jones\\
        999397426\\
        Professor Cheer\vspace{-2ex}}
\date{Spring 2015}

\begin{document}
  \maketitle

  \begin{enumerate}
    \item
      \begin{enumerate}
        \item
          True.

          \begin{proof}
            Assume $A = L_1U_1 = L_2U_2$ with $L_1, L_2$ lower triangular and unit diagonal,
            $U_1, U_2$ upper triangular with nonzero diagonal.

            So we have inverses for $L_1, L_2$ since the diagonal is all 1 with 0's above the diagonal,
            and we have inverses for $U_1, U_2$ since the diagonal is non-zero with 0's below the diagonal.

            Then we have:

            \begin{align*}
              L_1U_1 &= L_2U_2 \\
              L_2^{-1}\left(L_1U_1\right) &= L_2^{-1}\left(L_2U_2\right) \\
              L_2^{-1}\left(L_1U_1\right) &= \left(L_2^{-1}L_2\right)U_2 \\
              L_2^{-1}\left(L_1U_1\right) &= IU_2 \\
              L_2^{-1}\left(L_1U_1\right) &= U_2 \\
              L_2^{-1}\left(L_1U_1\right)U_1^{-1} &= U_2U_1^{-1} \\
              L_2^{-1}L_1\left(U_1U_1^{-1}\right) &= U_2U_1^{-1} \\
              L_2^{-1}L_1I &= U_2U_1^{-1} \\
              L_2^{-1}L_1 &= U_2U_1^{-1} \\
            \end{align*}

            Now, $L_2^{-1}L_1$ is a lower triangular matrix and
            $U_2U_1^{-1}$ is an upper triangular matrix.
            In order for these two to be equal they have to both be lower triangular and upper triangular at the same time.
            The only matrices with this property are diagonal matrices.

            So, $L_2^{-1}L_1, U_2U_1^{-1}$ are diagonal matrices.
            And since $L_2^{-1}L_1 = U_2U_1^{-1}$ we must have the same entries on the diagonal.
            And since $L_1, L_2$ have all 1 on the diagonal, $L_2^{-1}L_1$ has all 1 on the diagonal.

            So $L_2^{-1}L_1 = I$.

            Then,
            \begin{align*}
              L_2^{-1}L_1 &= I \\
              L_2\left(L_2^{-1}L_1\right) &= L_2I \\
              \left(L_2L_2^{-1}\right)L_1 &= L_2 \\
              IL_1 &= L_2 \\
              L_1 &= L_2
            \end{align*}

            And since $L_2^{-1}L_1 = I = U_2U_1^{-1}$, we have

            \begin{align*}
              I &= U_2U_1^{-1} \\
              IU_1 &= \left(U_2U_1^{-1}\right)U_1 \\
              U_1 &= U_2\left(U_1^{-1}U_1\right) \\
              U_1 &= U_2I \\
              U_1 &= U_2 \\
            \end{align*}

            So we have shown,

            if $A = L_1U_1 = L_2U_2$ with $L_1, L_2$ lower triangular and unit diagonal,
            $U_1, U_2$ upper triangular with nonzero diagonal,

            then $L_1 = L_2, U_1 = U_2$.
          \end{proof}
        \item
          True.

          \begin{proof}
            Assume $A^2 + A = I$
            \[
              A\left(A + I\right) = A^2 + A = I = A^2 + A = \left(A + I\right)A
            \]
            So $A + I$ is a left and right inverse of $A$,
            then $A^{-1} = A + I$.
          \end{proof}
        \item
          False.

          Let $A = \begin{bmatrix}0 & 1 \\ 1 & 0 \\\end{bmatrix}$.

          Then all the diagonal entries of $A$ are zero, but $\begin{vmatrix}0 & 1 \\ 1 & 0 \\\end{vmatrix} = -1$,
          so $A$ is non-singular.
      \end{enumerate}
    \item
      \begin{enumerate}
        \item
        \item
        \item
      \end{enumerate}
    \item
      \begin{enumerate}
        \item
        \item
        \item
        \item
        \item
        \item
      \end{enumerate}
    \item
      \begin{enumerate}
        \item
        \item
        \item
      \end{enumerate}
    \item
      \begin{enumerate}
        \item
        \item
        \item
      \end{enumerate}
  \end{enumerate}
\end{document}
