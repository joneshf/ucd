\documentclass[12pt,letterpaper]{article}
\usepackage{amsmath}
\usepackage{amsfonts}
\usepackage{amsthm}
\usepackage{mathtools}
\usepackage{cancel}
\usepackage[margin=1in]{geometry}
\usepackage{titling}
\usepackage{fp}
\usepackage{enumitem}
\usepackage[super]{nth}
\usepackage{dcolumn}
\usepackage[title]{appendix}
\usepackage{pgfplots}
\usepackage{tikz}
\pgfplotsset{compat=1.8}
\usepgfplotslibrary{statistics}
\usepackage[round-mode=figures,round-precision=3,scientific-notation=false]{siunitx}
\usepackage{color, colortbl}
\usepackage{systeme}
\definecolor{Gray}{gray}{0.8}
\newcolumntype{g}{>{\columncolor{Gray}}c}
\newcolumntype{d}{D{.}{.}{-1}}
\DeclarePairedDelimiter\ceil{\lceil}{\rceil}
\DeclarePairedDelimiter\floor{\lfloor}{\rfloor}

\newcommand*\circled[1]{
  \tikz[baseline=(char.base)]{
    \node[shape=circle,draw,inner sep=2pt] (char) {#1};
  }
}

\setlength{\droptitle}{-10ex}

\preauthor{\begin{flushright}\large \lineskip 0.5em}
\postauthor{\par\end{flushright}}
\predate{\begin{flushright}\large}
\postdate{\par\end{flushright}}

\title{MAT 167 HW 1\vspace{-2ex}}
\author{Hardy Jones\\
        999397426\\
        Professor Cheer\vspace{-2ex}}
\date{Spring 2015}

\begin{document}
  \maketitle

  \begin{enumerate}
    \item [$\S$ 1.4]
      \begin{enumerate}
        \item [2]
          \[
            \begin{bmatrix}
              4 & 1 \\
              5 & 1 \\
              6 & 1 \\
            \end{bmatrix}
            \begin{bmatrix}
              1 \\
              3
            \end{bmatrix}
            =
            1
            \begin{bmatrix}
              4 \\
              5 \\
              6 \\
            \end{bmatrix}
            + 3
            \begin{bmatrix}
              1 \\
              1 \\
              1 \\
            \end{bmatrix}
            =
            \begin{bmatrix}
              4 \\
              5 \\
              6 \\
            \end{bmatrix}
            +
            \begin{bmatrix}
              3 \\
              3 \\
              3 \\
            \end{bmatrix}
            =
            \begin{bmatrix}
              7 \\
              8 \\
              9 \\
            \end{bmatrix}
          \]
        \item [7]
          \begin{enumerate}[label=(\alph*)]
            \item
              \[
                \begin{bmatrix}
                  1 & 0 & 0 \\
                  0 & 2 & 0 \\
                  0 & 0 & 3 \\
                \end{bmatrix}
              \]
            \item
              \[
                \begin{bmatrix}
                  1 & 0 & 0 \\
                  0 & 2 & 0 \\
                  0 & 0 & 3 \\
                \end{bmatrix}
              \]
            \item
              \[
                \begin{bmatrix}
                  1 & 0 & 0 \\
                  0 & 2 & 0 \\
                  0 & 0 & 3 \\
                \end{bmatrix}
              \]
            \item
              \[
                \begin{bmatrix}
                  0  & 1  & 0 \\
                  -1 & 0  & 2 \\
                  0  & -2 & 0 \\
                \end{bmatrix}
              \]
          \end{enumerate}
        \item [9]
          Assuming $A$ has as many pivots as rows, we have the following results.
          \begin{enumerate}[label=(\alph*)]
            \item $a_{11}$
            \item $l_{i1} = \frac{a_{i1}}{a_{11}}$
            \item $a_{ij} - a_{1j}\left(\frac{a_{i1}}{a_{11}}\right)$
            \item $a_{22} - a_{12}\left(\frac{a_{21}}{a_{11}}\right)$
          \end{enumerate}
        \item [10]
          \begin{enumerate}[label=(\alph*)]
            \item True.
            \item
              False.

              $AB$ may not even have three rows.

              For example, let $A$ be a $1 \times 3$ matrix and $B$ be a $3 \times 3$ matrix.

              Then $AB$ is a $1 \times 3$ matrix, so it has no third row.
            \item True.
            \item

              False.

              Let
              \[
                A = \begin{bmatrix}
                  1 & 2 \\
                  1 & 2 \\
                \end{bmatrix}
                ,
                B = \begin{bmatrix}
                  1 & 1 \\
                  0 & 0 \\
                \end{bmatrix}
              \]

              Then
              \begin{align*}
                (AB)^2
                &= \left(\begin{bmatrix}
                  1 & 2 \\
                  1 & 2 \\
                \end{bmatrix}
                \begin{bmatrix}
                  1 & 1 \\
                  0 & 0 \\
                \end{bmatrix}\right)^2 \\
                &= \begin{bmatrix}
                  1 & 1 \\
                  1 & 1 \\
                \end{bmatrix}^2 \\
                &= \begin{bmatrix}
                  2 & 2 \\
                  2 & 2 \\
                \end{bmatrix} \\
              \end{align*}

              But
              \begin{align*}
                A^2B^2
                &= \begin{bmatrix}
                  1 & 2 \\
                  1 & 2 \\
                \end{bmatrix}^2
                \begin{bmatrix}
                  1 & 1 \\
                  0 & 0 \\
                \end{bmatrix}^2 \\
                &= \begin{bmatrix}
                  3 & 6 \\
                  3 & 6 \\
                \end{bmatrix}
                \begin{bmatrix}
                  1 & 1 \\
                  0 & 0 \\
                \end{bmatrix} \\
                &= \begin{bmatrix}
                  3 & 3 \\
                  3 & 3 \\
                \end{bmatrix} \\
              \end{align*}

              So $(AB)^2 \ne A^2B^2$.
          \end{enumerate}
        \item [12]
          \begin{itemize}
            \item

              Let
              \[
                A = \begin{bmatrix}
                  1 & 0 \\
                  1 & 1 \\
                \end{bmatrix}
                ,
                B = \begin{bmatrix}
                  1 & 0 \\
                  2 & 1 \\
                \end{bmatrix}
              \]

              \begin{align*}
                AB
                &=
                \begin{bmatrix}
                  1 & 0 & 0 \\
                  1 & 1 & 0 \\
                  1 & 1 & 1 \\
                \end{bmatrix}
                \begin{bmatrix}
                  1 & 0 & 0 \\
                  2 & 1 & 0 \\
                  3 & 2 & 1 \\
                \end{bmatrix}
                \\
                &=
                \begin{bmatrix}
                  1 & 0 & 0 \\
                  3 & 1 & 0 \\
                  6 & 3 & 1 \\
                \end{bmatrix}
              \end{align*}

            \item

              \begin{proof}
                For any lower triangular matrices $A, B$ with dimension $n \times n$,
                each entry $ab_{ij}$ in $AB$ it is computed by:

                \[
                  \sum_{k = 1}^{n} a_{ik}b_{kj}
                \]

                If $i < k, a_{ik} = 0$.

                If $k < j, b_{kj} = 0$.

                Each entry above the main diagonal has one of either $i < k$ or $k < j$.

                So for each entry above the main diagonal of $AB$,
                we have a sum of products where at least one of the factors is 0.

                So, each entry above the main diagonal is 0.

                Thus, the product of any two lower triangular matrices is lower triangular.
              \end{proof}
          \end{itemize}
        \item [13]
        \item [24]
        \item [42]
        \item [46]
      \end{enumerate}
    \item [$\S$ 1.5]
      \begin{enumerate}
        \item [11]
        \item [12]
        \item [18]
        \item [22]
        \item [28]
        \item [33]
        \item [42]
      \end{enumerate}
    \item [$\S$ 1.6]
      \begin{enumerate}
        \item [2]
        \item [4]
        \item [5]
        \item [10]
        \item [17]
        \item [21]
        \item [40]
        \item [49]
      \end{enumerate}
  \end{enumerate}
\end{document}
