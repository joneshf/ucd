\documentclass[12pt,letterpaper]{article}
\usepackage{amsmath}
\usepackage{amsfonts}
\usepackage{amsthm}
\usepackage{mathtools}
\usepackage{cancel}
\usepackage[margin=1in]{geometry}
\usepackage{titling}
\usepackage{fp}
\usepackage{enumitem}
\usepackage[super]{nth}
\usepackage{dcolumn}
\usepackage[title]{appendix}
\usepackage{pgfplots}
\pgfplotsset{compat=1.8}
\usepgfplotslibrary{statistics}
\usepackage[round-mode=figures,round-precision=3,scientific-notation=false]{siunitx}
\usepackage{color, colortbl}
\usepackage{systeme}
\definecolor{Gray}{gray}{0.8}
\newcolumntype{g}{>{\columncolor{Gray}}c}
\newcolumntype{d}{D{.}{.}{-1}}
\DeclarePairedDelimiter\ceil{\lceil}{\rceil}
\DeclarePairedDelimiter\floor{\lfloor}{\rfloor}

\setlength{\droptitle}{-10ex}

\preauthor{\begin{flushright}\large \lineskip 0.5em}
\postauthor{\par\end{flushright}}
\predate{\begin{flushright}\large}
\postdate{\par\end{flushright}}

\title{MAT 167 HW 1\vspace{-2ex}}
\author{Hardy Jones\\
        999397426\\
        Professor Cheer\vspace{-2ex}}
\date{Spring 2015}

\begin{document}
  \maketitle

  \begin{enumerate}
    \item [$\S$ 1.4]
      \begin{enumerate}
        \item [2]

          \begin{itemize}
            \item
              After reducing to a triangle system we have:

              \sysdelim..
              \systeme{
                2x + 3y = 1,
                -6y = 6
              }

              So $y = -1$, back-substituting and solving for $x$ we get

              $2x + 3(-1) = 1 \implies 2x - 3 = 1 \implies 2x = 4 \implies x = 2$.

              So we have $x = 2, y = -1$.
            \item
              We verify that

              \[
                2 \begin{bmatrix}2 \\ 10\end{bmatrix} + (-1) \begin{bmatrix}3 \\ 9\end{bmatrix}
                = \begin{bmatrix}4 \\ 20\end{bmatrix} + \begin{bmatrix}-3 \\ -9\end{bmatrix}
                = \begin{bmatrix}1 \\ 11\end{bmatrix}
              \]
            \item
              If the right hand side changed to $\begin{bmatrix}4 \\ 44\end{bmatrix} = 4\begin{bmatrix}1 \\ 11\end{bmatrix}$,
              then the $x$ and $y$ values increase accordingly.

              That is $x = 4(2) = 8, y = 4(-1) = -4$
          \end{itemize}

        \item [7]
          \begin{enumerate}
            \item
              If $a = 2$, then elimination breaks down permanently.
              As we will end up in an inconsistent system.
            \item
              If $a = 0$, then elimination breaks down temporarily
              until the first equation is swapped for the second.
          \end{enumerate}

          We can solve by elimination if we swap the equations

          \sysdelim.\}
          \systeme{
            4x + 6y = 6,
            3y = -3
          }
          $\implies$
          \systeme{
            4x + 6y = 6,
            y = -1
          }
          $\implies$
          \systeme{
            4x = 12,
            y = -1
          }
          $\implies$
          \systeme{
            x = 3,
            y = -1
          }

          So, $x = 3, y = -1$.
        \item [9]

          \begin{itemize}
            \item These two equations have a solution only when $2b_1 = b_2$.
            \item There are infinitely many solutions.
            \item
              \begin{tikzpicture}
                \begin{axis}[axis equal, axis x line=middle, axis y line=middle, grid=major]
                  \addplot[-stealth]
                    coordinates { (0, 0) (3, 6) }
                    [xshift=15pt, yshift=-10pt]
                    node[pos=1] { (3, 6) }
                  ;
                  \addplot[-stealth]
                    coordinates { (0, 0) (-2, -4) }
                    [xshift=-15pt, yshift=10pt]
                    node[pos=1] { (-2, -4) }
                  ;
                  \addplot[only marks, mark=*] coordinates { (0, 0) };
                \end{axis}
              \end{tikzpicture}
          \end{itemize}
        \item [10]
        \item [12]
        \item [13]
        \item [24]
        \item [42]
        \item [46]
      \end{enumerate}
    \item [$\S$ 1.5]
      \begin{enumerate}
        \item [11]
        \item [12]
        \item [18]
        \item [22]
        \item [28]
        \item [33]
        \item [42]
      \end{enumerate}
    \item [$\S$ 1.6]
      \begin{enumerate}
        \item [2]
        \item [4]
        \item [5]
        \item [10]
        \item [17]
        \item [21]
        \item [40]
        \item [49]
      \end{enumerate}
  \end{enumerate}
\end{document}
