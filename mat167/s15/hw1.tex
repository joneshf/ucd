\documentclass[12pt,letterpaper]{article}
\usepackage{amsmath}
\usepackage{amsfonts}
\usepackage{amsthm}
\usepackage{mathtools}
\usepackage{cancel}
\usepackage[margin=1in]{geometry}
\usepackage{titling}
\usepackage{fp}
\usepackage{enumitem}
\usepackage[super]{nth}
\usepackage{dcolumn}
\usepackage[title]{appendix}
\usepackage{pgfplots}
\usepackage{tikz}
\pgfplotsset{compat=1.8}
\usepgfplotslibrary{statistics}
\usepackage[round-mode=figures,round-precision=3,scientific-notation=false]{siunitx}
\usepackage{color, colortbl}
\usepackage{systeme}
\definecolor{Gray}{gray}{0.8}
\newcolumntype{g}{>{\columncolor{Gray}}c}
\newcolumntype{d}{D{.}{.}{-1}}
\DeclarePairedDelimiter\ceil{\lceil}{\rceil}
\DeclarePairedDelimiter\floor{\lfloor}{\rfloor}

\newcommand*\circled[1]{
  \tikz[baseline=(char.base)]{
    \node[shape=circle,draw,inner sep=2pt] (char) {#1};
  }
}

\makeatletter
\renewcommand*\env@matrix[1][*\c@MaxMatrixCols c]{%
  \hskip -\arraycolsep
  \let\@ifnextchar\new@ifnextchar
  \array{#1}}
\makeatother

\setlength{\droptitle}{-10ex}

\preauthor{\begin{flushright}\large \lineskip 0.5em}
\postauthor{\par\end{flushright}}
\predate{\begin{flushright}\large}
\postdate{\par\end{flushright}}

\title{MAT 167 HW 1\vspace{-2ex}}
\author{Hardy Jones\\
        999397426\\
        Professor Cheer\vspace{-2ex}}
\date{Spring 2015}

\begin{document}
  \maketitle

  \begin{enumerate}
    \item [$\S$ 1.4]
      \begin{enumerate}
        \item [2]
          \[
            \begin{bmatrix}
              4 & 1 \\
              5 & 1 \\
              6 & 1 \\
            \end{bmatrix}
            \begin{bmatrix}
              1 \\
              3
            \end{bmatrix}
            =
            1
            \begin{bmatrix}
              4 \\
              5 \\
              6 \\
            \end{bmatrix}
            + 3
            \begin{bmatrix}
              1 \\
              1 \\
              1 \\
            \end{bmatrix}
            =
            \begin{bmatrix}
              4 \\
              5 \\
              6 \\
            \end{bmatrix}
            +
            \begin{bmatrix}
              3 \\
              3 \\
              3 \\
            \end{bmatrix}
            =
            \begin{bmatrix}
              7 \\
              8 \\
              9 \\
            \end{bmatrix}
          \]
        \item [7]
          \begin{enumerate}[label=(\alph*)]
            \item
              \[
                \begin{bmatrix}
                  1 & 0 & 0 \\
                  0 & 2 & 0 \\
                  0 & 0 & 3 \\
                \end{bmatrix}
              \]
            \item
              \[
                \begin{bmatrix}
                  1 & 0 & 0 \\
                  0 & 2 & 0 \\
                  0 & 0 & 3 \\
                \end{bmatrix}
              \]
            \item
              \[
                \begin{bmatrix}
                  1 & 0 & 0 \\
                  0 & 2 & 0 \\
                  0 & 0 & 3 \\
                \end{bmatrix}
              \]
            \item
              \[
                \begin{bmatrix}
                  0  & 1  & 0 \\
                  -1 & 0  & 2 \\
                  0  & -2 & 0 \\
                \end{bmatrix}
              \]
          \end{enumerate}
        \item [9]
          Assuming $A$ has as many pivots as rows, we have the following results.
          \begin{enumerate}[label=(\alph*)]
            \item $a_{11}$
            \item $l_{i1} = \frac{a_{i1}}{a_{11}}$
            \item $a_{ij} - a_{1j}\left(\frac{a_{i1}}{a_{11}}\right)$
            \item $a_{22} - a_{12}\left(\frac{a_{21}}{a_{11}}\right)$
          \end{enumerate}
        \item [10]
          \begin{enumerate}[label=(\alph*)]
            \item True.
            \item
              False.

              $AB$ may not even have three rows.

              For example, let $A$ be a $1 \times 3$ matrix and $B$ be a $3 \times 3$ matrix.

              Then $AB$ is a $1 \times 3$ matrix, so it has no third row.
            \item True.
            \item

              False.

              Let
              \[
                A = \begin{bmatrix}
                  1 & 2 \\
                  1 & 2 \\
                \end{bmatrix}
                ,
                B = \begin{bmatrix}
                  1 & 1 \\
                  0 & 0 \\
                \end{bmatrix}
              \]

              Then
              \begin{align*}
                (AB)^2
                &= \left(\begin{bmatrix}
                  1 & 2 \\
                  1 & 2 \\
                \end{bmatrix}
                \begin{bmatrix}
                  1 & 1 \\
                  0 & 0 \\
                \end{bmatrix}\right)^2 \\
                &= \begin{bmatrix}
                  1 & 1 \\
                  1 & 1 \\
                \end{bmatrix}^2 \\
                &= \begin{bmatrix}
                  2 & 2 \\
                  2 & 2 \\
                \end{bmatrix} \\
              \end{align*}

              But
              \begin{align*}
                A^2B^2
                &= \begin{bmatrix}
                  1 & 2 \\
                  1 & 2 \\
                \end{bmatrix}^2
                \begin{bmatrix}
                  1 & 1 \\
                  0 & 0 \\
                \end{bmatrix}^2 \\
                &= \begin{bmatrix}
                  3 & 6 \\
                  3 & 6 \\
                \end{bmatrix}
                \begin{bmatrix}
                  1 & 1 \\
                  0 & 0 \\
                \end{bmatrix} \\
                &= \begin{bmatrix}
                  3 & 3 \\
                  3 & 3 \\
                \end{bmatrix} \\
              \end{align*}

              So $(AB)^2 \ne A^2B^2$.
          \end{enumerate}
        \item [12]
          \begin{itemize}
            \item

              Let
              \[
                A = \begin{bmatrix}
                  1 & 0 \\
                  1 & 1 \\
                \end{bmatrix}
                ,
                B = \begin{bmatrix}
                  1 & 0 \\
                  2 & 1 \\
                \end{bmatrix}
              \]

              \begin{align*}
                AB
                &=
                \begin{bmatrix}
                  1 & 0 & 0 \\
                  1 & 1 & 0 \\
                  1 & 1 & 1 \\
                \end{bmatrix}
                \begin{bmatrix}
                  1 & 0 & 0 \\
                  2 & 1 & 0 \\
                  3 & 2 & 1 \\
                \end{bmatrix}
                \\
                &=
                \begin{bmatrix}
                  1 & 0 & 0 \\
                  3 & 1 & 0 \\
                  6 & 3 & 1 \\
                \end{bmatrix}
              \end{align*}

            \item

              \begin{proof}
                For any lower triangular matrices $A, B$ with dimension $n \times n$,
                each entry $ab_{ij}$ in $AB$ it is computed by:

                \[
                  \sum_{k = 1}^{n} a_{ik}b_{kj}
                \]

                If $i < k, a_{ik} = 0$.

                If $k < j, b_{kj} = 0$.

                Each entry above the main diagonal has one of either $i < k$ or $k < j$.

                So for each entry above the main diagonal of $AB$,
                we have a sum of products where at least one of the factors is 0.

                So, each entry above the main diagonal is 0.

                Thus, the product of any two lower triangular matrices is lower triangular.
              \end{proof}
          \end{itemize}
        \item [13]
          \begin{enumerate}[label=(\alph*)]
            \item
              Let
              \[
                A =
                \frac{1}{\sqrt{3}}
                \begin{bmatrix}
                  1  & 2 \\
                  -2 & -1 \\
                \end{bmatrix}
              \]

              \begin{align*}
                A^2
                &=
                \left(
                \frac{1}{\sqrt{3}}
                \begin{bmatrix}
                  1  & 2 \\
                  -2 & -1 \\
                \end{bmatrix}
                \right)^2
                \\
                &=
                \frac{1}{3}
                \begin{bmatrix}
                  -3 & 0 \\
                  0  & -3 \\
                \end{bmatrix}
                \\
                &=
                \begin{bmatrix}
                  -1 & 0 \\
                  0  & -1 \\
                \end{bmatrix}
                \\
                &=
                -I
                \\
              \end{align*}
            \item

              Let
              \[
                B
                =
                \begin{bmatrix}
                  0 & 0 \\
                  1 & 0
                \end{bmatrix}
                \ne
                0
              \]

              \[
                B^2
                =
                \begin{bmatrix}
                  0 & 0 \\
                  1 & 0
                \end{bmatrix}^2
                =
                \begin{bmatrix}
                  0 & 0 \\
                  0 & 0
                \end{bmatrix}
                =
                0
              \]
            \item

              Let
              \[
                C
                =
                \begin{bmatrix}
                  0 & 1 \\
                  1 & 0 \\
                \end{bmatrix}
                ,
                D
                =
                \begin{bmatrix}
                  0 & -1 \\
                  1 & 0 \\
                \end{bmatrix}
              \]

              \[
                CD
                =
                \begin{bmatrix}
                  0 & 1 \\
                  1 & 0 \\
                \end{bmatrix}
                \begin{bmatrix}
                  0 & -1 \\
                  1 & 0 \\
                \end{bmatrix}
                =
                \begin{bmatrix}
                  1 & 0 \\
                  0 & -1 \\
                \end{bmatrix}
              \]

              \[
                DC
                =
                \begin{bmatrix}
                  0 & -1 \\
                  1 & 0 \\
                \end{bmatrix}
                \begin{bmatrix}
                  0 & 1 \\
                  1 & 0 \\
                \end{bmatrix}
                =
                \begin{bmatrix}
                  -1 & 0 \\
                  0  & 1 \\
                \end{bmatrix}
              \]

              \item

                Let
                \[
                  E = F =
                  \begin{bmatrix}
                    2  & 1 \\
                    -4 & -2 \\
                  \end{bmatrix}
                \]

                \[
                  EF
                  =
                  \begin{bmatrix}
                    2  & 1 \\
                    -4 & -2 \\
                  \end{bmatrix}
                  \begin{bmatrix}
                    2  & 1 \\
                    -4 & -2 \\
                  \end{bmatrix}
                  =
                  \begin{bmatrix}
                    0 & 0 \\
                    0 & 0 \\
                  \end{bmatrix}
                \]
          \end{enumerate}
        \item [24]
          We want
          \[
            E_{21}
            =
            \begin{bmatrix}
              1  & 0 & 0 \\
              -4 & 1 & 0 \\
              0  & 0 & 1 \\
            \end{bmatrix}
            ,
            E_{31}
            =
            \begin{bmatrix}
              1  & 0 & 0 \\
              0  & 1 & 0 \\
              2  & 0 & 1 \\
            \end{bmatrix}
            ,
            E_{32}
            =
            \begin{bmatrix}
              1  & 0  & 0 \\
              0  & 1  & 0 \\
              0  & -2 & 1 \\
            \end{bmatrix}
          \]

          \begin{align*}
            M
            &=
            E_{32}E_{31}E_{21}
            \\
            &=
            \begin{bmatrix}
              1  & 0  & 0 \\
              0  & 1  & 0 \\
              0  & -2 & 1 \\
            \end{bmatrix}
            \begin{bmatrix}
              1  & 0 & 0 \\
              0  & 1 & 0 \\
              2  & 0 & 1 \\
            \end{bmatrix}
            \begin{bmatrix}
              1  & 0 & 0 \\
              -4 & 1 & 0 \\
              0  & 0 & 1 \\
            \end{bmatrix}
            \\
            &=
            \begin{bmatrix}
              1  & 0  & 0 \\
              0  & 1  & 0 \\
              2  & -2 & 1 \\
            \end{bmatrix}
            \begin{bmatrix}
              1  & 0 & 0 \\
              -4 & 1 & 0 \\
              0  & 0 & 1 \\
            \end{bmatrix}
            \\
            &=
            \begin{bmatrix}
              1  & 0  & 0 \\
              -4 & 1  & 0 \\
              10 & -2 & 1 \\
            \end{bmatrix}
            \\
          \end{align*}

          \[
            MA
            =
            \begin{bmatrix}
              1  & 0  & 0 \\
              -4 & 1  & 0 \\
              10 & -2 & 1 \\
            \end{bmatrix}
            \begin{bmatrix}
              1  & 1  & 0 \\
              4  & 6  & 1 \\
              -2 & 2  & 0 \\
            \end{bmatrix}
            =
            \begin{bmatrix}
              1  & 1  & 0  \\
              0  & 2  & 1  \\
              0  & 0  & -2 \\
            \end{bmatrix}
          \]
        \item [42]
          \begin{enumerate}[label=(\alph*)]
            \item
              True.

              Since matrix multiplication is only defined for matrices $A, B$ with dimension $m \times n, n \times p$, respectively,
              $A^2 = A*A$ must have $m = n = p$.
              That is $A$ must be a square matrix.
            \item
              False.

              We can choose $A, B$ with dimension $m \times n, n \times m$, respectively, where $m \ne n$.
              So $A$ and $B$ are not square.

              Then $AB$ is defined, as well as $BA$, yet $A$ and $B$ are not square.
            \item
              True.

              We can choose $A, B$ with dimension $m \times n, n \times m$, respectively.

              Then $AB$ is defined, as well as $BA$.

              These two products have dimension $m \times m$ and $n \times n$ respectively,
              so $AB$ and $BA$ are square.
            \item
              False.

              Let $B = 0$.

              Then $A0 = 0$ for all appropriate matrices, but $A$ is not necessarily $I$.
          \end{enumerate}
        \item [46]
      \end{enumerate}
    \item [$\S$ 1.5]
      \begin{enumerate}
        \item [11]

          Forward-substituting from $Lc = b$ gives

          \begin{itemize}
            \item $c_1 = 2$
            \item $2 + c_2 = 0 \implies c_2 = -2$
            \item $2 + c_3 = 2 \implies c_3 = 0$
          \end{itemize}

          Now Back-substituting from $Ux = c$ gives

          \begin{itemize}
            \item $w = 0$
            \item $v + 2(0) = -2 \implies v = -2$
            \item $2u + 4(-2) + 4(0) = 2 \implies 2u - 8 = 2 \implies u = 5$
          \end{itemize}

          So we have $u = 5, v = -2, w = 0$.
        \item [12]
          \begin{itemize}
            \item

              We could factor $A$ into $UL$ if we reduced the rows with the pivots below all 0's.
              That is, we'd want to form the lower triangular factor first,
              then perform elementary row operations to create the upper triangular factor.

            \item
              No, these two decompositions do not necessarily produce the same factors,
              as $LU$-decomposition in general is not unique.
          \end{itemize}
        \item [18]
          We can solve the first two at the same time.
          \begin{itemize}
            \item
              Let
              \[
                A
                =
                \begin{bmatrix}
                  0  &  1 & -1 \\
                  1  & -1 &  0 \\
                  1  &  0 & -1 \\
                \end{bmatrix}
              \]

              Then if we attempt to reduce $A$ to an upper triangular we get
              \begin{align*}
                \begin{bmatrix}
                  0  &  1 & -1 \\
                  1  & -1 &  0 \\
                  1  &  0 & -1 \\
                \end{bmatrix}
                &=
                \begin{bmatrix}
                  1  & -1 &  0 \\
                  0  &  1 & -1 \\
                  1  &  0 & -1 \\
                \end{bmatrix}
                \\
                &=
                \begin{bmatrix}
                  1  & -1 &  0 \\
                  0  &  1 & -1 \\
                  0  &  1 & -1 \\
                \end{bmatrix}
                \\
                &=
                \begin{bmatrix}
                  1  & -1 &  0 \\
                  0  &  1 & -1 \\
                  0  &  0 &  0 \\
                \end{bmatrix}
                \\
              \end{align*}

              But this matrix is singular, so it has no solutions.

            \item

              Let
              \[
                A
                =
                \begin{bmatrix}
                  0  &  1 &  1 \\
                  1  &  1 &  0 \\
                  1  &  0 &  1 \\
                \end{bmatrix}
              \]

              Then if we attempt to reduce $A$ to an upper triangular we get
              \begin{align*}
                A
                &=
                \begin{bmatrix}
                  0  &  1 &  1 \\
                  1  &  1 &  0 \\
                  1  &  0 &  1 \\
                \end{bmatrix}
                \\
                E_1A
                &=
                \begin{bmatrix}
                   0 &  1 &  0 \\
                   1 &  0 &  0 \\
                   0 &  0 &  1 \\
                \end{bmatrix}
                A
                &=
                \begin{bmatrix}
                   1 &  1 &  0 \\
                   0 &  1 &  1 \\
                   1 &  0 &  1 \\
                \end{bmatrix}
                \\
                E_2E_1A
                &=
                \begin{bmatrix}
                   1 &  0 &  0 \\
                   0 &  1 &  0 \\
                  -1 &  0 &  1 \\
                \end{bmatrix}
                \begin{bmatrix}
                   0 &  1 &  0 \\
                   1 &  0 &  0 \\
                   0 &  0 &  1 \\
                \end{bmatrix}
                A
                &=
                \begin{bmatrix}
                   1 &  1 &  0 \\
                   0 &  1 &  1 \\
                   0 & -1 &  1 \\
                \end{bmatrix}
                \\
                E_3E_2E_1A
                &=
                \begin{bmatrix}
                   1 &  0 &  0 \\
                   0 &  1 &  0 \\
                   0 &  1 &  1 \\
                \end{bmatrix}
                \begin{bmatrix}
                   1 &  0 &  0 \\
                   0 &  1 &  0 \\
                  -1 &  0 &  1 \\
                \end{bmatrix}
                \begin{bmatrix}
                   0 &  1 &  0 \\
                   1 &  0 &  0 \\
                   0 &  0 &  1 \\
                \end{bmatrix}
                A
                &=
                \begin{bmatrix}
                   1 &  1 &  0 \\
                   0 &  1 &  1 \\
                   0 &  0 &  2 \\
                \end{bmatrix}
                \\
              \end{align*}

              Since $E_1$ is a permutation matrix,
              we have $E_1 = P \implies PA = LU$.

              Now we have

              \begin{align*}
                L &= E_2^{-1}E_3^{-1} \\
                &=
                \begin{bmatrix}
                   1 &  0 &  0 \\
                   0 &  1 &  0 \\
                   1 &  0 &  1 \\
                \end{bmatrix}
                \begin{bmatrix}
                   1 &  0 &  0 \\
                   0 &  1 &  0 \\
                   0 & -1 &  1 \\
                \end{bmatrix}
                \\
                &=
                \begin{bmatrix}
                   1 &  0 &  0 \\
                   0 &  1 &  0 \\
                   1 & -1 &  1 \\
                \end{bmatrix}
              \end{align*}

              So A is nonsingular, in fact we have one unique solution.

              We take $PAx = Pb = y$ and solve with our $LU$ factors.

              \[
                y
                =
                \begin{bmatrix}
                   0 &  1 &  0 \\
                   1 &  0 &  0 \\
                   0 &  0 &  1 \\
                \end{bmatrix}
                \begin{bmatrix}
                   1 \\
                   1 \\
                   1 \\
                \end{bmatrix}
                =
                \begin{bmatrix}
                   1 \\
                   1 \\
                   1 \\
                \end{bmatrix}
              \]

              We can compute the $c$ matrix with $Lc = y$.
              \begin{itemize}
                \item $c_1 = 1$
                \item $c_2 = 1$
                \item $1 - 1 + c_3 = 1 \implies c_3 = 1$
              \end{itemize}

              We can compute the $x$ matrix with $Ux = c$.
              \begin{itemize}
                \item $2w              = 1 \implies w = \frac{1}{2}$
                \item $v + \frac{1}{2} = 1 \implies v = \frac{1}{2}$
                \item $u + \frac{1}{2} = 1 \implies u = \frac{1}{2}$
              \end{itemize}

              Finally, we have our solution: $u = \frac{1}{2}, v = \frac{1}{2}, w = \frac{1}{2}$.
          \end{itemize}
        \item [22]
          The elementary operations we performed were
          \[
            E_1 =
            \begin{bmatrix}
               1 &  0 &  0 \\
              -1 &  1 &  0 \\
              -1 &  0 &  1 \\
            \end{bmatrix}
            ,
            E_2 =
            \begin{bmatrix}
               1 &  0 &  0 \\
               0 &  1 &  0 \\
               0 & -2 &  1 \\
            \end{bmatrix}
          \]

          \begin{itemize}
            \item
              So
              \[
                L
                =
                E_1^{-1} E_2^{-1}
                =
                \begin{bmatrix}
                   1 &  0 &  0 \\
                   1 &  1 &  0 \\
                   1 &  0 &  1 \\
                \end{bmatrix}
                \begin{bmatrix}
                   1 &  0 &  0 \\
                   0 &  1 &  0 \\
                   0 &  2 &  1 \\
                \end{bmatrix}
                =
                \begin{bmatrix}
                   1 &  0 &  0 \\
                   1 &  1 &  0 \\
                   1 &  2 &  1 \\
                \end{bmatrix}
              \]
              We have
              \begin{align*}
                Lc &= b \\
                \begin{bmatrix}
                   1 &  0 &  0 \\
                   1 &  1 &  0 \\
                   1 &  2 &  1 \\
                \end{bmatrix}
                \begin{bmatrix}
                  c_1 \\
                  c_2 \\
                  c_3 \\
                \end{bmatrix}
                &=
                \begin{bmatrix}
                   5 \\
                   7 \\
                  11 \\
                \end{bmatrix}
              \end{align*}
              and
              \begin{align*}
                Ux &= c \\
                \begin{bmatrix}
                   1 &  1 & 1 \\
                   0 &  1 & 2 \\
                   0 &  0 & 1 \\
                \end{bmatrix}
                \begin{bmatrix}
                  x \\
                  y \\
                  z \\
                \end{bmatrix}
                &=
                \begin{bmatrix}
                  c_1 \\
                  c_2 \\
                  c_3 \\
                \end{bmatrix}
              \end{align*}
            \item

              Plugging in the values for $c$ we compute:
              \[
                \begin{bmatrix}
                   1 &  0 &  0 \\
                   1 &  1 &  0 \\
                   1 &  2 &  1 \\
                \end{bmatrix}
                \begin{bmatrix}
                  5 \\
                  2 \\
                  2 \\
                \end{bmatrix}
                =
                5
                \begin{bmatrix}
                  1 \\
                  1 \\
                  1 \\
                \end{bmatrix}
                +
                2
                \begin{bmatrix}
                   0 \\
                   1 \\
                   2 \\
                \end{bmatrix}
                +
                2
                \begin{bmatrix}
                   0 \\
                   0 \\
                   1 \\
                \end{bmatrix}
                =
                \begin{bmatrix}
                  5 \\
                  5 \\
                  5 \\
                \end{bmatrix}
                +
                \begin{bmatrix}
                   0 \\
                   2 \\
                   4 \\
                \end{bmatrix}
                +
                \begin{bmatrix}
                   0 \\
                   0 \\
                   2 \\
                \end{bmatrix}
                =
                \begin{bmatrix}
                  5  \\
                  7  \\
                  11 \\
                \end{bmatrix}
              \]
              So this $c$ solves the first system.
            \item
              We compute
              \begin{itemize}
                \item $              z = 2$
                \item $    y    + 2(2) = 2 \implies y + 4 = 2 \implies y = -2$
                \item $x + (-2) + 2    = 5 \implies x = 5$
              \end{itemize}
              So the $x$ that solves the second system is:

              \[
                \begin{bmatrix}
                  x \\
                  y \\
                  z \\
                \end{bmatrix}
                =
                \begin{bmatrix}
                   5 \\
                  -2 \\
                   2 \\
                \end{bmatrix}
              \]
          \end{itemize}
        \item [28]
          \begin{align*}
            A
            &=
            \begin{bmatrix}
               2 &  4 \\
               4 & 11 \\
            \end{bmatrix}
            \\
            E_{A0}A
            &=
            \begin{bmatrix}
               1 &  0 \\
              -2 &  1 \\
            \end{bmatrix}
            A
            &=
            \begin{bmatrix}
               2 &  4 \\
               0 &  3 \\
            \end{bmatrix}
            \\
            E_{A0}A
            &=
            \begin{bmatrix}
               1 &  0 \\
              -2 &  1 \\
            \end{bmatrix}
            A
            &=
            \begin{bmatrix}
               2 &  0 \\
               0 &  3 \\
            \end{bmatrix}
            \begin{bmatrix}
               1 &  2 \\
               0 &  1 \\
            \end{bmatrix}
            \\
          \end{align*}

          So
          \[
            L_A
            =
            E_{A0}^{-1}
            =
            \begin{bmatrix}
               1 &  0 \\
               2 &  1 \\
            \end{bmatrix}
            ,
            D_A
            =
            \begin{bmatrix}
               2 &  0 \\
               0 &  3 \\
            \end{bmatrix}
            ,
            U_A
            =
            \begin{bmatrix}
               1 &  2 \\
               0 &  1 \\
            \end{bmatrix}
          \]

          \begin{align*}
            B
            &=
            \begin{bmatrix}
               1 &  4 &  0 \\
               4 & 12 &  4 \\
               0 &  4 &  0 \\
            \end{bmatrix}
            \\
            E_{B0}B
            &=
            \begin{bmatrix}
               1 &  0 &  0 \\
              -4 &  1 &  0 \\
               0 &  0 &  1 \\
            \end{bmatrix}
            B
            &=
            \begin{bmatrix}
               1 &  4 &  0 \\
               0 & -4 &  4 \\
               0 &  4 &  0 \\
            \end{bmatrix}
            \\
            E_{B1}E_{B0}B
            &=
            \begin{bmatrix}
               1 &  0 &  0 \\
               0 &  1 &  0 \\
               0 &  1 &  1 \\
            \end{bmatrix}
            \begin{bmatrix}
               1 &  0 &  0 \\
              -4 &  1 &  0 \\
               0 &  0 &  1 \\
            \end{bmatrix}
            B
            &=
            \begin{bmatrix}
               1 &  4 &  0 \\
               0 & -4 &  4 \\
               0 &  0 &  4 \\
            \end{bmatrix}
            \\
            E_{B1}E_{B0}B
            &=
            \begin{bmatrix}
               1 &  0 &  0 \\
               0 &  1 &  0 \\
               0 &  1 &  1 \\
            \end{bmatrix}
            \begin{bmatrix}
               1 &  0 &  0 \\
              -4 &  1 &  0 \\
               0 &  0 &  1 \\
            \end{bmatrix}
            B
            &=
            \begin{bmatrix}
               1 &  0 &  0 \\
               0 & -4 &  0 \\
               0 &  0 &  4 \\
            \end{bmatrix}
            \begin{bmatrix}
               1 &  4 &  0 \\
               0 &  1 & -1 \\
               0 &  0 &  1 \\
            \end{bmatrix}
            \\
          \end{align*}

          So
          \[
            L
            =
            E_{B0}^{-1}E_{B1}^{-1}
            =
            \begin{bmatrix}
               1 &  0 &  0 \\
               4 &  1 &  0 \\
               0 &  0 &  1 \\
            \end{bmatrix}
            \begin{bmatrix}
               1 &  0 &  0 \\
               0 &  1 &  0 \\
               0 & -1 &  1 \\
            \end{bmatrix}
            =
            \begin{bmatrix}
               1 &  0 &  0 \\
               4 &  1 &  0 \\
               0 & -1 &  1 \\
            \end{bmatrix}
            ,
          \]
          \[
            D
            =
            \begin{bmatrix}
               1 &  0 &  0 \\
               0 & -4 &  0 \\
               0 &  0 &  4 \\
            \end{bmatrix}
            ,
            U
            =
            \begin{bmatrix}
               1 &  4 &  0 \\
               0 &  1 & -1 \\
               0 &  0 &  1 \\
            \end{bmatrix}
          \]

          For these symmetric matrices, $L = U^T$
        \item [33]
          We find $c$
          \begin{itemize}
            \item $c_1             = 4$
            \item $4   + c_2       = 5 \implies c_2 = 1$
            \item $4   + 1   + c_3 = 6 \implies c_3 = 1$
          \end{itemize}

          So we have
          \[
            \begin{bmatrix}
              c_1 \\
              c_2 \\
              c_3 \\
            \end{bmatrix}
            =
            \begin{bmatrix}
              4 \\
              1 \\
              1 \\
            \end{bmatrix}
          \]

          Now we find $x$

          \begin{itemize}
            \item $            x_3 = 1$
            \item $      x_2 + 1   = 1 \implies x_2 = 0$
            \item $x_1 + 0   + 1   = 4 \implies x_1 = 3$
          \end{itemize}

          So we have
          \[
            \begin{bmatrix}
              x_1 \\
              x_2 \\
              x_3 \\
            \end{bmatrix}
            =
            \begin{bmatrix}
              3 \\
              0 \\
              1 \\
            \end{bmatrix}
          \]

          Finally,

          \[
            A
            =
            \begin{bmatrix}
               1 &  0 &  0 \\
               1 &  1 &  0 \\
               1 &  1 &  1 \\
            \end{bmatrix}
            \begin{bmatrix}
               1 &  1 &  1 \\
               1 &  1 &  1 \\
               1 &  0 &  1 \\
            \end{bmatrix}
            =
            \begin{bmatrix}
               1 &  1 &  1 \\
               1 &  2 &  2 \\
               1 &  2 &  3 \\
            \end{bmatrix}
          \]
        \item [42]
          \begin{itemize}
            \item
              Choose
              \[
                P_1
                =
                \begin{bmatrix}
                  1 & 0 & 0 \\
                  0 & 0 & 1 \\
                  0 & 1 & 0 \\
                \end{bmatrix}
                ,
                P_2
                =
                \begin{bmatrix}
                  0 & 1 & 0 \\
                  1 & 0 & 0 \\
                  0 & 0 & 1 \\
                \end{bmatrix}
              \]

              Then
              \[
                P_1 P_2
                =
                \begin{bmatrix}
                  0 & 1 & 0 \\
                  0 & 0 & 1 \\
                  1 & 0 & 0 \\
                \end{bmatrix}
                \ne
                \begin{bmatrix}
                  0 & 0 & 1 \\
                  1 & 0 & 0 \\
                  0 & 1 & 0 \\
                \end{bmatrix}
                =
                P_2 P_1
              \]
            \item
              Choose
              \[
                P_3
                =
                \begin{bmatrix}
                  1 & 0 \\
                  0 & 1 \\
                \end{bmatrix}
                ,
                P_4
                =
                \begin{bmatrix}
                  0 & 1 \\
                  1 & 0 \\
                \end{bmatrix}
              \]

              Then

              \[
                P_3 P_4
                =
                \begin{bmatrix}
                  0 & 1 \\
                  1 & 0 \\
                \end{bmatrix}
                =
                \begin{bmatrix}
                  0 & 1 \\
                  1 & 0 \\
                \end{bmatrix}
                =
                P_4 P_3
              \]
          \end{itemize}
      \end{enumerate}
    \item [$\S$ 1.6]
      \begin{enumerate}
        \item [2]
          \begin{enumerate}
            \item For all permutation matrices $P$, $P^{-1} = P^T$.

              So the first $P^{-1}$ is
              \[
                \begin{bmatrix}
                  0 & 0 & 1 \\
                  0 & 1 & 0 \\
                  1 & 0 & 0 \\
                \end{bmatrix}
              \]
              The second $P^{-1}$ is
              \[
                \begin{bmatrix}
                  0 & 1 & 0 \\
                  0 & 0 & 1 \\
                  1 & 0 & 0 \\
                \end{bmatrix}
              \]
            \item
              Intuitively we can think of $P^T$ as performing the inverse permutation of $P$.

              % \begin{proof}
              %   For any permutation matrix $P$ of dimension $n \times n$
              %   We can construct each $(PP^T)_{ij}$ as
              %   \[
              %     \sum_{k = 1}^{n} p_{ik}p_{kj}^T
              %   \]

              %   Since $P$ is a permutation matrix,
              %   each product in the sum consists of factors of either $0$ or $1$.
              %   $p_{ik} = 1$ when
              % \end{proof}
          \end{enumerate}
        \item [4]
          \begin{enumerate}
            \item
              \[
                AB = AC \implies A^{-1}AB = A^{-1}AC \implies B = C
              \]
            \item
              Choose
              \[
                B =
                \begin{bmatrix}
                  1 & 0 \\
                  0 & 1 \\
                \end{bmatrix}
                ,
                C =
                \begin{bmatrix}
                  1 & 0 \\
                  0 & 2 \\
                \end{bmatrix}
              \]
              Then
              \[
                AB =
                \begin{bmatrix}
                  1 & 0 \\
                  0 & 0 \\
                \end{bmatrix}
                \begin{bmatrix}
                  1 & 0 \\
                  0 & 1 \\
                \end{bmatrix}
                =
                \begin{bmatrix}
                  1 & 0 \\
                  0 & 0 \\
                \end{bmatrix}
                =
                \begin{bmatrix}
                  1 & 0 \\
                  0 & 0 \\
                \end{bmatrix}
                \begin{bmatrix}
                  1 & 0 \\
                  0 & 2 \\
                \end{bmatrix}
                =
                AC
              \]
               But $B \ne C$.
          \end{enumerate}
        \item [5]
          \begin{proof}
            Let $(A^2)^{-1} = B$

            \begin{align*}
              (A^2)^{-1} &= B \\
              (AA)^{-1} &= \\
              A^{-1}A^{-1} &= \\
              A^{-1} &= AB \\
            \end{align*}

            Thus, the inverse of $A$ is $AB$.
          \end{proof}
        \item [10]
          \begin{enumerate}[label=(\alph*)]
            \item

              \begin{align*}
                \begin{bmatrix}[c c c c | c c c c]
                   0 &  0 &  0 &  1 &  1 &  0 &  0 &  0 \\
                   0 &  0 &  2 &  0 &  0 &  1 &  0 &  0 \\
                   0 &  3 &  0 &  0 &  0 &  0 &  1 &  0 \\
                   4 &  0 &  0 &  0 &  0 &  0 &  0 &  1 \\
                \end{bmatrix}
                &=
                \begin{bmatrix}[c c c c | c c c c]
                   4 &  0 &  0 &  0 &  0 &  0 &  0 &  1 \\
                   0 &  3 &  0 &  0 &  0 &  0 &  1 &  0 \\
                   0 &  0 &  2 &  0 &  0 &  1 &  0 &  0 \\
                   0 &  0 &  0 &  1 &  1 &  0 &  0 &  0 \\
                \end{bmatrix}
                \\
                &=
                \begin{bmatrix}[c c c c | c c c c]
                    1 &  0 &  0 &  0 &  \frac{1}{4} &  0           &  0           &  0           \\
                    0 &  1 &  0 &  0 &  0           &  \frac{1}{3} &  0           &  0           \\
                    0 &  0 &  1 &  0 &  0           &  0           &  \frac{1}{2} &  0           \\
                    0 &  0 &  0 &  1 &  0           &  0           &  0           &  \frac{1}{1} \\
                \end{bmatrix}
                \\
              \end{align*}

              So

              \[
                A_1^{-1}
                =
                \begin{bmatrix}
                  \frac{1}{4} &  0           &  0           &  0           \\
                  0           &  \frac{1}{3} &  0           &  0           \\
                  0           &  0           &  \frac{1}{2} &  0           \\
                  0           &  0           &  0           &  \frac{1}{1} \\
                \end{bmatrix}
              \]
            \item
              \begin{align*}
                \begin{bmatrix}[c c c c | c c c c]
                   1            &  0            &  0            &  0 &  1 &  0 &  0 &  0 \\
                   -\frac{1}{2} &  1            &  0            &  0 &  0 &  1 &  0 &  0 \\
                   0            &  -\frac{2}{3} &  1            &  0 &  0 &  0 &  1 &  0 \\
                   0            &  0            &  -\frac{3}{4} &  1 &  0 &  0 &  0 &  1 \\
                \end{bmatrix}
                &=
                \begin{bmatrix}[c c c c | c c c c]
                   1            &  0            &  0            &  0 &  1 &  0 &  0 &  0 \\
                   0            &  1            &  0            &  0 &  2 &  1 &  0 &  0 \\
                   0            &  -\frac{2}{3} &  1            &  0 &  0 &  0 &  1 &  0 \\
                   0            &  0            &  -\frac{3}{4} &  1 &  0 &  0 &  0 &  1 \\
                \end{bmatrix}
                \\
                &=
                \begin{bmatrix}[c c c c | c c c c]
                   1            &  0            &  0            &  0 &  1 &  0           &  0 &  0 \\
                   0            &  1            &  0            &  0 &  2 &  1           &  0 &  0 \\
                   0            &  0            &  1            &  0 &  3 &  \frac{3}{2} &  1 &  0 \\
                   0            &  0            &  -\frac{3}{4} &  1 &  0 &  0           &  0 &  1 \\
                \end{bmatrix}
                \\
                &=
                \begin{bmatrix}[c c c c | c c c c]
                   1            &  0            &  0            &  0 &  1           &  0           &  0           &  0 \\
                   0            &  1            &  0            &  0 &  2           &  1           &  0           &  0 \\
                   0            &  0            &  1            &  0 &  3           &  \frac{3}{2} &  1           &  0 \\
                   0            &  0            &  0            &  1 &  \frac{9}{4} &  \frac{9}{8} &  \frac{3}{4} &  1 \\
                \end{bmatrix}
                \\
              \end{align*}

              So

              \[
                A_2^{-1}
                =
                \begin{bmatrix}
                  1           &  0           &  0           &  0 \\
                  2           &  1           &  0           &  0 \\
                  3           &  \frac{3}{2} &  1           &  0 \\
                  \frac{9}{4} &  \frac{9}{8} &  \frac{3}{4} &  1 \\
                \end{bmatrix}
              \]
            \item
              \begin{align*}
                \begin{bmatrix}[c c c c | c c c c]
                   a &  b &  0 &  0 &  1 &  0 &  0 &  0 \\
                   c &  d &  0 &  0 &  0 &  1 &  0 &  0 \\
                   0 &  0 &  a &  b &  0 &  0 &  1 &  0 \\
                   0 &  0 &  c &  d &  0 &  0 &  0 &  1 \\
                \end{bmatrix}
                &=
                \begin{bmatrix}[c c c c | c c c c]
                   a &  b               &  0 &  0               &  1           &  0 &  0           &  0 \\
                   0 &  \frac{ad-bc}{a} &  0 &  0               & -\frac{c}{a} &  1 &  0           &  0 \\
                   0 &  0               &  a &  b               &  0           &  0 &  1           &  0 \\
                   0 &  0               &  0 &  \frac{ad-bc}{a} &  0           &  0 & -\frac{c}{a} &  1 \\
                \end{bmatrix}
                \\
                &=
                \begin{bmatrix}[c c c c | c c c c]
                   a &  0               &  0 &  0               &  \frac{ad}{ad - bc} &  -\frac{ab}{ad - bc} &  0                  &  0                   \\
                   0 &  \frac{ad-bc}{a} &  0 &  0               & -\frac{c}{a}        &  1                   &  0                  &  0                   \\
                   0 &  0               &  a &  0               &  0                  &  0                   &  \frac{ad}{ad - bc} &  -\frac{ab}{ad - bc} \\
                   0 &  0               &  0 &  \frac{ad-bc}{a} &  0                  &  0                   & -\frac{c}{a}        &  1                   \\
                \end{bmatrix}
                \\
                &=
                \begin{bmatrix}[c c c c | c c c c]
                   1 &  0 &  0 &  0 &  \frac{d}{ad - bc} &  -\frac{b}{ad - bc} &  0                 &  0                  \\
                   0 &  1 &  0 &  0 & -\frac{c}{ad-bc}   &  \frac{a}{ad-bc}    &  0                 &  0                  \\
                   0 &  0 &  1 &  0 &  0                 &  0                  &  \frac{d}{ad - bc} &  -\frac{b}{ad - bc} \\
                   0 &  0 &  0 &  1 &  0                 &  0                  & -\frac{c}{ad-bc}   &  \frac{a}{ad-bc}    \\
                \end{bmatrix}
                \\
              \end{align*}

              So

              \[
                A_3^{-1}
                =
                \frac{1}{ad - bc}
                \begin{bmatrix}
                   d & -b &  0 &  0 \\
                  -c &  a &  0 &  0 \\
                   0 &  0 &  d & -b \\
                   0 &  0 & -c &  a \\
                \end{bmatrix}
              \]
          \end{enumerate}
        \item [17]
          \begin{enumerate}[label=(\alph*)]
            \item
              \begin{align*}
                A &= A \\
                L_2D_2U_2 &= L_1D_1U_1 \\
                L_1^{-1}L_2D_2U_2 &= D_1U_1 \\
                L_1^{-1}L_2D_2 &= D_1U_1U_2^{-1} \\
              \end{align*}

              Inverting a lower triangular matrix gives another lower triangular matrix.
              Inverting an upper triangular matrix gives another upper triangular matrix.

              Multiplication of one type of triangular matrix
              by a triangular matrix of the same type gives another triangular matrix of the same type.

              Since diagonal matrices are trivially both upper and lower triangular,
              the left side is a lower triangular matrix
              while the right side is an upper triangular matrix.
            \item
              For the equation $L_1^{-1}L_2D_2 = D_1U_1U_2^{-1}$ to be true,
              both sides must be diagonal matrices.
              This is because the left side is a lower triangular matrix
              and the right side is an upper triangular matrix.
              The only possible matrix of this type is a diagonal matrix.

              The main diagonal of the left hand side must be the same as the diagonal of $D_2$,
              as the triangular matrix on the left hand side has all 1's on the main diagonal.
              The main diagonal of the right hand side must be the same as the diagonal of $D_1$,
              as the triangular matrix on the right hand side has all 1's on the main diagonal.

              The elements off the diagonal must be all 0 since both sides are diagonal matrices.

              So we have that the left side is equal to $D_2$ and
              the right side is equal to $D_1$, thus $D_2 = D_1$.

              Since the left side is equal to $D_2$, we have:

              \[
                L_1^{-1}L_2D_2 = D_2 \implies L_1^{-1}L_2 = I \implies L_2 = L_1
              \]

              Since the left side is equal to $D_1$, we have:

              \[
                D_1U_1U_2^{-1} = D_1 \implies U_1U_2^{-1} = I \implies U_1 = U_2
              \]

              So we have shown if $A = L_1D_1U_1$ and $A = L_2D_2U_2$,
              $L_1 = L_2$, $D_1 = D_2$, $U_1 = U_2$.

              Thus the factorization is unique.
          \end{enumerate}
        \item [21]
          \begin{proof}
            Let $A, B$ be square matrices with $I - AB$ invertible.

            \begin{align*}
              B(I - AB) &= (I - BA)B \\
              B &= (I - BA)B(I - AB)^{-1} \\
              I &= (I - BA)B(I - AB)^{-1}B^{-1} \\
              (I - BA)^{-1} &= B(I - AB)^{-1}B^{-1} \\
            \end{align*}

            Then $(I - BA)$ is invertible.
          \end{proof}
        \item [40]
          \begin{enumerate}[label=(\alph*)]
            \item True. The matrix has less pivots than 4, and so is singular.
            \item

              False.

              Let
              \[
                A
                =
                \begin{bmatrix}
                  1 & 1 \\
                  1 & 1 \\
                \end{bmatrix}
              \]

              A has 1's down the main diagonal,
              but is singular and so not invertible.
            \item True. The inverse of $A^{-1}$ is $A$.
            \item True.
              \begin{proof}
                Given $A^T$ is invertible we have:

                \begin{align*}
                  A^{-1}A &= \left({A^T}^T\right)^{-1}\left({A^T}^T\right) \\
                  &= \left(\left({A^T}\right)^{-1}\right)^T\left({A^T}^T\right) \\
                  &= \left({A^T}\left({A^T}\right)^{-1}\right)^T \\
                  &= I^T \\
                  &= I \\
                \end{align*}

                and

                \begin{align*}
                  AA^{-1} &= \left({A^T}^T\right)\left({A^T}^T\right)^{-1} \\
                  &= \left({A^T}^T\right)\left(\left({A^T}\right)^{-1}\right)^T \\
                  &= \left(\left({A^T}\right)^{-1}{A^T}\right)^T \\
                  &= I^T \\
                  &= I \\
                \end{align*}

                So $A$ is invertible.
              \end{proof}
          \end{enumerate}
        \item [49]
          \begin{enumerate}
            \item

              \[
                A
                =
                \begin{bmatrix}
                   1 &  0 \\
                   9 &  3 \\
                \end{bmatrix}
              \]

              \[
                A^T
                =
                \begin{bmatrix}
                   1 &  9 \\
                   0 &  3 \\
                \end{bmatrix}
              \]
              \[
                A^{-1}
                =
                \frac{1}{1(3) - 9(0)}
                \begin{bmatrix}
                  -1 &  9 \\
                   0 & -3 \\
                \end{bmatrix}
                =
                \frac{1}{3}
                \begin{bmatrix}
                  -1 &  9 \\
                   0 & -3 \\
                \end{bmatrix}
              \]
              \[
                \left(A^{-1}\right)^T
                =
                \left(A^T\right)^{-1}
                =
                \frac{1}{3}
                \begin{bmatrix}
                  -1 &  0 \\
                   9 & -3 \\
                \end{bmatrix}
              \]
            \item
              \[
                A
                =
                \begin{bmatrix}
                   1 &  c \\
                   c &  0 \\
                \end{bmatrix}
              \]

              \[
                A^T
                =
                \begin{bmatrix}
                   1 &  c \\
                   c &  0 \\
                \end{bmatrix}
              \]
              \[
                A^{-1}
                =
                \frac{1}{1(0) - c(c)}
                \begin{bmatrix}
                  -1 &  c \\
                   c & -0 \\
                \end{bmatrix}
                =
                \frac{1}{c^2}
                \begin{bmatrix}
                  -1 &  c \\
                   c & -0 \\
                \end{bmatrix}
                , \text{ when } c \ne 0
              \]
              \[
                \left(A^{-1}\right)^T
                =
                \left(A^T\right)^{-1}
                =
                \frac{1}{c^2}
                \begin{bmatrix}
                  -1 &  c \\
                   c & -0 \\
                \end{bmatrix}
                , \text{ when } c \ne 0
              \]
          \end{enumerate}
      \end{enumerate}
  \end{enumerate}
\end{document}
