\documentclass[12pt,letterpaper]{article}
\usepackage{amsmath}
\usepackage{amsfonts}
\usepackage{amsthm}
\usepackage{mathtools}
\usepackage{cancel}
\usepackage[margin=1in]{geometry}
\usepackage{titling}
\usepackage{fp}
\usepackage{enumitem}
\usepackage[super]{nth}
\usepackage{dcolumn}
\usepackage[title]{appendix}
\usepackage{pgfplots}
\usepackage{tikz}
\pgfplotsset{compat=1.8}
\usepgfplotslibrary{statistics}
\usepackage[round-mode=figures,round-precision=3,scientific-notation=false]{siunitx}
\usepackage{color, colortbl}
\usepackage{systeme}
\definecolor{Gray}{gray}{0.8}
\newcolumntype{g}{>{\columncolor{Gray}}c}
\newcolumntype{d}{D{.}{.}{-1}}
\DeclarePairedDelimiter\ceil{\lceil}{\rceil}
\DeclarePairedDelimiter\floor{\lfloor}{\rfloor}

\newcommand*\circled[1]{
  \tikz[baseline=(char.base)]{
    \node[shape=circle,draw,inner sep=2pt] (char) {#1};
  }
}

\setlength{\droptitle}{-10ex}

\preauthor{\begin{flushright}\large \lineskip 0.5em}
\postauthor{\par\end{flushright}}
\predate{\begin{flushright}\large}
\postdate{\par\end{flushright}}

\title{MAT 167 HW 1\vspace{-2ex}}
\author{Hardy Jones\\
        999397426\\
        Professor Cheer\vspace{-2ex}}
\date{Spring 2015}

\begin{document}
  \maketitle

  \begin{enumerate}
    \item [$\S$ 1.4]
      \begin{enumerate}
        \item [2]
          \[
            \begin{bmatrix}
              4 & 1 \\
              5 & 1 \\
              6 & 1 \\
            \end{bmatrix}
            \begin{bmatrix}
              1 \\
              3
            \end{bmatrix}
            =
            1
            \begin{bmatrix}
              4 \\
              5 \\
              6 \\
            \end{bmatrix}
            + 3
            \begin{bmatrix}
              1 \\
              1 \\
              1 \\
            \end{bmatrix}
            =
            \begin{bmatrix}
              4 \\
              5 \\
              6 \\
            \end{bmatrix}
            +
            \begin{bmatrix}
              3 \\
              3 \\
              3 \\
            \end{bmatrix}
            =
            \begin{bmatrix}
              7 \\
              8 \\
              9 \\
            \end{bmatrix}
          \]
        \item [7]
          \begin{enumerate}[label=(\alph*)]
            \item
              \[
                \begin{bmatrix}
                  1 & 0 & 0 \\
                  0 & 2 & 0 \\
                  0 & 0 & 3 \\
                \end{bmatrix}
              \]
            \item
              \[
                \begin{bmatrix}
                  1 & 0 & 0 \\
                  0 & 2 & 0 \\
                  0 & 0 & 3 \\
                \end{bmatrix}
              \]
            \item
              \[
                \begin{bmatrix}
                  1 & 0 & 0 \\
                  0 & 2 & 0 \\
                  0 & 0 & 3 \\
                \end{bmatrix}
              \]
            \item
              \[
                \begin{bmatrix}
                  0  & 1  & 0 \\
                  -1 & 0  & 2 \\
                  0  & -2 & 0 \\
                \end{bmatrix}
              \]
          \end{enumerate}
        \item [9]
          Assuming $A$ has as many pivots as rows, we have the following results.
          \begin{enumerate}[label=(\alph*)]
            \item $a_{11}$
            \item $l_{i1} = \frac{a_{i1}}{a_{11}}$
            \item $a_{ij} - a_{1j}\left(\frac{a_{i1}}{a_{11}}\right)$
            \item $a_{22} - a_{12}\left(\frac{a_{21}}{a_{11}}\right)$
          \end{enumerate}
        \item [10]
          \begin{enumerate}[label=(\alph*)]
            \item True.
            \item
              False.

              $AB$ may not even have three rows.

              For example, let $A$ be a $1 \times 3$ matrix and $B$ be a $3 \times 3$ matrix.

              Then $AB$ is a $1 \times 3$ matrix, so it has no third row.
            \item True.
            \item

              False.

              Let
              \[
                A = \begin{bmatrix}
                  1 & 2 \\
                  1 & 2 \\
                \end{bmatrix}
                ,
                B = \begin{bmatrix}
                  1 & 1 \\
                  0 & 0 \\
                \end{bmatrix}
              \]

              Then
              \begin{align*}
                (AB)^2
                &= \left(\begin{bmatrix}
                  1 & 2 \\
                  1 & 2 \\
                \end{bmatrix}
                \begin{bmatrix}
                  1 & 1 \\
                  0 & 0 \\
                \end{bmatrix}\right)^2 \\
                &= \begin{bmatrix}
                  1 & 1 \\
                  1 & 1 \\
                \end{bmatrix}^2 \\
                &= \begin{bmatrix}
                  2 & 2 \\
                  2 & 2 \\
                \end{bmatrix} \\
              \end{align*}

              But
              \begin{align*}
                A^2B^2
                &= \begin{bmatrix}
                  1 & 2 \\
                  1 & 2 \\
                \end{bmatrix}^2
                \begin{bmatrix}
                  1 & 1 \\
                  0 & 0 \\
                \end{bmatrix}^2 \\
                &= \begin{bmatrix}
                  3 & 6 \\
                  3 & 6 \\
                \end{bmatrix}
                \begin{bmatrix}
                  1 & 1 \\
                  0 & 0 \\
                \end{bmatrix} \\
                &= \begin{bmatrix}
                  3 & 3 \\
                  3 & 3 \\
                \end{bmatrix} \\
              \end{align*}

              So $(AB)^2 \ne A^2B^2$.
          \end{enumerate}
        \item [12]
          \begin{itemize}
            \item

              Let
              \[
                A = \begin{bmatrix}
                  1 & 0 \\
                  1 & 1 \\
                \end{bmatrix}
                ,
                B = \begin{bmatrix}
                  1 & 0 \\
                  2 & 1 \\
                \end{bmatrix}
              \]

              \begin{align*}
                AB
                &=
                \begin{bmatrix}
                  1 & 0 & 0 \\
                  1 & 1 & 0 \\
                  1 & 1 & 1 \\
                \end{bmatrix}
                \begin{bmatrix}
                  1 & 0 & 0 \\
                  2 & 1 & 0 \\
                  3 & 2 & 1 \\
                \end{bmatrix}
                \\
                &=
                \begin{bmatrix}
                  1 & 0 & 0 \\
                  3 & 1 & 0 \\
                  6 & 3 & 1 \\
                \end{bmatrix}
              \end{align*}

            \item

              \begin{proof}
                For any lower triangular matrices $A, B$ with dimension $n \times n$,
                each entry $ab_{ij}$ in $AB$ it is computed by:

                \[
                  \sum_{k = 1}^{n} a_{ik}b_{kj}
                \]

                If $i < k, a_{ik} = 0$.

                If $k < j, b_{kj} = 0$.

                Each entry above the main diagonal has one of either $i < k$ or $k < j$.

                So for each entry above the main diagonal of $AB$,
                we have a sum of products where at least one of the factors is 0.

                So, each entry above the main diagonal is 0.

                Thus, the product of any two lower triangular matrices is lower triangular.
              \end{proof}
          \end{itemize}
        \item [13]
          \begin{enumerate}[label=(\alph*)]
            \item
              Let
              \[
                A =
                \frac{1}{\sqrt{3}}
                \begin{bmatrix}
                  1  & 2 \\
                  -2 & -1 \\
                \end{bmatrix}
              \]

              \begin{align*}
                A^2
                &=
                \left(
                \frac{1}{\sqrt{3}}
                \begin{bmatrix}
                  1  & 2 \\
                  -2 & -1 \\
                \end{bmatrix}
                \right)^2
                \\
                &=
                \frac{1}{3}
                \begin{bmatrix}
                  -3 & 0 \\
                  0  & -3 \\
                \end{bmatrix}
                \\
                &=
                \begin{bmatrix}
                  -1 & 0 \\
                  0  & -1 \\
                \end{bmatrix}
                \\
                &=
                -I
                \\
              \end{align*}
            \item

              Let
              \[
                B
                =
                \begin{bmatrix}
                  0 & 0 \\
                  1 & 0
                \end{bmatrix}
                \ne
                0
              \]

              \[
                B^2
                =
                \begin{bmatrix}
                  0 & 0 \\
                  1 & 0
                \end{bmatrix}^2
                =
                \begin{bmatrix}
                  0 & 0 \\
                  0 & 0
                \end{bmatrix}
                =
                0
              \]
            \item

              Let
              \[
                C
                =
                \begin{bmatrix}
                  0 & 1 \\
                  1 & 0 \\
                \end{bmatrix}
                ,
                D
                =
                \begin{bmatrix}
                  0 & -1 \\
                  1 & 0 \\
                \end{bmatrix}
              \]

              \[
                CD
                =
                \begin{bmatrix}
                  0 & 1 \\
                  1 & 0 \\
                \end{bmatrix}
                \begin{bmatrix}
                  0 & -1 \\
                  1 & 0 \\
                \end{bmatrix}
                =
                \begin{bmatrix}
                  1 & 0 \\
                  0 & -1 \\
                \end{bmatrix}
              \]

              \[
                DC
                =
                \begin{bmatrix}
                  0 & -1 \\
                  1 & 0 \\
                \end{bmatrix}
                \begin{bmatrix}
                  0 & 1 \\
                  1 & 0 \\
                \end{bmatrix}
                =
                \begin{bmatrix}
                  -1 & 0 \\
                  0  & 1 \\
                \end{bmatrix}
              \]

              \item

                Let
                \[
                  E = F =
                  \begin{bmatrix}
                    2  & 1 \\
                    -4 & -2 \\
                  \end{bmatrix}
                \]

                \[
                  EF
                  =
                  \begin{bmatrix}
                    2  & 1 \\
                    -4 & -2 \\
                  \end{bmatrix}
                  \begin{bmatrix}
                    2  & 1 \\
                    -4 & -2 \\
                  \end{bmatrix}
                  =
                  \begin{bmatrix}
                    0 & 0 \\
                    0 & 0 \\
                  \end{bmatrix}
                \]
          \end{enumerate}
        \item [24]
          We want
          \[
            E_{21}
            =
            \begin{bmatrix}
              1  & 0 & 0 \\
              -4 & 1 & 0 \\
              0  & 0 & 1 \\
            \end{bmatrix}
            ,
            E_{31}
            =
            \begin{bmatrix}
              1  & 0 & 0 \\
              0  & 1 & 0 \\
              2  & 0 & 1 \\
            \end{bmatrix}
            ,
            E_{32}
            =
            \begin{bmatrix}
              1  & 0  & 0 \\
              0  & 1  & 0 \\
              0  & -2 & 1 \\
            \end{bmatrix}
          \]

          \begin{align*}
            M
            &=
            E_{32}E_{31}E_{21}
            \\
            &=
            \begin{bmatrix}
              1  & 0  & 0 \\
              0  & 1  & 0 \\
              0  & -2 & 1 \\
            \end{bmatrix}
            \begin{bmatrix}
              1  & 0 & 0 \\
              0  & 1 & 0 \\
              2  & 0 & 1 \\
            \end{bmatrix}
            \begin{bmatrix}
              1  & 0 & 0 \\
              -4 & 1 & 0 \\
              0  & 0 & 1 \\
            \end{bmatrix}
            \\
            &=
            \begin{bmatrix}
              1  & 0  & 0 \\
              0  & 1  & 0 \\
              2  & -2 & 1 \\
            \end{bmatrix}
            \begin{bmatrix}
              1  & 0 & 0 \\
              -4 & 1 & 0 \\
              0  & 0 & 1 \\
            \end{bmatrix}
            \\
            &=
            \begin{bmatrix}
              1  & 0  & 0 \\
              -4 & 1  & 0 \\
              10 & -2 & 1 \\
            \end{bmatrix}
            \\
          \end{align*}

          \[
            MA
            =
            \begin{bmatrix}
              1  & 0  & 0 \\
              -4 & 1  & 0 \\
              10 & -2 & 1 \\
            \end{bmatrix}
            \begin{bmatrix}
              1  & 1  & 0 \\
              4  & 6  & 1 \\
              -2 & 2  & 0 \\
            \end{bmatrix}
            =
            \begin{bmatrix}
              1  & 1  & 0  \\
              0  & 2  & 1  \\
              0  & 0  & -2 \\
            \end{bmatrix}
          \]
        \item [42]
          \begin{enumerate}[label=(\alph*)]
            \item
              True.

              Since matrix multiplication is only defined for matrices $A, B$ with dimension $m \times n, n \times p$, respectively,
              $A^2 = A*A$ must have $m = n = p$.
              That is $A$ must be a square matrix.
            \item
              False.

              We can choose $A, B$ with dimension $m \times n, n \times m$, respectively, where $m \ne n$.
              So $A$ and $B$ are not square.

              Then $AB$ is defined, as well as $BA$, yet $A$ and $B$ are not square.
            \item
              True.

              We can choose $A, B$ with dimension $m \times n, n \times m$, respectively.

              Then $AB$ is defined, as well as $BA$.

              These two products have dimension $m \times m$ and $n \times n$ respectively,
              so $AB$ and $BA$ are square.
            \item
              False.

              Let $B = 0$.

              Then $A0 = 0$ for all appropriate matrices, but $A$ is not necessarily $I$.
          \end{enumerate}
        \item [46]
      \end{enumerate}
    \item [$\S$ 1.5]
      \begin{enumerate}
        \item [11]

          Forward-substituting from $Lc = b$ gives

          \begin{itemize}
            \item $c_1 = 2$
            \item $2 + c_2 = 0 \implies c_2 = -2$
            \item $2 + c_3 = 2 \implies c_3 = 0$
          \end{itemize}

          Now Back-substituting from $Ux = c$ gives

          \begin{itemize}
            \item $w = 0$
            \item $v + 2(0) = -2 \implies v = -2$
            \item $2u + 4(-2) + 4(0) = 2 \implies 2u - 8 = 2 \implies u = 5$
          \end{itemize}

          So we have $u = 5, v = -2, w = 0$.
        \item [12]
          \begin{itemize}
            \item

              We could factor $A$ into $UL$ if we reduced the rows with the pivots below all 0's.
              That is, we'd want to form the lower triangular factor first,
              then perform elementary row operations to create the upper triangular factor.

            \item
              No, these two decompositions do not necessarily produce the same factors,
              as $LU$-decomposition in general is not unique.
          \end{itemize}
        \item [18]
          We can solve the first two at the same time.
          \begin{itemize}
            \item
              Let
              \[
                A
                =
                \begin{bmatrix}
                  0  &  1 & -1 \\
                  1  & -1 &  0 \\
                  1  &  0 & -1 \\
                \end{bmatrix}
              \]

              Then if we attempt to reduce $A$ to an upper triangular we get
              \begin{align*}
                \begin{bmatrix}
                  0  &  1 & -1 \\
                  1  & -1 &  0 \\
                  1  &  0 & -1 \\
                \end{bmatrix}
                &=
                \begin{bmatrix}
                  1  & -1 &  0 \\
                  0  &  1 & -1 \\
                  1  &  0 & -1 \\
                \end{bmatrix}
                \\
                &=
                \begin{bmatrix}
                  1  & -1 &  0 \\
                  0  &  1 & -1 \\
                  0  &  1 & -1 \\
                \end{bmatrix}
                \\
                &=
                \begin{bmatrix}
                  1  & -1 &  0 \\
                  0  &  1 & -1 \\
                  0  &  0 &  0 \\
                \end{bmatrix}
                \\
              \end{align*}

              But this matrix is singular, so it has no solutions.

            \item

              Let
              \[
                A
                =
                \begin{bmatrix}
                  0  &  1 &  1 \\
                  1  &  1 &  0 \\
                  1  &  0 &  1 \\
                \end{bmatrix}
              \]

              Then if we attempt to reduce $A$ to an upper triangular we get
              \begin{align*}
                A
                &=
                \begin{bmatrix}
                  0  &  1 &  1 \\
                  1  &  1 &  0 \\
                  1  &  0 &  1 \\
                \end{bmatrix}
                \\
                E_1A
                &=
                \begin{bmatrix}
                   0 &  1 &  0 \\
                   1 &  0 &  0 \\
                   0 &  0 &  1 \\
                \end{bmatrix}
                A
                &=
                \begin{bmatrix}
                   1 &  1 &  0 \\
                   0 &  1 &  1 \\
                   1 &  0 &  1 \\
                \end{bmatrix}
                \\
                E_2E_1A
                &=
                \begin{bmatrix}
                   1 &  0 &  0 \\
                   0 &  1 &  0 \\
                  -1 &  0 &  1 \\
                \end{bmatrix}
                \begin{bmatrix}
                   0 &  1 &  0 \\
                   1 &  0 &  0 \\
                   0 &  0 &  1 \\
                \end{bmatrix}
                A
                &=
                \begin{bmatrix}
                   1 &  1 &  0 \\
                   0 &  1 &  1 \\
                   0 & -1 &  1 \\
                \end{bmatrix}
                \\
                E_3E_2E_1A
                &=
                \begin{bmatrix}
                   1 &  0 &  0 \\
                   0 &  1 &  0 \\
                   0 &  1 &  1 \\
                \end{bmatrix}
                \begin{bmatrix}
                   1 &  0 &  0 \\
                   0 &  1 &  0 \\
                  -1 &  0 &  1 \\
                \end{bmatrix}
                \begin{bmatrix}
                   0 &  1 &  0 \\
                   1 &  0 &  0 \\
                   0 &  0 &  1 \\
                \end{bmatrix}
                A
                &=
                \begin{bmatrix}
                   1 &  1 &  0 \\
                   0 &  1 &  1 \\
                   0 &  0 &  2 \\
                \end{bmatrix}
                \\
              \end{align*}

              Since $E_1$ is a permutation matrix,
              we have $E_1 = P \implies PA = LU$.

              Now we have

              \begin{align*}
                L &= E_2^{-1}E_3^{-1} \\
                &=
                \begin{bmatrix}
                   1 &  0 &  0 \\
                   0 &  1 &  0 \\
                   1 &  0 &  1 \\
                \end{bmatrix}
                \begin{bmatrix}
                   1 &  0 &  0 \\
                   0 &  1 &  0 \\
                   0 & -1 &  1 \\
                \end{bmatrix}
                \\
                &=
                \begin{bmatrix}
                   1 &  0 &  0 \\
                   0 &  1 &  0 \\
                   1 & -1 &  1 \\
                \end{bmatrix}
              \end{align*}

              So A is nonsingular, in fact we have one unique solution.

              We take $PAx = Pb = y$ and solve with our $LU$ factors.

              \[
                y
                =
                \begin{bmatrix}
                   0 &  1 &  0 \\
                   1 &  0 &  0 \\
                   0 &  0 &  1 \\
                \end{bmatrix}
                \begin{bmatrix}
                   1 \\
                   1 \\
                   1 \\
                \end{bmatrix}
                =
                \begin{bmatrix}
                   1 \\
                   1 \\
                   1 \\
                \end{bmatrix}
              \]

              We can compute the $c$ matrix with $Lc = y$.
              \begin{itemize}
                \item $c_1 = 1$
                \item $c_2 = 1$
                \item $1 - 1 + c_3 = 1 \implies c_3 = 1$
              \end{itemize}

              We can compute the $x$ matrix with $Ux = c$.
              \begin{itemize}
                \item $2w              = 1 \implies w = \frac{1}{2}$
                \item $v + \frac{1}{2} = 1 \implies v = \frac{1}{2}$
                \item $u + \frac{1}{2} = 1 \implies u = \frac{1}{2}$
              \end{itemize}

              Finally, we have our solution: $u = \frac{1}{2}, v = \frac{1}{2}, w = \frac{1}{2}$.
          \end{itemize}
        \item [22]
          The elementary operations we performed were
          \[
            E_1 =
            \begin{bmatrix}
               1 &  0 &  0 \\
              -1 &  1 &  0 \\
              -1 &  0 &  1 \\
            \end{bmatrix}
            ,
            E_2 =
            \begin{bmatrix}
               1 &  0 &  0 \\
               0 &  1 &  0 \\
               0 & -2 &  1 \\
            \end{bmatrix}
          \]

          \begin{itemize}
            \item
              So
              \[
                L
                =
                E_1^{-1} E_2^{-1}
                =
                \begin{bmatrix}
                   1 &  0 &  0 \\
                   1 &  1 &  0 \\
                   1 &  0 &  1 \\
                \end{bmatrix}
                \begin{bmatrix}
                   1 &  0 &  0 \\
                   0 &  1 &  0 \\
                   0 &  2 &  1 \\
                \end{bmatrix}
                =
                \begin{bmatrix}
                   1 &  0 &  0 \\
                   1 &  1 &  0 \\
                   1 &  2 &  1 \\
                \end{bmatrix}
              \]
              We have
              \begin{align*}
                Lc &= b \\
                \begin{bmatrix}
                   1 &  0 &  0 \\
                   1 &  1 &  0 \\
                   1 &  2 &  1 \\
                \end{bmatrix}
                \begin{bmatrix}
                  c_1 \\
                  c_2 \\
                  c_3 \\
                \end{bmatrix}
                &=
                \begin{bmatrix}
                   5 \\
                   7 \\
                  11 \\
                \end{bmatrix}
              \end{align*}
              and
              \begin{align*}
                Ux &= c \\
                \begin{bmatrix}
                   1 &  1 & 1 \\
                   0 &  1 & 2 \\
                   0 &  0 & 1 \\
                \end{bmatrix}
                \begin{bmatrix}
                  x \\
                  y \\
                  z \\
                \end{bmatrix}
                &=
                \begin{bmatrix}
                  c_1 \\
                  c_2 \\
                  c_3 \\
                \end{bmatrix}
              \end{align*}
            \item

              Plugging in the values for $c$ we compute:
              \[
                \begin{bmatrix}
                   1 &  0 &  0 \\
                   1 &  1 &  0 \\
                   1 &  2 &  1 \\
                \end{bmatrix}
                \begin{bmatrix}
                  5 \\
                  2 \\
                  2 \\
                \end{bmatrix}
                =
                5
                \begin{bmatrix}
                  1 \\
                  1 \\
                  1 \\
                \end{bmatrix}
                +
                2
                \begin{bmatrix}
                   0 \\
                   1 \\
                   2 \\
                \end{bmatrix}
                +
                2
                \begin{bmatrix}
                   0 \\
                   0 \\
                   1 \\
                \end{bmatrix}
                =
                \begin{bmatrix}
                  5 \\
                  5 \\
                  5 \\
                \end{bmatrix}
                +
                \begin{bmatrix}
                   0 \\
                   2 \\
                   4 \\
                \end{bmatrix}
                +
                \begin{bmatrix}
                   0 \\
                   0 \\
                   2 \\
                \end{bmatrix}
                =
                \begin{bmatrix}
                  5  \\
                  7  \\
                  11 \\
                \end{bmatrix}
              \]
              So this $c$ solves the first system.
            \item
              We compute
              \begin{itemize}
                \item $              z = 2$
                \item $    y    + 2(2) = 2 \implies y + 4 = 2 \implies y = -2$
                \item $x + (-2) + 2    = 5 \implies x = 5$
              \end{itemize}
              So the $x$ that solves the second system is:

              \[
                \begin{bmatrix}
                  x \\
                  y \\
                  z \\
                \end{bmatrix}
                =
                \begin{bmatrix}
                   5 \\
                  -2 \\
                   2 \\
                \end{bmatrix}
              \]
          \end{itemize}
        \item [28]
        \item [33]
        \item [42]
      \end{enumerate}
    \item [$\S$ 1.6]
      \begin{enumerate}
        \item [2]
        \item [4]
        \item [5]
        \item [10]
        \item [17]
        \item [21]
        \item [40]
        \item [49]
      \end{enumerate}
  \end{enumerate}
\end{document}
