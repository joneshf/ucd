\documentclass[12pt,letterpaper]{article}
\usepackage{amsmath}
\usepackage{amsfonts}
\usepackage{amsthm}
\usepackage{mathtools}
\usepackage{cancel}
\usepackage[margin=1in]{geometry}
\usepackage{titling}

\setlength{\droptitle}{-10ex}

\preauthor{\begin{flushright}\large \lineskip 0.5em}
\postauthor{\par\end{flushright}}
\predate{\begin{flushright}\large}
\postdate{\par\end{flushright}}

\title{MAT 167 Homework 3\vspace{-2ex}}
\author{Hardy Jones\\
        999397426\\
        Professor Cheer\vspace{-2ex}}
\date{Winter 2014}

\begin{document}
  \maketitle

  \section*{2.4}
    \subsection*{3}
      Find the dimension and basis for the four fundamental subspaces for
      \[
        A =
        \begin{bmatrix}
          1 & 2 & 0 & 1 \\
          0 & 1 & 1 & 0 \\
          1 & 2 & 0 & 1
        \end{bmatrix}
        \quad \text{and} \quad
        U =
        \begin{bmatrix}
          1 & 2 & 0 & 1 \\
          0 & 1 & 1 & 0 \\
          0 & 0 & 0 & 0
        \end{bmatrix}
      \]

      \begin{enumerate}
        \item
          C(A)

          dimension: 2

          basis:
          $\left\{\begin{bmatrix}1 \\ 0 \\ 1\end{bmatrix}, \begin{bmatrix}2 \\ 1 \\ 2\end{bmatrix}\right\}$

          C(U)

          dimension: 2

          basis:
          $\left\{\begin{bmatrix}1 \\ 0 \\ 0\end{bmatrix}, \begin{bmatrix}2 \\ 1 \\ 0\end{bmatrix}\right\}$

        \item
          N(A)

          dimension: 2

          basis:9
          $\left\{\begin{bmatrix}2 \\ -1 \\ 1 \\ 0\end{bmatrix}, \begin{bmatrix}-1 \\ 0 \\ 0 \\ 1\end{bmatrix}\right\}$

          N(U)

          dimension: 2

          basis:
          $\left\{\begin{bmatrix}2 \\ -1 \\ 1 \\ 0\end{bmatrix}, \begin{bmatrix}-1 \\ 0 \\ 0 \\ 1\end{bmatrix}\right\}$

        \item
          C(A$^T$)

          dimension: 2

          basis:
          $\left\{\begin{bmatrix}1 \\ 2 \\ 0 \\ 1\end{bmatrix}, \begin{bmatrix}0 \\ 1 \\ 1 \\ 0\end{bmatrix}\right\}$

          C(U$^T$)

          dimension: 2

          basis:
          $\left\{\begin{bmatrix}1 \\ 2 \\ 0 \\ 1\end{bmatrix}, \begin{bmatrix}0 \\ 1 \\ 1 \\ 0\end{bmatrix}\right\}$

        \item
          N(A$^T$)

          dimension: 1

          basis:
          $\left\{\begin{bmatrix}-1 \\ 0 \\ 1\end{bmatrix}\right\}$

          N(U$^T$)

          dimension: 1

          basis:
          $\left\{\begin{bmatrix}0 \\ 0 \\ 1\end{bmatrix}\right\}$
      \end{enumerate}
    \subsection*{7}
      Why is there no matrix whose row space and nullspace both contain (1,1,1)?

      The only vector in common between the row space and null space is the zero vector, since these two spaces are orthogonal by definition.
    \subsection*{12}
      Find the rank of A and write the matrix as $uv^T$

      \begin{enumerate}
        \item
          \[
            A = \begin{bmatrix}
              1 & 0 & 0 & 3 \\
              0 & 0 & 0 & 0 \\
              2 & 0 & 0 & 6 \\
            \end{bmatrix}
          \]
          rank: 1

          \[
            A = uv^T =
            \begin{bmatrix}
              1 \\
              0 \\
              2
            \end{bmatrix}
            \begin{bmatrix}
              1 & 0 & 0 & 3
            \end{bmatrix}
          \]

        \item
          \[
            A = \begin{bmatrix}
              2 & -2 \\
              6 & -6 \\
            \end{bmatrix}
          \]
          rank: 1

          \[
            A = uv^T =
            \begin{bmatrix}
              1 \\
              3
            \end{bmatrix}
            \begin{bmatrix}
              2 & -2
            \end{bmatrix}
          \]
      \end{enumerate}
    \subsection*{14}
      Find a left-inverse and/or a right-inverse (when they exist) for

      \[
        A =
        \begin{bmatrix}
          1 & 1 & 0 \\
          0 & 1 & 1
        \end{bmatrix}
      \]

      The rank of this matrix is 2, which is the same as the number of rows, and it is a rectangular matrix, so it only has a right inverse, though there many inverses.

      We can construct the ``best'' right inverse by $A^T(AA^T)^{-1}$

      \begin{align*}
        A^{-1} = A^T(AA^T)^{-1} &=
        \begin{bmatrix}
          1 & 0 \\
          1 & 1 \\
          0 & 1
        \end{bmatrix}
        \left(
        \begin{bmatrix}
          1 & 1 & 0 \\
          0 & 1 & 1
        \end{bmatrix}
        \begin{bmatrix}
          1 & 0 \\
          1 & 1 \\
          0 & 1
        \end{bmatrix}
        \right)^{-1} \\
        &=
        \begin{bmatrix}
          1 & 0 \\
          1 & 1 \\
          0 & 1
        \end{bmatrix}
        \begin{bmatrix}
          2 & 1 \\
          1 & 2
        \end{bmatrix}^{-1} \\
        &=
        \begin{bmatrix}
          1 & 0 \\
          1 & 1 \\
          0 & 1
        \end{bmatrix}
        \begin{bmatrix}
          \frac{2}{3} & -\frac{1}{3} \\
          -\frac{1}{3} & \frac{2}{3}
        \end{bmatrix} \\
        &=
        \begin{bmatrix}
          \frac{2}{3} & -\frac{1}{3} \\
          \frac{1}{3} & \frac{1}{3} \\
          -\frac{1}{3} & \frac{2}{3}
        \end{bmatrix}
      \end{align*}

      In this case, we have only one inverse.

      \[
        M =
        \begin{bmatrix}
          1 & 0 \\
          1 & 1 \\
          0 & 1
        \end{bmatrix}
      \]

      The rank of this matrix is 2, which is the same as the number of columns, and it is a rectangular matrix, so it only has a left inverse which is unique.

      We can use the fact that $M^T = A$ to use $(A^{-1})^T$ as the left inverse.

      \[
        M^{-1} = (A^{-1})^T =
        \begin{bmatrix}
          \frac{2}{3} & \frac{1}{3} & -\frac{1}{3} \\
          -\frac{1}{3} & \frac{1}{3} & \frac{2}{3}
        \end{bmatrix}
      \]

      \[
        T =
        \begin{bmatrix}
          a & b \\
          0 & a
        \end{bmatrix}
      \]

      The rank of this matrix is 2, which is the same as the number of rows and columns, so it has both a left and a right inverse which are the same.

      This we can use the closed form to calculate $T^{-1} = \frac{1}{\text{det}(T)}\text{adj}(T)$

      \[
        T^{-1} =
        \frac{1}{a^2}
        \begin{bmatrix}
          a & -b \\
          0 & a
        \end{bmatrix}
      \]
    \subsection*{21}
    \subsection*{33}

  \section*{2.5}
    \subsection*{1}
    \subsection*{2}
    \subsection*{3}
    \subsection*{4}
    \subsection*{5}
    \subsection*{10}

  \section*{2.6}
    \subsection*{6}
    \subsection*{14}
    \subsection*{19}
    \subsection*{21}
    \subsection*{28}
    \subsection*{34}
    \subsection*{48}
\end{document}
