\documentclass[12pt,letterpaper]{article}
\usepackage{amsmath}
\usepackage{amsfonts}
\usepackage{amsthm}
\usepackage{mathtools}
\usepackage{cancel}
\usepackage[bottom=1in,left=0.5in,right=1in,top=1in]{geometry}
\usepackage{titling}

% From https://code.google.com/p/linear-algebra/source/browse/linalgjh.sty#80
% Using brackets instead of parens.

%-------------bmat
% For matrices with arguments.
% Usage: \begin{bmat}{c|c|c} 1 &2 &3 \end{bmat}
\newenvironment{bmat}[1]{
  \left[\begin{array}{@{}#1@{}}
}{\end{array}\right]
}

\newcommand\numberthis{\addtocounter{equation}{1}\tag{\theequation}}
\newcommand{\GLNR}[0]{GL_n(\mathbb{R})}
\newcommand{\KER}[1]{\text{ker }#1}
\newcommand{\IM}[1]{\text{im }#1}
\newcommand{\ghg}[1]{g#1g^{-1}}

\setlength{\droptitle}{-10ex}

\preauthor{\begin{flushright}\large \lineskip 0.5em}
\postauthor{\par\end{flushright}}
\predate{\begin{flushright}\large}
\postdate{\par\end{flushright}}

\title{MAT 150A Homework 3\vspace{-2ex}}
\author{Hardy Jones\\
        999397426\\
        Professor Schilling\vspace{-2ex}}
\date{Fall 2014}

\begin{document}
  \maketitle

  \begin{enumerate}
    \item
      To show that $f$ is an automorphism,
      we need to show that $f$ is an isomorphism from $\GLNR \rightarrow \GLNR$.
      To show that $f$ is an isomorphism,
      we need to show that $f$ is a homomorphism and bijective.
      To show that $f$ is a homomorphism,
      we need to show that $f$ is closed, and that it preserves the group operation.

      \[
        \GLNR := \{A_n | A \text{ is an } n \times n \text{ matrix}, |A| \ne 0\}
      \]

      \begin{proof}
        We begin by showing that $f$ is a homomorphism.

        \begin{itemize}
          \item
            We first show closure.

            Choose any $A \in \GLNR$.
            $f(A) = (A^T)^{-1}$.

            The size of $f(A)$ has not changed.
            We know $|A^T| = |A|$ and $|A^{-1}| = |A|^{-1}$,
            so $|f(A)| = |(A^T)^{-1}| = |(A^T)|^{-1} = |A|^{-1} \ne 0$.

            So, $f$ maps $\GLNR \mapsto \GLNR$

            Thus, $f$ is closed.

          \item
            Now we show that $f$ preserves the group operation.

            Choose any $A, B \in \GLNR$.

            \[
              f(AB) = ((AB)^T)^{-1} = (B^TA^T)^{-1} = (A^T)^{-1}(B^T)^{-1} = f(A)f(B)
            \]

            So, $f$ preserves the group operation.

            Thus, $f : \GLNR \rightarrow \GLNR$ is a homomorphism.
        \end{itemize}

        Now we need to show that $f$ is bijective.

        \begin{itemize}
          \item
            We first show that $\text{ker } f = \{I_n\}$

            So we want to find all $A \in \GLNR$ such that $f(A) = (A^T)^{-1} = I_n$.

            But we know that for any group the identity is its own inverse, so $A^T = I_n$.

            We also know that $I_n^T = I_n$, so $A = I_n$.

            And since we know that the identity is unique,
            we have that $\text{ker } f = \{I_n\}$.

          \item
            Now we show that $\text{im } f = \GLNR$

            It suffices to show that $f$ has an inverse.
            Namely, $f^{-1}(A) = A^T$.
        \end{itemize}

        So, we have shown that $f : \GLNR \rightarrow \GLNR$ is an isomorphism.
        And since $f$'s domain is its co-domain, $f$ is an automorphism.
      \end{proof}

    \item
      We want to show that $\forall \varphi : G -> G'$ that are group homomorphisms,
      $\KER{\varphi} \le G$ and $\IM{\varphi} \le G$.

      For both of these possible subgroups, it suffices to show two things:
        \begin{enumerate}
          \item The possible subgroup is non-empty.
          \item For all $a, b$ in the possible subgroup, $ab^{-1}$ is also in the subgroup.
        \end{enumerate}

      \begin{itemize}
        \item $\KER{\varphi} \le G$

          \begin{proof}
            We need to show the two conditions above.

            \begin{enumerate}
              \item
                Since $\varphi$ is a homomorphism, $\varphi(e_G) = e_{G'}$,
                so $\KER{\varphi}$ is non-empty (as $e_G \in \KER{\varphi}$).
              \item
                Choose $a, b \in \KER{\varphi}$.

                So we have, $\varphi(a) = e_{G'}$ and $\varphi(b) = e_{G'}$,
                and since $\varphi$ is a homomorphism.
                \begin{align*}
                  \varphi(ab^{-1}) &= \varphi(a)\varphi(b^{-1}) \\
                  &= \varphi(a)\varphi(b)^{-1} \\
                  &= e_{G'}\varphi(b)^{-1} \\
                  &= e_{G'}e_{G'}^{-1} \\
                  &= e_{G'}e_{G'} \\
                  &= e_{G'}
                \end{align*}
            \end{enumerate}

            Thus, $\KER{\varphi} \le G$.
          \end{proof}

        \item $\IM{\varphi} \le G$

          \begin{proof}
            We need to show the two conditions above.

            \begin{enumerate}
              \item
                Since $\varphi$ is a homomorphism, $\varphi(e_G) = e_{G'}$,
                so $\IM{\varphi}$ is non-empty (as $e_G \in \IM{\varphi}$).
              \item
                Choose $a, b \in \IM{\varphi}$.

                This means $\exists a', b' \in G \text{ s.t. } \varphi(a') = a, \varphi(b') = b$.

                Since $\varphi$ is a homomorphism and $a'b'^{-1} \in G$.
                \begin{align*}
                  \varphi(a'b'^{-1}) &= \varphi(a')\varphi(b'^{-1}) \\
                  &= \varphi(a')\varphi(b')^{-1} \\
                  &= a\varphi(b)^{-1} \\
                  &= ab^{-1} \in \IM{\varphi} \\
                \end{align*}
            \end{enumerate}

            Thus, $\IM{\varphi} \le G$.
          \end{proof}
      \end{itemize}

    \item
      The subgroups of $S_3$ are:
      \begin{align*}
        & \{id\} \\
        & \{id, (1,2)\}, \{id, (1,3)\}, \{id, (2,3)\} \\
        & \{id, (1,2,3)\}, \{id, (1,3,2)\} \\
        & \{id, (1,2), (1,3), (2,3), (1,2,3), (1,3,2)\}
      \end{align*}

      The trivial subgroup and the group itself are normal.

    \item
      Want to show $\varphi(x) = \varphi(y) \iff xy^{-1} \in \KER{\varphi}$

      \begin{proof}
        \begin{itemize}
          \item ($\Rightarrow$)

            Since $\varphi$ is a homomorphism and $\varphi(x) = \varphi(y)$.
            \[
              \varphi(xy^{-1}) = \varphi(x)\varphi(y^{-1}) = \varphi(x)\varphi(y)^{-1} = \varphi(x)\varphi(x)^{-1} = \varphi(x)\varphi(x^{-1}) = \varphi(xx^{-1}) = \varphi(e_G) = e_{G'}
            \]

            By the definition of the kernel, $xy^{-1} \in \KER{\varphi}$.

          \item ($\Leftarrow$)

            Since $\varphi$ is a homomorphism and $xy^{-1} \in \KER{\varphi}$.

            \begin{align*}
              \varphi(xy^{-1}) &= e_{G'} \\
              \varphi(x)\varphi(y^{-1}) &= e_{G'} \\
              \varphi(x)\varphi(y)^{-1} &= e_{G'} \\
              \varphi(x)\varphi(y)^{-1}\varphi(y) &= e_{G'}\varphi(y) \\
              \varphi(x)e_{G'} &= e_{G'}\varphi(y) \\
              \varphi(x) &= e_{G'}\varphi(y) \\
              \varphi(x) &= \varphi(y) \\
            \end{align*}

            So, $\varphi(x) = \varphi(y)$
        \end{itemize}

        Thus, we have shown both directions and $\varphi(x) = \varphi(y) \iff xy^{-1} \in \KER{\varphi}$.
      \end{proof}

    \item
      \begin{enumerate}
        \item
          We need to show that $gHg^{-1}$ is closed,
          has an identity and has inverses.

          \begin{proof}
            \begin{itemize}
              \item

                Choose $x = \ghg{h}, y = \ghg{h'} \in \ghg{h}$.
                \[
                  xy = (\ghg{h})(\ghg{h'}) = \ghg{h(g^{-1}g)h'} = \ghg{heh'} = \ghg{hh'}
                \]
                Now, since $hh' \in H$, we have that $\ghg{hh'} = xy \in H$.

                Thus, $\ghg{H}$ is closed.

              \item

                Choose $e \in G$ as the identity.

                \[
                  \ghg{h}e = \ghg{h} = e\ghg{h}
                \]

                Thus, $\ghg{H}$ has an identity.

              \item

                $\forall x = \ghg{h} in \ghg{H}$, choose $x^{-1} = \ghg{h^{-1}}$

                \[
                  (\ghg{h})(\ghg{h^{-1}}) = \ghg{h(g^{-1}g)h^{-1}} = \ghg{heh^{-1}} = \ghg{(hh^{-1})} = \ghg{e} = \ghg{} = e
                \]

                Thus, $\ghg{H}$ has inverses.
            \end{itemize}

            From the three results shown, $\ghg{H} \le G$ is a subgroup.
          \end{proof}

        \item
          We want to prove:

          $H \triangleleft G$ is normal $\iff$ $\forall g \in G, \ghg{H} = H$

          \begin{proof}

            We need to show both sides of this equivalence.

            \begin{itemize}
              \item[$(\Rightarrow)$]
                Choose $h \in H$, we know that $\forall g \in G, gH = Hg$,
                since $H \triangleleft G$ is normal.

                So, choose $g \in G$, we have:
                \begin{align*}
                  gh &= hg \\
                  ghg^{-1} &= hgg^{-1} \\
                  ghg^{-1} &= he \\
                  ghg^{-1} &= h \\
                \end{align*}

                Since our choices for $g, h$ were arbitrary,
                we have that this result holds for all $g \in G, h \in H$.

                Thus we have that $\ghg{H} = H$.

              \item[$(\Leftarrow)$]

                Choose $g \in G$, we know $\forall g \in G, \ghg{H} = H$.

                So, choose $h \in H$, we have:
                \begin{align*}
                  \ghg{h} &= h \\
                  \ghg{h}g &= hg \\
                  \ghg{h}g &= hg \\
                  gh(g^{-1}g) &= hg \\
                  ghe &= hg \\
                  gh &= hg \\
                \end{align*}

                Since our choices for $g, h$ were arbitrary,
                we have that this result holds for all $g \in G, h \in H$.

                Thus we have that $gH = Hg$, in other words, $H \triangleleft G$ is normal.
            \end{itemize}
          \end{proof}
      \end{enumerate}

    \item
      We need to show two things:
      \begin{itemize}
        \item The center of a group is a subgroup.

          For this we need to show three things.
          \begin{itemize}
            \item The subgroup is closed.
            \item The subgroup has an identity.
            \item the subgroup has inverses.
          \end{itemize}
        \item The center of a group is normal.
      \end{itemize}

      \begin{proof}
        \begin{itemize}
          \item We show that $Z(G) \le G$ is a subgroup.

            \begin{itemize}
              \item \textbf{Closure}

                Choose $z, z' \in Z(G)$.

                Since $Z(G)$ is the center of $G$, we know that $z, z' \in G$.

                So, we know that $zz' = z'z$.

                Thus, $Z(G)$ is closed.

              \item \textbf{Identity}

                Choose $e \in G$.

                Since $eg = g = ge \forall g \in G, e \in Z(G)$.

                Thus, $Z(G)$ has an identity.

              \item \textbf{Inverse}

                Choose $g \in G$.

                Since $G$ is a group,
                $\forall g \in G, \exists g^{-1} \in G \text{ s.t. } g^{-1}g = e = gg^{-1}$.

                So, $\forall g \in G, g^{-1} \in Z(G)$.

                Thus, $Z(G)$ has inverses.
            \end{itemize}

            From these results, we see that $Z(G) \le G$ is a subgroup.

          \item

            We show that $Z(G) \triangleleft G$ is normal.
            Equivalently, $\forall g \in G, gZ(G) = Z(G)g$.

            By definition of $Z(G) = \{z \in G | \forall g \in G, zg = gz\}$.

            This is exactly the definition of the normal.

            Thus, $Z(G) \triangleleft G$ is normal by construction.
        \end{itemize}

        From these two results,
        we have shown that the center of a group is a normal subgroup.
      \end{proof}

    \item
      We need to consider four cases here.

      \begin{enumerate}
        \item $|G|$ is infinite, $|H|$ is infinite.

          In this case, since the two sets are infinite,
          they have infinite order.

          Thus, $\forall x \in G, x \ne e_G, |\varphi(x)| = |x|$.

        \item $|G|$ is infinite, $|H|$ is finite.

          The two sets cannot have the same order,
          as $\forall x \in G, x \ne e_G, |x| = \infty$,

          While $\forall x \in H, |x| \le |H|$

        \item $|G|$ is finite, $|H|$ is infinite.

          A similar argument holds here as before.

        \item $|G|$ is finite, $|H|$ is finite.

          \begin{itemize}
            \item
              Since $G$ is finite, $\forall x \in G \text{ of order } n \in \mathbb{Z^+}$,
              we have $x^n = e_G$.

              So, we have:
              \begin{align*}
                \varphi(x^n) &= \varphi(x)^n \\
                \varphi(e_G) &= \varphi(x)^n \\
                e_H &= \varphi(x)^n \\
              \end{align*}

              Thus $\forall x \in G \text{ of order } n \in \mathbb{Z^+}, |\varphi(x)| = |x|$

            \item
              Since $\varphi : G \rightarrow H$ is an isomorphism,
              every element in $G$ maps to exactly one element in $H$ and vice versa.

              We know that $|x| = |\varphi(x)|$,
              so $H$ has exactly the same number of elements of order $n$ for each $n \in \mathbb{Z^+}$.
              Since our $G$ and $H$ are arbitrary groups,
              the result holds for all isomorphic groups.

            \item

              No, for instance, choose the homomorphism
              \[
                h : (\{-1, 0, 1\}, +) \rightarrow (\{0\}, +)
              \]
              \[
                n \mapsto 0
              \]

              The domain has two elements of order 2 and one element of order 1,
              and the co-domain has just one element of order 1.

              This cannot be an isomorphism, and the map is not bijective.
          \end{itemize}
      \end{enumerate}
  \end{enumerate}
\end{document}
