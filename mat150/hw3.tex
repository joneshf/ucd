\documentclass[12pt,letterpaper]{article}
\usepackage{amsmath}
\usepackage{amsfonts}
\usepackage{amsthm}
\usepackage{mathtools}
\usepackage{cancel}
\usepackage[bottom=1in,left=0.5in,right=1in,top=1in]{geometry}
\usepackage{titling}

% From https://code.google.com/p/linear-algebra/source/browse/linalgjh.sty#80
% Using brackets instead of parens.

%-------------bmat
% For matrices with arguments.
% Usage: \begin{bmat}{c|c|c} 1 &2 &3 \end{bmat}
\newenvironment{bmat}[1]{
  \left[\begin{array}{@{}#1@{}}
}{\end{array}\right]
}

\newcommand\numberthis{\addtocounter{equation}{1}\tag{\theequation}}
\newcommand{\GLNR}[0]{GL_n(\mathbb{R})}
\newcommand{\KER}[1]{\text{ker }#1}
\newcommand{\IM}[1]{\text{im }#1}

\setlength{\droptitle}{-10ex}

\preauthor{\begin{flushright}\large \lineskip 0.5em}
\postauthor{\par\end{flushright}}
\predate{\begin{flushright}\large}
\postdate{\par\end{flushright}}

\title{MAT 150A Homework 3\vspace{-2ex}}
\author{Hardy Jones\\
        999397426\\
        Professor Schilling\vspace{-2ex}}
\date{Fall 2014}

\begin{document}
  \maketitle

  \begin{enumerate}
    \item
      To show that $f$ is an automorphism,
      we need to show that $f$ is an isomorphism from $\GLNR \rightarrow \GLNR$.
      To show that $f$ is an isomorphism,
      we need to show that $f$ is a homomorphism and bijective.
      To show that $f$ is a homomorphism,
      we need to show that $f$ is closed, and that it preserves the group operation.

      \[
        \GLNR := \{A_n | A \text{ is an } n \times n \text{ matrix}, |A| \ne 0\}
      \]

      \begin{proof}
        We begin by showing that $f$ is a homomorphism.

        \begin{itemize}
          \item
            We first show closure.

            Choose any $A \in \GLNR$.
            $f(A) = (A^T)^{-1}$.

            The size of $f(A)$ has not changed.
            We know $|A^T| = |A|$ and $|A^{-1}| = |A|^{-1}$,
            so $|f(A)| = |(A^T)^{-1}| = |(A^T)|^{-1} = |A|^{-1} \ne 0$.

            So, $f$ maps $\GLNR \mapsto \GLNR$

            Thus, $f$ is closed.

          \item
            Now we show that $f$ preserves the group operation.

            Choose any $A, B \in \GLNR$.

            \[
              f(AB) = ((AB)^T)^{-1} = (B^TA^T)^{-1} = (A^T)^{-1}(B^T)^{-1} = f(A)f(B)
            \]

            So, $f$ preserves the group operation.

            Thus, $f : \GLNR \rightarrow \GLNR$ is a homomorphism.
        \end{itemize}

        Now we need to show that $f$ is bijective.

        \begin{itemize}
          \item
            We first show that $\text{ker } f = \{I_n\}$

            So we want to find all $A \in \GLNR$ such that $f(A) = (A^T)^{-1} = I_n$.

            But we know that for any group the identity is its own inverse, so $A^T = I_n$.

            We also know that $I_n^T = I_n$, so $A = I_n$.

            And since we know that the identity is unique,
            we have that $\text{ker } f = \{I_n\}$.

          \item
            Now we show that $\text{im } f = \GLNR$

            It suffices to show that $f$ has an inverse.
            Namely, $f^{-1}(A) = A^T$.
        \end{itemize}

        So, we have shown that $f : \GLNR \rightarrow \GLNR$ is an isomorphism.
        And since $f$'s domain is its co-domain, $f$ is an automorphism.
      \end{proof}

    \item
      We want to show that $\forall \varphi : G -> G'$ that are group homomorphisms,
      $\KER{\varphi} \le G$ and $\IM{\varphi} \le G$.

      For both of these possible subgroups, it suffices to show two things:
        \begin{enumerate}
          \item The possible subgroup is non-empty.
          \item For all $a, b$ in the possible subgroup, $ab^{-1}$ is also in the subgroup.
        \end{enumerate}

      \begin{itemize}
        \item $\KER{\varphi} \le G$

          \begin{proof}
            We need to show the two conditions above.

            \begin{enumerate}
              \item
                Since $\varphi$ is a homomorphism, $\varphi(e_G) = e_{G'}$,
                so $\KER{\varphi}$ is non-empty (as $e_G \in \KER{\varphi}$).
              \item
                Choose $a, b \in \KER{\varphi}$.

                So we have, $\varphi(a) = e_{G'}$ and $\varphi(b) = e_{G'}$,
                and since $\varphi$ is a homomorphism.
                \begin{align*}
                  \varphi(ab^{-1}) &= \varphi(a)\varphi(b^{-1}) \\
                  &= \varphi(a)\varphi(b)^{-1} \\
                  &= e_{G'}\varphi(b)^{-1} \\
                  &= e_{G'}e_{G'}^{-1} \\
                  &= e_{G'}e_{G'} \\
                  &= e_{G'}
                \end{align*}
            \end{enumerate}

            Thus, $\KER{\varphi} \le G$.
          \end{proof}

        \item $\IM{\varphi} \le G$

          \begin{proof}
            We need to show the two conditions above.

            \begin{enumerate}
              \item
                Since $\varphi$ is a homomorphism, $\varphi(e_G) = e_{G'}$,
                so $\IM{\varphi}$ is non-empty (as $e_G \in \IM{\varphi}$).
              \item
                Choose $a, b \in \IM{\varphi}$.

                This means $\exists a', b' \in G \text{ s.t. } \varphi(a') = a, \varphi(b') = b$.

                Since $\varphi$ is a homomorphism and $a'b'^{-1} \in G$.
                \begin{align*}
                  \varphi(a'b'^{-1}) &= \varphi(a')\varphi(b'^{-1}) \\
                  &= \varphi(a')\varphi(b')^{-1} \\
                  &= a\varphi(b)^{-1} \\
                  &= ab^{-1} \in \IM{\varphi} \\
                \end{align*}
            \end{enumerate}

            Thus, $\IM{\varphi} \le G$.
          \end{proof}
      \end{itemize}

    \item
      The subgroups of $S_3$ are:
      \begin{align*}
        & \{id\} \\
        & \{id, (1,2)\}, \{id, (1,3)\}, \{id, (2,3)\} \\
        & \{id, (1,2,3)\}, \{id, (1,3,2)\} \\
        & \{id, (1,2), (1,3), (2,3), (1,2,3), (1,3,2)\}
      \end{align*}

      The trivial subgroup and the group itself are normal.

    \item
      Want to show $\varphi(x) = \varphi(y) \iff xy^{-1} \in \KER{\varphi}$

      \begin{proof}
        \begin{itemize}
          \item ($\Rightarrow$)

            Since $\varphi$ is a homomorphism and $\varphi(x) = \varphi(y)$.
            \[
              \varphi(xy^{-1}) = \varphi(x)\varphi(y^{-1}) = \varphi(x)\varphi(y)^{-1} = \varphi(x)\varphi(x)^{-1} = \varphi(x)\varphi(x^{-1}) = \varphi(xx^{-1}) = \varphi(e_G) = e_{G'}
            \]

            By the definition of the kernel, $xy^{-1} \in \KER{\varphi}$.

          \item ($\Leftarrow$)

            Since $\varphi$ is a homomorphism and $xy^{-1} \in \KER{\varphi}$.

            \begin{align*}
              \varphi(xy^{-1}) &= e_{G'} \\
              \varphi(x)\varphi(y^{-1}) &= e_{G'} \\
              \varphi(x)\varphi(y)^{-1} &= e_{G'} \\
              \varphi(x)\varphi(y)^{-1}\varphi(y) &= e_{G'}\varphi(y) \\
              \varphi(x)e_{G'} &= e_{G'}\varphi(y) \\
              \varphi(x) &= e_{G'}\varphi(y) \\
              \varphi(x) &= \varphi(y) \\
            \end{align*}

            So, $\varphi(x) = \varphi(y)$
        \end{itemize}

        Thus, we have shown both directions and $\varphi(x) = \varphi(y) \iff xy^{-1} \in \KER{\varphi}$.
      \end{proof}
  \end{enumerate}
\end{document}
