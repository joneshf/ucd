\documentclass[12pt,letterpaper]{article}
\usepackage{amsmath}
\usepackage{amsfonts}
\usepackage{amsthm}
\usepackage{mathtools}
\usepackage{cancel}
\usepackage[margin=1in]{geometry}
\usepackage{titling}

% From https://code.google.com/p/linear-algebra/source/browse/linalgjh.sty#80
% Using brackets instead of parens.

%-------------bmat
% For matrices with arguments.
% Usage: \begin{bmat}{c|c|c} 1 &2 &3 \end{bmat}
\newenvironment{bmat}[1]{
  \left[\begin{array}{@{}#1@{}}
}{\end{array}\right]
}

\setlength{\droptitle}{-10ex}

\preauthor{\begin{flushright}\large \lineskip 0.5em}
\postauthor{\par\end{flushright}}
\predate{\begin{flushright}\large}
\postdate{\par\end{flushright}}

\title{MAT 150A Homework 1\vspace{-2ex}}
\author{Hardy Jones\\
        999397426\\
        Professor Schilling\vspace{-2ex}}
\date{Fall 2014}

\begin{document}
  \maketitle

  \begin{enumerate}
    \item
      \[X =
        \begin{bmatrix}
          1 & a & b \\
          0 & 1 & c \\
          0 & 0 & 1
        \end{bmatrix}
        ,
        Y =
        \begin{bmatrix}
          1 & d & e \\
          0 & 1 & f \\
          0 & 0 & 1
        \end{bmatrix}
      \]
      \begin{enumerate}
        \item[(1)]
          \[XY =
            \begin{bmatrix}
              1 & a & b \\
              0 & 1 & c \\
              0 & 0 & 1
            \end{bmatrix}
            \begin{bmatrix}
              1 & d & e \\
              0 & 1 & f \\
              0 & 0 & 1
            \end{bmatrix}
            =
            \begin{bmatrix}
              1 & a + d & b + e + af \\
              0 & 1     & c + f \\
              0 & 0     & 1
            \end{bmatrix}
          \]

          Since $\mathbb{F}$ is a field, it is closed under addition and multiplication.

          So, $a+d, b+e+af, c+f \in \mathbb{F}$.

          Thus, $XY \in H(F)$.

        \item[(2)]
          Given some array
          \[A =
            \begin{bmatrix}
              1 & a & b \\
              0 & 1 & c \\
              0 & 0 & 1
            \end{bmatrix}
          \]

          \begin{align*}
            \begin{bmat}{c c c | c c c}
              1 & a & b & 1 & 0 & 0 \\
              0 & 1 & c & 0 & 1 & 0 \\
              0 & 0 & 1 & 0 & 0 & 1
            \end{bmat}
            &=
            \begin{bmat}{c c c | c c c}
              1 & a & b & 1 & 0 & 0 \\
              0 & 1 & 0 & 0 & 1 & -c \\
              0 & 0 & 1 & 0 & 0 & 1
            \end{bmat} \\
            &=
            \begin{bmat}{c c c | c c c}
              1 & a & 0 & 1 & 0 & -b \\
              0 & 1 & 0 & 0 & 1 & -c \\
              0 & 0 & 1 & 0 & 0 & 1
            \end{bmat} \\
            &=
            \begin{bmat}{c c c | c c c}
              1 & 0 & 0 & 1 & -a & ac-b \\
              0 & 1 & 0 & 0 & 1 & -c \\
              0 & 0 & 1 & 0 & 0 & 1
            \end{bmat}
          \end{align*}

          So, we assume our inverse is
          \[A^{-1} =
            \begin{bmatrix}
              1 & -a & ac-b \\
              0 & 1 & -c \\
              0 & 0 & 1
            \end{bmatrix}
          \].

          Since $\mathbb{F}$ is a field, it has additive inverses, is closed under addition and multiplication.

          So, $-a, ac-b, -c \in \mathbb{F}$, and $A^{-1} \in H(F)$.

          We need to check that $A^{-1}$ is the inverse by showing that $AA^{-1} = I = A^{-1}A$

          \[AA^{-1} =
            \begin{bmatrix}
              1 & a & b \\
              0 & 1 & c \\
              0 & 0 & 1
            \end{bmatrix}
            \begin{bmatrix}
              1 & -a & ac-b \\
              0 & 1 & -c \\
              0 & 0 & 1
            \end{bmatrix}
            =
            \begin{bmatrix}
              1 & 0 & 0 \\
              0 & 1 & 0 \\
              0 & 0 & 1
            \end{bmatrix}
          \]
          \[A^{-1}A =
            \begin{bmatrix}
              1 & -a & ac-b \\
              0 & 1 & -c \\
              0 & 0 & 1
            \end{bmatrix}
            \begin{bmatrix}
              1 & a & b \\
              0 & 1 & c \\
              0 & 0 & 1
            \end{bmatrix}
            =
            \begin{bmatrix}
              1 & 0 & 0 \\
              0 & 1 & 0 \\
              0 & 0 & 1
            \end{bmatrix}
          \]

          So, the closed form of the inverse is given by
          \[A^{-1} =
            \begin{bmatrix}
              1 & -a & ac-b \\
              0 & 1 & -c \\
              0 & 0 & 1
            \end{bmatrix}
          \].

        \item[(3)]
          Given
      \end{enumerate}
  \end{enumerate}
\end{document}
