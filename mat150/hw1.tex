\documentclass[12pt,letterpaper]{article}
\usepackage{amsmath}
\usepackage{amsfonts}
\usepackage{amsthm}
\usepackage{mathtools}
\usepackage{cancel}
\usepackage[margin=1in]{geometry}
\usepackage{titling}

% From https://code.google.com/p/linear-algebra/source/browse/linalgjh.sty#80
% Using brackets instead of parens.

%-------------bmat
% For matrices with arguments.
% Usage: \begin{bmat}{c|c|c} 1 &2 &3 \end{bmat}
\newenvironment{bmat}[1]{
  \left[\begin{array}{@{}#1@{}}
}{\end{array}\right]
}

\setlength{\droptitle}{-10ex}

\preauthor{\begin{flushright}\large \lineskip 0.5em}
\postauthor{\par\end{flushright}}
\predate{\begin{flushright}\large}
\postdate{\par\end{flushright}}

\title{MAT 150A Homework 1\vspace{-2ex}}
\author{Hardy Jones\\
        999397426\\
        Professor Schilling\vspace{-2ex}}
\date{Fall 2014}

\begin{document}
  \maketitle

  \begin{enumerate}
    \item
      \[X =
        \begin{bmatrix}
          1 & a & b \\
          0 & 1 & c \\
          0 & 0 & 1
        \end{bmatrix}
        ,
        Y =
        \begin{bmatrix}
          1 & d & e \\
          0 & 1 & f \\
          0 & 0 & 1
        \end{bmatrix}
      \]
      \begin{enumerate}
        \item[(1)]
          \[XY =
            \begin{bmatrix}
              1 & a & b \\
              0 & 1 & c \\
              0 & 0 & 1
            \end{bmatrix}
            \begin{bmatrix}
              1 & d & e \\
              0 & 1 & f \\
              0 & 0 & 1
            \end{bmatrix}
            =
            \begin{bmatrix}
              1 & a + d & b + e + af \\
              0 & 1     & c + f \\
              0 & 0     & 1
            \end{bmatrix}
          \]

          Since $\mathbb{F}$ is a field, it is closed under addition and multiplication.

          So, $a+d, b+e+af, c+f \in \mathbb{F}$.

          Thus, $XY \in H(F)$.

        \item[(2)]
          Given some matrix
          \[A =
            \begin{bmatrix}
              1 & a & b \\
              0 & 1 & c \\
              0 & 0 & 1
            \end{bmatrix}
          \]

          \begin{align*}
            \begin{bmat}{c c c | c c c}
              1 & a & b & 1 & 0 & 0 \\
              0 & 1 & c & 0 & 1 & 0 \\
              0 & 0 & 1 & 0 & 0 & 1
            \end{bmat}
            &=
            \begin{bmat}{c c c | c c c}
              1 & a & b & 1 & 0 & 0 \\
              0 & 1 & 0 & 0 & 1 & -c \\
              0 & 0 & 1 & 0 & 0 & 1
            \end{bmat} \\
            &=
            \begin{bmat}{c c c | c c c}
              1 & a & 0 & 1 & 0 & -b \\
              0 & 1 & 0 & 0 & 1 & -c \\
              0 & 0 & 1 & 0 & 0 & 1
            \end{bmat} \\
            &=
            \begin{bmat}{c c c | c c c}
              1 & 0 & 0 & 1 & -a & ac-b \\
              0 & 1 & 0 & 0 & 1 & -c \\
              0 & 0 & 1 & 0 & 0 & 1
            \end{bmat}
          \end{align*}

          So, we assume our inverse is
          \[A^{-1} =
            \begin{bmatrix}
              1 & -a & ac-b \\
              0 & 1 & -c \\
              0 & 0 & 1
            \end{bmatrix}
          \].

          Since $\mathbb{F}$ is a field, it has additive inverses, is closed under addition and multiplication.

          So, $-a, ac-b, -c \in \mathbb{F}$, and $A^{-1} \in H(F)$.

          We need to check that $A^{-1}$ is the inverse by showing that $AA^{-1} = I = A^{-1}A$

          \[AA^{-1} =
            \begin{bmatrix}
              1 & a & b \\
              0 & 1 & c \\
              0 & 0 & 1
            \end{bmatrix}
            \begin{bmatrix}
              1 & -a & ac-b \\
              0 & 1 & -c \\
              0 & 0 & 1
            \end{bmatrix}
            =
            \begin{bmatrix}
              1 & 0 & 0 \\
              0 & 1 & 0 \\
              0 & 0 & 1
            \end{bmatrix}
          \]
          \[A^{-1}A =
            \begin{bmatrix}
              1 & -a & ac-b \\
              0 & 1 & -c \\
              0 & 0 & 1
            \end{bmatrix}
            \begin{bmatrix}
              1 & a & b \\
              0 & 1 & c \\
              0 & 0 & 1
            \end{bmatrix}
            =
            \begin{bmatrix}
              1 & 0 & 0 \\
              0 & 1 & 0 \\
              0 & 0 & 1
            \end{bmatrix}
          \]

          So, the closed form of the inverse is given by
          \[A^{-1} =
            \begin{bmatrix}
              1 & -a & ac-b \\
              0 & 1 & -c \\
              0 & 0 & 1
            \end{bmatrix}
          \].

        \item[(3)]
          Given
          \[A =
            \begin{bmatrix}
              1 & a & b \\
              0 & 1 & c \\
              0 & 0 & 1
            \end{bmatrix}
            , B =
            \begin{bmatrix}
              1 & d & e \\
              0 & 1 & f \\
              0 & 0 & 1
            \end{bmatrix}
            , C =
            \begin{bmatrix}
              1 & g & h \\
              0 & 1 & i \\
              0 & 0 & 1
            \end{bmatrix}
            \in H(F)
          \].

          We want to show that $A(BC) = (AB)C$.

          \begin{align*}
            A(BC) &= A
              \left(
                \begin{bmatrix}
                  1 & d & e \\
                  0 & 1 & f \\
                  0 & 0 & 1
                \end{bmatrix}
                \begin{bmatrix}
                  1 & g & h \\
                  0 & 1 & i \\
                  0 & 0 & 1
                \end{bmatrix}
              \right) \\
            &= A
              \left(
                \begin{bmatrix}
                  1 & d+g & e+h+id \\
                  0 & 1   & f+i \\
                  0 & 0   & 1
                \end{bmatrix}
              \right) \\
            &=
              \begin{bmatrix}
                1 & a & b \\
                0 & 1 & c \\
                0 & 0 & 1
              \end{bmatrix}
              \begin{bmatrix}
                1 & d+g & e+h+id \\
                0 & 1   & f+i \\
                0 & 0   & 1
              \end{bmatrix} \\
            &=
              \begin{bmatrix}
                1 & a+d+g & af+ai+b+e+h+id \\
                0 & 1     & c+f+i \\
                0 & 0     & 1
              \end{bmatrix}
          \end{align*}

          \begin{align*}
            (AB)C &=
              \left(
                \begin{bmatrix}
                  1 & a & b \\
                  0 & 1 & c \\
                  0 & 0 & 1
                \end{bmatrix}
                \begin{bmatrix}
                  1 & d & e \\
                  0 & 1 & f \\
                  0 & 0 & 1
                \end{bmatrix}
              \right) C \\
            &=
              \left(
                \begin{bmatrix}
                  1 & a+d & af+b+e \\
                  0 & 1   & c+f \\
                  0 & 0   & 1
                \end{bmatrix}
              \right) C \\
            &=
              \begin{bmatrix}
                1 & a+d & af+b+e \\
                0 & 1   & c+f \\
                0 & 0   & 1
              \end{bmatrix}
              \begin{bmatrix}
                1 & g & h \\
                0 & 1 & i \\
                0 & 0 & 1
              \end{bmatrix} \\
            &=
              \begin{bmatrix}
                1 & a+d+g & af+ai+b+e+h+id \\
                0 & 1   & c+f+i \\
                0 & 0   & 1
              \end{bmatrix}
          \end{align*}

          So, $A(BC) = (AB)C$.

          Thus, $H(F)$ is associative under matrix multiplication.

        \item[(4)]
          The elements of $\mathbb{F}_2$ are $\{0,1\}$,
          with $+$ and $\cdot$ defined: \\

          \begin{tabular}{c | c | c |}
            $+$ & 0 & 1 \\
            \hline
            0 & 0 & 1 \\
            \hline
            1 & 1 & 0 \\
            \hline
          \end{tabular}
          \qquad
          \begin{tabular}{c | c | c |}
            $\cdot$ & 0 & 1 \\
            \hline
            0 & 0 & 0 \\
            \hline
            1 & 0 & 1 \\
            \hline
          \end{tabular}

          The elements of $H(\mathbb{F}_2)$ are:
          \begin{align*}
            H(\mathbb{F}_2) &=
              \left\{
                \begin{bmatrix}
                  1 & 0 & 0 \\
                  0 & 1 & 0 \\
                  0 & 0 & 1
                \end{bmatrix},
                \begin{bmatrix}
                  1 & 1 & 0 \\
                  0 & 1 & 0 \\
                  0 & 0 & 1
                \end{bmatrix},
                \begin{bmatrix}
                  1 & 0 & 1 \\
                  0 & 1 & 0 \\
                  0 & 0 & 1
                \end{bmatrix},
                \begin{bmatrix}
                  1 & 1 & 1 \\
                  0 & 1 & 0 \\
                  0 & 0 & 1
                \end{bmatrix},
                \right. \\
              &
              \qquad \left.
                \begin{bmatrix}
                  1 & 0 & 0 \\
                  0 & 1 & 1 \\
                  0 & 0 & 1
                \end{bmatrix},
                \begin{bmatrix}
                  1 & 1 & 0 \\
                  0 & 1 & 1 \\
                  0 & 0 & 1
                \end{bmatrix},
                \begin{bmatrix}
                  1 & 0 & 1 \\
                  0 & 1 & 1 \\
                  0 & 0 & 1
                \end{bmatrix},
                \begin{bmatrix}
                  1 & 1 & 1 \\
                  0 & 1 & 1 \\
                  0 & 0 & 1
                \end{bmatrix}
              \right\}
          \end{align*}

          For simplicity we enumerate these as $\{H_0, H_1, \dots, H_7\}$

          We see that $H_0$ has order 1.

          $
            \begin{bmatrix}
              1 & 1 & 0 \\
              0 & 1 & 0 \\
              0 & 0 & 1
            \end{bmatrix}
            \begin{bmatrix}
              1 & 1 & 0 \\
              0 & 1 & 0 \\
              0 & 0 & 1
            \end{bmatrix}
            =
            \begin{bmatrix}
              1 & 0 & 0 \\
              0 & 1 & 0 \\
              0 & 0 & 1
            \end{bmatrix}
          $, so $H_1$ has order 2.

          $
            \begin{bmatrix}
              1 & 0 & 1 \\
              0 & 1 & 0 \\
              0 & 0 & 1
            \end{bmatrix}
            \begin{bmatrix}
              1 & 0 & 1 \\
              0 & 1 & 0 \\
              0 & 0 & 1
            \end{bmatrix}
            =
            \begin{bmatrix}
              1 & 0 & 0 \\
              0 & 1 & 0 \\
              0 & 0 & 1
            \end{bmatrix}
          $, so $H_2$ has order 2.

          $
            \begin{bmatrix}
              1 & 1 & 1 \\
              0 & 1 & 0 \\
              0 & 0 & 1
            \end{bmatrix}
            \begin{bmatrix}
              1 & 1 & 1 \\
              0 & 1 & 0 \\
              0 & 0 & 1
            \end{bmatrix}
            =
            \begin{bmatrix}
              1 & 0 & 0 \\
              0 & 1 & 0 \\
              0 & 0 & 1
            \end{bmatrix}
          $, so $H_3$ has order 2.

          $
            \begin{bmatrix}
              1 & 0 & 0 \\
              0 & 1 & 1 \\
              0 & 0 & 1
            \end{bmatrix}
            \begin{bmatrix}
              1 & 0 & 0 \\
              0 & 1 & 1 \\
              0 & 0 & 1
            \end{bmatrix}
            =
            \begin{bmatrix}
              1 & 0 & 0 \\
              0 & 1 & 0 \\
              0 & 0 & 1
            \end{bmatrix}
          $, so $H_4$ has order 2.

          $
            \begin{bmatrix}
              1 & 1 & 0 \\
              0 & 1 & 1 \\
              0 & 0 & 1
            \end{bmatrix}
            \begin{bmatrix}
              1 & 1 & 0 \\
              0 & 1 & 1 \\
              0 & 0 & 1
            \end{bmatrix}
            =
            \begin{bmatrix}
              1 & 0 & 1 \\
              0 & 1 & 0 \\
              0 & 0 & 1
            \end{bmatrix}
          $ and
          $
            \begin{bmatrix}
              1 & 0 & 1 \\
              0 & 1 & 0 \\
              0 & 0 & 1
            \end{bmatrix}
            \begin{bmatrix}
              1 & 1 & 0 \\
              0 & 1 & 1 \\
              0 & 0 & 1
            \end{bmatrix}
            =
            \begin{bmatrix}
              1 & 0 & 0 \\
              0 & 1 & 0 \\
              0 & 0 & 1
            \end{bmatrix}
          $, so $H_5$ has order 3.

          $
            \begin{bmatrix}
              1 & 0 & 1 \\
              0 & 1 & 1 \\
              0 & 0 & 1
            \end{bmatrix}
            \begin{bmatrix}
              1 & 0 & 1 \\
              0 & 1 & 1 \\
              0 & 0 & 1
            \end{bmatrix}
            =
            \begin{bmatrix}
              1 & 0 & 0 \\
              0 & 1 & 0 \\
              0 & 0 & 1
            \end{bmatrix}
          $, so $H_6$ has order 2.

          $
            \begin{bmatrix}
              1 & 1 & 1 \\
              0 & 1 & 1 \\
              0 & 0 & 1
            \end{bmatrix}
            \begin{bmatrix}
              1 & 1 & 1 \\
              0 & 1 & 1 \\
              0 & 0 & 1
            \end{bmatrix}
            =
            \begin{bmatrix}
              1 & 0 & 1 \\
              0 & 1 & 0 \\
              0 & 0 & 1
            \end{bmatrix}
          $ and
          $
            \begin{bmatrix}
              1 & 0 & 1 \\
              0 & 1 & 0 \\
              0 & 0 & 1
            \end{bmatrix}
            \begin{bmatrix}
              1 & 1 & 1 \\
              0 & 1 & 1 \\
              0 & 0 & 1
            \end{bmatrix}
            =
            \begin{bmatrix}
              1 & 0 & 0 \\
              0 & 1 & 0 \\
              0 & 0 & 1
            \end{bmatrix}
          $, so $H_7$ has order 3.
      \end{enumerate}

    \item
      Let's look at a few cases first.

      $n = 2$,
      \[
        \begin{vmatrix}
          2 & -1 \\
          -1 & 2
        \end{vmatrix}
        = 3
      \]
      $n = 3$,
      \[
        \begin{vmatrix}
          2  & -1 &    \\
          -1 & 2  & -1 \\
             & -1 & 2
        \end{vmatrix}
        = 4
      \]
      $n = 4$,
      \[
        \begin{vmatrix}
          2  & -1 &    &    \\
          -1 & 2  & -1 &    \\
             & -1 & 2  & -1 \\
             &    & -1 & 2
        \end{vmatrix}
        = 5
      \]

      Looks like the determinate of the $n \times n$ matrix is $n+1$

      \begin{proof}

        \textbf{Base Case} $n = 2$
        \[
          \begin{vmatrix}
            2 & -1 \\
            -1 & 2
          \end{vmatrix}
          = 2(2) - (-1)(-1) = 4 - 1 = 3
        \]
        and $n + 1 = 3$.

        So our base case holds.

        \textbf{Inductive Case}

        Assume determinate of an $n \times n$ matrix is $n + 1$.
        Need to show, the determinate of an $(n+1) \times (n+1)$ matrix is $(n+1) + 1 = n + 2$

        \[
          \begin{vmatrix}
            2  & -1 &        &        &        &        \\
            -1 & 2  & -1     &        &        &        \\
               & -1 & 2      & -1     &        &        \\
               &    & -1     & \ddots & \ddots &        \\
               &    &        & \ddots & 2      & -1     \\
               &    &        &        & -1     & 2      \\
          \end{vmatrix}
        \]

        We first expand along the first column.

        \begin{align*}
          \begin{vmatrix}
            2  & -1 &        &        &        &        \\
            -1 & 2  & -1     &        &        &        \\
               & -1 & 2      & -1     &        &        \\
               &    & -1     & \ddots & \ddots &        \\
               &    &        & \ddots & 2      & -1     \\
               &    &        &        & -1     & 2      \\
          \end{vmatrix}
          &= 2
            \begin{vmatrix}
              2  & -1     &        &        &        \\
              -1 & 2      & -1     &        &        \\
                 & -1     & \ddots & \ddots &        \\
                 &        & \ddots & 2      & -1     \\
                 &        &        & -1     & 2      \\
            \end{vmatrix}
            - (-1)
            \begin{vmatrix}
              -1 &        &        &        &        \\
              -1 & 2      & -1     &        &        \\
                 & -1     & \ddots & \ddots &        \\
                 &        & \ddots & 2      & -1     \\
                 &        &        & -1     & 2      \\
            \end{vmatrix} \\
          & \qquad + 0 \begin{vmatrix}\dots\end{vmatrix}
            - 0 \begin{vmatrix}\dots\end{vmatrix}
            + \dots \\
          &= 2
            \begin{vmatrix}
              2  & -1     &        &        &        \\
              -1 & 2      & -1     &        &        \\
                 & -1     & \ddots & \ddots &        \\
                 &        & \ddots & 2      & -1     \\
                 &        &        & -1     & 2      \\
            \end{vmatrix}
            +
            \begin{vmatrix}
              -1 &        &        &        &        \\
              -1 & 2      & -1     &        &        \\
                 & -1     & \ddots & \ddots &        \\
                 &        & \ddots & 2      & -1     \\
                 &        &        & -1     & 2      \\
            \end{vmatrix}
        \end{align*}

        Looking at the first matrix expansion, we see that it is an $n \times n$ matrix of the same form.
        By our assumption, it has determinate of $n + 1$.

        \begin{align*}
          \begin{vmatrix}
            2  & -1 &        &        &        &        \\
            -1 & 2  & -1     &        &        &        \\
               & -1 & 2      & -1     &        &        \\
               &    & -1     & \ddots & \ddots &        \\
               &    &        & \ddots & 2      & -1     \\
               &    &        &        & -1     & 2      \\
          \end{vmatrix}
          &= 2(n + 1)
            +
            \begin{vmatrix}
              -1 &        &        &        &        \\
              -1 & 2      & -1     &        &        \\
                 & -1     & \ddots & \ddots &        \\
                 &        & \ddots & 2      & -1     \\
                 &        &        & -1     & 2      \\
            \end{vmatrix}
        \end{align*}

        We expand the remaining determinate across the first column.

        \begin{align*}
          \begin{vmatrix}
            2  & -1 &        &        &        &        \\
            -1 & 2  & -1     &        &        &        \\
               & -1 & 2      & -1     &        &        \\
               &    & -1     & \ddots & \ddots &        \\
               &    &        & \ddots & 2      & -1     \\
               &    &        &        & -1     & 2      \\
          \end{vmatrix}
          &= 2(n + 1)
            + (-1)
            \begin{vmatrix}
              2      & -1     &        &        \\
              -1     & \ddots & \ddots &        \\
                     & \ddots & 2      & -1     \\
                     &        & -1     & 2      \\
            \end{vmatrix} \\
          & \qquad - 0 \begin{vmatrix}\dots\end{vmatrix}
            + 0 \begin{vmatrix}\dots\end{vmatrix}
            + \dots \\
          &= 2(n + 1)
            -
            \begin{vmatrix}
              2      & -1     &        &        \\
              -1     & \ddots & \ddots &        \\
                     & \ddots & 2      & -1     \\
                     &        & -1     & 2      \\
            \end{vmatrix}
        \end{align*}

        This is now an $(n - 1) \times (n - 1)$ matrix of the same form.
        Again, by our assumption, the determinate is $(n-1) + 1 = n$.

        \begin{align*}
          \begin{vmatrix}
            2  & -1 &        &        &        &        \\
            -1 & 2  & -1     &        &        &        \\
               & -1 & 2      & -1     &        &        \\
               &    & -1     & \ddots & \ddots &        \\
               &    &        & \ddots & 2      & -1     \\
               &    &        &        & -1     & 2      \\
          \end{vmatrix}
          &= 2(n + 1) - n = 2n + 2 - n = n + 2
        \end{align*}

        So, we have shown by induction, that the determinate of an $n \times n$ matrix of this form is $n + 1$.
      \end{proof}

    \item
      Given the permutation $(1,3,4,2)$.
      \begin{enumerate}
        \item
          \[
            P =
            \begin{bmatrix}
              0 & 1 & 0 & 0 \\
              0 & 0 & 0 & 1 \\
              1 & 0 & 0 & 0 \\
              0 & 0 & 1 & 0 \\
            \end{bmatrix}
          \]
        \item
          \begin{align*}
            p &= (1, 2)(1, 4)(1, 3) \\
            &=
              \begin{bmatrix}
                0 & 1 & 0 & 0 \\
                1 & 0 & 0 & 0 \\
                0 & 0 & 1 & 0 \\
                0 & 0 & 0 & 1 \\
              \end{bmatrix}
              \begin{bmatrix}
                0 & 0 & 0 & 1 \\
                0 & 1 & 0 & 0 \\
                0 & 0 & 1 & 0 \\
                1 & 0 & 0 & 0 \\
              \end{bmatrix}
              \begin{bmatrix}
                0 & 0 & 1 & 0 \\
                0 & 1 & 0 & 0 \\
                1 & 0 & 0 & 0 \\
                0 & 0 & 0 & 1 \\
              \end{bmatrix} \\
            &=
              \begin{bmatrix}
                0 & 1 & 0 & 0 \\
                0 & 0 & 0 & 1 \\
                0 & 0 & 1 & 0 \\
                1 & 0 & 0 & 0 \\
              \end{bmatrix}
              \begin{bmatrix}
                0 & 0 & 1 & 0 \\
                0 & 1 & 0 & 0 \\
                1 & 0 & 0 & 0 \\
                0 & 0 & 0 & 1 \\
              \end{bmatrix} \\
            &=
              \begin{bmatrix}
                0 & 1 & 0 & 0 \\
                0 & 0 & 0 & 1 \\
                1 & 0 & 0 & 0 \\
                0 & 0 & 1 & 0 \\
              \end{bmatrix}
          \end{align*}
        \item
          Since $p$ can be written as the product of three transpositions, the sign is odd.
      \end{enumerate}
    \item
      Given some $n \times n$ permutation matrix $P$, by definition, each row and column has exactly one $1$ and the rest $0$.
      This means that $P^T$ also has every row and column with exactly one $1$ and the rest $0$.

      Without loss of generality, we can compute the product of these matrices by:

      \begin{align*}
        PP^T &=
          \begin{bmatrix}
            {PP^T}_{1 1} & {PP^T}_{1 2} & \cdots & {PP^T}_{1 n} \\
            {PP^T}_{2 1} & {PP^T}_{2 2} & \cdots & {PP^T}_{2 n} \\
            \vdots       & \vdots       & \ddots & \vdots       \\
            {PP^T}_{n 1} & {PP^T}_{n 2} & \cdots & {PP^T}_{n n} \\
          \end{bmatrix}\\
      \end{align*}

      Where each ${PP^T}_{i j}$ is the inner product between row $i$ of $P$ and column $j$ of $P^T$.

      By definition of transposition, when $i = j$ the inner product is between row $i$ of $P$ and the transpose of row $i$ of $P$.

      In other words:
        \[
          \begin{bmatrix}
            0, 0, \cdots, 1_i, \cdots, 0
          \end{bmatrix}
          \begin{bmatrix}
            0 \\
            0 \\
            \vdots \\
            1_i \\
            \vdots \\
            0
          \end{bmatrix}
        \]
      where $1_i$ is in the same row/column number.

      The inner product of this is $1$.

      However, when $i \ne j$ the inner product is between a row and a column which are orthogonal to each other.

      In other words:
        \[
          \begin{bmatrix}
            0, 0, \cdots, 1_i, \cdots, 0
          \end{bmatrix}
          \begin{bmatrix}
            0 \\
            0 \\
            \vdots \\
            1_j \\
            \vdots \\
            0
          \end{bmatrix}
        \]
      where $1_i$ and $1_j$ are in different row/column numbers.

      Since these vectors are orthogonal, the inner product is $0$.

      So, for each ${PP^T}_{i j}$, if $i = j$, the value is $1$,
      and if $i \ne j$, the value is $0$.

      In other words, it produces all $0$'s except on the main diagonal.
      This is the Identity matrix.

      So, $PP^T = I$.

      A similar argument holds for $P^TP = I$.

      Thus, the transpose of a permutation matrix is its inverse.
  \end{enumerate}
\end{document}
