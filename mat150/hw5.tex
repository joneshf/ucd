\documentclass[12pt,letterpaper]{article}
\usepackage{amsmath}
\usepackage{amsfonts}
\usepackage{amsthm}
\usepackage{mathtools}
\usepackage{cancel}
\usepackage[bottom=1in,left=0.5in,right=1in,top=1in]{geometry}
\usepackage{titling}

% From https://code.google.com/p/linear-algebra/source/browse/linalgjh.sty#80
% Using brackets instead of parens.

%-------------bmat
% For matrices with arguments.
% Usage: \begin{bmat}{c|c|c} 1 &2 &3 \end{bmat}
\newenvironment{bmat}[1]{
  \left[\begin{array}{@{}#1@{}}
}{\end{array}\right]
}

\newcommand\numberthis{\addtocounter{equation}{1}\tag{\theequation}}
\newcommand{\GLNR}[0]{GL_n(\mathbb{R})}
\newcommand{\KER}[1]{\text{ker }#1}
\newcommand{\IM}[1]{\text{im }#1}
\newcommand{\ghg}[1]{g#1g^{-1}}

\setlength{\droptitle}{-10ex}

\preauthor{\begin{flushright}\large \lineskip 0.5em}
\postauthor{\par\end{flushright}}
\predate{\begin{flushright}\large}
\postdate{\par\end{flushright}}

\title{MAT 150A Homework 5\vspace{-2ex}}
\author{Hardy Jones\\
        999397426\\
        Professor Schilling\vspace{-2ex}}
\date{Fall 2014}

\begin{document}
  \maketitle

  \begin{enumerate}
    \item
      Since $G$ has subgroups of order 3 and 5, we know that we can reconstruct at least part of $G$ by taking the composition of each element from the subgroups.

    \item
    \item
      \begin{enumerate}
        \item We can show this by enumerating the table.

          \begin{tabular}{c | c | c | c | c | c | c |}
            $\cdot$ & 0 & 1 & 2 & 3 & 4 & 5 \\
            \hline
            0 & 0 & 0 & 0 & 0 & 0 & 0 \\
            \hline
            1 & 0 & 1 & 2 & 3 & 4 & 5 \\
            \hline
            2 & 0 & 2 & 4 & 0 & 2 & 4 \\
            \hline
            3 & 0 & 3 & 0 & 3 & 0 & 3 \\
            \hline
            4 & 0 & 4 & 2 & 0 & 4 & 2 \\
            \hline
            5 & 0 & 5 & 4 & 3 & 2 & 1 \\
            \hline
          \end{tabular}

          As we see, there is not a 1 in either the row for 2 or the column for 2.
          So, there is no inverse for 2.
        \item
          2 has an inverse modulo $n$ for all odd $n > 1$. In particular, the inverse is $\frac{n + 1}{2}$

          Choose some odd $n > 1$.

          \begin{align*}
            2\left(\frac{n + 1}{2}\right) \text{ mod } n &= (n + 1) \text{ mod } n \\
            &= 1
          \end{align*}

          For even $n$, there is no such $n$ as we can generalize the result above.

          So, for each odd $n > 1$, 2 mod $n$ has an inverse.
      \end{enumerate}
    \item
      We need to show Closure, Identity, and Inverse
      \begin{enumerate}
        \item \textbf{Closure}

          Choose $h, h' \in H$, then we just need to show the determinate is positive.

          $|hh'| = |h||h'| > 0$ as multiplying two positive numbers is always positive.

          So $H$ is closed.

        \item \textbf{Identity}

          $|I_n| = 1$, so $H$ has an identity.

        \item \textbf{Inverse}

          Since each element has a positive determinate, each element is invertible.

          Choose $h \in H$, then there exists some $h^{-1} \in H$ such that:

          \[hh^{-1} = I_n = h^{-1}h\]

          And the determinate is:

          \begin{align*}
            |hh^{-1}| &= |I_n| \\
            |h||h^{-1}| &= 1 \\
            |h^{-1}| &= \frac{1}{|h|} > 0 \\
          \end{align*}

          So, $h^{-1}$ is an element of $H$.

          Thus, $H$ has inverses.
      \end{enumerate}

      So we know that $H$ is a subgroup.

      Now we just need to show normalcy.

      Choose $g \in \GLNR, h \in H$,
      then we want to show $\ghg{h} \in H$

      \[
        |\ghg{h}| = |g||h||g^{-1}| = |g||g^{-1}||h| = |gg^{-1}||h| = |I_n||h| = |h| > 0
      \]

      So, we know that $\ghg{h} \in H$.

      Thus $H \trianglelefteq \GLNR$.

      The quotient group would be one left coset where $|g| \mapsto \text{abs}(|g|)$.

      The other would be a left coset where $|g| \mapsto -\text{abs}(|g|)$.
    \item
    \item

      For every group of even order, there are an odd number of elements with order $\ge 1$--as the identity is the only element of order 1.

      Since each element has exactly one inverse, this means there must be some element which is its own inverse. In other words, it has order 2.

    \item

      From Lagrange's theorem, we know that \[H \le G \implies [G : H] = \frac{|G|}{|H|}\]

      So, we can substitute,

      \begin{align*}
        [G : H][H : K] &= \frac{|G|}{|H|} \frac{|H|}{|K|} \\
        &= \frac{|G|}{|K|} \\
        &= [G : K]
      \end{align*}
  \end{enumerate}
\end{document}
