\documentclass[12pt,letterpaper]{article}
\usepackage{amsmath}
\usepackage{amsfonts}
\usepackage{amsthm}
\usepackage{mathtools}
\usepackage{cancel}
\usepackage[bottom=1in,left=0.5in,right=1in,top=1in]{geometry}
\usepackage{titling}

% From https://code.google.com/p/linear-algebra/source/browse/linalgjh.sty#80
% Using brackets instead of parens.

%-------------bmat
% For matrices with arguments.
% Usage: \begin{bmat}{c|c|c} 1 &2 &3 \end{bmat}
\newenvironment{bmat}[1]{
  \left[\begin{array}{@{}#1@{}}
}{\end{array}\right]
}

\newcommand\numberthis{\addtocounter{equation}{1}\tag{\theequation}}
\newcommand{\GLNR}[0]{GL_n(\mathbb{R})}
\newcommand{\KER}[1]{\text{ker }#1}
\newcommand{\IM}[1]{\text{im }#1}
\newcommand{\ghg}[1]{g#1g^{-1}}

\setlength{\droptitle}{-10ex}

\preauthor{\begin{flushright}\large \lineskip 0.5em}
\postauthor{\par\end{flushright}}
\predate{\begin{flushright}\large}
\postdate{\par\end{flushright}}

\title{MAT 150A Homework 6\vspace{-2ex}}
\author{Hardy Jones\\
        999397426\\
        Professor Schilling\vspace{-2ex}}
\date{Fall 2014}

\begin{document}
  \maketitle

  \begin{enumerate}
    \item
      The matrix is:
      \[
        \begin{bmatrix}
          \cos(\theta)  & 0 & \sin(\theta) \\
          0             & 1 & 0            \\
          -\sin(\theta) & 0 & \cos(\theta) \\
        \end{bmatrix}
      \]

    \item
      \begin{enumerate}
        \item
          To show that $\mathcal{O}_n \le GL_n(\mathbb{R})$ is a subgroup, we need to show that it is closed, has the identity and has inverses.

          \begin{proof}
            \begin{enumerate}
              \item \textbf{Closure}

                Choose $A, B \in \mathcal{O}_n$.

                $AB$ is in $\mathcal{O}_n$ if $A^TAB^TB = I_n$.
                \[
                  A^TAB^TB = (A^TA)(B^TB) = I_nI_n = I_n
                \]
                So $\mathcal{O}_n$ has closure.

              \item \textbf{Identity}

                \[
                  I_n^TI_n = I_nI_n = I_n
                \]

                So $\mathcal{O}_n$ has the identity.

              \item \textbf{Inverse}

                Choose $A \in \mathcal{O}_n$.

                Since $A^TA = I_n$ $A^T = A^{-1}$,
                so $\mathcal{O}_n$ has inverses.
            \end{enumerate}

            From these three, $\mathcal{O}_n \le GL_n(\mathbb{R})$ is a subgroup.
          \end{proof}

        \item
          If we show that $\mathcal{SO}_n \le \mathcal{O}_n$,
          then from above $\mathcal{SO}_n \le GL_n(\mathbb{R})$.

          Again we need to show it's closed, has the identity and has inverses.

          \begin{proof}
            \begin{enumerate}
              \item \textbf{Closure}

                Choose $A, B \in \mathcal{SO}_n$.

                $AB$ is in $\mathcal{SO}_n$ if $|AB| = 1$
                \[
                  |AB| = |A||B| = 1 \cdot 1 = 1
                \]

                So, $\mathcal{SO}_n$ is closed.

              \item \textbf{Identity}

                \[
                  |I_n| = 1
                \]

                So, $\mathcal{SO}_n$ has the identity.
              \item \textbf{Inverse}

                Choose $A \in \mathcal{SO}_n$.

                Since $|A| \ne 0$, we know that $A$ is invertible.
                So, $A^{-1} \in \mathcal{SO}_n$.
                Thus, $\mathcal{SO}_n$ has inverses.
            \end{enumerate}

            From these three, $\mathcal{O}_n \le \mathcal{O}_n$ is a subgroup.
          \end{proof}
      \end{enumerate}
  \end{enumerate}
\end{document}
