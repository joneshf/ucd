\documentclass[12pt,letterpaper]{article}
\usepackage{amsmath}
\usepackage{amsfonts}
\usepackage{amsthm}
\usepackage{mathtools}
\usepackage{cancel}
\usepackage[bottom=1in,left=0.5in,right=1in,top=1in]{geometry}
\usepackage{titling}

% From https://code.google.com/p/linear-algebra/source/browse/linalgjh.sty#80
% Using brackets instead of parens.

%-------------bmat
% For matrices with arguments.
% Usage: \begin{bmat}{c|c|c} 1 &2 &3 \end{bmat}
\newenvironment{bmat}[1]{
  \left[\begin{array}{@{}#1@{}}
}{\end{array}\right]
}

\newcommand\numberthis{\addtocounter{equation}{1}\tag{\theequation}}

\setlength{\droptitle}{-10ex}

\preauthor{\begin{flushright}\large \lineskip 0.5em}
\postauthor{\par\end{flushright}}
\predate{\begin{flushright}\large}
\postdate{\par\end{flushright}}

\title{MAT 150A Homework 2\vspace{-2ex}}
\author{Hardy Jones\\
        999397426\\
        Professor Schilling\vspace{-2ex}}
\date{Fall 2014}

\begin{document}
  \maketitle

  \begin{enumerate}
    \item
      We need to show that $a(bc) = (ab)c$

      \begin{proof}
        \begin{align*}
          a(bc) &= a(b) \\
          &= ab \\
          &= a \tag{left} \label{left}
        \end{align*}
        \begin{align*}
          (ab)c &= (ab) \\
          &= ab \\
          &= a \tag{right} \label{right}
        \end{align*}

        Since \ref{left} = \ref{right}, we have shown that the operation is associative.
      \end{proof}

      This law is an identity for sets with exactly one element.

      \begin{proof}
        Assume that a set with more than one element had this law.

        Choose $a \in S$ with $e$ as the identity.

        Then we want that $ae = a = ea$.

        But we see that $ea = e \ne a$.

        It can be shown that if $e = a$ then the identity law holds.
        As $ee = e = ee$.
      \end{proof}
    \item
      We need to show
      \begin{itemize}
        \item $\star$ is closed
          \begin{proof}
            Choose $a, b \in G^O$.
            $a \star b = ba$ and we know that $ba \in G$,
            so since the set is the same between $G$ and $G^O$, we also know $ba \in G^O$.

            Thus, $\star$ is closed.
          \end{proof}
        \item $\forall a, b, c \in G^O, a \star (b \star c) = (a \star b) \star c$
          \begin{proof}
            Choose $a, b, c \in G^O$.

            \begin{align*}
              a \star (b \star c) &= a \star (cb) \\
              &= (cb)a \tag{left} \label{left}
            \end{align*}
            \begin{align*}
              (a \star b) \star c) &= c(a \star b) \\
              &= c(ba) \tag{right} \label{right}
            \end{align*}

            Since we know the underlying group $G$, we know that it is associative.
            So \ref{left} = \ref{right} since $G$ is associative.

            Thus, we have shown that the associativity law holds.
          \end{proof}
        \item $\exists e \in G^O$ s.t. $\forall a \in G^O, a \star e = a = e \star a$
          \begin{proof}
            Choose $a \in G^O$.

            $a \star e = ea = a$ and $e \star a = ae = a$.

            Thus, we have shown that the identity law holds.
          \end{proof}
        \item $\forall a \in G^O, \exists a^{-1} \in G^O$ s.t. $a \star a^{-1} = e = a^{-1} \star a$

          \begin{proof}
            Choose $a \in G^O$.

            $a \star a^{-1} = a^{-1}a = e$ and $a^{-1} \star a = aa^{-1} = e$.

            Thus, we have shown that the inverse law holds.
          \end{proof}
      \end{itemize}

      Since we have shown all four properties of a group,
      we conclude $G^O$ is a group.

    \item
      Let's name our matrix.
      \[
        A =
        \begin{bmatrix}
          1  & 1  \\
          -1 & 0
        \end{bmatrix}
      \]
      \[
        A^2 = AA =
        \begin{bmatrix}
          1  & 1  \\
          -1 & 0
        \end{bmatrix}
        \begin{bmatrix}
          1  & 1  \\
          -1 & 0
        \end{bmatrix}
        =
        \begin{bmatrix}
          0  & 1  \\
          -1 & -1
        \end{bmatrix}
      \]
      \[
        A^3 = A^2A =
        \begin{bmatrix}
          0  & 1  \\
          -1 & -1
        \end{bmatrix}
        \begin{bmatrix}
          1  & 1  \\
          -1 & 0
        \end{bmatrix}
        =
        \begin{bmatrix}
          -1 & 0  \\
          0  & -1
        \end{bmatrix}
      \]
      \[
        A^4 = A^3A =
        \begin{bmatrix}
          -1 & 0  \\
          0  & -1
        \end{bmatrix}
        \begin{bmatrix}
          1  & 1  \\
          -1 & 0
        \end{bmatrix}
        =
        \begin{bmatrix}
          -1 & -1 \\
          1  & 0
        \end{bmatrix}
      \]
      \[
        A^5 = A^4A =
        \begin{bmatrix}
          -1 & -1 \\
          1  & 0
        \end{bmatrix}
        \begin{bmatrix}
          1  & 1  \\
          -1 & 0
        \end{bmatrix}
        =
        \begin{bmatrix}
          0  & -1 \\
          1  & 1
        \end{bmatrix}
      \]
      \[
        A^6 = A^5A =
        \begin{bmatrix}
          0  & -1 \\
          1  & 1
        \end{bmatrix}
        \begin{bmatrix}
          1  & 1  \\
          -1 & 0
        \end{bmatrix}
        =
        \begin{bmatrix}
          1  & 0  \\
          0  & 1
        \end{bmatrix}
      \]

      Since we have generated the identity,
      we have generated all possible elements of this cyclic group.

    \item
      wat

    \item
      \begin{enumerate}
        \item[(b)]
          We need to show:
          \begin{itemize}
            \item Closure
              \begin{proof}
                We can actually prove this by enumeration.
                \[1 \times 1 = 1 \in H\]
                \[1 \times -1 = -1 \in H\]
                \[-1 \times 1 = -1 \in H\]
                \[-1 \times -1 = 1 \in H\]

                So every element is in $H$, thus we have closure.
              \end{proof}
            \item Identity
              \begin{proof}
                Again, we can prove by enumeration that $e = 1$.
                \[1 \times 1 = 1 = 1 \times 1\]
                \[-1 \times 1 = -1 = 1 \times -1\]

                Thus, the identity exists.
              \end{proof}
            \item Inverse
              \begin{proof}
                Once again, we prove by enumeration.
                \[1 \times 1 = 1 = 1 \times 1\]
                \[-1 \times -1 = 1 = -1 \times -1\]

                Thus, each element in $H$ has an inverse.
              \end{proof}
          \end{itemize}

          From these three we have shown that $H$ is a subgroup of $G$.

        \item[(c)]
          $H$ is not a subgroup of $G$ as it lacks an identity element and it lacks inverses.
        \item[(d)]
          We need to show:
          \begin{itemize}
            \item Closure
              \begin{proof}
                Choose $a, b \in H$.

                $a \times b$ is a positive real number.
                So, we have shown closure.
              \end{proof}
            \item Identity
              \begin{proof}
                We want $e = 1$ to be the identity.
                Choose $a \in H$.

                $1 \times a = a = a \times 1$.

                So, we have shown the identity exists.
              \end{proof}
            \item Inverse
              \begin{proof}
                Choose $a \in H$.

                Since $a$ is a real number there exists $\frac{1}{a} \in H$.

                $a \times \frac{1}{a} = 1 = \frac{1}{a} \times a$.

                So we have shown that inverses exist.
              \end{proof}
          \end{itemize}

          From these three we have shown that $H$ is a subgroup of $G$.
        \item[(e)]
          $H$ is not a subgroup of $G$ as $H \not \subseteq G$ since every element of $H$ is not invertible.
      \end{enumerate}

    \item
    \item
      \begin{enumerate}
        \item
          We can enumerate the possibilities with this group.

          \begin{tabular}{| c | c | c | c | c | c |}
            \hline
            $a^0$          & $a^1$          & $a^2$          & $a^3$          & $a^4$          & $a^5$          \\
            $a^0a^0 = a^0$ & $a^1a^1 = a^2$ & $a^2a^2 = a^4$ & $a^3a^3 = a^0$ & $a^4a^4 = a^2$ & $a^5a^5 = a^4$ \\
            $a^0a^0 = a^0$ & $a^2a^1 = a^3$ & $a^4a^2 = a^2$ & $a^0a^3 = a^3$ & $a^2a^4 = a^4$ & $a^4a^5 = a^3$ \\
            $a^0a^0 = a^0$ & $a^3a^1 = a^4$ & $a^2a^2 = a^4$ & $a^3a^3 = a^0$ & $a^4a^4 = a^2$ & $a^3a^5 = a^2$ \\
            $a^0a^0 = a^0$ & $a^4a^1 = a^5$ & $a^4a^2 = a^2$ & $a^0a^3 = a^3$ & $a^2a^4 = a^4$ & $a^2a^5 = a^1$ \\
            $a^0a^0 = a^0$ & $a^5a^1 = a^0$ & $a^2a^2 = a^4$ & $a^3a^3 = a^0$ & $a^4a^4 = a^2$ & $a^1a^5 = a^0$ \\
            \hline
          \end{tabular}

          So, we see two of its elements generate the group.
          Namely, $a^1$ and $a^5$.

        \item
          We again enumerate the possibilities.

          First for order 5.

          \begin{tabular}{| c | c | c | c | c |}
            \hline
            $a^0$          & $a^1$          & $a^2$          & $a^3$          & $a^4$          \\
            $a^0a^0 = a^0$ & $a^1a^1 = a^2$ & $a^2a^2 = a^4$ & $a^3a^3 = a^1$ & $a^4a^4 = a^3$ \\
            $a^0a^0 = a^0$ & $a^2a^1 = a^3$ & $a^4a^2 = a^1$ & $a^1a^3 = a^4$ & $a^3a^4 = a^2$ \\
            $a^0a^0 = a^0$ & $a^3a^1 = a^4$ & $a^1a^2 = a^3$ & $a^4a^3 = a^2$ & $a^2a^4 = a^1$ \\
            $a^0a^0 = a^0$ & $a^4a^1 = a^0$ & $a^3a^2 = a^0$ & $a^2a^3 = a^0$ & $a^1a^4 = a^0$ \\
            \hline
          \end{tabular}

          So, we see 4 of its elements generate the group.
          Namely, $a^1, a^2, a^3,$ and $a^4$

          And for order 8.

          \begin{tabular}{| c | c | c | c | c | c | c | c |}
            \hline
            $a^0$          & $a^1$          & $a^2$          & $a^3$          & $a^4$          & $a^5$          & $a^6$          & $a^7$          \\
            $a^0a^0 = a^0$ & $a^1a^1 = a^2$ & $a^2a^2 = a^4$ & $a^3a^3 = a^6$ & $a^4a^4 = a^0$ & $a^5a^5 = a^2$ & $a^6a^6 = a^4$ & $a^7a^7 = a^6$ \\
            $a^0a^0 = a^0$ & $a^2a^1 = a^3$ & $a^4a^2 = a^6$ & $a^6a^3 = a^1$ & $a^0a^4 = a^4$ & $a^2a^5 = a^7$ & $a^4a^6 = a^2$ & $a^6a^7 = a^5$ \\
            $a^0a^0 = a^0$ & $a^3a^1 = a^4$ & $a^6a^2 = a^0$ & $a^1a^3 = a^4$ & $a^4a^4 = a^0$ & $a^7a^5 = a^4$ & $a^2a^6 = a^0$ & $a^5a^7 = a^4$ \\
            $a^0a^0 = a^0$ & $a^4a^1 = a^5$ & $a^0a^2 = a^2$ & $a^4a^3 = a^7$ & $a^0a^4 = a^4$ & $a^4a^5 = a^1$ & $a^0a^6 = a^6$ & $a^4a^7 = a^3$ \\
            $a^0a^0 = a^0$ & $a^5a^1 = a^6$ & $a^2a^2 = a^4$ & $a^7a^3 = a^2$ & $a^4a^4 = a^0$ & $a^1a^5 = a^6$ & $a^6a^6 = a^4$ & $a^3a^7 = a^2$ \\
            $a^0a^0 = a^0$ & $a^6a^1 = a^7$ & $a^4a^2 = a^6$ & $a^2a^3 = a^5$ & $a^0a^4 = a^4$ & $a^6a^5 = a^3$ & $a^4a^6 = a^2$ & $a^2a^7 = a^1$ \\
            $a^0a^0 = a^0$ & $a^7a^1 = a^0$ & $a^6a^2 = a^0$ & $a^5a^3 = a^0$ & $a^4a^4 = a^0$ & $a^3a^5 = a^0$ & $a^2a^6 = a^0$ & $a^1a^7 = a^0$ \\
            \hline
          \end{tabular}

          So, we see 4 of its elements generate the group.
          Namely, $a^1, a^3, a^5,$ and $a^7$.

        \item
          If we look at the generators for each of the previous groups,
          we notice that the elements are generators when $gcd(i, n) = 1$,
          where $i$ is the element and $n$ is the order of the group.
          In other words, it is the count of the number of coprimes of $n$.

          But we know that Euler's totient, $\varphi(n)$ provides us with this number.

          Euler's totient is defined as:
          \[\varphi(n) = n\prod_{p|n}^n\left(1-\frac{1}{p}\right)\]

          where $p$ are distinct prime numbers.

          We can double check this for the cases above.

          \begin{itemize}
            \item $n = 6$
              \begin{align*}
                \varphi(6) &= 6\prod_{p|6}^6\left(1-\frac{1}{p}\right) \\
                &= 6\left(\left(1-\frac{1}{2}\right)\left(1-\frac{1}{3}\right)\right) \\
                &= 6\left(\frac{1}{2}\frac{2}{3}\right) \\
                &= 6\left(\frac{1}{3}\right) \\
                &= 2 \\
              \end{align*}
            \item $n = 5$
              \begin{align*}
                \varphi(5) &= 5\prod_{p|5}^5\left(1-\frac{1}{p}\right) \\
                &= 5\left(1-\frac{1}{5}\right) \\
                &= 5\left(\frac{4}{5}\right) \\
                &= 4 \\
              \end{align*}
            \item $n = 8$
              \begin{align*}
                \varphi(8) &= 8\prod_{p|8}^8\left(1-\frac{1}{p}\right) \\
                &= 8\left(1-\frac{1}{2}\right) \\
                &= 8\left(\frac{1}{2}\right) \\
                &= 4 \\
              \end{align*}
          \end{itemize}

          So, in general, we have $\varphi(n)$ generators in a cyclic group.
      \end{enumerate}
  \end{enumerate}
\end{document}
