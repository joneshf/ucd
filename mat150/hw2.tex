\documentclass[12pt,letterpaper]{article}
\usepackage{amsmath}
\usepackage{amsfonts}
\usepackage{amsthm}
\usepackage{mathtools}
\usepackage{cancel}
\usepackage[margin=1in]{geometry}
\usepackage{titling}

% From https://code.google.com/p/linear-algebra/source/browse/linalgjh.sty#80
% Using brackets instead of parens.

%-------------bmat
% For matrices with arguments.
% Usage: \begin{bmat}{c|c|c} 1 &2 &3 \end{bmat}
\newenvironment{bmat}[1]{
  \left[\begin{array}{@{}#1@{}}
}{\end{array}\right]
}

\newcommand\numberthis{\addtocounter{equation}{1}\tag{\theequation}}

\setlength{\droptitle}{-10ex}

\preauthor{\begin{flushright}\large \lineskip 0.5em}
\postauthor{\par\end{flushright}}
\predate{\begin{flushright}\large}
\postdate{\par\end{flushright}}

\title{MAT 150A Homework 2\vspace{-2ex}}
\author{Hardy Jones\\
        999397426\\
        Professor Schilling\vspace{-2ex}}
\date{Fall 2014}

\begin{document}
  \maketitle

  \begin{enumerate}
    \item
      We need to show that $a(bc) = (ab)c$

      \begin{proof}
        \begin{align*}
          a(bc) &= a(b) \\
          &= ab \\
          &= a \tag{left} \label{left}
        \end{align*}
        \begin{align*}
          (ab)c &= (ab) \\
          &= ab \\
          &= a \tag{right} \label{right}
        \end{align*}

        Since \ref{left} = \ref{right}, we have shown that the operation is associative.
      \end{proof}

      This law is an identity for sets with exactly one element.

      \begin{proof}
        Assume that a set with more than one element had this law.

        Choose $a \in S$ with $e$ as the identity.

        Then we want that $ae = a = ea$.

        But we see that $ea = e \ne a$.

        It can be shown that if $e = a$ then the identity law holds.
        As $ee = e = ee$.
      \end{proof}
    \item
      We need to show
      \begin{itemize}
        \item $\star$ is closed
          \begin{proof}
            Choose $a, b \in G^O$.
            $a \star b = ba$ and we know that $ba \in G$,
            so since the set is the same between $G$ and $G^O$, we also know $ba \in G^O$.

            Thus, $\star$ is closed.
          \end{proof}
        \item $\forall a, b, c \in G^O, a \star (b \star c) = (a \star b) \star c$
          \begin{proof}
            Choose $a, b, c \in G^O$.

            \begin{align*}
              a \star (b \star c) &= a \star (cb) \\
              &= (cb)a \tag{left} \label{left}
            \end{align*}
            \begin{align*}
              (a \star b) \star c) &= c(a \star b) \\
              &= c(ba) \tag{right} \label{right}
            \end{align*}

            Since we know the underlying group $G$, we know that it is associative.
            So \ref{left} = \ref{right} since $G$ is associative.

            Thus, we have shown that the associativity law holds.
          \end{proof}
        \item $\exists e \in G^O$ s.t. $\forall a \in G^O, a \star e = a = e \star a$
          \begin{proof}
            Choose $a \in G^O$.

            $a \star e = ea = a$ and $e \star a = ae = a$.

            Thus, we have shown that the identity law holds.
          \end{proof}
        \item $\forall a \in G^O, \exists a^{-1} \in G^O$ s.t. $a \star a^{-1} = e = a^{-1} \star a$

          \begin{proof}
            Choose $a \in G^O$.

            $a \star a^{-1} = a^{-1}a = e$ and $a^{-1} \star a = aa^{-1} = e$.

            Thus, we have shown that the inverse law holds.
          \end{proof}
      \end{itemize}

      Since we have shown all four properties of a group,
      we conclude $G^O$ is a group.

    \item
      Let's name our matrix.
      \[
        A =
        \begin{bmatrix}
          1  & 1  \\
          -1 & 0
        \end{bmatrix}
      \]
      \[
        A^2 = AA =
        \begin{bmatrix}
          1  & 1  \\
          -1 & 0
        \end{bmatrix}
        \begin{bmatrix}
          1  & 1  \\
          -1 & 0
        \end{bmatrix}
        =
        \begin{bmatrix}
          0  & 1  \\
          -1 & -1
        \end{bmatrix}
      \]
      \[
        A^3 = A^2A =
        \begin{bmatrix}
          0  & 1  \\
          -1 & -1
        \end{bmatrix}
        \begin{bmatrix}
          1  & 1  \\
          -1 & 0
        \end{bmatrix}
        =
        \begin{bmatrix}
          -1 & 0  \\
          0  & -1
        \end{bmatrix}
      \]
      \[
        A^4 = A^3A =
        \begin{bmatrix}
          -1 & 0  \\
          0  & -1
        \end{bmatrix}
        \begin{bmatrix}
          1  & 1  \\
          -1 & 0
        \end{bmatrix}
        =
        \begin{bmatrix}
          -1 & -1 \\
          1  & 0
        \end{bmatrix}
      \]
      \[
        A^5 = A^4A =
        \begin{bmatrix}
          -1 & -1 \\
          1  & 0
        \end{bmatrix}
        \begin{bmatrix}
          1  & 1  \\
          -1 & 0
        \end{bmatrix}
        =
        \begin{bmatrix}
          0  & -1 \\
          1  & 1
        \end{bmatrix}
      \]
      \[
        A^6 = A^5A =
        \begin{bmatrix}
          0  & -1 \\
          1  & 1
        \end{bmatrix}
        \begin{bmatrix}
          1  & 1  \\
          -1 & 0
        \end{bmatrix}
        =
        \begin{bmatrix}
          1  & 0  \\
          0  & 1
        \end{bmatrix}
      \]

      Since we have generated the identity,
      we have generated all possible elements of this cyclic group.
  \end{enumerate}
\end{document}
