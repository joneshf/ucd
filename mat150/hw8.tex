\documentclass[12pt,letterpaper]{article}
\usepackage{amsmath}
\usepackage{amsfonts}
\usepackage{amsthm}
\usepackage{mathtools}
\usepackage{cancel}
\usepackage[bottom=1in,left=0.5in,right=1in,top=1in]{geometry}
\usepackage{titling}

% From https://code.google.com/p/linear-algebra/source/browse/linalgjh.sty#80
% Using brackets instead of parens.

%-------------bmat
% For matrices with arguments.
% Usage: \begin{bmat}{c|c|c} 1 &2 &3 \end{bmat}
\newenvironment{bmat}[1]{
  \left[\begin{array}{@{}#1@{}}
}{\end{array}\right]
}

\newcommand\numberthis{\addtocounter{equation}{1}\tag{\theequation}}
\newcommand{\GLNR}[0]{GL_n(\mathbb{R})}
\newcommand{\KER}[1]{\text{ker }#1}
\newcommand{\IM}[1]{\text{im }#1}
\newcommand{\ghg}[1]{g#1g^{-1}}
\newcommand{\abs}[1]{\lvert#1\rvert}
\newcommand{\norm}[1]{\lVert#1\rVert}

\setlength{\droptitle}{-10ex}

\preauthor{\begin{flushright}\large \lineskip 0.5em}
\postauthor{\par\end{flushright}}
\predate{\begin{flushright}\large}
\postdate{\par\end{flushright}}

\title{MAT 150A Homework 8\vspace{-2ex}}
\author{Hardy Jones\\
        999397426\\
        Professor Schilling\vspace{-2ex}}
\date{Fall 2014}

\begin{document}
  \maketitle

  \begin{enumerate}
    \item
      Choose a subgroup $H \le G$.

      Then $\exists \varepsilon > 0$ s.t.:
      \begin{enumerate}
        \item $\forall t_a \in H, \norm{a} \ge \varepsilon$ as $t_a \in G$
        \item $\forall \rho_\theta \in H, \abs{\theta} \ge \varepsilon$ as $\rho_\theta \in G$
      \end{enumerate}
      Since our choice of $H$ was arbitrary,
      any subgroup $H \le G$ is discrete.

    \item
      If a discrete group $G$ is rotations about the origin,
      let $\rho_\theta \in G$ be the smallest rotation.

      Choose $\rho_\psi \in G$,
      then $\exists \varphi \in \mathbb{R}, n \in \mathbb{N}$
      such that $\psi = n \theta + \varphi$ and $0 \le \varphi < \theta$.

      Then
      \begin{align*}
        \rho_\psi &= \rho_{n \theta + \varphi} \\
        &= \rho_{n \theta} \rho_\varphi \\
        &= \rho_\theta^n \rho_\varphi \\
        \rho_\theta^{-n} \rho_\psi &= (\rho_\theta^{-n} \rho_\theta^n) \rho_\varphi \\
        \rho_\theta^{-n} \rho_\psi &= \rho_\varphi \\
      \end{align*}

      Since the LHS is in $G$, the RHS is also in $G$.
      But, as we chose $\theta$ as the smallest rotation angle,
      this implies $\varphi = 0$.

      \begin{align*}
        \rho_\theta^{-n} \rho_\psi &= \rho_0 \\
        (\rho_\theta^{n} \rho_\theta^{-n}) \rho_\psi &= \rho_\theta^{n} \rho_0 \\
        \rho_\psi &= \rho_\theta^{n} \\
        \rho_\psi &= \rho_{n\theta} \\
      \end{align*}

      So we have that $\psi = n \theta$ for some $n \in \mathbb{N}$.
      Since our choice of $\rho_\psi$ was arbitrary,
      this shows that every rotation in $G$ is generated by the smallest rotation.
      In other words, a discrete group $G$ of rotations about the origin is a cyclic group generated by the smallest rotation $\rho_\theta \in G$.

    \item
      Choose two elements of $G$ as $a = t_a \rho_\theta, b = t_b \rho_\psi$.

      \begin{align*}
        aba^{-1}b^{-1} &= (t_a \rho_\theta) (t_b \rho_\psi) (t_a \rho_\theta)^{-1} (t_b \rho_\psi)^{-1} \\
        &= (t_a \rho_\theta) (t_b \rho_\psi) (\rho_\theta^{-1} t_a^{-1}) (\rho_\psi^{-1} t_b^{-1}) \\
        &= (t_a \rho_\theta) (t_b \rho_\psi) (\rho_{-\theta} t_{-a}) (\rho_{-\psi} t_{-b}) \\
        &= (t_a \rho_\theta) t_b (\rho_\psi \rho_{-\theta}) t_{-a} (\rho_{-\psi} t_{-b}) \\
        &= t_a (\rho_\theta t_b) \rho_{\psi - \theta} t_{-a} (\rho_{-\psi} t_{-b}) \\
        &= t_a (t_{b'} \rho_\theta) \rho_{\psi - \theta} t_{-a} (\rho_{-\psi} t_{-b}) \\
        &= t_{a + b'} (\rho_\theta \rho_{\psi - \theta}) t_{-a} (\rho_{-\psi} t_{-b}) \\
        &= t_{a + b'} (\rho_{\theta + \psi - \theta} t_{-a}) (\rho_{-\psi} t_{-b}) \\
        &= t_{a + b'} (\rho_{\psi} t_{-a}) (\rho_{-\psi} t_{-b}) \\
        &= (t_{a + b'} t_{-a'}) (\rho_{\psi} \rho_{-\psi}) t_{-b} \\
        &= t_{a + b' - a'} (\rho_{\psi} \rho_{-\psi}) t_{-b} \\
        &= t_{a + b' - a'} \rho_{\psi - \psi} t_{-b} \\
        &= t_{a + b' - a'} t_{-b} \\
        &= t_{a + b' - a' -b} \\
      \end{align*}

      So $G$ contains a translation.

    \item
      Since $\mathcal{O}_2$ does not contain translations,
      we only need concern ourselves with showing:
      \[
        \forall \text{ discrete subgroups } G \le \mathcal{O}_2, \exists \varepsilon > 0 \text{ s.t. } \forall \rho_\theta \in G, \abs{\theta} \ge \varepsilon
      \]

      Choose a discrete subgroup $G \le \mathcal{O}_2$.

      Suppose $G$ is infinite.
      That is, there are an infinite number of angles.

      Then, for any $\varepsilon > 0$ we can choose an $n \in \mathbb{N}$
      and construct $n$ divisions from 0 to $2\pi$ each of size $\frac{2 \pi \varepsilon}{n}$.

      Since $G$ is infinite, there must be at least two non-zero angles $\theta, \psi$
      such that $\abs{\theta - \psi} < \frac{2 \pi}{n} < \varepsilon \le \abs{\theta + \psi`'}$.

      This means that $\rho_\theta \rho_\psi \in G$.

      Which also means that $\rho_\theta \rho_\psi^{-1} = \rho_\theta \rho_{-\psi} = \rho_{\theta - \psi} \in G$.

      But $\theta - \psi < \varepsilon$, which would make this not a discrete subgroup by construction.

      So our assumption was incorrect and $G$ must not be infinite.

      Thus any discrete subgroup $G \le \mathcal{O}_2$ is finite.

    \item
      $G$ acts transitively on $G$-set $S$ if,
      $\forall s_1, s_2 \in S, s_2$ is the orbit of $s_1$ in $S$.

      That is, $O_{s_1} = \{s_2 \in S | s_2 = g s_1 \text{ for some } g \in G\}$

    \item
      We want to show that $G$ acts faithfully on $S$ $\iff$
      $\forall g, g' \in G, s \in S, (g \ne g' \implies g s \ne g' s)$
      \begin{itemize}
        \item ($\Longrightarrow$)

          Choose $e, g \in G$ where $e \ne g$.
          \begin{align*}
            e s &\ne g s \\
            s &\ne g s
          \end{align*}
          Since $G$ acts faithfully on $S$.

          So, we have shown for any $g, g' \in G, s \in S, (g \ne g' \implies g s \ne g' s)$

        \item ($\Longleftarrow$)

          Choose $e, g \in G$ where $e \ne g$ and any $s \in S$.

          We know that $g s \ne e s$ so $g s \ne s$.

          This is the contrapositive of $G$ acting faithfully on $G$-set $S$.

          So we have that $G$ acts faithfully on $G$-set $S$.
      \end{itemize}

      From these two, we have shown both directions.

      Thus, $G$ acts faithfully on $S$ $\iff$
      $\forall g, g' \in G, s \in S, (g \ne g' \implies g s \ne g' s)$
  \end{enumerate}
\end{document}
