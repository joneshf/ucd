\documentclass[12pt,letterpaper]{article}
\usepackage{amsmath}
\usepackage{amsfonts}
\usepackage{amsthm}
\usepackage{cancel}
\usepackage[bottom=1in,left=0.4in,right=1in,top=1in]{geometry}
\usepackage{titling}
\usepackage{multirow}
\usepackage{amssymb}
\usepackage{algorithm2e}
\usepackage{wasysym}
\usepackage{qtree}

\newcommand{\lb}[0]{\text{lg}}
\newcommand{\parens}[1]{\left(#1\right)}
\newcommand{\Ans}[0]{\text{Ans}}
\newcommand{\Car}[0]{\text{Car}}
\newcommand{\EV}[0]{\text{EV}}
\newcommand{\Pickup}[0]{\text{Pickup}}
\newcommand{\Product}[0]{\text{Product}}
\newcommand{\AND}[0]{\text{ AND }}
\newcommand{\OR}[0]{\text{ OR }}

\setlength{\droptitle}{-10ex}

\preauthor{\begin{flushright}\large \lineskip 0.5em}
\postauthor{\par\end{flushright}}
\predate{\begin{flushright}\large}
\postdate{\par\end{flushright}}

\title{ECS 165A Homework 1\vspace{-2ex}}
\author{Hardy Jones\\
        999397426\\
        Professor Nitta\vspace{-2ex}}
\date{Fall 2014}

\begin{document}
  \maketitle

  \begin{enumerate}
    \item
      \begin{enumerate}
        \item \[\Pi_{\parens{model}}\parens{\sigma_{city > 30}\parens{\Car}}\]
        \item \[\Pi_{\parens{model}}\parens{\sigma_{towing > 15000 \AND highway > 18}\parens{\Pickup}}\]
        \item
          \begin{align*}
            C(make)  &:= \Pi_{\parens{make}}\parens{\Car \bowtie_{\Car.model = \Product.model \AND price < 20000} \Product} \\
            P(make)  &:= \Pi_{\parens{make}}\parens{\Pickup \bowtie_{\Pickup.model = \Product.model \AND price < 20000} \Product} \\
            E(make)  &:= \Pi_{\parens{make}}\parens{\EV \bowtie_{\EV.model = \Product.model \AND price < 20000} \Product} \\
            C'(make) &:= \Pi_{\parens{make}}\parens{\Car \bowtie_{\Car.model = \Product.model \AND price > 50000} \Product} \\
            P'(make) &:= \Pi_{\parens{make}}\parens{\Pickup \bowtie_{\Pickup.model = \Product.model \AND price > 50000} \Product} \\
            E'(make) &:= \Pi_{\parens{make}}\parens{\EV \bowtie_{\EV.model = \Product.model \AND price > 50000} \Product} \\
            C''(make) &:= \parens{C \bowtie C'} \cup \parens{C \bowtie P'} \cup \parens{C \bowtie E'} \\
            P''(make) &:= \parens{P \bowtie C'} \cup \parens{P \bowtie P'} \cup \parens{P \bowtie E'} \\
            E''(make) &:= \parens{E \bowtie C'} \cup \parens{E \bowtie P'} \cup \parens{E \bowtie E'} \\
            \Ans(make) &:= C'' \cup P'' \cup E''
          \end{align*}
        \item
          \begin{align*}
            C(passengers) &:= \Pi_{\parens{passengers}}\parens{\rho_C\parens{\Car} \bowtie_{C.passengers = C'.passengers \AND C.model \ne C'.model} \rho_{C'}\parens{\Car}} \\
            E(passengers) &:= \Pi_{\parens{passengers}}\parens{\rho_E\parens{\EV} \bowtie_{E.passengers = E'.passengers \AND E.model \ne E'.model} \rho_{E'}\parens{\EV}} \\
            P(passengers) &:= \Pi_{\parens{passengers}}\parens{\rho_P\parens{\Pickup} \bowtie_{P.passengers = P'.passengers \AND P.model \ne P'.model} \rho_{P'}\parens{\Pickup}} \\
            CE(passengers) &:= \Pi_{\parens{passengers}}\parens{\Car \bowtie_{\Car.passengers = \EV.passengers} \EV} \\
            CP(passengers) &:= \Pi_{\parens{passengers}}\parens{\Car \bowtie_{\Car.passengers = \Pickup.passengers} \Pickup} \\
            PE(passengers) &:= \Pi_{\parens{passengers}}\parens{\Pickup \bowtie_{\Pickup.passengers = \EV.passengers} \EV} \\
            \Ans(passengers) &:= C \cup E \cup P \cup CE \cup CP \cup PE
          \end{align*}
        \item
          \begin{align*}
            C(city, highway, model) &:= \rho_C\parens{\Car} \bowtie_{C.city * 0.55 + C.highway * 0.45 > C'.city * 0.55 + C'.highway * 0.45} \rho_{C'}\parens{\Car} \\
            C'(city, highway, model) &:= \Pi_{\parens{city, highway, model}}\parens{\Car} - C \\
            P(city, highway, model) &:= \rho_P\parens{\Pickup} \bowtie_{P.city * 0.55 + P.highway * 0.45 > P'.city * 0.55 + P'.highway * 0.45} \rho_{P'}\parens{\Pickup} \\
            P'(city, highway, model) &:= \Pi_{\parens{city, highway, model}}\parens{\Pickup} - P \\
            Both(model) &:= \Pi_{\parens{model}}\parens{C' \bowtie_{C'.city * 0.55 + C'.highway * 0.45 \ge P'.city * 0.55 + P'.highway * 0.45} P'} \\
            Both'(model) &:= \Pi_{\parens{model}}\parens{P' \bowtie_{P'.city * 0.55 + P'.highway * 0.45 \ge C'.city * 0.55 + C'.highway * 0.45} C'} \\
            \Ans(make) &:= \Pi_{\parens{make}}\parens{\parens{Both \cup Both'} \bowtie \Product}
          \end{align*}
        \item
          \begin{align*}
            C(city, highway, model) &:= \rho_C\parens{\Car} \bowtie_{C.city * 0.55 + C.highway * 0.45 > C'.city * 0.55 + C'.highway * 0.45} \rho_{C'}\parens{\Car} \\
            C'(city, highway, model) &:= \Pi_{\parens{city, highway, model}}\parens{C} - C \\
            E(battery, model, range) &:= \rho_E\parens{\EV} \bowtie_{\frac{E.range * 33.1}{E.battery} > \frac{E'.range * 33.1}{E'.battery}} \rho_{E'}\parens{\EV} \\
            E'(battery, model, range) &:= \Pi_{\parens{battery, model, range}}\parens{E} - E \\
            P(city, highway, model) &:= \rho_P\parens{\Pickup} \bowtie_{P.city * 0.55 + P.highway * 0.45 > P'.city * 0.55 + P'.highway * 0.45} \rho_{P'}\parens{\Pickup} \\
            P'(city, highway, model) &:= \Pi_{\parens{city, highway, model}}\parens{P} - P \\
            CP(city, highway, model) &:= C' \bowtie_{C'.city * 0.55 + C'.highway * 0.45 \ge P'.city * 0.55 + P'.highway * 0.45} P' \\
            CP'(city, highway, model) &:= P' \bowtie_{P'.city * 0.55 + P'.highway * 0.45 \ge C'.city * 0.55 + C'.highway * 0.45} C' \\
            CP''(city, highway, model) &:= CP \cup CP' \\
            CEP(model) &:= \Pi_{\parens{model}}\parens{E' \bowtie_{\frac{E'.range * 33.1}{E'.battery} \ge CP''.city * 0.55 + CP''.highway * 0.45} CP''} \\
            CEP'(model) &:= \Pi_{\parens{model}}\parens{CP'' \bowtie_{CP''.city * 0.55 + CP''.highway * 0.45 \ge \frac{E'.range * 33.1}{E'.battery}} E'} \\
            \Ans(make) &:= \Pi_{\parens{make}}\parens{\parens{CEP \cup CEP'} \bowtie \Product}
          \end{align*}
        \item
          \begin{align*}
            P(make) &:= \Pi_{\parens{make}}\parens{\Pickup \bowtie_{city < 10 \AND \Pickup.model = \Product.model} \Product} \\
            E(make) &:= \Pi_{\parens{make}}\parens{\EV \bowtie \Product} \\
            \Ans(make) &:= E \cap P
          \end{align*}
        \item
          \begin{align*}
            C(make, vol) &:= \rho_{C(make, vol)}\parens{\Pi_{\parens{make, trunk}}\parens{\Car \bowtie \Product}} \\
            P(make, vol) &:= \rho_{P(make, vol)}\parens{\Pi_{\parens{make, cargo}}\parens{\Pickup \bowtie \Product}} \\
            CP(make, vol) &:= C \cup P \\
            CP'(make, vol) &:= \rho_{Left}CP \bowtie_{Left.vol > Right.vol \AND Left.make = Right.make} \rho_{Right}CP \\
            CP''(make, vol) &:= \rho_{Left}CP' \bowtie_{Left.vol > Right.vol \AND Left.make = Right.make} \rho_{Right}CP' \\
          \end{align*}
      \end{enumerate}

    \item
      \begin{enumerate}
        \item
          \[
            \sigma_{towing < 10000}\parens{\Pickup} \subseteq \sigma_{price \le 22000}\parens{\Pickup}
          \]
        \item
          \[
            \EV \subseteq \sigma_{range = 100}\parens{\EV}
          \]
        \item
          \[
            \sigma_{city \ge 50}\parens{\Car} \subseteq \sigma_{highway \ge 40 \OR price < 18000}\parens{\Car}
          \]
        \item
          \[
            \parens{\Pi_{make}\parens{\Pickup \bowtie \Product}} \cap \parens{\Pi_{make}\parens{\EV \bowtie \Product}} = \emptyset
          \]
        \item
          \begin{align*}
            C(city, make, model) &:= \Pi_{city, make, model}\parens{\Car \bowtie \Product} \\
            P(city, make, model) &:= \Pi_{city, make, model}\parens{\Pickup \bowtie \Product} \\
          \end{align*}
          \[
            C \bowtie_{C.make = P.make \AND C.city \le P.city} P = \emptyset
          \]
        \item
          \[
            \Car \bowtie_{\Car.highway < \Pickup.highway \AND \Pickup.price \le \Car.price} \Pickup = \emptyset
          \]
      \end{enumerate}

    \item
      \begin{enumerate}
        \item We expect the following functional dependencies to hold:
          \begin{itemize}
            \item $ID \rightarrow P_X \ P_Y \ P_Z \ V_X \ V_Y \ V_Z$
            \item $P_X \ P_Y \ P_Z \rightarrow ID \ V_X \ V_Y \ V_Z$

              Since no two distinct molecules can occupy the same position in space.
          \end{itemize}
        \item The keys are $\{ID\}$ and $\{P_X, P_Y, P_Z\}$,
          as each of these sets can fully determine the other attributes of the relation,
          and no proper subset of either of these can also fully determine the other attributes of the relation
        \item Three possible superkeys are:
          \begin{itemize}
            \item $\{ID, P_X\}$
            \item $\{ID, V_X\}$
            \item $\{P_X, P_Y, P_Z, ID\}$
          \end{itemize}
        \item
          Since we have two keys, we have to be careful calculating this.

          Let's look first at the superkeys generated by the key $\{ID\}$.
          We want to find all the possible sets that include at least this key.
          We can begin by enumerating some possible superkeys.

          \[\{\{ID, P_X\}, \{ID, P_Y\}, \{ID, P_Z\}, \{ID, V_X\}, \{ID, V_Y\}, \{ID, V_Z\}, \{ID, P_X, P_Y\} \dots \}\]

          If we're careful in our enumeration, we can see that we're almost generating the powerset of the remaining attributes.
          We do not generate $\{ID\}$ again, as this is the key.

          So, this set generates $|\mathcal{P}\left(\{P_X, P_Y, P_Z, V_X, V_Y, V_Z\}\right)| - 1 = 2^6 - 1 = 63$ superkeys.

          Following similar logic, we see the other key generates $|\mathcal{P}\{ID, V_X, V_Y, V_Z\}| - 1 = 2^4 - 1 = 15$ superkeys.

          Now, we just need to combine the number of superkeys generated.
          We have to be careful to remove duplicates.
          These occur when the superkey contains $\{ID, P_X, P_Y, P_Z\}$.

          While we can actually enumerate these duplicates fairly easily,
          we can also see that this will produce
          \[|\mathcal{P}\left(\{ID, P_X, P_Y, P_Z, V_X, V_Y, V_Z\} - \{ID, P_X, P_Y, P_Z\}\right)| = |\mathcal{P}\left(\{V_X, V_Y, V_Z\}\right)| = 8\]
          duplicates.

          Finally, we see that there are $63 + 15 - 8 = 70$ superkeys not including the keys.
      \end{enumerate}
  \end{enumerate}
\end{document}
