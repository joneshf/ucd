\documentclass[12pt,letterpaper]{article}
\usepackage[utf8]{inputenc}
\usepackage{amsmath}
\usepackage{amsfonts}
\usepackage{amsthm}
\usepackage{mathtools}
\usepackage{cancel}
\usepackage[margin=1in]{geometry}
\usepackage{titling}
\usepackage{minted}
\usepackage[tone]{tipa}
\usepackage{qtree}
\usepackage[super]{nth}
\usepackage[title]{appendix}
\usepackage[backend=biber]{biblatex}
\addbibresource{final.bib}

\usepackage{newunicodechar}
\newunicodechar{ɾ}{\textfishhookr}

\setlength{\droptitle}{-10ex}

\preauthor{\begin{flushright}\large \lineskip 0.5em}
\postauthor{\par\end{flushright}}
\predate{\begin{flushright}\large}
\postdate{\par\end{flushright}}

\title{LIN 177 Final\vspace{-2ex}}
\author{Hardy Jones\\
        999397426\\
        Professor Ojeda\vspace{-2ex}}
\date{Winter 2015}

\begin{document}
  \maketitle

  \newmintedfile[prologfile]{prolog}{ fontsize=\footnotesize
                                    , frame=single
                                    }

  For this final, I decided to implement part of SLIPA and use that to translate between English and ASL. \autocite{slipa}
  The program works similarly to \texttt{spanish.swipl}.

  SLIPA was invented to be a version of IPA for sign language.
  The system allows creating new signed languages building upon primitives.
  There are many different signed languages in written format, however most are hard to work with with computerized fonts.
  The idea of SLIPA was to provide an ASCII realizable format for signed languages.
  SLIPA provides primitives for places, movements, hand signs, etc.

  The \texttt{asl} predicate supports places, movement, hand shapes and single handed signs.
  This program doesn't encode all of SLIPA. For example, it does not support facial expressions, such as raising or lowering eye brows.
  However, it should be fairly simple to improve the program to add this.

  The \texttt{asl} predicate uses an \texttt{under\_asl} predicate which creates possible subjects and verbs.
  \texttt{asl} then combines the possible subjects and verbs into valid sentences.
  Each possible \texttt{under\_asl} ensures that the sign is grammatically possible.
  There are five general grammar rules that all signed languages appear to follow.
  These rules are defined in \texttt{sign.swipl}.

  More information can be found with each module.

  \begin{appendices}
    \section{R code}

        \subsection*{Problem 1}
            \rData{prob1.R}
            \subsubsection*{(a)}
                \rData{prob1a.R}
            \subsubsection*{(b)}
                \rData{prob1b.R}

        \subsection*{Problem 2}
            \rData{prob2.R}
            \subsubsection*{(a)}
                \rData{prob2a.R}
            \subsubsection*{(b)}
                \rData{prob2b.R}

        \subsection*{Problem 3}
            \rData{prob3.R}
            \subsubsection*{(a)}
                \rData{prob3a.R}
            \subsubsection*{(b)}
                \rData{prob3b.R}

\end{appendices}


  \printbibliography

\end{document}
