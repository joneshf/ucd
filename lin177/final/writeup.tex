\documentclass[12pt,letterpaper]{article}
\usepackage[utf8]{inputenc}
\usepackage{amsmath}
\usepackage{amsfonts}
\usepackage{amsthm}
\usepackage{mathtools}
\usepackage{cancel}
\usepackage[margin=1in]{geometry}
\usepackage{titling}
\usepackage{minted}
\usepackage[tone]{tipa}
\usepackage{qtree}
\usepackage[super]{nth}

\usepackage{newunicodechar}
\newunicodechar{ɾ}{\textfishhookr}

\setlength{\droptitle}{-10ex}

\preauthor{\begin{flushright}\large \lineskip 0.5em}
\postauthor{\par\end{flushright}}
\predate{\begin{flushright}\large}
\postdate{\par\end{flushright}}

\title{LIN 177 Term Project Proposal\vspace{-2ex}}
\author{Hardy Jones\\
        999397426\\
        Professor Ojeda\vspace{-2ex}}
\date{Winter 2015}

\begin{document}
  \maketitle

  For my term project I would like to create a program that translates between ASL and English.
  This seems like an interesting project for a few reasons:

  \begin{itemize}
    \item ASL is not actually a written language. More on that below.
    \item I will learn how to implement a language from first principles.
    \item I get a chance to learn two additional languages.
  \end{itemize}

  To implement this program in Prolog, I need a textual representation of ASL.
  Unfortunately, no standard language exists.
  However, there are at least two choices that should work.

  \begin{description}
    \item [Stokoe Notation] \

      \begin{itemize}
        \item This is considered the original ASL language.
        \item It has the benefit of being well documented and lots of material is available.
      \end{itemize}

    \item [SLIPA] \

      \begin{itemize}
        \item This is the sign language equivalent of IPA.
        \item This seems more fundamental, and could later be extended to work with other sign languages
      \end{itemize}
  \end{description}

  Both of these options have ASCII formats,
  so it should be possible to use them in prolog.

  For the program, I would like it to translate at least simple sentences.
  In the same style as \texttt{spanish.swipl},
  I plan to associate a sign and category with an english sound and meaning.

\end{document}
