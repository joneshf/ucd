\documentclass[12pt,letterpaper]{article}
\usepackage{amsmath}
\usepackage{amsfonts}
\usepackage{amsthm}
\usepackage{mathtools}
\usepackage{cancel}
\usepackage[margin=1in]{geometry}
\usepackage{titling}
\usepackage{minted}

\setlength{\droptitle}{-10ex}

\preauthor{\begin{flushright}\large \lineskip 0.5em}
\postauthor{\par\end{flushright}}
\predate{\begin{flushright}\large}
\postdate{\par\end{flushright}}

\title{LIN 177 Homework 1\vspace{-2ex}}
\author{Hardy Jones\\
        999397426\\
        Professor Ojeda\vspace{-2ex}}
\date{Winter 2015}

\begin{document}
  \maketitle

  \newmintedfile[prolog]{prolog}{ fontsize=\footnotesize
                                , frame=single
                                }

  \begin{enumerate}
    \item
      \prolog{hah.pl}

    \item
      \prolog{blather.txt}

    \item
      \begin{itemize}
        \item
          Two options are the phrases, ``Get the Led out.'' and ``Get the lead out.''

          The first means to listen to the band \textit{Led Zeppelin}.
          The second means to remove the metal ``lead'' from something.

        \item
          Two options are the phrases, ``Susan's mother is Carol.'' and ``The mother of Susan is Carol.''

          These two sound different, but have the same meaning of identifying Susan's mother.
      \end{itemize}

    \pagebreak

    \item
      \prolog{japanese.swipl}

      This grammar is principled.

      We give rules for proper conjugation of accusative noun-phrases, for instance.
      Rather than enumerating all possible ways to construct an accusative noun-phrase,
      we instead give a rule that it's a normal noun-phrase with ``o'' appended to it.

    \item
      \prolog{spanish_japanese.swipl}
  \end{enumerate}
\end{document}
