\documentclass[12pt,letterpaper]{article}
\usepackage[utf8]{inputenc}
\usepackage{amsmath}
\usepackage{amsfonts}
\usepackage{amsthm}
\usepackage{mathtools}
\usepackage{cancel}
\usepackage[margin=1in]{geometry}
\usepackage{titling}
\usepackage{minted}
\usepackage[tone]{tipa}
\usepackage{qtree}

\usepackage{newunicodechar}
\newunicodechar{ɾ}{\textfishhookr}

\setlength{\droptitle}{-10ex}

\preauthor{\begin{flushright}\large \lineskip 0.5em}
\postauthor{\par\end{flushright}}
\predate{\begin{flushright}\large}
\postdate{\par\end{flushright}}

\title{LIN 177 Midterm\vspace{-2ex}}
\author{Hardy Jones\\
        999397426\\
        Professor Ojeda\vspace{-2ex}}
\date{Winter 2015}

\begin{document}
  \maketitle

  \newmintedfile[prologfile]{prolog}{ fontsize=\footnotesize
                                    , frame=single
                                    }
  \newmint[prolog]{prolog}{ fontsize=\footnotesize
                          , frame=single
                          }

  \begin{enumerate}
    \item
      The program that follows generates Castilian Spanish syllables.

      It is important to generalize the generation to natural classes for many reasons.
      \begin{itemize}
        \item We are explicit in all possible syllables we can create.
        \item We generate well formed syllables.
        \item We do not generate incorrect syllables.
        \item It is easier to reason about which syllables we can generate when we talk abstractly about natural classes.
        \item We can make a simpler translation from the natural language rules to our program by thinking with natural classes.
        \item Our program is simpler by not enumerating edge cases.
        \item Our program is more maintainable and gives more reuse by allowing us to simply change which phones exist, and the properties they satisfy.
      \end{itemize}

      \prologfile{syllable.swipl}
  \end{enumerate}
\end{document}
