\documentclass[12pt,letterpaper]{article}
\usepackage[utf8]{inputenc}
\usepackage{amsmath}
\usepackage{amsfonts}
\usepackage{amsthm}
\usepackage{mathtools}
\usepackage{cancel}
\usepackage[margin=1in]{geometry}
\usepackage{titling}
\usepackage{minted}
\usepackage[tone]{tipa}
\usepackage{qtree}

\setlength{\droptitle}{-10ex}

\preauthor{\begin{flushright}\large \lineskip 0.5em}
\postauthor{\par\end{flushright}}
\predate{\begin{flushright}\large}
\postdate{\par\end{flushright}}

\title{LIN 177 Homework 4\vspace{-2ex}}
\author{Hardy Jones\\
        999397426\\
        Professor Ojeda\vspace{-2ex}}
\date{Winter 2015}

\begin{document}
  \maketitle

  \newmintedfile[prologfile]{prolog}{ fontsize=\footnotesize
                                    , frame=single
                                    }
  \newmint[prolog]{prolog}{ fontsize=\footnotesize
                          , frame=single
                          }

  \begin{enumerate}
    \item
      % I adore golfing.
      % Unfortunately, I have not been golfing in years.
      % My scores were abysmal.
      % I routinely shot in the hundreds, but I would practice religiously.
      % When I lived in Washington State, I used to go to the driving range at least three times a week.
      % Sun, rain, whatever, I would be at the range practicing my putting, driving or chipping.
      % I never took professional lessons, though I suspect it would have helped quite a bit.
      % I did take a course while at community college that seemed to improve my game though.
      % I would like to get back into golfing again.

      \begin{IPA}
        aI l2v gOlfIN.
        @nfOrtSun@tli, aI h\ae v nAt bIn gOlfIN In jIrz.
        maI skOrz w3r @bIzm@l.
        aI rutinli SAt In D@ h2ndr@dz, b2t aI wUd pr\ae kt@s rIlIdZ@sli.
        wEn aI lIvd In waSINt@n steIt, aI juzd tu goU tu D@ draIvIN reIndZ \ae t list Tri taImz @ wik.
        s2n, reIn, w@tEv@r, aI wUd bi \ae t D@ reIndZ pr\ae ktesIN maI p@tIN, draIvIN, Or \textteshlig IpIN.
        aI nEv@r tUk profES@n@l lEs@nz, DoU aI s2spEkt it wUd h\ae v hElpt kwaIt @ bIt.
        aI dId teIk @ kOrs waIl \ae t k@mjunIti kAlIdZ D\ae t simd tu Impruv maI geIm DoU.
        aI wUd laIk tU gEt b\ae k Intu gOlfIN @gEn.
      \end{IPA}

      % aɪ əˈdɔr ˈgɑlfɪŋ.
      % ənˈfɔrʧənətli, aɪ hæv nɑt bɪn ˈgɑlfɪŋ ɪn jɪrz.
      % maɪ skɔrz wɜr əˈbɪzməl.
      % aɪ ruˈtinli ʃɑt ɪn ðə ˈhʌndrədz, bʌt aɪ wʊd ˈpræktəs rɪˈlɪʤəsli.
      % wɛn aɪ laɪvd ɪn ˈwɑʃɪŋtən steɪt, aɪ juzd tu goʊ tu ðə ˈdraɪvɪŋ reɪnʤ æt list θri taɪmz ə wik.
      % sʌn, reɪn, ˌwʌˈtɛvər, aɪ wʊd bi æt ðə reɪnʤ ˈpræktəsɪŋ maɪ ˈpʌtɪŋ, ˈdraɪvɪŋ ɔr ˈʧɪpɪŋ.
      % aɪ ˈnɛvər tʊk prəˈfɛʃənəl ˈlɛsənz, ðoʊ aɪ ˈsʌˌspɛkt ɪt wʊd hæv hɛlpt kwaɪt ə bɪt.
      % aɪ dɪd teɪk ə kɔrs waɪl æt kəmˈjunəti ˈkɑlɪʤ ðæt simd tu ɪmˈpruv maɪ geɪm ðoʊ.
      % aɪ wʊd laɪk tu gɛt bæk ˈɪntu ˈgɑlfɪŋ əˈgɛn.
    \item
      \begin{enumerate}
        \item The phones are \begin{IPA}[p,b,t,d,k,g]\end{IPA}
        \item The phones are \begin{IPA}[\ae,a]\end{IPA}
        \item The phones are \begin{IPA}[\textturnr]\end{IPA}
        \item The phones are \begin{IPA}[j,w,i,I,e,\ae,u,U,o,a,@,2]\end{IPA}
      \end{enumerate}
    \item
      \begin{enumerate}
        \item
          One minimal property is
          \prolog @ phone(X), sib(X), not(voi(X)). @
        \item
          One minimal property is
          \prolog @ phone(X), str(X), not(bck(X)), not(ctr(X)). @
        \item
          One minimal property is
          \prolog @ phone(X), lab(X), not(cnt(X)). @
      \end{enumerate}
    \item
      The structure is:

      \Tree [.syllable [.onset [.s ] [.p ] ] [.rhyme [.nucleus [.o ] [.j ] ] [.coda [.l ] ] ] ]

    \item
      \begin{itemize}
        \item We can solve this with some combinatorics.

          We want to find how many sequences of 6 phones or less there are.
          We can break this down to finding the number sequences of 0, 1, 2, 3, 4, 5, and 6 phones and summing them.

          So we need to find the permutations with replacement,
          as it's possible to have more than one phone in a sequence.

          Since there are 34 possible phones we want to find the following sum:

          \begin{align*}
            \sum_{i = 0}^6 34^i &= 34^0 + 34^1 + 34^2 + 34^3 + 34^4 + 34^5 + 34^6 \\
            &= 1 + 34 + 1156 + 39304 + 1336336 + 45435424 + 1544804416 \\
            &= 1591616671 \\
          \end{align*}

          So, there are $1591616671$ possible sequences with 6 phones or less.

        \item
          Using the query:

          \prolog @ findall(X, (syllable(X), length(X, Len), Len =< 6), _Y), length(_Y, YLen). @

          The result is:

          \prolog @ YLen = 20608. @

          So there are $20608$ English syllables according to \texttt{syllable.swipl}.

        \item
          $20608$ is 0.00129\% of the $1591616671$ possible syllables.

        \item
          This percentage is important because it means the vast majority of sounds that can be made are not English syllables.
          It shows that English is a very small language in the scheme of things.
      \end{itemize}

    \pagebreak

    \item
      \prologfile{senufo.swipl}

      Some output from running this:

      \prologfile{senufo_output.swipl}

  \end{enumerate}
\end{document}
