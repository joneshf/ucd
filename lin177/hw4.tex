\documentclass[12pt,letterpaper]{article}
\usepackage[utf8]{inputenc}
\usepackage{amsmath}
\usepackage{amsfonts}
\usepackage{amsthm}
\usepackage{mathtools}
\usepackage{cancel}
\usepackage[margin=1in]{geometry}
\usepackage{titling}
\usepackage{minted}
\usepackage[tone]{tipa}
\usepackage{qtree}

\setlength{\droptitle}{-10ex}

\preauthor{\begin{flushright}\large \lineskip 0.5em}
\postauthor{\par\end{flushright}}
\predate{\begin{flushright}\large}
\postdate{\par\end{flushright}}

\title{LIN 177 Homework 4\vspace{-2ex}}
\author{Hardy Jones\\
        999397426\\
        Professor Ojeda\vspace{-2ex}}
\date{Winter 2015}

\begin{document}
  \maketitle

  \newmintedfile[prologfile]{prolog}{ fontsize=\footnotesize
                                    , frame=single
                                    }
  \newmint[prolog]{prolog}{ fontsize=\footnotesize
                          , frame=single
                          }

  \begin{enumerate}
    \item
    \item
      \begin{enumerate}
        \item The phones are \begin{IPA}[p,b,t,d,k,g]\end{IPA}
        \item The phones are \begin{IPA}[\ae,a]\end{IPA}
        \item The phones are \begin{IPA}[\textturnr]\end{IPA}
        \item The phones are \begin{IPA}[j,w,i,I,e,\ae,u,U,o,a,@,2]\end{IPA}
      \end{enumerate}
    \item
      \begin{enumerate}
        \item
          One minimal property is
          \prolog @ phone(X), sib(X), not(voi(X)). @
        \item
          One minimal property is
          \prolog @ phone(X), str(X), not(bck(X)), not(ctr(X)). @
        \item
          One minimal property is
          \prolog @ phone(X), lab(X), not(cnt(X)). @
      \end{enumerate}
    \item
      The structure is:

      \Tree [.syllable [.onset [.s ] [.p ] ] [.rhyme [.nucleus [.o ] [.j ] ] [.coda [.l ] ] ] ]

    \item
      \begin{itemize}
        \item We can solve this with some combinatorics.

          We want to find how many sequences of 6 phones or less there are.
          We can break this down to finding the number sequences of 0, 1, 2, 3, 4, 5, and 6 phones and summing them.

          So we need to find the permutations with replacement,
          as it's possible to have more than one phone in a sequence.

          Since there are 34 possible phones we want to find the following sum:

          \begin{align*}
            \sum_{i = 0}^6 34^i &= 34^0 + 34^1 + 34^2 + 34^3 + 34^4 + 34^5 + 34^6 \\
            &= 1 + 34 + 1156 + 39304 + 1336336 + 45435424 + 1544804416 \\
            &= 1591616671 \\
          \end{align*}

          So, there are $1591616671$ possible sequences with 6 phones or less.

        \item
          Using the query:

          \prolog @ findall(X, (syllable(X), length(X, Len), Len =< 6), _Y), length(_Y, YLen). @

          The result is:

          \prolog @ YLen = 20608. @

          So there are $20608$ English syllables according to \texttt{syllable.swipl}.

        \item
          $20608$ is 0.00129\% of the $1591616671$ possible syllables.

        \item
          This percentage is important because it means the vast majority of sounds that can be made are not English syllables.
          It shows that English is a very small language in the scheme of things.
      \end{itemize}

    \pagebreak

    \item
      \prologfile{senufo.swipl}

      Some output from running this:

      \prologfile{senufo_output.swipl}

  \end{enumerate}
\end{document}
