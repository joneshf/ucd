\documentclass[12pt,letterpaper]{article}
\usepackage{amsmath}
\usepackage{amsfonts}
\usepackage{amsthm}
\usepackage{mathtools}
\usepackage{cancel}
\usepackage[margin=1in]{geometry}
\usepackage{titling}
\usepackage{minted}

\setlength{\droptitle}{-10ex}

\preauthor{\begin{flushright}\large \lineskip 0.5em}
\postauthor{\par\end{flushright}}
\predate{\begin{flushright}\large}
\postdate{\par\end{flushright}}

\title{LIN 177 Homework 2\vspace{-2ex}}
\author{Hardy Jones\\
        999397426\\
        Professor Ojeda\vspace{-2ex}}
\date{Winter 2015}

\begin{document}
  \maketitle

  \newmintedfile[prolog]{prolog}{ fontsize=\footnotesize
                                , frame=single
                                }

  \begin{enumerate}
    \item
      Given the query

      \texttt{?- spanish([eva, mira, a, adan], SM, ES, EM).}

      Prolog attempts to match against the first rule with a variable as none of the facts match.
      It finds this rule:
      \prolog[firstline=13, lastline=14]{Programs/Programs/Unix/spanish.swipl}
      It attempts to unify \texttt{A = [eva, mira, a, adan]}.
      To unify it needs to prove the body.
      So prolog searches for

      \texttt{spanish([eva, mira, a, adan], [verb, intransitive], ES, [property])}

      There are no facts or rules for this, so it backtracks and tries the next rule.
      \prolog[firstline=16, lastline=18]{Programs/Programs/Unix/spanish.swipl}
      Again, it attempts to unify \texttt{A = [eva, mira, a, adan]}.
      To unify it needs to prove the body.
      So prolog searches for

      \texttt{spanish(C, [nounphrase], ES, [entity]), append([a], C, [eva, mira, a, adan])}

      There are no facts or rules for this, so it backtracks and tries the next rule.
      \prolog[firstline=20, lastline=24]{Programs/Programs/Unix/spanish.swipl}
      Again, it attempts to unify \texttt{A = [eva, mira, a, adan]}.
      To unify it needs to prove the body.
      The important searches are

      \texttt{spanish(C, [verb, transitive], D, [relation]), spanish(E, [nounphrase, accusative], F, [entity])}

      It is also constrained by

      \texttt{append(C, E, A)}

      There are no facts or rules that satisfy all constraints, so it backtracks and tries the next rule.
      \prolog[firstline=26, lastline=31]{Programs/Programs/Unix/spanish.swipl}
      Again, it attempts to unify \texttt{A = [eva, mira, a, adan]}.
      To unify it needs to prove the body.

      So prolog searches for

      \texttt{spanish(C, [nounphrase], D, [entity]), spanish(E, [verbphrase], F, [property])}

      These are constrained also by

      \texttt{append(C, E, A)}

      The fact
      \prolog[firstline=3, lastline=3]{Programs/Programs/Unix/spanish.swipl}
      matches the first part of the body,
      so prolog unifys \texttt{C = [eva], D = [eve]}.

      This means prolog is trying to solve the constraint

      \texttt{append([eva], E, [eva, mira, a, adan])}

      So it attempts to unify \texttt{E = [mira, a, adan]}.
      Now prolog just needs to find a fact or rule satisfying

      \texttt{spanish([mira, a, adan], [verbphrase], F, [property])}

      It finds this rule:
      \prolog[firstline=13, lastline=14]{Programs/Programs/Unix/spanish.swipl}
      It attempts to unify \texttt{A = [mira, a, adan]}.
      To unify it needs to prove the body.
      So prolog searches for

      \texttt{spanish([mira, a, adan], [verb, intransitive], F, [property])}

      There are no facts or rules for this, so it backtracks and tries the next rule that matches.
      \prolog[firstline=20, lastline=24]{Programs/Programs/Unix/spanish.swipl}
      Again, it attempts to unify \texttt{A = [mira, a, adan]}.
      To unify it needs to prove the body.
      The important searches are

      \texttt{spanish(C, [verb, transitive], D, [relation]), spanish(E, [nounphrase, accusative], F, [entity])}

      It is also constrained by

      \texttt{append(C, E, A)}

      The fact
      \prolog[firstline=9, lastline=9]{Programs/Programs/Unix/spanish.swipl}
      matches the first part of the body,
      so prolog unifies \texttt{C = [mira], D = [watches]}.

      Given the append constraint,
      prolog then needs to solve

      \texttt{append([mira], E, [mira, a, adan])}

      So it attempts to unify \texttt{E = [a, adan]}
      which means it needs to solve

      \texttt{spanish([a, adan], [nounphrase, accusative], F, [entity])}

      The only matching rule is
      \prolog[firstline=16, lastline=18]{Programs/Programs/Unix/spanish.swipl}
      So it attempts to unify \texttt{A = [a, adan]}.
      To unify it needs to prove the body.
      So prolog searches for

      \texttt{spanish(C, [nounphrase], F, [entity]), append([a], C, [a, adan])}

      Prolog can instantiate \texttt{C = [adan]} assuming it can solve

      \texttt{spanish([adan], [nounphrase], F, [entity])}

      This is a fact
      \prolog[firstline=1, lastline=1]{Programs/Programs/Unix/spanish.swipl}
      So prolog unifies \texttt{F = [adam]}

      And we can start filling in the blanks.

      When prolog makes to back to the toplevel, we have unified the query with the rule
      \prolog[firstline=26, lastline=30]{Programs/Programs/Unix/spanish.swipl}
      And the unifications are

      \prolog{hw2prob1.swipl}
    \item
      \prolog{trilingual.swipl}
    \item
      \begin{enumerate}
        \item This fact is ill-formed as the second argument is two atoms separated by a space.
        \item This fact is ill-formed as the functor name starts with an uppercase letter.
        \item This fact is ill-formed as the fact lacks a period.
        \item This fact is ill-formed as the functor name starts with a number.
        \item This fact is ill-formed as the functor name starts with a number.
      \end{enumerate}

    \pagebreak

    \item
      \prolog{english.swipl}
    \item
      \prolog{spanish-hw2.swipl}
  \end{enumerate}
\end{document}
