\documentclass[12pt,letterpaper]{article}
\usepackage[utf8]{inputenc}
\usepackage{amsmath}
\usepackage{amsfonts}
\usepackage{amsthm}
\usepackage{mathtools}
\usepackage{cancel}
\usepackage[margin=1in]{geometry}
\usepackage{titling}
\usepackage{minted}
\usepackage[tone]{tipa}

\setlength{\droptitle}{-10ex}

\preauthor{\begin{flushright}\large \lineskip 0.5em}
\postauthor{\par\end{flushright}}
\predate{\begin{flushright}\large}
\postdate{\par\end{flushright}}

\title{LIN 177 Homework 3\vspace{-2ex}}
\author{Hardy Jones\\
        999397426\\
        Professor Ojeda\vspace{-2ex}}
\date{Winter 2015}

\begin{document}
  \maketitle

  \newmintedfile[prolog]{prolog}{ fontsize=\footnotesize
                                , frame=single
                                }

  \begin{enumerate}
    \item
      % \prolog{hw2_phone.pl}
      \texttt{
        ?- findall(X, phone(X), Y). \\
        Y =
          \begin{IPA}
            [p,b,m,t,d,n,k,g,N,f,v,T,D,s,z,S,Z,\v{c},\v{j},l,\textturnr,j,w,h,i,I,e,\ae,u,U,o,a,@,2].
          \end{IPA}
      }

    \item

      \begin{enumerate}
        \setcounter {enumii}{1}
        \item
          % \texttt{[piro, quechua, german] = [piro, quechua, german | []]}

          Yes.

          The left side can be desugared a bit to read:

          \texttt{[piro, quechua, german | []]}
        \item
          % \texttt{[piro, quechua, german, []] = [piro, quechua, german|[]]}

          No.

          The left side is a list with 4 elements and the right side is a list with 3 elements.
        \item
          % \texttt{[Head, Head1, Head2] = [french, german, english]}

          Yes.

          It unifies the following variables:

          \texttt{
            Head = french, Head1 = german, Head2 = english
          }
        \item
          % \texttt{[Head, Head1] =[french, german, english]}

          No.

          The left side is a list with 2 elements and the right side is a list with 3 elements.
        \item
          % \texttt{[Head, Head1 | Tail] = [Head | Tail1]}

          Yes.

          Clearly the variable \texttt{Head} unifies with itself.
          Then the other unification takes place:

          \texttt{[Head1 | Tail] = Tail1}
        \item
          % \texttt{[] = [Head | Tail]}
          No.

          The left side is the empty list,
          while the right side has at least one element.
        \item
          % \texttt{[english] = [english | []]}
          Yes.

          These are the same lists, the left side has more sugar than the right side.
        \item
          % \texttt{[[english]] = [english | []]}
          No.

          These are different lists.
          The left side is a list with a single element that is the list containing \texttt{english}.
          The right side is a list with a single element \texttt{english}
        \item
          % \texttt{[[english, french], spanish] = [Head | Tail]}
          Yes. with the following instantiations:

          \texttt{
            [english, french] = Head, [spanish] = Tail
          }
        \item
          % \texttt{[[english, Item], piro] = [[Item1,quechua] | Tail]}

          Yes.

          It has the following instantiations:

          \texttt{
            Item = quechua, Item1 = english, [piro] = Tail
          }
        \item
          % \texttt{[french, [quechua, german]] = [Head | Tail]}

          Yes.

          It has the following instantiations

          \texttt{
            french = Head, [[quechua, german]] = Tail
          }
      \end{enumerate}

    \item

      For the given definition of ``the sublist relation'', this predicate holds.
      However, this predicate does not properly detect that one list is a sublist of the other in the normal sense of the word, as \texttt{sublist([1,1,1], [1])} should fail.
      However, that example passes the predicate.

      The \texttt{sublist} fact states that the empty list is a sublist of any list.

      This fact is true, as the empty list is a sublist of itself, and all other lists are made by cons-ing onto another list that is eventually the empty list.

      The \texttt{sublist} rule states that for any cons-ed list, if the first element of the list is an element of some list \texttt{List} and the rest of the list is a sublist of \texttt{List}, then the cons-ed list is a sublist of \texttt{List}.

      This rule is true as it ensures each element of the first list is a member in the second list. The recursive call within the rule is always working with a list containing exactly one less element (in particular the head element) so we are guaranteed to check each and every element of the first list.

    \item

      The \texttt{member} rule states that an \texttt{Element} is a member of a \texttt{List} if we can append some \texttt{List1} to the front of another list where \texttt{Element} is the head, and we get out the \texttt{List}.

      There are three cases we can consider (though the first and third are special cases of the second.)

      \begin{itemize}
        \item
          In the case where \texttt{Element} is the head of \texttt{List}, we unify \texttt{List1 = []} and we unify \texttt{Tail} with the tail of \texttt{List}

        \item
          When \texttt{Element} is in the middle of \texttt{List} at some index $i$, we unify \texttt{List1} with the list consisting of the first $i - 1$ elements of \texttt{List} and \texttt{Tail} with the last $n - i$ elements of \texttt{List}--where $n$ is the length of \texttt{List}.

        \item
          When \texttt{Element} is the last element of \texttt{List}, we unify \texttt{List1} with the list consisting of all elements of \texttt{List} except the last, and we unify \texttt{List1} with the empty list.
      \end{itemize}

      From these cases, we can see that the predicate \texttt{member} describes ``the member relation''.
  \end{enumerate}
\end{document}
