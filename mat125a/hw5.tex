\documentclass[12pt,letterpaper]{article}

\usepackage[margin=1in]{geometry}
\usepackage[round-mode=figures,round-precision=3,scientific-notation=false]{siunitx}
\usepackage[super]{nth}
\usepackage[title]{appendix}
\usepackage{amsfonts}
\usepackage{amsmath}
\usepackage{amsthm}
\usepackage{cancel}
\usepackage{color, colortbl}
\usepackage{dcolumn}
\usepackage{enumitem}
\usepackage{fp}
\usepackage{mathtools}
\usepackage{pgfplots}
\usepackage{titling}

\usepgfplotslibrary{statistics}

\pgfplotsset{compat=1.8}

\definecolor{Gray}{gray}{0.8}

\newcolumntype{g}{>{\columncolor{Gray}}c}
\newcolumntype{d}{D{.}{.}{-1}}

\DeclarePairedDelimiter\ceil{\lceil}{\rceil}
\DeclarePairedDelimiter\floor{\lfloor}{\rfloor}

\newcommand\epsdelta[5]{
  We want to prove:
  \[
    \lim_{x \to #1} #2 = #3
  \]

  \begin{proof}
    Given $\epsilon > 0$,
    we want to find $\delta > 0$ such that

    \[
      0 < \left|x - #1\right| < \delta \implies \left|#2 - #3\right| < \epsilon
    \].

    #5

    So, choose $\delta = #4$.

    Then we have
    \[
      0 < \left|x - #1\right| < \delta \implies \left|#2 - #3\right| < \epsilon
    \]
    as was to be shown.
  \end{proof}
}
\newcommand\epsdeltaconsequent[5]{
  \epsdelta{#1}{#2}{#3}{#4}{
    We can simplify the consequent a bit.

    #5

    If we notice, this is exactly the form of the antecedent,
    assuming $\delta = #4$.
  }
}
\newcommand\e{e}
\newcommand\uc{uniformly continuous }

\renewcommand{\labelenumi}{6.\arabic*}
\renewcommand{\labelenumii}{\arabic*}
\renewcommand{\labelenumiii}{(\alph*)}

\setlength{\droptitle}{-10ex}

\preauthor{\begin{flushright}\large \lineskip 0.5em}
\postauthor{\par\end{flushright}}
\predate{\begin{flushright}\large}
\postdate{\par\end{flushright}}

\title{MAT 125A HW 5\vspace{-2ex}}
\author{Hardy Jones\\
        999397426\\
        Professor Slivken\vspace{-2ex}}
\date{Spring 2015}

\begin{document}
  \maketitle

  \begin{enumerate}
    \setcounter{enumi}{1}
    \item
      \begin{enumerate}
        \item
          Given: $f_n(x) = \frac{nx}{1 + nx^2}$
          \begin{enumerate}
            \item
              \begin{align*}
                \lim_{n \to \infty}\frac{nx}{1 + nx^2}
                &= \lim_{n \to \infty}\frac{\frac{1}{x}\left(nx^2\right)}{1 + nx^2} \\
                &= \lim_{n \to \infty}\frac{\frac{1}{x}\left(1 - 1 + nx^2\right)}{1 + nx^2} \\
                &= \lim_{n \to \infty}\frac{\frac{1}{x}\left(1 + nx^2\right) - \frac{1}{x}}{1 + nx^2} \\
                &= \lim_{n \to \infty}\left[\frac{\frac{1}{x}\left(1 + nx^2\right)}{1 + nx^2} - \frac{\frac{1}{x}}{1 + nx^2}\right] \\
                &= \lim_{n \to \infty}\left[\frac{1}{x} - \frac{\frac{1}{x}}{1 + nx^2}\right] \\
                &= \lim_{n \to \infty}\frac{1}{x} - \lim_{n \to \infty}\frac{\frac{1}{x}}{1 + nx^2} \\
                &= \lim_{n \to \infty}\frac{1}{x} - 0 \\
                &= \lim_{n \to \infty}\frac{1}{x} \\
                &= \frac{1}{x} \\
              \end{align*}

              So the point-wise limit of $(f_n)$ is $\frac{1}{x}$
            \item
              On $\left(0, \infty\right)$ we check:

              \begin{align*}
                \left|f_n(x) - f(x)\right|
                &= \left|\frac{nx}{1 + nx^2} - \frac{1}{x}\right| \\
                &= \left|\frac{nx^2 - (1 + nx^2)}{x + nx^3}\right| \\
                &= \left|\frac{-1}{x + nx^3}\right| \\
                &= \frac{1}{x + nx^3} \\
              \end{align*}

              And we need to find some $N \in \mathbb{N}$ for any $\epsilon$,
              such that, if $n \geq N$, then $\frac{1}{x + nx^3} < \epsilon$

              We can rewrite a bit:

              \begin{align*}
                \frac{1}{x + nx^3} &< \epsilon \\
                \frac{1}{\epsilon} &< x + nx^3 \\
                \frac{1}{\epsilon} - x &< nx^3 \\
                \frac{1}{\epsilon x^3} - \frac{1}{x^2} &< n \\
                \frac{1}{\epsilon x^3} - \frac{1}{x^2} &< N \leq n \\
              \end{align*}

              But when $x \to 0$, $\frac{1}{\epsilon x^3} - \frac{1}{x^2} \to \infty$.

              So there's no $N$ we can choose for all $\epsilon$ on $(0, \infty)$.

              Thus, the convergence is not uniform on $(0, \infty)$.
            \item
              On $(0, 1)$ we run into the same issues as previous.

              So the convergence is not uniform on $(0, 1)$.
            \item
              On $(1, \infty)$ we check:

              The computation is the same up to $\frac{1}{\epsilon x^3} - \frac{1}{x^2} < N \leq n$.

              For all $x$ in $(1, \infty)$, $\frac{1}{\epsilon x^3} - \frac{1}{x^2} < \frac{1}{\epsilon} - 1$

              So we can choose $N > \frac{1}{\epsilon} - 1$.

              Then we have $\left|f_n(x) - f(x)\right| < \epsilon$.

              So the convergence is uniform on $(1, \infty)$.

          \end{enumerate}
        \item
          \begin{itemize}
            \item
              \begin{align*}
                \lim_{n \to \infty} \frac{nx + \sin(nx)}{2n}
                &= \lim_{n \to \infty} \left(\frac{nx}{2n} + \frac{\sin(nx)}{2n}\right) \\
                &= \lim_{n \to \infty} \left(\frac{x}{2} + \frac{\sin(nx)}{2n}\right) \\
                &= \lim_{n \to \infty} \frac{x}{2} + \lim_{n \to \infty} \frac{\sin(nx)}{2n} \\
                &= \frac{x}{2} + 0 \\
                &= \frac{x}{2} \\
              \end{align*}

              So the pointwise limit of $(g_n)$ is $\frac{x}{2}$

            \item
              On $[-10, 10]$ we check:

              \begin{align*}
                \left|g_n(x) - g(x)\right|
                &= \left|\frac{x}{2} - \frac{nx + \sin(nx)}{2n} \right| \\
                &= \left|\frac{nx - nx + \sin(nx)}{2n} \right| \\
                &= \left|\frac{\sin(nx)}{2n} \right| \\
              \end{align*}

              Since $-1 \leq \sin(nx) \leq 1$, we have:

              \begin{align*}
                \left|g_n(x) - g(x)\right|
                &= \left|\frac{\sin(nx)}{2n} \right| \\
                &\leq \left| \frac{1}{2n} \right| \\
              \end{align*}

              And we want $\frac{1}{2n} < \epsilon \implies \frac{1}{2\epsilon} < n$.

              So choose $N > \frac{1}{2\epsilon}$, then we have $\left|g_n(x) - g(x)\right| < \epsilon$.

              So the convergence is uniform on $[-10, 10]$.
            \item

              Since $x$ does not affect our decision of $N$,
              the same reasoning can be used to choose $N > \frac{1}{2\epsilon}$ on all of $\mathbb{R}$.

              So the convergence is uniform on $\mathbb{R}$.
          \end{itemize}
        \item
          \begin{enumerate}
            \item
              We have to consider three cases:
              \begin{itemize}
                \item $0 \leq x < 1$
                \item $x = 1$
                  \begin{align*}
                    \lim_{n \to \infty} \frac{1}{1 + 1^n} = \lim_{n \to \infty} \frac{1}{2} = \frac{1}{2}
                  \end{align*}
                \item $1 < x$
                  \begin{align*}
                    \lim_{n \to \infty} \frac{x}{1 + x^n} = 0
                  \end{align*}
              \end{itemize}

            \item
              Since there is a jump discontinuity on $h_n(x)$ when $x = 1$, and when $x > 1$,
              $h_n(x)$ is not continuous.
              And from Theorem 6.2.6, $h_n$ is not continuous.

              Since $h_n$ is not continuous, it cannot have convergence on $[0, \infty)$
            \item
              Choose $(1, \infty)$.

              Then we check:

              \begin{align*}
                \left| h_n(x) - h(x) \right|
                &= \left| \frac{x}{1 + x^n} - 0 \right| \\
                &= \left| \frac{x}{1 + x^n} \right| \\
                &= \frac{x}{1 + x^n} \\
                &< 1
              \end{align*}

              Then we can choose $N = 1$.

              And for any $\epsilon$
          \end{enumerate}
        \item
          We take some derivatives to find maxima and minima.

          \begin{align*}
            f_n'(x)
            &= \frac{\left(1 + nx^2\right) \cdot 1 - x (2nx)}{\left(1 + nx^2\right)^2}\\
            &= \frac{1 + nx^2 - 2nx^2}{\left(1 + nx^2\right)^2}\\
            &= \frac{1 - nx^2}{\left(1 + nx^2\right)^2}\\
          \end{align*}
          \begin{align*}
            f_n''(x)
            &= \frac{\left(1 + 2nx^2 + n^2x^4\right)\left(-2nx\right) - \left(4nx + 4n^2x^3\right)\left(1 - nx^2\right)}{\left(1 + nx^2\right)^4} \\
            &= \frac{-6nx - 4n^2x^3 + 2n^3x^5}{\left(1 + nx^2\right)^4} \\
          \end{align*}

          Setting the first derivative to zero, we can find possible maxima and minima.

          \begin{align*}
            \frac{1 - nx^2}{\left(1 + nx^2\right)^2} &= 0 \\
            1 - nx^2 &= 0 \\
            1 &= nx^2 \\
            \frac{1}{n} &= x^2 \\
            \pm\frac{1}{\sqrt{n}} &= x \\
          \end{align*}

          Now, we can plus into the second derivative.

          \begin{align*}
            f_n''\left(\frac{1}{\sqrt{n}}\right)
            &= \frac{-6n\left(\frac{1}{\sqrt{n}}\right) - 4n^2\left(\frac{1}{\sqrt{n}}\right)^3 + 2n^3\left(\frac{1}{\sqrt{n}}\right)^5}{\left(1 + n\left(\frac{1}{\sqrt{n}}\right)^2\right)^4} \\
            &= \frac{-6\sqrt{n} - 4n^{2 - \frac{3}{2}} + 2n^{3 - \frac{5}{2}}}{\left(1 + 1\right)^4} \\
            &= \frac{-6\sqrt{n} - 4\sqrt{n} + 2\sqrt{n}}{16} \\
            &= \frac{-8\sqrt{n}}{16} \\
            &= \frac{-\sqrt{n}}{2} \\
            &< 0 \\
          \end{align*}

          \begin{align*}
            f_n''\left(\frac{-1}{\sqrt{n}}\right)
            &= \frac{-6n\left(\frac{-1}{\sqrt{n}}\right) - 4n^2\left(\frac{-1}{\sqrt{n}}\right)^3 + 2n^3\left(\frac{-1}{\sqrt{n}}\right)^5}{\left(1 + n\left(\frac{-1}{\sqrt{n}}\right)^2\right)^4} \\
            &= \frac{6\sqrt{n} + 4n^{2 - \frac{3}{2}} - 2n^{3 - \frac{5}{2}}}{\left(1 + 1\right)^4} \\
            &= \frac{6\sqrt{n} + 4\sqrt{n} - 2\sqrt{n}}{16} \\
            &= \frac{8\sqrt{n}}{16} \\
            &= \frac{\sqrt{n}}{2} \\
            &> 0 \\
          \end{align*}
          So the maximum occurs at $\frac{-1}{\sqrt{n}}$ and the minimum occurs at $\frac{1}{\sqrt{n}}$
        \item
        \item
        \item
      \end{enumerate}
  \end{enumerate}
\end{document}
