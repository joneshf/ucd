\documentclass[12pt,letterpaper]{article}

\usepackage[margin=1in]{geometry}
\usepackage[round-mode=figures,round-precision=3,scientific-notation=false]{siunitx}
\usepackage[super]{nth}
\usepackage[title]{appendix}
\usepackage{amsfonts}
\usepackage{amsmath}
\usepackage{amsthm}
\usepackage{cancel}
\usepackage{color, colortbl}
\usepackage{dcolumn}
\usepackage{enumitem}
\usepackage{fp}
\usepackage{mathtools}
\usepackage{pgfplots}
\usepackage{titling}

\usepgfplotslibrary{statistics}

\pgfplotsset{compat=1.8}

\definecolor{Gray}{gray}{0.8}

\newcolumntype{g}{>{\columncolor{Gray}}c}
\newcolumntype{d}{D{.}{.}{-1}}

\DeclarePairedDelimiter\ceil{\lceil}{\rceil}
\DeclarePairedDelimiter\floor{\lfloor}{\rfloor}

\newcommand\epsdelta[5]{
  We want to prove:
  \[
    \lim_{x \to #1} #2 = #3
  \]

  \begin{proof}
    Given $\epsilon > 0$,
    we want to find $\delta > 0$ such that

    \[
      0 < \left|x - #1\right| < \delta \implies \left|#2 - #3\right| < \epsilon
    \].

    #5

    So, choose $\delta = #4$.

    Then we have
    \[
      0 < \left|x - #1\right| < \delta \implies \left|#2 - #3\right| < \epsilon
    \]
    as was to be shown.
  \end{proof}
}
\newcommand\epsdeltaconsequent[5]{
  \epsdelta{#1}{#2}{#3}{#4}{
    We can simplify the consequent a bit.

    #5

    If we notice, this is exactly the form of the antecedent,
    assuming $\delta = #4$.
  }
}
\newcommand\e{e}
\newcommand\uc{uniformly continuous }

\renewcommand{\labelenumi}{6.\arabic*}
\renewcommand{\labelenumii}{\arabic*}
\renewcommand{\labelenumiii}{(\alph*)}

\setlength{\droptitle}{-10ex}

\preauthor{\begin{flushright}\large \lineskip 0.5em}
\postauthor{\par\end{flushright}}
\predate{\begin{flushright}\large}
\postdate{\par\end{flushright}}

\title{MAT 125A HW 6\vspace{-2ex}}
\author{Hardy Jones\\
        999397426\\
        Professor Slivken\vspace{-2ex}}
\date{Spring 2015}

\begin{document}
  \maketitle

  \begin{enumerate}
    \setcounter{enumi}{1}
    \item
      \begin{enumerate}
        \setcounter{enumii}{10}
        \item
          \begin{enumerate}
            \item
              Let $h_n = f_n + g_n$
              \begin{proof}
                For any $\epsilon > 0$,
                since $f_n, g_n$ are uniformly convergent,
                we know that there exists some $M, N \in \mathbb{N}$ such that
                for all $m, n \in \mathbb{N}$, with $m \geq M, n \geq N$,
                we have $\left|f_m(x) - f(x)\right| < \frac{\epsilon}{2}, \left|g_n(x) - g(x)\right| < \frac{\epsilon}{2}$

                Choose $P = max(M, N)$.

                Then we have for all $p \geq P \in \mathbb{N}$.

                \begin{align*}
                  \left|h_p(x) - h(x)\right|
                  &= \left|\left(f_p + g_p\right)(x) - \left(f + g\right)(x)\right| \\
                  &= \left|f_p(x) + g_p(x) - f(x) - g(x)\right| \\
                  &= \left|f_p(x) - f(x) + g_p(x) - g(x)\right| \\
                  &\leq \left|f_p(x) - f(x)\right| + \left|g_p(x) - g(x)\right| \\
                  &< \frac{\epsilon}{2} + \frac{\epsilon}{2} \\
                  &= \epsilon \\
                \end{align*}

                So $h_n$ is uniformly convergent.
              \end{proof}
            \item
              \begin{proof}
                Let $f_n(x) = x, g_n(x) = \frac{1}{n}$.

                So $\lim\limits_{n \to \infty}f_n(x) = \lim\limits_{n \to \infty}x = x$,
                and $\lim\limits_{n \to \infty}g_n(x) = \lim\limits_{n \to \infty}\frac{1}{n} = 0$.

                Then, for any $\epsilon > 0$, choose $M = 1$, then for all $m > 1 \in \mathbb{N}$
                \[\left|f_m(x) - f(x)\right| = \left|x - x\right| = 0 < \epsilon\]

                Also, for any $\epsilon > 0$, choose $N = 1$, then for all $n > 1 \in \mathbb{N}$
                \[\left|g_n(x) - f(x)\right| = \left|x - x\right| = 0 < \epsilon\]
              \end{proof}
            \item
          \end{enumerate}
        \setcounter{enumii}{14}
        \item
          \begin{enumerate}
            \item
            \item
          \end{enumerate}
        \item
          \begin{enumerate}
            \item
            \item
            \item
          \end{enumerate}
      \end{enumerate}
    \item
      \begin{enumerate}
        \setcounter{enumii}{1}
        \item
          \begin{enumerate}
            \item
            \item
          \end{enumerate}
        \setcounter{enumii}{4}
        \item
      \end{enumerate}
    \item
      \begin{enumerate}
        \item
          \begin{enumerate}
            \item
            \item
          \end{enumerate}
        \item
          \begin{enumerate}
            \item
            \item
            \item
            \item
          \end{enumerate}
        \setcounter{enumii}{4}
        \item
        \item
        \item
          \begin{enumerate}
            \item
            \item
          \end{enumerate}
      \end{enumerate}
  \end{enumerate}
\end{document}
