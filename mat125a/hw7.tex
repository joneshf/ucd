\documentclass[12pt,letterpaper]{article}

\usepackage[margin=1in]{geometry}
\usepackage[round-mode=figures,round-precision=3,scientific-notation=false]{siunitx}
\usepackage[super]{nth}
\usepackage[title]{appendix}
\usepackage{amsfonts}
\usepackage{amsmath}
\usepackage{amsthm}
\usepackage{cancel}
\usepackage{color, colortbl}
\usepackage{commath}
\usepackage{dcolumn}
\usepackage{enumitem}
\usepackage{fp}
\usepackage{mathtools}
\usepackage{pgfplots}
\usepackage{titling}

\usepgfplotslibrary{statistics}

\pgfplotsset{compat=1.8}

\definecolor{Gray}{gray}{0.8}

\newcolumntype{g}{>{\columncolor{Gray}}c}
\newcolumntype{d}{D{.}{.}{-1}}

\DeclarePairedDelimiter\ceil{\lceil}{\rceil}
\DeclarePairedDelimiter\floor{\lfloor}{\rfloor}

\newcommand\epsdelta[5]{
  We want to prove:
  \[
    \lim_{x \to #1} #2 = #3
  \]

  \begin{proof}
    Given $\epsilon > 0$,
    we want to find $\delta > 0$ such that

    \[
      0 < \left|x - #1\right| < \delta \implies \left|#2 - #3\right| < \epsilon
    \].

    #5

    So, choose $\delta = #4$.

    Then we have
    \[
      0 < \left|x - #1\right| < \delta \implies \left|#2 - #3\right| < \epsilon
    \]
    as was to be shown.
  \end{proof}
}
\newcommand\epsdeltaconsequent[5]{
  \epsdelta{#1}{#2}{#3}{#4}{
    We can simplify the consequent a bit.

    #5

    If we notice, this is exactly the form of the antecedent,
    assuming $\delta = #4$.
  }
}
\newcommand\e{e}
\newcommand\uc{uniformly continuous }

\renewcommand{\labelenumi}{6.\arabic*}
\renewcommand{\labelenumii}{\arabic*}
\renewcommand{\labelenumiii}{(\alph*)}

\setlength{\droptitle}{-10ex}

\preauthor{\begin{flushright}\large \lineskip 0.5em}
\postauthor{\par\end{flushright}}
\predate{\begin{flushright}\large}
\postdate{\par\end{flushright}}

\title{MAT 125A HW 7\vspace{-2ex}}
\author{Hardy Jones\\
        999397426\\
        Professor Slivken\vspace{-2ex}}
\date{Spring 2015}

\begin{document}
  \maketitle

  \begin{enumerate}
    \setcounter{enumi}{4}
    \item
      \begin{enumerate}
        \setcounter{enumii}{1}
        \item
          \begin{enumerate}
            \item
            \item
            \item
            \item
              No, this is not possible.
              If the series converged absolutely at $x = 1$,
              then it the series would converge for all $x_0$ where $|x_0| \leq 1$.
              This includes $x_0 = -1$.

              So the series would have to converge absolutely at $x_0 = -1$ as well.
          \end{enumerate}
        \item
          From Theorem 6.5.1, we get the set of points a power series converges to must be one of $\{0\}, \mathbb{R}$, or a bounded interval about 0 (really the first two are specific cases of the last one).
          \begin{itemize}
            \item
              If the set of convergence is $\{0\}$, then the series converges only at one point, which is less than 2 points.
            \item
              If the set of convergence is $\mathbb{R}$, then the series converges absolutely at every point.
            \item
              If the set of convergence is some interval $(-x, x), (-x, x], [-x, x)$, or $[-x, x]$, then the series converges absolutely at every point $x_0$ where $|x_0| < |x|$, and only at the end points is it possible to converge conditionally.
          \end{itemize}

          From these three cases, we see that at most two points can have conditional convergence.
        \item
          \begin{enumerate}
            \item
            \item
          \end{enumerate}
        \setcounter{enumii}{8}
        \item
          \begin{proof}
            First let's look at some derivatives.

            Since we know

            \[
              f(x) = \sum_{n = 0}^{\infty} a_n x^n = \sum_{n = 0}^{\infty} b_n x^n = g(x)
            \]

            We can find the first few derivatives.

            \begin{enumerate}
              \item
                \nth{0} derivative:
                \[
                  f(x) = \sum_{n = 0}^{\infty} a_n x^n = \sum_{n = 0}^{\infty} b_n x^n = g(x)
                \]
              \item
                \nth{1} derivative:
                \[
                  f'(x) = \sum_{n = 1}^{\infty} n a_n x^{n - 1} = \sum_{n = 1}^{\infty} n b_n x^{n - 1} = g'(x)
                \]
              \item
                \nth{2} derivative:
                \[
                  f''(x) = \sum_{n = 2}^{\infty} n(n - 1) a_n x^{n - 2} = \sum_{n = 2}^{\infty} n(n - 1) b_n x^{n - 2} = g''(x)
                \]
              \item
                \nth{3} derivative:
                \[
                  f'''(x) = \sum_{n = 3}^{\infty} n(n - 1)(n - 2) a_n x^{n - 3} = \sum_{n = 3}^{\infty} n(n - 1)(n - 2) b_n x^{n - 3} = g'''(x)
                \]
            \end{enumerate}

            Looks like there's a pattern.

            The $k$-th derivative of $f, g$ is

            \[
              f^{(k)}(x) = \sum_{n = k}^{\infty} \left(\prod_{m = 0}^{k - 1} n - m\right) a_n x^{n - k} = \sum_{n = k}^{\infty} \left(\prod_{m = 0}^{k - 1} n - m\right) b_n x^{n - k} = g^{(k)}(x)
            \]

            We can prove this by induction:

            \begin{itemize}
              \item Base Case: $k = 0$

                \begin{align*}
                  f^{(0)}(x) = f(x) &= \sum_{n = 0}^{\infty} a_n x_n \\
                  &= \sum_{n = 0}^{\infty} (1) a_n x_n \\
                  &= \sum_{n = 0}^{\infty} \left(\prod_{m = 0}^{0 - 1}n - m\right) a_n x_n \\
                  &= \sum_{n = 0}^{\infty} \left(\prod_{m = 0}^{0 - 1}n - m\right) b_n x_n \\
                  &= \sum_{n = 0}^{\infty} (1) b_n x_n \\
                  g^{(0)}(x) = g(x) &= \sum_{n = 0}^{\infty} b_n x_n \\
                \end{align*}
              \item Inductive Case:

                Assume
                \[
                  f^{(k)}(x) = \sum_{n = k}^{\infty} \left(\prod_{m = 0}^{k - 1} n - m\right) a_n x^{n - k} = \sum_{n = k}^{\infty} \left(\prod_{m = 0}^{k - 1} n - m\right) b_n x^{n - k} = g^{(k)}(x)
                \]

                \begin{align*}
                  f^{(k + 1)}(x) &= \od{}{x}\left(f^{(k)}(x)\right) \\
                  &= \od{}{x}\left(\sum_{n = k}^{\infty} \left(\prod_{m = 0}^{k - 1} n - m\right) a_n x^{n - k}\right) \\
                  &= \sum_{n = k + 1}^{\infty} (n - k)\left(\prod_{m = 0}^{k - 1} n - m\right) a_n x^{n - k - 1} \\
                  &= \sum_{n = k + 1}^{\infty} \left(\prod_{m = 0}^{k} n - m\right) a_n x^{n - (k + 1)} \\
                  &= \sum_{n = k + 1}^{\infty} \left(\prod_{m = 0}^{k} n - m\right) b_n x^{n - (k + 1)} \\
                  &= \sum_{n = k + 1}^{\infty} (n - k)\left(\prod_{m = 0}^{k - 1} n - m\right) b_n x^{n - k - 1} \\
                  &= \od{}{x}\left(\sum_{n = k}^{\infty} \left(\prod_{m = 0}^{k - 1} n - m\right) b_n x^{n - k}\right) \\
                  g^{(k + 1)}(x) &= \od{}{x}\left(g^{(k)}(x)\right) \\
                \end{align*}
            \end{itemize}

            By induction we have proved our conjecture.

            Now we can move on to the actual proof.

            Since we know

            \[
              f(x) = \sum_{n = 0}^{\infty} a_n x^n = \sum_{n = 0}^{\infty} b_n x^n = g(x)
            \]

            for all $x \in (-R, R)$, choose $x = 0$.

            Then we have:

            \begin{align*}
              f(0) &= \sum_{n = 0}^{\infty} a_n 0^n \\
              &= a_0 0^0 + a_1 0^1 + a_2 0^2 + \dots \\
              &= a_0 \\
              &= b_0 \\
              &= b_0 0^0 + b_1 0^1 + b_2 0^2 + \dots \\
              g(0) &= \sum_{n = 0}^{\infty} b_n 0^n \\
            \end{align*}

            So $a_0 = b_0$.

            Now, assuming $a_k = b_k$:

            \begin{align*}
              f^{(k + 1)}(0) &= \sum_{n = k + 1}^{\infty} \left(\prod_{m = 0}^{k} n - m\right) a_n 0^{n - (k + 1)} \\
              &= \left(\prod_{m = 0}^{k} (k + 1) - m \right) a_{k + 1} 0^0 + \left(\prod_{m = 0}^{k} (k + 2) - m \right) a_{k + 2} 0^1 + \dots \\
              &= \left(\prod_{m = 0}^{k} (k + 1) - m \right) a_{k + 1} \\
              &= \left(\prod_{m = 0}^{k} (k + 1) - m \right) b_{k + 1} \\
              &= \left(\prod_{m = 0}^{k} (k + 1) - m \right) b_{k + 1} 0^0 + \left(\prod_{m = 0}^{k} (k + 2) - m \right) b_{k + 2} 0^1 + \dots \\
              g^{(k + 1)}(0) &= \sum_{n = k + 1}^{\infty} \left(\prod_{m = 0}^{k} n - m\right) b_n 0^{n - (k + 1)} \\
            \end{align*}

            So we have shown by induction that $a_n = b_n, \forall n \in \{0, 1, 2, \dots\}$.
          \end{proof}
      \end{enumerate}
    \item
      \begin{enumerate}
        \item
          At the point $x = 1$, the series is:
          \[
            1 - \frac{1}{3} + \frac{1}{5} - \frac{1}{7} + \dots
          \]
          and this series converges by the alternating series test.

          By Abel's Theorem, the series converges uniformly on $[0, 1]$.

          Since we assume $\arctan(x)$ is continuous on $[0, 1]$,
          we must necessarily have $\arctan(1) = 1 - \frac{1}{3} + \frac{1}{5} - \frac{1}{7} + \dots$.

          After trying and failing miserably to find a nice identity, I plugged $\arctan(1)$ into a calculator and found $\frac{\pi}{4}$.
        \item
          Following the example (in reverse), we want
          \begin{align*}
            \od{\ln(1 + x)}{x} &= \frac{1}{1 + x}\\
            \ln(1 + x) &= \int_{0}^{x} \frac{1}{1 + t} \dif t\\
          \end{align*}

          So we need to substitute $-t$ for $t$:

          \[
            \frac{1}{1 + t} = 1 - t + t^2 - t^3 + t^4 - \dots
          \]

          So if we integrate this, we get

          \[
            \ln(1 + x) = x - \frac{x^2}{2} + \frac{x^3}{3} - \frac{x^4}{4} + \frac{x^5}{5} - \dots
          \]

          So this expression if valid for all $x \in \intoo{-1, \infty}$.
          Though it only converges for $x \in \intoc{-1, 1}$.
        \setcounter{enumii}{4}
        \item
          \begin{proof}
            To prove that $S_N(x)$ converges uniformly to $\sin(x)$ on $\intcc{-2, 2}$,
            we need to show:

            $\forall \epsilon > 0, \exists M \in \mathbb{N}$,
            such that $\forall m \geq M, x \in \intcc{-2, 2}, \envert{S_m(x) - \sin(x)} < \epsilon$.

            It suffices to show that $E_N(x) \to 0$ as $N \to \infty$,
            so what we really want is:

            $\forall \epsilon > 0, \exists M \in \mathbb{N}$,
            such that $\forall m \geq M, x \in \intcc{-2, 2}, \envert{E_m(x)} < \epsilon$.

            From Lagrange's Remainder Theorem, and recalling that derivatives of $\sin, \cos$ cycle between each other, we have:
            \begin{align*}
              \envert{E_N(x)} &= \envert{\frac{\sin^{(N + 1)}(c)}{(N + 1)!}x^{N + 1}} \\
              &\leq \envert{\frac{1}{(N + 1)!}x^{N + 1}} && \text{since } \envert{\sin(x)}, \envert{\cos(x)} \leq 1, \forall x \\
              &= \frac{1}{(N + 1)!}\envert{x^{N + 1}} \\
              &\leq \frac{1}{(N + 1)!}\envert{2^{N + 1}} && \text{since } x \in \intcc{-2, 2} \\
              &\leq \frac{1}{(N + 1)!}2^{N + 1} && \text{since } x \in \intcc{-2, 2} \\
            \end{align*}

            So for any $\epsilon > 0$, choose $M \in \mathbb{N}$ such that $\frac{1}{M + 1}2^{M + 1} < \epsilon$.

            Then we have, for any $m \geq M$ and for all $x \in \intcc{-2, 2}$,

            $\envert{E_m(x)} \leq \frac{1}{M + 1}2^{M + 1} < \epsilon$.

            So $S_N(x)$ converges uniformly to $\sin(x)$ on $\intcc{-2, 2}$.
          \end{proof}

          We can generalize this proof to any interval $\intcc{-R, R}$ by substituting $R$ for $2$.

          \begin{proof}

            For any $\epsilon > 0$,

            choose $M \in \mathbb{N}$ such that $\frac{1}{M + 1}R^{M + 1} < \epsilon$.

            Then we have, for any $m \geq M$ and for all $x \in \intcc{-R, R}$,

            $\envert{E_m(x)} \leq \frac{1}{M + 1}R^{M + 1} < \epsilon$.

            So $S_N(x)$ converges uniformly to $\sin(x)$ on $\intcc{-R, R}$.
          \end{proof}
        \setcounter{enumii}{9}
        \item
          Let's rewrite this first.

          \[g(x) = e^{-x^{-2}}\]

          Now we have:

          \[g'(x) = e^{-x^{-2}}\left(2x^{-3}\right)\]

          \begin{align*}
            g''(x) &= e^{-x^{-2}}\left(2x^{-3}\right)\left(2x^{-3}\right) + e^{-x^{-2}}\left(-6x^{-4}\right) \\
            &= e^{-x^{-2}}\left(4x^{-6}-6x^{-4}\right) \\
          \end{align*}

          \begin{align*}
            g'''(x) &= e^{-x^{-2}}\left(2x^{-3}\right)\left(4x^{-6}-6x^{-4}\right) + e^{-x^{-2}}\left(-24x^{-7}+24x^{-5}\right) \\
            &= e^{-x^{-2}}\left(8x^{-9}-36x^{-7}+24x^{-5}\right) \\
          \end{align*}
        \item
          \begin{align*}
            g''(0) &= \lim_{x \to 0} \frac{g'(x)}{x} \\
            &= \lim_{x \to 0} \frac{e^{-x^{-2}}\left(2x^{-3}\right)}{x} \\
            &= \lim_{x \to 0} e^{-x^{-2}}\left(2x^{-4}\right) \\
          \end{align*}
      \end{enumerate}
  \end{enumerate}
\end{document}
