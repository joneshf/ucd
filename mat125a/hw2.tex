\documentclass[12pt,letterpaper]{article}

\usepackage[margin=1in]{geometry}
\usepackage[round-mode=figures,round-precision=3,scientific-notation=false]{siunitx}
\usepackage[super]{nth}
\usepackage[title]{appendix}
\usepackage{amsfonts}
\usepackage{amsmath}
\usepackage{amsthm}
\usepackage{cancel}
\usepackage{color, colortbl}
\usepackage{dcolumn}
\usepackage{enumitem}
\usepackage{fp}
\usepackage{mathtools}
\usepackage{pgfplots}
\usepackage{titling}

\usepgfplotslibrary{statistics}

\pgfplotsset{compat=1.8}

\definecolor{Gray}{gray}{0.8}

\newcolumntype{g}{>{\columncolor{Gray}}c}
\newcolumntype{d}{D{.}{.}{-1}}

\DeclarePairedDelimiter\ceil{\lceil}{\rceil}
\DeclarePairedDelimiter\floor{\lfloor}{\rfloor}

\newcommand\epsdelta[5]{
  We want to prove:
  \[
    \lim_{x \to #1} #2 = #3
  \]

  \begin{proof}
    Given $\epsilon > 0$,
    we want to find $\delta > 0$ such that

    \[
      0 < \left|x - #1\right| < \delta \implies \left|#2 - #3\right| < \epsilon
    \].

    #5

    So, choose $\delta = #4$.

    Then we have
    \[
      0 < \left|x - #1\right| < \delta \implies \left|#2 - #3\right| < \epsilon
    \]
    as was to be shown.
  \end{proof}
}
\newcommand\epsdeltaconsequent[5]{
  \epsdelta{#1}{#2}{#3}{#4}{
    We can simplify the consequent a bit.

    #5

    If we notice, this is exactly the form of the antecedent,
    assuming $\delta = #4$.
  }
}
\newcommand\e{e}

\setlength{\droptitle}{-10ex}

\preauthor{\begin{flushright}\large \lineskip 0.5em}
\postauthor{\par\end{flushright}}
\predate{\begin{flushright}\large}
\postdate{\par\end{flushright}}

\title{MAT 125A HW 2\vspace{-2ex}}
\author{Hardy Jones\\
        999397426\\
        Professor Slivken\vspace{-2ex}}
\date{Spring 2015}

\begin{document}
  \maketitle

  \begin{enumerate}[label=Exercise 4.3.\arabic*]
    \setcounter{enumi}{5}
    \item
      \begin{enumerate}
        \item
          \begin{proof}
            We have Dirichlet's function
            \[
              f(x) =
              \begin{cases}
                1 & \text{if } x \in \mathbb{Q} \\
                0 & \text{if } x \notin \mathbb{Q} \\
              \end{cases}
            \]

            This is a function from $\mathbb{R} \to \mathbb{R}$.

            Choose 0 as a limit point in $\mathbb{R}$.

            Now we construct a sequence from $x_n = \frac{\e}{n}$.
            So $(x_n) \to 0$, by The Algebraic Limit Theorem.

            Now, $f(x_n) \to 0$, but $f(0) = 1$.

            So, by Corollary 4.3.3, Dirichlet's function is not continuous at 0.

            We can extend this to any $c \in \mathbb{Q}$
            by constructing a new sequence $y_n = x_n + c$.

            Following similar arguments,
            it can be shown that Dirichlet's function is not continuous at any point in $\mathbb{Q}$.

            A similar argument holds for showing that Dirichlet's function is not continuous on $\mathbb{I}$.
            We choose some sequence of rationals such that for any limit point $c$,
            $(z_n) \to c$.
            Then we have that $f(z_n) \to 1$, but $f(c) = 0$.
            So Dirichlet's function is not continuous on $\mathbb{I}$ either.

            Thus, Dirichlet's function is not continuous on $\mathbb{R}$.
          \end{proof}
        \item
          \begin{proof}
            We have
            \[
              f(x) =
              \begin{cases}
                1           & \text{if } x = 0 \\
                \frac{1}{n} & \text{if } x = \frac{m}{n} \in \mathbb{Q} \setminus \{0\} \text{ is in lowest terms with } n > 0 \\
                0           & \text{if } x \notin \mathbb{Q} \\
              \end{cases}
            \].

            Choose some rational number $c$ in lowest terms $\frac{m}{n}$.

            Now construct any sequence $(x_n)$ from $\mathbb{I}$ such that,
            $(x_n) \to c$.

            So, $f(x_n) \to 0$, but $f(c) = \frac{1}{n}$.

            Now, since $f : \mathbb{R} \to \mathbb{R}$,
            $c$ is a limit point in $\mathbb{R}$,
            $(x_n) \to c$,
            but $f(x_n) \neq f(c)$,
            we conclude that Thomae's function is not continuous at any rational point.
          \end{proof}
        \item
          \begin{proof}
            We have
            \[
              f(x) =
              \begin{cases}
                1           & \text{if } x = 0 \\
                \frac{1}{n} & \text{if } x = \frac{m}{n} \in \mathbb{Q} \setminus \{0\} \text{ is in lowest terms with } n > 0 \\
                0           & \text{if } x \notin \mathbb{Q} \\
              \end{cases}
            \].

            Choose some $c \in \mathbb{I}$.
            Then we know that $f(c) = 0$, since $c \notin \mathbb{Q}$.

            Now given, some $\epsilon > 0$, choose $\delta = \epsilon$.

            So for $x \in \mathbb{I}$ where $|x - c| < \delta$,
            we have $|f(x) - f(c)| = |0 - 0| = 0 < \epsilon$.

            Thus, by Theorem 4.3.2, $f$ is continuous on $\mathbb{I}$.
          \end{proof}
      \end{enumerate}

  \end{enumerate}

\end{document}
