\documentclass[12pt,letterpaper]{article}

\usepackage[margin=1in]{geometry}
\usepackage[round-mode=figures,round-precision=3,scientific-notation=false]{siunitx}
\usepackage[super]{nth}
\usepackage[title]{appendix}
\usepackage{amsfonts}
\usepackage{amsmath}
\usepackage{amsthm}
\usepackage{cancel}
\usepackage{color, colortbl}
\usepackage{dcolumn}
\usepackage{enumitem}
\usepackage{fp}
\usepackage{mathtools}
\usepackage{pgfplots}
\usepackage{titling}

\usepgfplotslibrary{statistics}

\pgfplotsset{compat=1.8}

\definecolor{Gray}{gray}{0.8}

\newcolumntype{g}{>{\columncolor{Gray}}c}
\newcolumntype{d}{D{.}{.}{-1}}

\DeclarePairedDelimiter\ceil{\lceil}{\rceil}
\DeclarePairedDelimiter\floor{\lfloor}{\rfloor}

\newcommand\epsdelta[5]{
  We want to prove:
  \[
    \lim_{x \to #1} #2 = #3
  \]

  \begin{proof}
    Given $\epsilon > 0$,
    we want to find $\delta > 0$ such that

    \[
      0 < \left|x - #1\right| < \delta \implies \left|#2 - #3\right| < \epsilon
    \].

    #5

    So, choose $\delta = #4$.

    Then we have
    \[
      0 < \left|x - #1\right| < \delta \implies \left|#2 - #3\right| < \epsilon
    \]
    as was to be shown.
  \end{proof}
}
\newcommand\epsdeltaconsequent[5]{
  \epsdelta{#1}{#2}{#3}{#4}{
    We can simplify the consequent a bit.

    #5

    If we notice, this is exactly the form of the antecedent,
    assuming $\delta = #4$.
  }
}
\newcommand\e{e}

\setlength{\droptitle}{-10ex}

\preauthor{\begin{flushright}\large \lineskip 0.5em}
\postauthor{\par\end{flushright}}
\predate{\begin{flushright}\large}
\postdate{\par\end{flushright}}

\title{MAT 125A HW 2\vspace{-2ex}}
\author{Hardy Jones\\
        999397426\\
        Professor Slivken\vspace{-2ex}}
\date{Spring 2015}

\begin{document}
  \maketitle

  \begin{enumerate}[label=Exercise 4.3.\arabic*]
    \setcounter{enumi}{5}
    \item
      \begin{enumerate}
        \item
          \begin{proof}
            This proof was partially inspired by John Hunter's lecture notes.

            We have Dirichlet's function
            \[
              f(x) =
              \begin{cases}
                1 & \text{if } x \in \mathbb{Q} \\
                0 & \text{if } x \notin \mathbb{Q} \\
              \end{cases}
            \]

            This is a function from $\mathbb{R} \to \mathbb{R}$.

            Choose 0 as a limit point in $\mathbb{R}$.

            Now we construct a sequence from $x_n = \frac{\e}{n}$.
            So $(x_n) \to 0$, by The Algebraic Limit Theorem.

            Now, $f(x_n) \to 0$, but $f(0) = 1$.

            So, by Corollary 4.3.3, Dirichlet's function is not continuous at 0.

            We can extend this to any $c \in \mathbb{Q}$
            by constructing a new sequence $y_n = x_n + c$.

            Following similar arguments,
            it can be shown that Dirichlet's function is not continuous at any point in $\mathbb{Q}$.

            A similar argument holds for showing that Dirichlet's function is not continuous on $\mathbb{I}$.
            We choose some sequence of rationals such that for any limit point $c$,
            $(z_n) \to c$.
            Then we have that $f(z_n) \to 1$, but $f(c) = 0$.
            So Dirichlet's function is not continuous on $\mathbb{I}$ either.

            Thus, Dirichlet's function is not continuous on $\mathbb{R}$.
          \end{proof}
        \item
          \begin{proof}
            We have
            \[
              f(x) =
              \begin{cases}
                1           & \text{if } x = 0 \\
                \frac{1}{n} & \text{if } x = \frac{m}{n} \in \mathbb{Q} \setminus \{0\} \text{ is in lowest terms with } n > 0 \\
                0           & \text{if } x \notin \mathbb{Q} \\
              \end{cases}
            \].

            Choose some rational number $c$ in lowest terms $\frac{m}{n}$.

            Now construct any sequence $(x_n)$ from $\mathbb{I}$ such that,
            $(x_n) \to c$.

            So, $f(x_n) \to 0$, but $f(c) = \frac{1}{n}$.

            Now, since $f : \mathbb{R} \to \mathbb{R}$,
            $c$ is a limit point in $\mathbb{R}$,
            $(x_n) \to c$,
            but $f(x_n) \neq f(c)$,
            we conclude that Thomae's function is not continuous at any rational point.
          \end{proof}
        \item
          \begin{proof}
            We have
            \[
              f(x) =
              \begin{cases}
                1           & \text{if } x = 0 \\
                \frac{1}{n} & \text{if } x = \frac{m}{n} \in \mathbb{Q} \setminus \{0\} \text{ is in lowest terms with } n > 0 \\
                0           & \text{if } x \notin \mathbb{Q} \\
              \end{cases}
            \].

            Choose some $c \in \mathbb{I}$.
            Then we know that $f(c) = 0$, since $c \notin \mathbb{Q}$.

            Now given, some $\epsilon > 0$, choose $\delta = \epsilon$.

            So for $x \in \mathbb{I}$ where $|x - c| < \delta$,
            we have $|f(x) - f(c)| = |0 - 0| = 0 < \epsilon$.

            Thus, by Theorem 4.3.2, $f$ is continuous on $\mathbb{I}$.
          \end{proof}
      \end{enumerate}

    \item
      \begin{proof}

        Let $c$ be a limit point in $K$.
        Then there exists some sequence $(x_n)$ in $K$ such that
        $\lim\limits_{x \to c}x_n = c$

        Since $h$ is continuous on $\mathbb{R}$,
        $\lim\limits_{x \to c} h(x_n) = h(c)$.

        But since all $x_n$ are in $K$, all $h(x_n) = 0$.

        So $\lim\limits_{x \to c} h(x_n) = h(c) = 0$.

        Since $h(c) = 0, c \in K$, so $K$ contains its limit points.

        Thus $K$ is a closed set.
      \end{proof}

    \item
      \begin{enumerate}
        \item
          \begin{proof}
            Let the continuous function be $f$.

            Let $c$ be an arbitrary point in $\mathbb{I}$.

            Then there exists some sequence $(x_n)$ in $\mathbb{Q}$ such that
            $\lim\limits_{x \to c} x_n = c$.
            Since $f$ is continuous, we know that $\lim\limits_{x \to c} f(x_n) = f(c)$.
            And since all $x_n \in \mathbb{Q}$, all $f(x_n) = 0$.

            So $\lim\limits_{x \to c} f(x_n) = f(c) = 0$.

            Since our choice of $c$ was arbitrary,
            we have that for all $c \in \mathbb{I}, f(c) = 0$.

            So $f$ is 0 on all of $\mathbb{I}$ and $\mathbb{Q}$.

            Thus $f$ is 0 on all of $\mathbb{R}$.
          \end{proof}
        \item
          No, the two functions do not have to be the same since there is not restriction that the functions be continuous.

          Let

          \[
            f(x) = 1,
            g(x) =
            \begin{cases}
              1 & \text{if } \in    \mathbb{Q} \\
              0 & \text{if } \notin \mathbb{Q}
            \end{cases}
          \]

          Then these two functions are not the same,
          yet both equal each other when $x \in \mathbb{Q}$.
      \end{enumerate}
    \item
      \begin{enumerate}
        \item
          Since the given information looks quite similar to the definition of continuity,
          we should try to manipulate it a bit.

          If we had $|f(x) - f(y)| \leq c|x - y| < \epsilon$ for any $\epsilon > 0$,
          we'd be all set.

          \begin{proof}
            Let $y$ be a limit point of $\mathbb{R}$.

            For any $\epsilon > 0$, choose $\delta = \frac{\epsilon}{c}$.

            Then we have $0 < |x - y| < \delta = \frac{\epsilon}{c} \implies |f(x) - f(y)| \leq c|x - y|$.

            Since $|x - y| < \frac{\epsilon}{c} \implies |f(x) - f(y)| \leq c|x - y| < c\frac{\epsilon}{c} = \epsilon$.

            So $f$ is continuous at $y$.
            But our choice of $y$ was arbitrary, so $f$ is continuous on all of $\mathbb{R}$.
          \end{proof}
        \item
        \item
        \item
      \end{enumerate}
    \item
      \begin{enumerate}
        \item
          \begin{itemize}
            \item
              Choose $x = y = 0$.

              Then:
              \begin{align*}
                f(0 + 0) &= f(0) + f(0) \\
                f(0) &= f(0) + f(0) \\
                0 &= f(0) \\
              \end{align*}

              Thus, $f(0) = 0$.
            \item
              Since $f(0) = 0$.

              Then:
              \begin{align*}
                f(0) &= 0 \\
                f(x - x) &= 0, \text{for any } x \in \mathbb{R} \\
                f(x + (- x)) &= 0 \\
                f(x) + f(- x) &= 0 \\
                f(- x) &= -f(x) \\
              \end{align*}

              Thus, $f(- x) = -f(x)$ for any $x \in \mathbb{R}$.
          \end{itemize}
        \item
          \begin{proof}
            Assume $f$ is continuous at 0.

            Choose some $c \in \mathbb{R}$ as a limit point.

            Then there exists a sequence $(x_n) \to c$.
            We also know that $(c) \to c$.

            So we can compute $(x_n - c) \to c - c = 0$.

            Since $f$ is continuous at 0, we have
            \begin{align*}
              \lim\limits_{x \to 0} f(x_n - c) &= f(0) \\
              \lim\limits_{x \to 0} f(x_n + (- c)) &= 0 \\
              \lim\limits_{x \to 0} f(x_n) + f(- c) &= \\
              \lim\limits_{x \to 0} f(x_n) - f(c) &= \\
              \lim\limits_{x \to 0} f(x_n) - \lim\limits_{x \to 0} f(c) &= \\
              \lim\limits_{x \to 0} f(x_n) &= \lim\limits_{x \to 0} f(c) \\
            \end{align*}

            So $f$ is continuous at $c$.

            Since our choice of $c$ was arbitrary, the result holds for all $c \in \mathbb{R}$.

            Thus, if $f$ is continuous at 0 then $f$ is continuous at all points in $\mathbb{R}$.
          \end{proof}
        \item
          \begin{itemize}
            \item
              \begin{proof}
                We can show this by induction.

                \begin{itemize}
                  \item Base
                    $f(1) = k = k(1)$, so the base case is true.

                  \item Inductive
                    Assume $f(n) = kn$.
                    Then $f(n + 1) = f(n) + f(1) = kn + k = k(n + 1)$.
                    So the inductive case in true.
                \end{itemize}

                Thus we have shown by induction that $f(n) = kn, \forall n \in \mathbb{N}$
              \end{proof}
            \item
              We need to look at three case:
              \begin{itemize}
                \item
                  $n > 0$

                  This is exactly the case proven above.

                \item
                  $n = 0$

                  We are given by definition $f(0) = 0 = k(0)$

                \item
                  $n < 0$

                  In this case we can let $-m = n \implies m = -n$ and $m >0$.

                  Then we have $f(-m) = -f(m) = -k(m) = -k(-n) = k(n)$
              \end{itemize}

              From these three we have shown that $f(n) = kn, \forall n \in \mathbb{Z}$
            \item
              Given some $f\left(\frac{p}{q}\right)$, we can rewrite this as
              \begin{align*}
                f\underbrace{\left(\frac{1}{q} + \frac{1}{q} + \dots + \frac{1}{q}\right)}_{p\text{ times}}
                &= \underbrace{f\left(\frac{1}{q}\right) + f\left(\frac{1}{q}\right) + \dots + f\left(\frac{1}{q}\right)}_{p\text{ times}} \\
                &= p f\left(\frac{1}{q}\right)
              \end{align*}

              So we can rewrite
              \begin{align*}
                f(1)
                &= f\underbrace{\left(\frac{1}{n} + \frac{1}{n} + \dots + \frac{1}{n}\right)}_{n\text{ times}} \\
                &= \underbrace{f\left(\frac{1}{n}\right) + f\left(\frac{1}{n}\right) + \dots + f\left(\frac{1}{n}\right)}_{n\text{ times}} \\
                &= n f\left(\frac{1}{n}\right)
                &= k
              \end{align*}
              So, $k = n f\left(\frac{1}{n}\right) \implies f\left(\frac{1}{n}\right) = \frac{k}{n}$

              Then, $f\left(\frac{p}{q}\right) = p f\left(\frac{1}{q}\right) = p \frac{k}{q} = k \frac{p}{q}$.

              Thus $f(n) = k(n), \forall n \in \mathbb{Q}$
          \end{itemize}
        \item
          \begin{proof}
            Choose $x \in \mathbb{I}$ as a limit point.

            Then there exists some $(x_n) \to x$ with all $x_n \in \mathbb{Q}$.

            Since $f$ is continuous on $\mathbb{R}$,
            $f$ is continuous at $x$.

            So, $\lim_{n \to \infty} f(x_n) = f(x)$ and $\lim_{n \to \infty} f(x_n) = \lim_{n \to \infty} k x_n = k x$

            Then $f(x) = k x$.

            Thus we conclude any additive function that is continuous at $x = 0$ must necessarily be a linear function through the origin.
          \end{proof}
      \end{enumerate}
    % \item
    %   \begin{enumerate}
    %     \item
    %     \item
    %     \item
    %     \item
    %   \end{enumerate}
  \end{enumerate}

\end{document}
