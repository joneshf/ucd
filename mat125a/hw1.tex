\documentclass[12pt,letterpaper]{article}
\usepackage{amsmath}
\usepackage{amsfonts}
\usepackage{amsthm}
\usepackage{mathtools}
\usepackage{cancel}
\usepackage[margin=1in]{geometry}
\usepackage{titling}
\usepackage{fp}
\usepackage{enumitem}
\usepackage[super]{nth}
\usepackage{dcolumn}
\usepackage[title]{appendix}
\usepackage{pgfplots}
\pgfplotsset{compat=1.8}
\usepgfplotslibrary{statistics}
\usepackage[round-mode=figures,round-precision=3,scientific-notation=false]{siunitx}
\usepackage{color, colortbl}
\definecolor{Gray}{gray}{0.8}
\newcolumntype{g}{>{\columncolor{Gray}}c}
\newcolumntype{d}{D{.}{.}{-1}}
\DeclarePairedDelimiter\ceil{\lceil}{\rceil}
\DeclarePairedDelimiter\floor{\lfloor}{\rfloor}

\newcommand\epsdelta[5]{
  We want to prove:
  \[
    \lim_{x \to #1} #2 = #3
  \]

  \begin{proof}
    Given $\epsilon > 0$,
    we want to find $\delta > 0$ such that

    \[
      0 < \left|x - #1\right| < \delta \implies \left|#2 - #3\right| < \epsilon
    \].

    #5

    So, choose $\delta = #4$.

    Then we have
    \[
      0 < \left|x - #1\right| < \delta \implies \left|#2 - #3\right| < \epsilon
    \]
    as was to be shown.
  \end{proof}
}
\newcommand\epsdeltaconsequent[5]{
  \epsdelta{#1}{#2}{#3}{#4}{
    We can simplify the consequent a bit.

    #5

    If we notice, this is exactly the form of the antecedent,
    assuming $\delta = #4$.
  }
}
\newcommand\e{e}

\setlength{\droptitle}{-10ex}

\preauthor{\begin{flushright}\large \lineskip 0.5em}
\postauthor{\par\end{flushright}}
\predate{\begin{flushright}\large}
\postdate{\par\end{flushright}}

\title{MAT 125A HW 1\vspace{-2ex}}
\author{Hardy Jones\\
        999397426\\
        Professor Slivken\vspace{-2ex}}
\date{Spring 2015}

\begin{document}
  \maketitle

  \begin{enumerate}[label=Exercise 4.2.\arabic*]
    \item
      \begin{enumerate}
        \item
          \epsdeltaconsequent{2}{(2x + 4)}{8}{\frac{\epsilon}{2}}{
            \begin{align*}
              \left|(2x + 4) - 8\right| &< \epsilon \\
              \left|2x - 4\right| &< \\
              2\left|x - 2\right| &< \\
              \left|x - 2\right| &< \frac{\epsilon}{2} \\
            \end{align*}
          }

        \item

          \epsdeltaconsequent{0}{x^3}{0}{\sqrt[3]{\epsilon}}{
            \begin{align*}
              \left|x^3 - 0\right| &< \epsilon \\
              \left|x^3\right| &< \\
              \left|x\right|^3 &< \\
              \left|x\right| &< \sqrt[3]{\epsilon} \\
              \left|x - 0\right| &< \sqrt[3]{\epsilon} \\
            \end{align*}
          }

        \item

          \epsdelta{2}{x^3}{8}{\text{min}\{1, \frac{\epsilon}{19}\}}{

            Start by manipulating the consequent a bit.

            \begin{align*}
              \left|x^3 - 8\right| &< \epsilon \\
              \left|(x - 2)(x^2 + 2x + 4)\right| &< \\
              \left|x - 2\right|\left|x^2 + 2x + 4\right| &< \\
            \end{align*}

            As in example 4.2.2 (ii),
            we can control the size of $\left|x - 2\right|$ but not $\left|x^2 + 2x + 4\right|$.

            We arbitrarily choose some upper bound for $\delta$, say $\delta \le 1$.
            This gives us a delta neighborhood between 1 and 3.

            Since $\left|x^2 + 2x + 4\right|$ is strictly increasing in the delta neighborhood,
            we only need compute the upper bound.

            So we have $\forall x \in V_\delta(2), \left|x^2 + 2x + 4\right| \le \left|3^2 + 2(3) + 4\right| = 19$ as our upper bound.

            Continuing with the method used in the example,
            we choose $\delta = \text{min}\{1, \frac{\epsilon}{19}\}$.

            So if $0 < \left|x - 2\right| < \delta$, then we have:
            \begin{align*}
              \left|x^3 - 8\right| &= \left|x - 2\right|\left|x^2 + 2x + 4\right| \\
              &< \frac{\epsilon}{19}(19) \\
              &= \epsilon
            \end{align*}
          }

        \item

          \epsdelta{\pi}{\floor{x}}{3}{\pi - 3}{
            We begin by noting that if $\floor{x} = 3$ then $\floor{x} - 3 = 0 < \epsilon$ for any choice of $\epsilon$.

            So, we restrict $\delta$ to only produce $x$ such that $\floor{x} = 3$.
            This happens for any $x$ in the interval $[3, 4)$.
            We can restrict this further to get an exact value of $\delta$ by choosing the neighborhood to be at $\pi$ with a delta of the fractional part of $\pi$.
            That is, $\delta = \pi - 3$.
          }
      \end{enumerate}

    \item

      Any $\delta_0$ smaller than $\delta$ will suffice, as it implies a stronger statement.
      This is because if $0 < \left|x - c\right| < \delta_0$ is true,
      then the following is also true: $0 < \left|x - c\right| < \delta_0 < \delta$.
      From which it follows $\left|f(x) - L\right| < \epsilon$.

    \item
      \begin{enumerate}
        \item
          We have $f(x) = \frac{\left|x\right|}{x}$.
          It is helpful to enumerate some values of this function.
          \begin{tabular}{c | c}
            $x$ & $f(x)$ \\
            \hline
            -3 & -1 \\
            -2 & -1 \\
            -1 & -1 \\
             0 & $\frac{0}{0}$ \\
             1 & 1 \\
             2 & 1 \\
             3 & 1 \\
          \end{tabular}

          So we can see that $f(0)$ is a problem.
          We'll need to construct two sequences that approach 0--so they have the same limit,
          but have different limits when $f$ is applied to them element-wise.

          \begin{proof}
            Choose $x_n = \frac{1}{n}, y_n = -\frac{1}{n}$.

            So $(x_n) = \{1, \frac{1}{2}, \frac{1}{3}, \dots\}$,
            and $(y_n) = \{-1, -\frac{1}{2}, -\frac{1}{3}, \dots\}$.

            Then $\lim(x_n) = 0$ and $\lim(y_n) = 0$.

            Now, $f(x_n) = \{1, 1, 1, \dots\}$ and $f(y_n) = \{-1, -1, -1, \dots\}$.

            So,  $\lim f(x_n) = 1$ and $\lim f(y_n) = -1$.

            Thus,
            we have our function $f(x) = \frac{\left|x\right|}{x}$,
            with $c = 0$.
            We have constructed two sequences $(x_n), (y_n)$
            with $x_n \ne 0, y_n \ne 0, \lim(x_n) = \lim(y_n) = 0,$
            and $\lim f(x_n) \ne \lim f(y_n)$.

            So we conclude by Corollary 4.2.5 that $\lim_{x \to 0} f(x)$ does not exist.
          \end{proof}
        \item
          We have
          \[
            g(x) =
            \begin{cases}
              1 & \text{if } x \in \mathbb{Q} \\
              0 & \text{if } x \notin \mathbb{Q} \\
            \end{cases}
          \].

          We'll need to construct two sequences that approach 1--so they have the same limit,
          but have different limits when $g$ is applied to them element-wise.

          \begin{proof}
            Choose $x_n = \frac{n + 1}{n}, y_n = \frac{n + \e}{n}$.

            So $(x_n) = \{2, \frac{3}{2}, \frac{4}{3}, \dots\}$,
            and $(y_n) = \{1 + \e, \frac{2 + \e}{2}, \frac{3 + \e}{3}, \dots\}$.

            Then $\lim(x_n) = 1$ and $\lim(y_n) = 1$.

            If we look at each element of $(x_n)$ we see that every element is in $\mathbb{Q}$,
            as $n + 1 \in \mathbb{Q}, n \in \mathbb{Q}, n \ne 0$ and $\mathbb{Q}$ is closed under division where the quotient does not equal 0.

            If we look at each element of $(y_n)$ we see that every element is not in $\mathbb{Q}$,
            as $\e \notin \mathbb{Q} \implies n + \e \notin \mathbb{Q} \implies \frac{n + \e}{n} \notin \mathbb{Q}$.

            Now, $g(x_n) = \{1, 1, 1, \dots\}$ and $g(y_n) = \{0, 0, 0, \dots\}$.

            So,  $\lim g(x_n) = 1$ and $\lim g(y_n) = 0$.

            Thus,
            we have our function
            \[
              g(x) =
              \begin{cases}
                1 & \text{if } x \in \mathbb{Q} \\
                0 & \text{if } x \notin \mathbb{Q} \\
              \end{cases},
            \]
            with $c = 1$.

            We have constructed two sequences $(x_n), (y_n)$
            with $x_n \ne 1, y_n \ne 1, \lim(x_n) = \lim(y_n) = 1,$
            and $\lim g(x_n) \ne \lim g(y_n)$.

            So we conclude by Corollary 4.2.5 that $\lim_{x \to 1} g(x)$ does not exist.
          \end{proof}
      \end{enumerate}

    \item

      We have
      \[
        t(x) =
        \begin{cases}
          1           & \text{if } x = 0 \\
          \frac{1}{n} & \text{if } x = \frac{m}{n} \in \mathbb{Q} \setminus \{0\} \text{ is in lowest terms with } n > 0 \\
          0           & \text{if } x \notin \mathbb{Q} \\
        \end{cases}
      \].

      \begin{enumerate}
        \item

          We can choose

          \begin{itemize}
            \item
              $x_n = \frac{n + 1}{n}$,

              $(x_n) = \{2, \frac{3}{2}, \frac{4}{3}, \dots\}$
            \item
              $y_n = \frac{n + \e}{n}$,

              $(y_n) = \{1 + \e, \frac{2 + \e}{2}, \frac{3 + \e}{3}, \dots\}$
            \item
              $z_n = \frac{n + \e}{n + 1}$,

              $(z_n) = \{\frac{1 + \e}{2}, \frac{2 + \e}{3}, \frac{3 + \e}{4}, \dots\}$
          \end{itemize}

          So all of these sequences converge to 1 and no sequence contains 1.
        \item
          Now we compute the limits.

          \begin{itemize}
            \item
              Taken element-wise
              $t(x_n) = \{1, \frac{1}{2}, \frac{1}{3}, \dots\}$.

              So, $\lim_{x \to 1} t(x_n) = 0$
            \item
              Taken element-wise
              $t(y_n) = \{0, 0, 0, \dots\}$.

              So, $\lim_{x \to 1} t(y_n) = 0$
            \item
              Taken element-wise
              $t(z_n) = \{0, 0, 0, \dots\}$.

              So, $\lim_{x \to 1} t(z_n) = 0$
          \end{itemize}

        \item

          We conject that $\lim_{x \to 1} t(x) = 0$.

      \end{enumerate}

    \setcounter{enumi}{5}

    \item

      \begin{proof}
        Since $f(x)$ is bounded by some $M > 0$ such that $\forall x \in A, \left|f(x)\right| \le M$,
        we know that $\left|f(x)\right| > 0$.

        If $\lim_{x \to c} g(x) = 0$,
        then we know that for any $\epsilon > 0$, there exists some $\delta > 0$,
        such that $0 < \left|x - c\right| < \delta \implies \left|g(x) - 0\right| = \left|g(x)\right| < \epsilon$.

        We can choose $\epsilon_0 = \frac{\epsilon}{M}$, and we have some $\delta_0$ such that:

        $0 < \left|x - c\right| < \delta_0 \implies \left|g(x)\right| < \frac{\epsilon_0}{M}$.

        From this we can show:
        \begin{align*}
          \left|g(x)\right| &< \frac{\epsilon}{M} \\
          \left|g(x)\right|\left|f(x)\right| &< \left(\frac{\epsilon}{M}\right)M \\
          \left|g(x)\right|\left|f(x)\right| &< \epsilon \\
          \left|g(x) f(x)\right| &< \epsilon \\
          \left|g(x) f(x) - 0\right| &< \epsilon \\
        \end{align*}

        So, $\lim_{x \to c} g(x)f(x) = 0$.

        Thus, if $\lim_{x \to c} g(x) = 0$, then $\lim_{x \to c}g(x)f(x) = 0$ as well.
      \end{proof}

    \item

      \begin{enumerate}
        \item

          Let $f : A \to R$, and let $c$ be a limit point of the domain $A$.
          We say that $\lim_{x \to c} f(x) = \infty$ provided that, for all arbitrarily large $\epsilon$, there exists a $\delta > 0$
          such that whenever $0 < |x - c| < \delta$ (and $x \in A$) it follows that $f(x) > \epsilon$.

          \begin{proof}
            We have $f(x) = \frac{1}{x^2}, c = 0$.

            Given an arbitrarily large $\epsilon$,
            we want to find $\delta > 0$ such that

            \[
              0 < \left|x - 0\right| = \left|x\right| < \delta \implies \frac{1}{x^2} > \epsilon
            \].

            We can simplify the consequent to:

            \begin{align*}
              \frac{1}{x^2} &> \epsilon \\
              \frac{1}{\epsilon} &> x^2 \\
              \frac{1}{\sqrt{\epsilon}} &> \left|x\right| \\
            \end{align*}

            So, choose $\delta = \frac{1}{\sqrt{\epsilon}}$.

            Then we have
            \[
              0 < \left|x\right| < \delta \implies \frac{1}{x^2} > \epsilon
            \]
            as was to be shown.

          \end{proof}

        \item

          Let $f : A \to R$.
          We say that $\lim_{x \to \infty} f(x) = L$ provided that, for all $\epsilon > 0$, there exists an arbitrarily large $\delta$
          such that whenever $x > \delta$ (and $x \in A$) it follows that $\left|f(x) - L\right| < \epsilon$.

          \begin{proof}
            We have $f(x) = \frac{1}{x}$.

            Given an $\epsilon > 0$,
            we want to find an arbitrarily large $\delta$ such that

            \[
              x > \delta \implies \left|\frac{1}{x} - 0\right| = \left|\frac{1}{x}\right| < \epsilon
            \].

            Since we know $x > 0$, we can simplify the consequent to:

            \begin{align*}
              \left|\frac{1}{x}\right| &< \epsilon \\
              \frac{1}{x} &< \epsilon \\
              \frac{1}{\epsilon} &< x \\
            \end{align*}

            So, choose $\delta = \frac{1}{\epsilon}$.

            Then we have
            \[
              x > \delta \implies \left|\frac{1}{x} - 0\right| < \epsilon
            \]
            as was to be shown.

          \end{proof}

        \item
          Let $f : A \to R$.

          We say that $\lim_{x \to \infty} f(x) = \infty$ provided that,
          for all arbitrarily large $\epsilon$,
          there exists an arbitrarily large $\delta$
          such that whenever $x > \delta$ (and $x \in A$) it follows that $f(x) > \epsilon$.

          An example of such a limit is

          \[
            \lim_{x \to \infty} x = \infty
          \]

          \begin{proof}
            We have $f(x) = x$.

            Given an arbitrarily large $\epsilon$,
            we want to find an arbitrarily large $\delta$ such that

            \[
              x > \delta \implies x > \epsilon
            \].

            So, choose $\delta = \epsilon$.

            Then we have
            \[
              x > \delta \implies x > \epsilon
            \]
            as was to be shown.

          \end{proof}

      \end{enumerate}

    \item

      \begin{proof}

        Choose some sequence $(x_n)$ such that $(x_n) \to c$ and $x_n \ne c$.

        Then we have $f(x_n) \ge g(x_n)$ for all $x \in A$.

        Let $\lim_{x \to c} f(x_n) = L$ and $\lim_{x \to c} g(x_n) = M$.

        Now the Order Limit Theorem states:

        If $f(x_n) \ge g(x_n)$ for all $n$, then $L \ge M$.

        Thus we have
        \[
          \lim_{x \to c} f(x) \ge \lim_{x \to c} g(x)
        \]
        as was to be shown.
      \end{proof}

    \item

        We have $\lim_{x \to c} f(x) = L$ and $\lim_{x \to c} h(x) = L$.

        So for any $\epsilon > 0$, there exists $\delta_0 > 0$ such that

        $0 < \left|x - c\right| < \delta_0 \implies \left|f(x) - L\right| < \epsilon$ and

        for any $\epsilon > 0$, there exists $\delta_1 > 0$ such that

        $0 < \left|x - c\right| < \delta_1 \implies \left|h(x) - L\right| < \epsilon$.

        We need to show for any $\epsilon > 0$, there exists $\delta > 0$ such that

        $0 < \left|x - c\right| < \delta \implies \left|g(x) - L\right| < \epsilon$.

      \begin{proof}
        Given $\epsilon > 0$, we have $\delta_0, \delta_1 > 0$ such that

        $0 < \left|x - c\right| < \delta_0 \implies \left|f(x) - L\right| < \epsilon$ and

        $0 < \left|x - c\right| < \delta_1 \implies \left|h(x) - L\right| < \epsilon$.

        We can manipulate $\left|f(x) - L\right| < \epsilon$ to
        $-\epsilon < f(x) - L < \epsilon$, and

        $\left|h(x) - L\right| < \epsilon$ to
        $-\epsilon < h(x) - L < \epsilon$.

        Now choose $\delta = \text{min}\{\delta_0, \delta_1\}$

        Since we know $f(x) \le g(x) \le h(x)$
        it follows that $f(x) - L \le g(x) - L \le h(x) - L$,

        from this and our choice of $\delta$ it follows that $-\epsilon < f(x) - L \le g(x) - L \le h(x) - L < \epsilon$,

        which simplifies to $-\epsilon < g(x) - L < \epsilon = \left|g(x) - L\right| < \epsilon$.

        Then we have
        \[
          0 < \left|x - c\right| < \delta \implies \left|g(x) - L\right| < \epsilon
        \].

        Thus, we have $\lim_{x \to c} g(x) = L$, as was to be shown.

      \end{proof}

  \end{enumerate}

\end{document}
