\documentclass[12pt,letterpaper]{article}
\usepackage{amsmath}
\usepackage{amsfonts}
\usepackage{amsthm}
\usepackage{mathtools}
\usepackage{cancel}
\usepackage[margin=1in]{geometry}
\usepackage{titling}
\usepackage{fp}
\usepackage{enumitem}
\usepackage[super]{nth}
\usepackage{dcolumn}
\usepackage[title]{appendix}
\usepackage{pgfplots}
\pgfplotsset{compat=1.8}
\usepgfplotslibrary{statistics}
\usepackage[round-mode=figures,round-precision=3,scientific-notation=false]{siunitx}
\usepackage{color, colortbl}
\definecolor{Gray}{gray}{0.8}
\newcolumntype{g}{>{\columncolor{Gray}}c}
\newcolumntype{d}{D{.}{.}{-1}}
\DeclarePairedDelimiter\ceil{\lceil}{\rceil}
\DeclarePairedDelimiter\floor{\lfloor}{\rfloor}

\newcommand\epsdelta[5]{
  We want to prove:
  \[
    \lim_{x \to #1} #2 = #3
  \]

  \begin{proof}
    Choose $\epsilon > 0$,
    we want to find $\delta > 0$ such that

    \[
      0 < \left|x - #1\right| < \delta \implies \left|#2 - #3\right| < \epsilon
    \].

    We can simplify the consequent a bit.

    #5

    If we notice, this is exactly the form of the antecedent,
    assuming $\delta = #4$.

    So, choose $\delta = #4$.

    Then we have
    \[
      0 < \left|x - #1\right| < \delta \implies \left|#2 - #3\right| < \epsilon
    \]
    as was to be shown.
  \end{proof}
}
\newcommand\e{e}

\setlength{\droptitle}{-10ex}

\preauthor{\begin{flushright}\large \lineskip 0.5em}
\postauthor{\par\end{flushright}}
\predate{\begin{flushright}\large}
\postdate{\par\end{flushright}}

\title{MAT 125A HW 1\vspace{-2ex}}
\author{Hardy Jones\\
        999397426\\
        Professor Slivken\vspace{-2ex}}
\date{Spring 2015}

\begin{document}
  \maketitle

  \begin{enumerate}[label=Exercise 4.2.\arabic*]
    \item
      \begin{enumerate}
        \item
          \epsdelta{2}{(2x + 4)}{8}{\frac{\epsilon}{2}}{
            \begin{align*}
              \left|(2x + 4) - 8\right| &< \epsilon \\
              \left|2x - 4\right| &< \\
              2\left|x - 2\right| &< \\
              \left|x - 2\right| &< \frac{\epsilon}{2} \\
            \end{align*}
          }

        \item

          \epsdelta{0}{x^3}{0}{\sqrt[3]{\epsilon}}{
            \begin{align*}
              \left|x^3 - 0\right| &< \epsilon \\
              \left|x^3\right| &< \\
              \left|x\right|^3 &< \\
              \left|x\right| &< \sqrt[3]{\epsilon} \\
              \left|x - 0\right| &< \sqrt[3]{\epsilon} \\
            \end{align*}
          }

        \item

          \epsdelta{2}{x^3}{8}{\sqrt[3]{\epsilon}}{
            \begin{align*}
              \left|x^3 - 8\right| &< \epsilon \\
              \left|(x - 2)(x^2 + 2x + 4)\right| &< \\
            \end{align*}
          }

        \item

          \epsdelta{\pi}{\floor{x}}{3}{0}{}
      \end{enumerate}

    \item

      Any $\delta_0$ smaller than $\delta$ will suffice, as it implies a stronger statement.
      This is because if $0 < \left|x - c\right| < \delta_0$ is true,
      then the following is also true: $0 < \left|x - c\right| < \delta_0 < \delta$.
      From which it follows $\left|f(x) - L\right| < \epsilon$.

    \item
      \begin{enumerate}
        \item
          We have $f(x) = \frac{\left|x\right|}{x}$.
          It is helpful to enumerate some values of this function.
          \begin{tabular}{c | c}
            $x$ & $f(x)$ \\
            \hline
            -3 & -1 \\
            -2 & -1 \\
            -1 & -1 \\
             0 & $\frac{0}{0}$ \\
             1 & 1 \\
             2 & 1 \\
             3 & 1 \\
          \end{tabular}

          So we can see that $f(0)$ is a problem.
          We'll need to construct two sequences that approach 0--so they have the same limit,
          but have different limits when $f$ is applied to them element-wise.

          \begin{proof}
            Choose $x_n = \frac{1}{n}, y_n = -\frac{1}{n}$.

            So $(x_n) = \{1, \frac{1}{2}, \frac{1}{3}, \dots\}$,
            and $(y_n) = \{-1, -\frac{1}{2}, -\frac{1}{3}, \dots\}$.

            Then $\lim(x_n) = 0$ and $\lim(y_n) = 0$.

            Now, $f(x_n) = \{1, 1, 1, \dots\}$ and $f(y_n) = \{-1, -1, -1, \dots\}$.

            So,  $\lim f(x_n) = 1$ and $\lim f(y_n) = -1$.

            Thus,
            we have our function $f(x) = \frac{\left|x\right|}{x}$,
            with $c = 0$.
            We have constructed two sequences $(x_n), (y_n)$
            with $x_n \ne 0, y_n \ne 0, \lim(x_n) = \lim(y_n) = 0,$
            and $\lim f(x_n) \ne \lim f(y_n)$.

            So we conclude by Corollary 4.2.5 that $\lim_{x \to 0} f(x)$ does not exist.
          \end{proof}
        \item
          We have
          \[
            g(x) =
            \begin{cases}
              1 & \text{if } x \in \mathbb{Q} \\
              0 & \text{if } x \notin \mathbb{Q} \\
            \end{cases}
          \].

          We'll need to construct two sequences that approach 1--so they have the same limit,
          but have different limits when $g$ is applied to them element-wise.

          \begin{proof}
            Choose $x_n = \frac{n + 1}{n}, y_n = \frac{n + \e}{n}$.

            So $(x_n) = \{2, \frac{3}{2}, \frac{4}{3}, \dots\}$,
            and $(y_n) = \{1 + \e, \frac{2 + \e}{2}, \frac{3 + \e}{3}, \dots\}$.

            Then $\lim(x_n) = 1$ and $\lim(y_n) = 1$.

            If we look at each element of $(x_n)$ we see that every element is in $\mathbb{Q}$,
            as $n + 1 \in \mathbb{Q}, n \in \mathbb{Q}, n \ne 0$ and $\mathbb{Q}$ is closed under division where the quotient does not equal 0.

            If we look at each element of $(y_n)$ we see that every element is not in $\mathbb{Q}$,
            as $\e \notin \mathbb{Q} \implies n + \e \notin \mathbb{Q} \implies \frac{n + \e}{n} \notin \mathbb{Q}$.

            Now, $g(x_n) = \{1, 1, 1, \dots\}$ and $g(y_n) = \{0, 0, 0, \dots\}$.

            So,  $\lim g(x_n) = 1$ and $\lim g(y_n) = 0$.

            Thus,
            we have our function
            \[
              g(x) =
              \begin{cases}
                1 & \text{if } x \in \mathbb{Q} \\
                0 & \text{if } x \notin \mathbb{Q} \\
              \end{cases},
            \]
            with $c = 1$.

            We have constructed two sequences $(x_n), (y_n)$
            with $x_n \ne 1, y_n \ne 1, \lim(x_n) = \lim(y_n) = 1,$
            and $\lim g(x_n) \ne \lim g(y_n)$.

            So we conclude by Corollary 4.2.5 that $\lim_{x \to 1} g(x)$ does not exist.
          \end{proof}
      \end{enumerate}

    \item

    \setcounter{enumi}{5}

    \item

    \item

    \item

    \item

  \end{enumerate}

\end{document}
