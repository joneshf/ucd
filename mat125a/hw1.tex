\documentclass[12pt,letterpaper]{article}
\usepackage{amsmath}
\usepackage{amsfonts}
\usepackage{amsthm}
\usepackage{mathtools}
\usepackage{cancel}
\usepackage[margin=1in]{geometry}
\usepackage{titling}
\usepackage{fp}
\usepackage{enumitem}
\usepackage[super]{nth}
\usepackage{dcolumn}
\usepackage[title]{appendix}
\usepackage{pgfplots}
\pgfplotsset{compat=1.8}
\usepgfplotslibrary{statistics}
\usepackage[round-mode=figures,round-precision=3,scientific-notation=false]{siunitx}
\usepackage{color, colortbl}
\definecolor{Gray}{gray}{0.8}
\newcolumntype{g}{>{\columncolor{Gray}}c}
\newcolumntype{d}{D{.}{.}{-1}}

\newcommand\epsdelta[3]{
  We want to prove:
  \[
    \lim_{x \to #1} #2 = #3
  \]

  \begin{proof}
    Choose $\epsilon > 0$,
    we want to find $\delta > 0$ such that

    \[
      0 < \left|x - #1\right| < \delta \implies \left|#2 - #3\right| < \epsilon
    \].

    We can simplify the consequent a bit.

    \begin{align*}
      \left|#2 - #3\right| &< \epsilon \\
      \left|2x - 4\right| &< \\
      2\left|x - 2\right| &< \\
      \left|x - 2\right| &< \frac{\epsilon}{2} \\
    \end{align*}

    If we notice, this is exactly the form of the antecedent,
    assuming $\delta = \frac{\epsilon}{2}$.

    So, choose $\delta = \frac{\epsilon}{2}$.
    Then we have
    \[
      0 < \left|x - #1\right| < \delta \implies \left|#2 - #3\right| < \epsilon
    \]
    as was to be shown.
  \end{proof}
}

\setlength{\droptitle}{-10ex}

\preauthor{\begin{flushright}\large \lineskip 0.5em}
\postauthor{\par\end{flushright}}
\predate{\begin{flushright}\large}
\postdate{\par\end{flushright}}

\title{MAT 125A HW 1\vspace{-2ex}}
\author{Hardy Jones\\
        999397426\\
        Professor Slivken\vspace{-2ex}}
\date{Spring 2015}

\begin{document}
  \maketitle

  \begin{enumerate}
    \item $\S 4.2$
      \begin{enumerate}[label=Exercise 4.2.\arabic*]
        \item

          We want to prove:
          \[
            \lim_{x \to 2} (2x + 4) = 8
          \]

          \begin{proof}
            Choose $\epsilon > 0$,
            we want to find $\delta > 0$ such that

            \[
              0 < \left|x - 2\right| < \delta \implies \left|(2x + 4) - 8\right| < \epsilon
            \].

            We can simplify the consequent a bit.

            \begin{align*}
              \left|(2x + 4) - 8\right| &< \epsilon \\
              \left|2x - 4\right| &< \\
              2\left|x - 2\right| &< \\
              \left|x - 2\right| &< \frac{\epsilon}{2} \\
            \end{align*}

            If we notice, this is exactly the form of the antecedent,
            assuming $\delta = \frac{\epsilon}{2}$.

            So, choose $\delta = \frac{\epsilon}{2}$.
            Then we have
            \[
              0 < \left|x - 2\right| < \delta \implies \left|(2x + 4) - 8\right| < \epsilon
            \]
            as was to be shown.
          \end{proof}

        \item

        \item

        \item

        \setcounter{enumii}{5}

        \item

        \item

        \item

        \item

      \end{enumerate}
  \end{enumerate}

\end{document}
