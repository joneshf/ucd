\documentclass[12pt,letterpaper]{article}

\usepackage[margin=1in]{geometry}
\usepackage[round-mode=figures,round-precision=3,scientific-notation=false]{siunitx}
\usepackage[super]{nth}
\usepackage[title]{appendix}
\usepackage{amsfonts}
\usepackage{amsmath}
\usepackage{amsthm}
\usepackage{cancel}
\usepackage{color, colortbl}
\usepackage{dcolumn}
\usepackage{enumitem}
\usepackage{fp}
\usepackage{mathtools}
\usepackage{pgfplots}
\usepackage{titling}

\usepgfplotslibrary{statistics}

\pgfplotsset{compat=1.8}

\definecolor{Gray}{gray}{0.8}

\newcolumntype{g}{>{\columncolor{Gray}}c}
\newcolumntype{d}{D{.}{.}{-1}}

\DeclarePairedDelimiter\ceil{\lceil}{\rceil}
\DeclarePairedDelimiter\floor{\lfloor}{\rfloor}

\newcommand\epsdelta[5]{
  We want to prove:
  \[
    \lim_{x \to #1} #2 = #3
  \]

  \begin{proof}
    Given $\epsilon > 0$,
    we want to find $\delta > 0$ such that

    \[
      0 < \left|x - #1\right| < \delta \implies \left|#2 - #3\right| < \epsilon
    \].

    #5

    So, choose $\delta = #4$.

    Then we have
    \[
      0 < \left|x - #1\right| < \delta \implies \left|#2 - #3\right| < \epsilon
    \]
    as was to be shown.
  \end{proof}
}
\newcommand\epsdeltaconsequent[5]{
  \epsdelta{#1}{#2}{#3}{#4}{
    We can simplify the consequent a bit.

    #5

    If we notice, this is exactly the form of the antecedent,
    assuming $\delta = #4$.
  }
}
\newcommand\e{e}
\newcommand\uc{uniformly continuous }

\renewcommand{\labelenumi}{4.\arabic*}
\renewcommand{\labelenumii}{\arabic*}
\renewcommand{\labelenumiii}{(\alph*)}

\setlength{\droptitle}{-10ex}

\preauthor{\begin{flushright}\large \lineskip 0.5em}
\postauthor{\par\end{flushright}}
\predate{\begin{flushright}\large}
\postdate{\par\end{flushright}}

\title{MAT 125A HW 3\vspace{-2ex}}
\author{Hardy Jones\\
        999397426\\
        Professor Slivken\vspace{-2ex}}
\date{Spring 2015}

\begin{document}
  \maketitle

  \begin{enumerate}
    \setcounter{enumi}{3}
    \item
      \begin{enumerate}
        \setcounter{enumii}{3}
        \item

          \begin{proof}
            Assume $f$ is continuous on $[a, b]$ and for all $x \in [a, b], f(x) > 0$.

            So we have $[a, b]$ is compact,
            then we know that $f$ is compact on $[a, b]$, since $f$ is continuous on $[a, b]$.

            By the extreme value theorem, $f$ has a minimum and maximum value.
            So there is some $x_0 \in [a, b]$ such that for all $x \in [a, b], f(x_0) \leq f(x)$.

            Now, since we know for all $x \in [a, b], f(x) > 0$,

            we have $f(x_0) \leq f(x) \implies \frac{1}{f(x)} \leq \frac{1}{f(x_0)}$.

            Then for all $x \in [a, b], \frac{1}{f(x)}$ is bounded by $\frac{1}{f(x_0)}$.
          \end{proof}
        \setcounter{enumii}{5}
        \item
          \begin{enumerate}
            \item
              Let $f(x) = \frac{1}{x}$.
              This function is continuous on $(0, 1)$.

              If we take the sequence $(x_n) = \frac{1}{n}$,
              then $(x_n)$ is Cauchy.

              But $f(x_n) = \frac{1}{\frac{1}{n}} = n$, and this sequence is unbounded.
              So it is not Cauchy.
            \item
            \item
            \item
              Let $f(x) = -\left(x - \frac{1}{2}\right)^2$.

              Then $f$ has a maximum at the root of the polynomial, but no minimum value.
          \end{enumerate}
        \setcounter{enumii}{7}
        \item
          \begin{enumerate}
            \item
              We want to show
              if there exists some $b > 0$ such that $f$ is \uc on $[b, \infty)$,
              then $f$ is \uc on $[0, \infty)$.
              \begin{proof}
                Given $f : [0, \infty) \to \mathbb{R}$ is continuous at every point in $[0, \infty)$.

                Assume that $f$ is not \uc on $[0, \infty)$,
                then we want to show that for any $b > 0$ $f$ is not \uc on $[b, \infty)$.

                We can rephrase the first part as:

                There exists some $\epsilon_0 > 0$ such that for all $\delta_0 > 0$,
                $|x - y| < \delta_0$ and $|f(x) - f(y)| \geq \epsilon_0$.

                And the second part as for any $b > 0$, there exists some $\epsilon_1 > 0$ such that
                for all $\delta_1 > 0$,
                $|x - y| < \delta_1$ and $|f(x) - f(y)| \geq \epsilon_1$.

                If we choose $\epsilon_0 = \epsilon_1$,
                then for any $\delta_1 > 0$ there exists some $0 < \delta_0 < \delta_1$,

                where $|x - y| < \delta_0$ and $|f(x) - f(y)| \geq \epsilon_0$.

                Then we have that $f$ is not \uc, as we desired.

                Thus, by contraposition:


                Given $f : [0, \infty) \to \mathbb{R}$ is continuous at every point in $[0, \infty)$,
                if there exists some $b > 0$ such that $f$ is \uc on $[b, \infty)$,
                then $f$ is \uc on $[0, \infty)$.
              \end{proof}
            \item
              \begin{proof}
                From part (a), we need an interval $[b, \infty)$ with $f$ continuous at every point.

                If we choose $b = 1$,
                then for any $\epsilon > 0$ we can choose $\delta = \epsilon$.

                We also know that $\sqrt{x} + \sqrt{y} \geq 1 + 1 = 2$

                Now,

                \begin{align*}
                  |x - y| &< \epsilon \\
                  |\sqrt{x} - \sqrt{y}||\sqrt{x} + \sqrt{y}| &< \epsilon \\
                  |\sqrt{x} - \sqrt{y}| &< \frac{\epsilon}{|\sqrt{x} + \sqrt{y}|} \\
                  |\sqrt{x} - \sqrt{y}| &\leq \frac{\epsilon}{2} \\
                  |\sqrt{x} - \sqrt{y}| &< \epsilon \\
                \end{align*}

                So $f$ is \uc on $[1, \infty)$.

                From part (a), we conclude that $f$ is \uc on $[0, \infty)$.

                Thus, $f(x) = \sqrt{x}$ \uc on $[0, \infty)$.
              \end{proof}
          \end{enumerate}
        \item
          \begin{enumerate}
            \item
              \begin{proof}
                Assume $f : A \to \mathbb{R}$ is Lipschitz.

                Then there exists some $M > 0$ such that for all $x, y \in A$,

                \[\left|\frac{f(x) - f(y)}{x - y}\right| \leq M\].

                For any $\epsilon > 0$, choose $\delta = \frac{\epsilon}{M}$.

                Then, when $|x - y| < \delta$,

                \begin{align*}
                  \left|\frac{f(x) - f(y)}{x - y}\right| &\leq M \\
                  \frac{\left|f(x) - f(y)\right|}{\left|x - y\right|} &\leq M \\
                  \left|f(x) - f(y)\right| &\leq \left|x - y\right|M \\
                  \left|f(x) - f(y)\right| &< \frac{\epsilon}{M}M \\
                  \left|f(x) - f(y)\right| &< \epsilon \\
                \end{align*}

                So, $f$ is \uc.

                Thus, if $f : A \to \mathbb{R}$ is Lipschitz,
                then $f$ is \uc.
              \end{proof}
            \item
              No, not all \uc functions are Lipschitz.

              Let $f(x) = \sqrt{x}$ on $[0, \infty)$, then we know this is \uc by a problem above.

              But if $y = 0$ then

              \[
                \left|\frac{\sqrt{x} - \sqrt{0}}{x - 0}\right|
                =
                \left|\frac{\sqrt{x}}{x}\right|
                =
                \frac{\sqrt{x}}{x}
              \]

              and this grows unbounded when $x \to 0$.

              So $f$ is not Lipschitz.
          \end{enumerate}
        \setcounter{enumii}{12}
        \item
          \begin{enumerate}
            \item
              \begin{proof}
                Assume $f : A \to \mathbb{R}$ is \uc,
                and $(x_n) \subseteq A$ is Cauchy.

                Then we have,
                for all $\epsilon > 0$ there exists some $\delta > 0$ such that,
                if $|x - y| < \delta$ then $|f(x) - f(y)| < \epsilon$.

                And also,
                for all $\epsilon > 0$ there exists some $N \in \mathbb{N}$ such that,
                for all $m, n \geq N, |x_m - x_n| < \epsilon$.

                So, given some $\epsilon > 0$, choose $\delta = \epsilon$.

                Since $(x_n)$ is Cauchy, there exists $N_0 \in \mathbb{N}$ such that,
                for all $m, n \geq N, |x_m - x_n| < \delta$.

                And since we know $f$ is \uc, we have $|f(x_m) - f(x_n)| < \epsilon$.

                So $f(x_n)$ is also Cauchy.
              \end{proof}
            \item
              \begin{proof}
                We need to prove both directions.

                \begin{itemize}
                  \item $(\Longrightarrow)$

                  \item $(\Longleftarrow)$

                    Since $g$ is continuous and $[a, b]$ is compact,
                    $g$ is \uc on $[a, b]$.

                    And Since $(a, b) \subseteq [a, b]$, $g$ is \uc on $(a, b)$.
                \end{itemize}
              \end{proof}
          \end{enumerate}
      \end{enumerate}
    \item
      \begin{enumerate}
        \item
          \begin{proof}
            The closed interval $[a, b] = E$ is connected.

            Given Theorem 4.5.2, $f(E)$ is also connected.

            If we have some $L$ between $f(a)$ and $f(b)$,
            then $L \in f(E)$.
            And if $L \in f(E)$ there must be some $c \in E$ such that $f(c) = L$.

            This is the Intermediate Value Theorem.
          \end{proof}
        \item
          \begin{enumerate}
            \item False.

              Let $f(x) = \frac{1}{x}$.

              Then on the bounded open interval $(0, 1)$,
              the range of $f$ is $(0, \infty)$, which is unbounded.
            \item False.

              Let $f(x) = x^2$.

              Then on the bounded open interval $(-1, 1)$,
              the range of $f$ is $[0, 1)$, which is a not an open set.
            \item True.

              Let $f$ be continuous on a bounded closed interval $A$.
              Then this interval is compact, and so $f(A)$ is also compact.

              And since $f(A)$ is compact, $f(A)$ is an interval.
          \end{enumerate}
        \setcounter{enumii}{6}
        \item
          \begin{proof}
            Assume $f$ is continuous on $[0, 1]$.
            We can construct a function $g(x) = x$ which is also continuous with the same domain and range.

            $g$ is continuous if we take $\epsilon = \delta$ for any $\epsilon > 0$, then $|x - y| < \delta = \epsilon$.

            Then we can construct another continuous function, $h(x) = f(x) - g(x)$.

            Now $h(x)$ has domain $[0, 1]$ and range $[-1, 1]$.

            Where at $x = 0$, $h(0) = f(0) - g(0) = f(0)$ so $h(0) \in [0, 1]$.
            And at $x = 1$, $h(1) = f(1) - g(1) = f(1) - 1$ so $h(1) \in [-1, 0]$.

            So by the Intermediate Value Theorem, there must exist some $c \in [0, 1]$
            such that $h(c) = 0 = f(c) - g(c) \implies f(c) = g(c) \implies f(c) = c$.
          \end{proof}
      \end{enumerate}
    % \item
    %   \begin{enumerate}
    %     \setcounter{enumii}{3}
    %     \item
    %     \item
    %     \setcounter{enumii}{8}
    %     \item
    %     \item
    %   \end{enumerate}
  \end{enumerate}
\end{document}
