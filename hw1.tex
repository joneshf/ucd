\documentclass[12pt,letterpaper]{article}
\usepackage{amsmath}
\usepackage{amsfonts}
\usepackage{amsthm}
\usepackage{cancel}
\usepackage[margin=1in]{geometry}
\usepackage{titling}

\setlength{\droptitle}{-10ex}

\preauthor{\begin{flushright}\large \lineskip 0.5em}
\postauthor{\par\end{flushright}}
\predate{\begin{flushright}\large}
\postdate{\par\end{flushright}}

\title{MAT 67 Homework 1\vspace{-2ex}}
\author{Hardy Jones\\
        999397426\\
        Professor Bandyopadhyay\vspace{-2ex}}
\date{Fall 2013}

\begin{document}

  \maketitle

  \begin{enumerate}
    \item
      Let $a$, $b$, $c$ and $d$ be fixed real numbers and
      consider the following system of linear equations in two real variables
      $x_1$ and $x_2$
      \begin{align*}
        ax_1 + bx_2 &= 0 \\
        cx_1 + dx_2 &= 0
      \end{align*}
      Note that $x_1 = x_2 = 0$ is a solution of the above equations for any
      choice of $a$, $b$, $c$, and $d$.

      Prove that if $ad - bc \neq 0$, then $x_1 = x_2 = 0$ is the only solution.

      \begin{proof}
        Assume $ad - bc \neq 0$
        \begin{alignat*}{3}
          ax_1 & {} + {} & bx_2 & {} = {} & 0 \\
          cx_1 & {} + {} & dx_2 & {} = {} & 0 \tag{1}
        \end{alignat*}

        Let's multiply the first equation by $d$ and the second equation by $b$.
        \begin{alignat*}{3}
          adx_1 & {} + {} & bdx_2 & {} = {} & 0 \\
          bcx_1 & {} + {} & bdx_2 & {} = {} & 0
        \end{alignat*}

        Now let's subtract the second equation from the first.
        \[adx_1 - bcx_1 = (ad - bc)x_1 = 0\]

        We assumed that $ad-bc \neq 0$ so we can divide through by $ad - bc$.
        \[x_1 = 0\]

        Now we need $x_2$.  Let's go back to the system in $(1)$.
        \begin{alignat*}{3}
          ax_1 & {} + {} & bx_2 & {} = {} & 0 \\
          cx_1 & {} + {} & dx_2 & {} = {} & 0 \tag{1}
        \end{alignat*}

        Now let's multiply the first equation by $c$ and the second by $a$.
        \begin{alignat*}{3}
          acx_1 & {} + {} & bcx_2 & {} = {} & 0 \\
          acx_1 & {} + {} & adx_2 & {} = {} & 0
        \end{alignat*}

        Now let's subtract the first equation from the second.
        \[adx_2 - bcx_2 = (ad - bc)x_2 = 0\]

        Again, we assumed that $ad - bc \neq 0$ so we divide by $ad - bc$.
        \[x_2 = 0\]

        So $x_1 = x_2 = 0$.
        Thus if $ad - bc \neq 0$, then $x_1 = x_2 = 0$.

      \end{proof}

    \item
      Let $z, \omega \in \mathbb{C}$.

      Prove:
      \[|z - \omega|^2 + |z + \omega|^2 = 2(|z|^2 + |\omega|^2)\]

      \begin{proof}
        Given $z, \omega \in \mathbb{C}$,

        $\exists a, b, c, d \in \mathbb{R} : a + ib = z, c + id = \omega$
        \begin{align*}
          |z - \omega|^2 + |z + \omega|^2 &=
            |(a + ib) - (c + id)|^2 + |(a + ib) + (c + id)|^2 \\
          &= |(a - c) + i(b - d)|^2 + |(a + c) + i(b + d)|^2 \\
          &= \sqrt{(a - c)^2 + (b - d)^2}^2 + \sqrt{(a + c)^2 + (b + d)^2}^2 \\
          &= (a - c)^2 + (b - d)^2 + (a + c)^2 + (b + d)^2 \\
          &= (a^2 - 2ac + c^2) + (b^2 - 2bd + d^2) + (a^2 + 2ac + c^2) + (b^2 + 2bd + d^2) \\
          &= a^2 + a^2 + b^2 + b^2 + c^2 + c^2 + d^2 + d^2 + \cancel{2ac - 2ac} + \cancel{2bd - 2bd} \\
          &= 2a^2 + 2b^2 + 2c^2 + 2d^2 \\
          &= 2(a^2 + b^2 + c^2 + d^2) \\
          &= 2(\sqrt{a^2 + b^2}^2 + \sqrt{c^2 + d^2}^2) \\
          &= 2(|a + ib|^2 + |c + id|^2) \\
          &= 2(|z|^2 + |\omega|^2)
        \end{align*}
        So $|z - \omega|^2 + |z + \omega|^2 = 2(|z|^2 + |\omega|^2)$.
      \end{proof}
  \end{enumerate}

\end{document}
