\documentclass[12pt]{article}
\usepackage{amsmath}
\usepackage{amsfonts}
\usepackage[margin=1in]{geometry}

\begin{document}

\title{MAT 25 Homework 1}
\author{Hardy Jones\\
        999397426\\
        Professor Bae}
\date{Fall 2013}

\maketitle

\begin{enumerate}
    \item Let $x$ be a natural number.
    \begin{enumerate}
        \item Prove that $x^2$ is even if and only if $x$ is even. \\

            We must show two statements are true
            \begin{enumerate}
                \item If $x^2$ is even, then $x$ is even.
                \item If $x$ is even, then $x^2$ is even.
            \end{enumerate}

            Proof

            \begin{enumerate}
                \item
                    We can show this by contraposition.
                    So we must show: \\
                    If $x$ is not even, then $x^2$ is not even
                    Assume $x$ is odd.
                    \[\exists r \in \mathbb{Z} : x = 2p + 1\]
                    \[
                        x^2 = x \cdot x = (2p + 1) (2p + 1) = 4p^2 + 4p + 1 =
                        2(2p^2 + 2p) + 1
                    \]
                    Since $\mathbb{Z}$ is closed under addition and multiplication,
                    $2p^2 + 2p \in \mathbb{Z}$.
                    So we can rename it $q$ for clarity.
                    \[2(2p^2 + 2p) + 1 = 2q + 1 = x^2\]
                    So $x^2$ is odd.
                    This is equivalent by contraposition to the statement: \\
                    If $x^2$ is even, then $x$ is even.

                \item
                    Assume $x$ is even.
                    \[\exists p \in \mathbb{Z} : x = 2r\]
                    \[x^2 = x \cdot x = 2r \cdot 2r = 2 \cdot 2r^2\]
                    Since $\mathbb{Z}$ is closed under multiplication,
                    $2r^2 \in \mathbb{Z}$ so we can rename it $s$ for clarity.
                    \[2 \cdot 2r^2 = 2 \cdot s = x^2\]
                    So $x^2$ is even.
            \end{enumerate}

            By $i$ and $ii$, we have shown both cases and thus
            $x^2$ is even if and only if $x$ is even. \\

        \item Prove that $x^2$ is odd if and only if $x$ is odd. \\

            We can take a nearly identical approach to this as we did the previous.
            We must show two statements are true
            \begin{enumerate}
                \item If $x^2$ is odd, then $x$ is odd.
                \item If $x$ is odd, then $x^2$ is odd.
            \end{enumerate}

            Proof

            \begin{enumerate}
                \item
                    We can show this by contraposition.
                    So we must show: \\
                    If $x$ is not odd, then $x^2$ is not odd
                    Assume $x$ is even.
                    \[\exists r \in \mathbb{Z} : x = 2p\]
                    \[
                        x^2 = x \cdot x = 2p \cdot 2p = 2 \cdot 2p^2
                    \]
                    Since $\mathbb{Z}$ is closed under multiplication,
                    $2p^2 \in \mathbb{Z}$.
                    So we can rename it $q$ for clarity.
                    \[2 \cdot 2p^2 = 2q = x^2\]
                    So $x^2$ is even.
                    This is equivalent by contraposition to the statement: \\
                    If $x^2$ is odd, then $x$ is odd.

                \item
                    Assume $x$ is odd.
                    \[\exists p \in \mathbb{Z} : x = 2r\]
                    \[x^2 = x \cdot x = 2r \cdot 2r = 2 \cdot 2r^2\]
                    Since $\mathbb{Z}$ is closed under multiplication,
                    $2r^2 \in \mathbb{Z}$ so we can rename it $s$ for clarity.
                    \[2 \cdot 2r^2 = 2 \cdot s = x^2\]
                    So $x^2$ is odd.
            \end{enumerate}

            By $i$ and $ii$, we have shown both cases and thus
            $x^2$ is even if and only if $x$ is even. \\

        \item Prove that $x^2$ is divisible by $3$ if and only if
              $x$ is divisible by $3$. \\

            Again, we follow the same template
            We must show two statements are true
            \begin{enumerate}
                \item If $x^2$ is divisible by $3$, then $x$ is divisible by $3$.
                \item If $x$ is divisible by $3$, then $x^2$ is divisible by $3$.
            \end{enumerate}

            Proof

            \begin{enumerate}
                \item
                    Here we can prove by contraposition once again.
                    That is:
                    If $x$ is not divisible by $3$,
                    then $x^2$ is not divisible by $3$.
                    Assume $x$ is not divisible by $3$.
                    \[\exists p \in \mathbb{Z} : x = 3p \pm 1\]
                    For the sake of clarity,
                    we handle each instance of $x$ separately.
                    \begin{enumerate}
                        \item
                            \[\exists p \in \mathbb{Z} : x = 3p + 1\]
                            \[
                                x^2 = x \cdot x = (3p + 1)(3p + 1) =
                                9p^2 + 6p + 1 = 3(3p^2 + 2p) + 1
                            \]
                            Since $\mathbb{Z}$ is is closed under
                            addition and multiplication,
                            $3p^2 + 2p \in \mathbb{Z}$ so we can rename it $q$
                            for clarity.
                            \[3(3p^2 + 2p) + 1 = 3q + 1 = x^2\]
                            So $x^2$ is not divisible by $3$.

                        \item
                            \[\exists p \in \mathbb{Z} : x = 3p - 1\]
                            \[
                                x^2 = x \cdot x = (3p - 1)(3p - 1) =
                                9p^2 - 6p + 1 = 3(3p^2 - 2p) + 1
                            \]
                            Since $\mathbb{Z}$ is is closed under
                            addition and multiplication,
                            $3p^2 - 2p \in \mathbb{Z}$ so we can rename it $q$
                            for clarity.
                            \[3(3p^2 - 2p) + 1 = 3q + 1 = x^2\]
                            So $x^2$ is not divisible by $3$.
                    \end{enumerate}
                    From $A$ and $B$ we have shown that if $x$ is not divisible
                    by $3$, then $x^2$ is not divisible by $3$.
                    By contraposition, this is equivalent to the statement:
                    if $x^2$ is divisible by $3$, then $x$ is divisible by $3$.
                \item
                    Assume $x$ is divisible by $3$.
                    \[\exists r \in \mathbb{Z} : x = 3r\]
                    \[x^2 = x \cdot x = 3r \cdot 3r = 3 \cdot 3r^2\]
                    Since $\mathbb{Z}$ is closed under multiplication,
                    $3r^2 \in \mathbb{Z}$ so we can rename it $s$ for clarity.
                    \[3 \cdot 3r^2 = 3 \cdot s = x^2\]
                    So $x^2$ is divisible by $3$.
            \end{enumerate}

            From $i$ and $ii$ we have shown both cases and thus
            $x^2$ is divisible by $3$ if and only if $x$ is divisible by $3$.

    \end{enumerate}

    \item Solve exercise 1.2.1.
        \begin{enumerate}
            \item
                Prove that $\sqrt{3}$ is irrational.
                Does a similar argument work to show
                $\sqrt{6}$ is irrational? \\

            \begin{itemize}

                \item
                    Proof that $\sqrt{3}$ is irrational.

                    We can prove this by contradiction. \\

                    Assume $\sqrt{3}$ is rational.
                    \[
                        \exists p, q \in \mathbb{Z} :
                        \sqrt{3} = \frac{p}{q},
                        p \text{ and } q \text{ are relatively prime}
                    \]
                    \begin{align*}
                        \left(\sqrt{3}\right)^2 &=
                            \left(\frac{p}{q}\right)^2 \\
                        3 &= \frac{p^2}{q^2} \\
                        3q^2 &= p^2 \tag{1}
                    \end{align*}

                    Since $\mathbb{Z}$ is closed under multiplication,
                    $q^2 \in \mathbb{Z}$ we can rename it $r$ for clarity.

                    \begin{align*}
                        3q^2 &= p^2 \\
                        3r &= p^2 \tag{2}
                    \end{align*}

                    This shows that $p^2$ is divisible by $3$.
                    From 1.c, it follows that $p$ is divisible by $3$.

                    We can substitute the value of $p^2$
                    from $(2)$ into $(1)$.

                    \begin{align*}
                        3q^2 &= p^2 \\
                        3q^2 &= \left(3r\right)^2 \\
                        3q^2 &= 9r^2 \\
                        q^2 &= 3r^2
                    \end{align*}

                    Since $\mathbb{Z}$ is closed under multiplication,
                    $r^2 \in \mathbb{Z}$ we can rename it $s$ for clarity.

                    \begin{align*}
                        q^2 &= 3r^2 \\
                        q^2 &= 3s
                    \end{align*}

                    This shows that $q^2$ is divisible by $3$.
                    From 1.c, it follows that $q$ is divisible by $3$.

                    So we have that $p$ and $q$ are both divisible by $3$.
                    But this contradicts part of our assumption:
                    $p$ and $q$ are relatively prime.
                    So our assumption was false,
                    thus $\sqrt{3}$ is not rational.

                \item
                    Applying a similar proof to $\sqrt{6}$
                    would depend on the validity of the following statement: \\

                    $x^2$ is divisible by $6$ if and only if
                    $x$ is divisible by $6$. \\

                    Until that is proven, or disproven,
                    we cannot say one way or the other with our method.

                \end{itemize}

            \item
                Where does the proof of Theorem 1.1.1 break down if we try
                to use it to prove $\sqrt{4}$ is irrational. \\

                The proof of Theorem 1.1.1 breaks down when we attempt to say
                $x^2$ is divisible by $4$ if and only if $x$ is divisible by $4$.
                One such counter example is when $x^2 = 4$.
                In this case, $x^2$ is divisible by $4$.
                However, $x$ (which is $2$) is not divisible by $4$.
                Thus we cannot use Theorem 1.1.1 to prove $\sqrt{4}$ is
                irrational.

        \end{enumerate}
\end{enumerate}

\end{document}
