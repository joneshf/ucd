\documentclass[12pt]{article}
\usepackage{amsmath}
\usepackage{amsfonts}
\usepackage{fullpage}

\begin{document}

\title{MAT 25 Homework 1}
\author{Hardy Jones\\
        999397426}
\date{October 1, 2013}

\maketitle

\begin{enumerate}
    \item Let $x$ be a natural number.
    \begin{enumerate}
        \item Prove that $x^2$ is even if and only if $x$ is even. \\

            We must show two statements are true
            \begin{enumerate}
                \item If $x^2$ is even, then $x$ is even.
                \item If $x$ is even, then $x^2$ is even.
            \end{enumerate}

            Proof

            \begin{enumerate}
                \item
                    We can show this by contraposition.
                    So we must show: \\
                    If $x$ is not even, then $x^2$ is not even
                    Assume $x$ is odd.
                    \[\exists r \in \mathbb{Z} : x = 2p + 1\]
                    \[
                        x^2 = x \cdot x = (2p + 1) (2p + 1) = 4p^2 + 4p + 1 =
                        2(2p^2 + 2p) + 1
                    \]
                    Since $\mathbb{Z}$ is closed under addition and multiplication,
                    $2p^2 + 2p \in \mathbb{Z}$.
                    So we can rename it $q$ for clarity.
                    \[2(2p^2 + 2p) + 1 = 2q + 1 = x^2\]
                    So $x^2$ is odd.
                    This is equivalent by contraposition to the statement: \\
                    If $x^2$ is even, then $x$ is even.

                \item
                    Assume $x$ is even.
                    \[\exists p \in \mathbb{Z} : x = 2r\]
                    \[x^2 = x \cdot x = 2r \cdot 2r = 2 \cdot 2r^2\]
                    Since $\mathbb{Z}$ is closed under multiplication,
                    $2r^2 \in \mathbb{Z}$ so we can rename it $s$ for clarity.
                    \[2 \cdot 2r^2 = 2 \cdot s = x^2\]
                    So $x^2$ is even.
            \end{enumerate}

            By $i$ and $ii$, we have shown both cases and thus
            $x^2$ is even if and only if $x$ is even. \\

        \item Prove that $x^2$ is odd if and only if $x$ is odd. \\

            We can take a nearly identical approach to this as we did the previous.
            We must show two statements are true
            \begin{enumerate}
                \item If $x^2$ is odd, then $x$ is odd.
                \item If $x$ is odd, then $x^2$ is odd.
            \end{enumerate}

            Proof

            \begin{enumerate}
                \item
                    We can show this by contraposition.
                    So we must show: \\
                    If $x$ is not odd, then $x^2$ is not odd
                    Assume $x$ is even.
                    \[\exists r \in \mathbb{Z} : x = 2p + 1\]
                    \[
                        x^2 = x \cdot x = (2p + 1) (2p + 1) = 4p^2 + 4p + 1 =
                        2(2p^2 + 2p) + 1
                    \]
                    Since $\mathbb{Z}$ is closed under addition and multiplication,
                    $2p^2 + 2p \in \mathbb{Z}$.
                    So we can rename it $q$ for clarity.
                    \[2(2p^2 + 2p) + 1 = 2q + 1 = x^2\]
                    So $x^2$ is even.
                    This is equivalent by contraposition to the statement: \\
                    If $x^2$ is odd, then $x$ is odd.

                \item
                    Assume $x$ is odd.
                    \[\exists p \in \mathbb{Z} : x = 2r\]
                    \[x^2 = x \cdot x = 2r \cdot 2r = 2 \cdot 2r^2\]
                    Since $\mathbb{Z}$ is closed under multiplication,
                    $2r^2 \in \mathbb{Z}$ so we can rename it $s$ for clarity.
                    \[2 \cdot 2r^2 = 2 \cdot s = x^2\]
                    So $x^2$ is odd.
            \end{enumerate}

            By $i$ and $ii$, we have shown both cases and thus
            $x^2$ is even if and only if $x$ is even. \\

        \item Prove that $x^2$ is divisible by $3$ if and only if
              $x$ is divisible by $3$.
    \end{enumerate}

    \item Solve exercise 1.2.1.
\end{enumerate}

\end{document}
