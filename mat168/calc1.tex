\documentclass[12pt,letterpaper]{article}

\usepackage[margin=1in]{geometry}
\usepackage[round-mode=figures,round-precision=3,scientific-notation=false]{siunitx}
\usepackage[super]{nth}
\usepackage[title]{appendix}
\usepackage{amsfonts}
\usepackage{amsmath}
\usepackage{amsthm}
\usepackage{cancel}
\usepackage{caption}
\usepackage{color, colortbl}
\usepackage{dcolumn}
\usepackage{enumitem}
\usepackage{fp}
\usepackage{float}
\usepackage{listings}
\usepackage{mathtools}
\usepackage{pgfplots}
\usepackage{subcaption}
\usepackage{systeme}
\usepackage{tikz}
\usepackage{titling}

\usepgfplotslibrary{statistics}

\usetikzlibrary{intersections}
\usetikzlibrary{patterns}

\pgfplotsset{compat=1.8}

\definecolor{Gray}{gray}{0.8}
\newcolumntype{g}{>{\columncolor{Gray}}c}
\newcolumntype{d}{D{.}{.}{-1}}
\DeclarePairedDelimiter\ceil{\lceil}{\rceil}
\DeclarePairedDelimiter\floor{\lfloor}{\rfloor}

\newcommand*\circled[1]{
  \tikz[baseline=(char.base)]{
    \node[shape=circle,draw,inner sep=2pt] (char) {#1};
  }
}

\makeatletter
\renewcommand*\env@matrix[1][*\c@MaxMatrixCols c]{%
  \hskip -\arraycolsep
  \let\@ifnextchar\new@ifnextchar
  \array{#1}}
\makeatother

\newcommand*\constant[1]{
  Each product has a different #1,
  but these are constant values not constrained by anything else in the process.
}

\newcommand*\genericconstraint[2]{
  Each product has a #1 constraint on #2.
}

\newcommand*\maximumconstraint[1]{
  \genericconstraint{maximum}{#1}
}

\newcommand*\minimumconstraint[1]{
  \genericconstraint{minimum}{#1}
}

\newcommand*\genericobjective[2]{
  The objective is to #1 #2.
}

\newcommand*\maximumobjective[1]{
  \genericobjective{maximum}{#1}
}

\newcommand*\minimumobjective[1]{
  \genericobjective{minimum}{#1}
}

% Simplex commands

\newcommand*\continueopt[1]{
  Now, we can continue optimizing since #1 has a positive coefficient.
}

\newcommand*\enterexit[2]{
  So we can let #1 enter and #2 exit.
}

\newcommand*\morerestrictive[2]{
  The more restrictive constraint is that #1 $\leq$ #2, so set #1 $=$ #2.
}

\newcommand*\newdict{
  So we have a new dictionary:
}

\newcommand*\newvalue[1]{
  This gives a new value for #1.
}

\lstdefinelanguage{zimpl}{
  morekeywords={forall,in,maximize,minimize,param,set,subto,sum,var},
  sensitive=true,
  morecomment=[l]{\#},
  morestring=[b]",
}
\lstset{basicstyle=\scriptsize, frame=single, language=zimpl}

\setlength{\droptitle}{-10ex}

\preauthor{\begin{flushright}\large \lineskip 0.5em}
\postauthor{\par\end{flushright}}
\predate{\begin{flushright}\large}
\postdate{\par\end{flushright}}

\title{MAT 168 Calculation 1\vspace{-2ex}}
\author{Hardy Jones\\
        999397426\\
        Professor K\"{o}ppe\vspace{-2ex}}
\date{Spring 2015}

\begin{document}
  \maketitle

  \begin{enumerate}
    \item [2.1]
      We first start by rewriting as a dictionary:

      \begin{alignat*}{17}
        \text{maximize}   \quad & \zeta & {}={} & 0 & {}+{} & 6 & x_1 & {}+{} & 8 & x_2 & {}+{} & 5 & x_3 & {}+{} & 9 & x_4 \\
        \text{subject to} \quad & x_5   & {}={} & 5 & {}-{} & 2 & x_1 & {}-{} &   & x_2 & {}-{} &   & x_3 & {}-{} & 3 & x_4 \\
                                & x_6   & {}={} & 3 & {}-{} &   & x_1 & {}-{} & 3 & x_2 & {}-{} &   & x_3 & {}-{} & 2 & x_4 \\
      \end{alignat*}

      Then we can begin by entering with the largest variable, $x_4$.

      We look at the constraints and see:
      \begin{itemize}
        \item $x_5 \implies x_4 \leq \frac{5}{3}$
        \item $x_6 \implies x_4 \leq \frac{3}{2}$
      \end{itemize}

      \morerestrictive{$x_4$}{$\frac{3}{2}$}

      \enterexit{$x_4$}{$x_6$}

      \[
        x_4 = \frac{3}{2} - \frac{1}{2}x_1 - \frac{3}{2}x_2 - \frac{1}{2}x_3 - \frac{1}{2}x_6
      \]

      \newvalue{$x_5$}

      \[
        x_5 = \frac{1}{2} - \frac{1}{2}x_1 + \frac{7}{2}x_2 + \frac{1}{2}x_3 + \frac{3}{2}x_6
      \]

      \newvalue{$\zeta$}

      \[
        \zeta = \frac{27}{2} + \frac{3}{2}x_1 - \frac{11}{2}x_2 + \frac{1}{2}x_3 - \frac{9}{2}x_6
      \]

      \newdict

      \begin{alignat*}{17}
        \text{maximize}   \quad & \zeta & {}={} & \frac{27}{2} & {}+{} & \frac{3}{2} & x_1 & {}-{} & \frac{11}{2} & x_2 & {}+{} & \frac{1}{2} & x_3 & {}-{} & \frac{9}{2} & x_6 \\
        \text{subject to} \quad & x_5   & {}={} & \frac{1}{2}  & {}-{} & \frac{1}{2} & x_1 & {}+{} & \frac{7}{2}  & x_2 & {}+{} & \frac{1}{2} & x_3 & {}+{} & \frac{3}{2} & x_6 \\
                                & x_4   & {}={} & \frac{3}{2}  & {}-{} & \frac{1}{2} & x_1 & {}-{} & \frac{3}{2}  & x_2 & {}-{} & \frac{1}{2} & x_3 & {}-{} & \frac{1}{2} & x_6 \\
      \end{alignat*}

      \continueopt{$x_1$}

      We look at the constraints and see:
      \begin{itemize}
        \item $x_5 \implies x_1 \leq 1$
        \item $x_4 \implies x_1 \leq 3$
      \end{itemize}

      \morerestrictive{$x_1$}{$1$}

      \enterexit{$x_1$}{$x_5$}

      \[
        x_1 = 1 + 7x_2 + x_3 + 3x_6 - 2x_5
      \]

      \newvalue{$x_4$}

      \[
        x_4 = 1 - 5x_2 - x_3 - 2x_6 + x_5
      \]

      \newvalue{$\zeta$}

      \[
        \zeta = 15 + 5x_2 + 2x_3 - 3x_5
      \]

      \newdict

      \begin{alignat*}{17}
        \text{maximize}   \quad & \zeta & {}={} & 15 & {}+{} & 5 & x_2 & {}+{} & 2 & x_3 &       &   &     & {}-{} & 3 & x_5 \\
        \text{subject to} \quad & x_1   & {}={} & 1  & {}+{} & 7 & x_2 & {}+{} &   & x_3 & {}+{} & 3 & x_6 & {}-{} & 2 & x_5 \\
                                & x_4   & {}={} & 1  & {}-{} & 5 & x_2 & {}-{} &   & x_3 & {}-{} & 2 & x_6 & {}+{} &   & x_5 \\
      \end{alignat*}

      \continueopt{$x_2$}

      We look at the constraints and see:
      \begin{itemize}
        \item $x_1 \implies x_2 \geq 0$
        \item $x_4 \implies x_2 \leq \frac{1}{5}$
      \end{itemize}

      \morerestrictive{$x_2$}{$\frac{1}{5}$}

      \enterexit{$x_2$}{$x_4$}

      \[
        x_2 = \frac{1}{5} - \frac{1}{5}x_4 - \frac{1}{5}x_3 - \frac{2}{5}x_6 + \frac{1}{5}x_5
      \]

      \newvalue{$x_1$}

      \[
        x_1 = \frac{12}{5} - \frac{7}{5}x_4 - \frac{2}{5}x_3 + \frac{1}{5}x_6 - \frac{3}{5}x_5
      \]

      \newvalue{$\zeta$}

      \[
        \zeta = 16 - x_4 + x_3 - 2x_6 - 2x_5
      \]

      \newdict

      \begin{alignat*}{17}
        \text{maximize}   \quad & \zeta & {}={} & 16           & {}-{} &             & x_4 & {}+{} &             & x_3 & {}-{} & 2           & x_6 & {}-{} & 2            & x_5 \\
        \text{subject to} \quad & x_1   & {}={} & \frac{12}{5} & {}-{} & \frac{7}{5} & x_4 & {}-{} & \frac{2}{5} & x_3 & {}+{} & \frac{1}{5} & x_6 & {}-{} & \frac{3}{5} & x_5 \\
                                & x_2   & {}={} & \frac{1}{5}  & {}-{} & \frac{1}{5} & x_4 & {}-{} & \frac{1}{5} & x_3 & {}-{} & \frac{2}{5} & x_6 & {}+{} & \frac{1}{5} & x_5 \\
      \end{alignat*}

      \continueopt{$x_3$}

      We look at the constraints and see:
      \begin{itemize}
        \item $x_1 \implies x_3 \leq 6$
        \item $x_2 \implies x_3 \leq 1$
      \end{itemize}

      \morerestrictive{$x_3$}{$1$}

      \enterexit{$x_3$}{$x_2$}

      \[
        x_3 = 1 - x_4 - 5x_2 - 2x_6 + x_5
      \]

      \newvalue{$x_1$}

      \[
        x_1 = 2 - x_4 + 2x_2 + x_6 - x_5
      \]

      \newvalue{$\zeta$}

      \[
        \zeta = 17 - 2x_4 - 5x_2 - 4x_6 - x_5
      \]

      \newdict

      \begin{alignat*}{17}
        \text{maximize}   \quad & \zeta & {}={} & 17 & {}-{} & 2 & x_4 & {}-{} & 5 & x_2 & {}-{} & 4 & x_6 & {}-{} &  & x_5 \\
        \text{subject to} \quad & x_1   & {}={} & 2  & {}-{} &   & x_4 & {}+{} & 2 & x_2 & {}+{} &   & x_6 & {}-{} &  & x_5 \\
                                & x_3   & {}={} & 1  & {}-{} &   & x_4 & {}-{} & 5 & x_2 & {}-{} & 2 & x_6 & {}+{} &  & x_5 \\
      \end{alignat*}

      Since we have no more optimizable variables
      (all variable coefficients of $\zeta$ are non-positive),
      we can no longer maximize $\zeta$.

      Then we have an optimal solution with $x_1 = 2, x_3 = 1, x_2 = x_4 = x_5 = x_6 = 0$,
      and value 17.
    \item [2.2]
      We first start by rewriting as a dictionary:

      \begin{alignat*}{17}
        \text{maximize}   \quad & \zeta & {}={} & 0 & {}+{} & 2 & x_1 & {}+{} &   & x_2 \\
        \text{subject to} \quad & x_3   & {}={} & 4 & {}-{} & 2 & x_1 & {}-{} &   & x_2 \\
                                & x_4   & {}={} & 3 & {}-{} & 2 & x_1 & {}-{} & 3 & x_2 \\
                                & x_5   & {}={} & 5 & {}-{} & 4 & x_1 & {}-{} &   & x_2 \\
                                & x_6   & {}={} & 1 & {}-{} &   & x_1 & {}-{} & 5 & x_2 \\
      \end{alignat*}

      Then we can begin by entering with the largest variable, $x_1$.

      We look at the constraints and see:
      \begin{itemize}
        \item $x_3 \implies x_1 \leq 2$
        \item $x_4 \implies x_1 \leq \frac{3}{2}$
        \item $x_5 \implies x_1 \leq \frac{5}{4}$
        \item $x_6 \implies x_1 \leq 1$
      \end{itemize}

      \morerestrictive{$x_1$}{$1$}

      \enterexit{$x_1$}{$x_6$}

      \[
        x_1 = 1 + 5x_2 - x_6
      \]

      \newvalue{$x_3$}

      \[
        x_3 = 2 - 9x_2 + 2x_6
      \]

      \newvalue{$x_4$}

      \[
        x_4 = 1 + 7x_2 + 2x_6
      \]

      \newvalue{$x_5$}

      \[
        x_5 = 1 + 19x_2 + 4x_6
      \]

      \newvalue{$\zeta$}

      \[
        \zeta = 2 - 9x_2 - 2x_6
      \]

      \newdict

      \begin{alignat*}{17}
        \text{maximize}   \quad & \zeta & {}={} & 2 & {}-{} & 9  & x_2 & {}-{} & 2 & x_6 \\
        \text{subject to} \quad & x_3   & {}={} & 2 & {}-{} & 9  & x_2 & {}+{} & 2 & x_6 \\
                                & x_4   & {}={} & 1 & {}+{} & 7  & x_2 & {}+{} & 2 & x_6 \\
                                & x_5   & {}={} & 1 & {}+{} & 19 & x_2 & {}+{} & 4 & x_6 \\
                                & x_1   & {}={} & 1 & {}+{} & 5  & x_2 & {}-{} &   & x_6 \\
      \end{alignat*}

      Since we have no more optimizable variables
      (all variable coefficients of $\zeta$ are non-positive),
      we can no longer maximize $\zeta$.

      Then we have an optimal solution with $x_1 = 1, x_2 = x_3 = x_4 = x_5 = x_6 = 0$,
      and value 2.
    \item [2.3]
  \end{enumerate}
\end{document}
