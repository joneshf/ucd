\documentclass[12pt,letterpaper]{article}

\usepackage[margin=1in]{geometry}
\usepackage[group-separator={,},round-mode=figures,round-precision=3,scientific-notation=false]{siunitx}
\usepackage[super]{nth}
\usepackage[title]{appendix}
\usepackage{amsfonts}
\usepackage{amsmath}
\usepackage{amsthm}
\usepackage{cancel}
\usepackage{caption}
\usepackage{color, colortbl}
\usepackage{dcolumn}
\usepackage{enumitem}
\usepackage{fp}
\usepackage{float}
\usepackage{listings}
\usepackage{mathtools}
\usepackage{pgfplots}
\usepackage{subcaption}
\usepackage{systeme}
\usepackage{tikz}
\usepackage{titling}

\usepgfplotslibrary{statistics}

\usetikzlibrary{intersections}
\usetikzlibrary{patterns}

\pgfplotsset{compat=1.8}

\definecolor{Gray}{gray}{0.8}
\newcolumntype{g}{>{\columncolor{Gray}}c}
\newcolumntype{d}{D{.}{.}{-1}}
\DeclarePairedDelimiter\ceil{\lceil}{\rceil}
\DeclarePairedDelimiter\floor{\lfloor}{\rfloor}

\newcommand*\circled[1]{%
  \tikz[baseline=(char.base)]{
    \node[shape=circle,draw,inner sep=2pt] (char) {#1};
  }
}

\makeatletter
\renewcommand*\env@matrix[1][*\c@MaxMatrixCols c]{%
  \hskip -\arraycolsep
  \let\@ifnextchar\new@ifnextchar
  \array{#1}}
\makeatother

\newcommand*\constant[1]{%
  Each product has a different #1,
  but these are constant values not constrained by anything else in the process.
}

\newcommand*\genericconstraint[2]{%
  Each product has a #1 constraint on #2.
}

\newcommand*\maximumconstraint[1]{%
  \genericconstraint{maximum}{#1}
}

\newcommand*\minimumconstraint[1]{%
  \genericconstraint{minimum}{#1}
}

\newcommand*\genericobjective[2]{%
  The objective is to #1 #2.
}

\newcommand*\maximumobjective[1]{%
  \genericobjective{maximum}{#1}
}

\newcommand*\minimumobjective[1]{%
  \genericobjective{minimum}{#1}
}

\newcommand*\buck[1]{%
  \$\num{#1}%
}

\lstdefinelanguage{zimpl}{
  morekeywords={forall,in,maximize,minimize,param,set,subto,sum,var},
  sensitive=true,
  morecomment=[l]{\#},
  morestring=[b]",
}
\lstset{basicstyle=\scriptsize, frame=single, language=zimpl}

\setlength{\droptitle}{-10ex}

\preauthor{\begin{flushright}\large \lineskip 0.5em}
\postauthor{\par\end{flushright}}
\predate{\begin{flushright}\large}
\postdate{\par\end{flushright}}

\title{MAT 168 Modeling 3\vspace{-2ex}}
\author{Hardy Jones\\
        999397426\\
        Professor K\"{o}ppe\vspace{-2ex}}
\date{Spring 2015}

\begin{document}
  \maketitle

  \begin{enumerate}
    \item [r4-25)]

      Let's try to understand what the problem is saying.

      Calving is the process of a cow giving birth.
      Cows calve in a certain month, $c$,
      with a maintenance cost, $m_c$.

      The milk production of each cow is based on when it calved, $c$, and the month of the year it is, $d$.
      This production can be estimated by $p_{dc}$.
      For example, if a cow calved in month 3, then its milk production will be close to zero in month 1.
      The production is close to zero because, according to the problem,
      milk production drops almost to zero by the \nth{10} month of calving.
      Since there are 12 months in a year, $c, d \in \{1, 2, \dots, 12\}$.

      We use the decision variables:

      \begin{itemize}
        \item $x_c$ := Number of cows calving in month $c$. These are non negative integers, since it doesn't make sense to have a fractional cow calving.
        \item $y_d$ := Pounds of excess milk produced in month $d$.
      \end{itemize}

      Each month, $d$, has a certain demand for milk, $r_d$.
      So we can compute the excess amount of milk for each month as
      $y_d = \left(\sum\limits_{c = 1}^{12} x_c p_{dc}\right) - r_d$
      $\implies$
      $r_d = \left(\sum\limits_{c = 1}^{12} x_c p_{dc}\right) - y_d$.
      And this equation can be modeled as two inequalities.

      Any excess milk produced must be sold at a loss of $b$ per pound.

      We're asked to minimize the cost of a calving schedule,
      which is the same as maximizing the negative cost.
      The cost for some month $i$ is $m_i x_i + b y_i$

      We can now model this in standard form (after a little algebraic manipulation):

      \begin{alignat*}{9}
        \text{maximize}   \quad \mathrlap{-\sum_{i = 1}^{12}m_i x_i + b y_i}  \\
        \text{subject to} \quad &       &\left(\sum_{c = 1}^{12} x_c p_{dc}\right) & {}-{} & y_d & {}\leq{} &       & r_d            && \text{, for } d \in \{1, 2, \dots, 12\} \\
                                & {}-{} &\left(\sum_{c = 1}^{12} x_c p_{dc}\right) & {}+{} & y_d & {}\leq{} & {}-{} & r_d            && \text{, for } d \in \{1, 2, \dots, 12\} \\
                                &       &                                          &       &     &          &       & \mathclap{x_c} && \in  \{0, 1, 2, \dots\} \\
                                &       &                                          &       &     &          &       & \mathclap{y_d} && \in  [0, \infty) \\
      \end{alignat*}

    \pagebreak

    \item [r11-35)]

      Let's try to understand what this problem is saying.

      AFL is planning their production and distribution network.
      They produce cases at plants, $i \in \{1, 2, \dots, 7\}$.
      The cases are transported to warehouses, $j \in \{1, 2, \dots, 13\}$.
      From warehouses, the cases are transported to customer regions, $k \in \{1, 2, \dots, 219\}$.

      Each plant has a fixed construction cost \buck{50000000}.
      Each warehouse has a fixed construction cost \buck{12000000}.

      Each plant can produce up to \num{30000} cases per year.
      Each warehouse can handle up to \num{10000} cases per year.

      Cases are transported via train from plants to warehouses, the cost per case is $r_{ij}$.
      Cases are transported via truck from warehouses to regions, the cost per case is $t_{jk}$.

      Each region has a certain demand for the number of cases, $d_k$.

      We use the decision variables:

      \begin{itemize}
        \item $x_{ijk}$ := Thousands of cases transported from plant $i$ to warehouse $j$ to region $k$.
        \item $y_i$ :=
          $
            \begin{cases*}
              1 & if plant $i$ is opened \\
              0 & otherwise \\
            \end{cases*}
          $
        \item $w_i$ :=
          $
            \begin{cases*}
              1 & if warehouse $i$ is opened \\
              0 & otherwise \\
            \end{cases*}
          $
      \end{itemize}

      So, for all plants $i$, warehouses $j$, and regions $k$, we need:
      $x_{ijk} \leq 30 y_i$ (since we cannot produce more than \num{30000} cases from plant $i$), $x_{ijk} \leq 10 w_j$ (since we cannot store more than \num{10000} cases at warehouse $j$), and $1000 x_{ijk} \geq d_k$ (since we transport cases in increments of \num{1000} to region $k$, and there is no penalty for exceeding demand).

      We're asked to model this network to meet all demands.
      We assume AFL wants to minimize the cost, or maximize the negative cost.
      We compute the shipping cost for all plants $i$, warehouses $j$, and regions $k$ as
      $\num{1000}x_{ijk}y_iw_j\left(r_{ij} + t_{jk}\right)$, since we only transport if both plant $i$ and warehouse $j$ are opened.

      We can now model this in standard form (after a little algebraic manipulation):
      \[
      \begin{alignedat}{7}
        \text{maximize}   \quad & {-1000\left[\num{50000} \sum_{i = 1}^{7} y_i + \num{12000} \sum_{j = 1}^{13} w_j + \sum_{i = 1}^{7} \sum_{j = 1}^{13} \sum_{k = 1}^{219} x_{ijk}y_iw_j\left(r_{ij} + t_{jk}\right)\right]} \span \span \span \span \span \span \\
        \text{subject to} \quad &       & \sum_{j = 1}^{13} \sum_{k = 1}^{219} x_{ijk} & {}\leq{} & 30 & y_i                     && \text{, for } i \in \{1, 2, \dots, 7\} \\
                                &       & \sum_{i = 1}^{7}  \sum_{k = 1}^{219} x_{ijk} & {}\leq{} & 10 & w_j                     && \text{, for } j \in \{1, 2, \dots, 13\} \\
                                & -1000 & \sum_{i = 1}^{7}  \sum_{j = 1}^{13}  x_{ijk} & {}\leq{} &  - & d_k                     && \text{, for } k \in \{1, 2, \dots, 219\} \\
                                &       &                                              &          &    & \mathclap{x_{ijk}, d_k} && \in  \{0, 1, 2, \dots\} \\
                                &       &                                              &          &    & \mathclap{y_i, w_j}     && \in  \{0, 1\} \\
      \end{alignedat}
      \]
  \end{enumerate}
\end{document}
