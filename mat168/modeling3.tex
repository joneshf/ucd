\documentclass[12pt,letterpaper]{article}

\usepackage[margin=1in]{geometry}
\usepackage[round-mode=figures,round-precision=3,scientific-notation=false]{siunitx}
\usepackage[super]{nth}
\usepackage[title]{appendix}
\usepackage{amsfonts}
\usepackage{amsmath}
\usepackage{amsthm}
\usepackage{cancel}
\usepackage{caption}
\usepackage{color, colortbl}
\usepackage{dcolumn}
\usepackage{enumitem}
\usepackage{fp}
\usepackage{float}
\usepackage{listings}
\usepackage{mathtools}
\usepackage{pgfplots}
\usepackage{subcaption}
\usepackage{systeme}
\usepackage{tikz}
\usepackage{titling}

\usepgfplotslibrary{statistics}

\usetikzlibrary{intersections}
\usetikzlibrary{patterns}

\pgfplotsset{compat=1.8}

\definecolor{Gray}{gray}{0.8}
\newcolumntype{g}{>{\columncolor{Gray}}c}
\newcolumntype{d}{D{.}{.}{-1}}
\DeclarePairedDelimiter\ceil{\lceil}{\rceil}
\DeclarePairedDelimiter\floor{\lfloor}{\rfloor}

\newcommand*\circled[1]{
  \tikz[baseline=(char.base)]{
    \node[shape=circle,draw,inner sep=2pt] (char) {#1};
  }
}

\makeatletter
\renewcommand*\env@matrix[1][*\c@MaxMatrixCols c]{%
  \hskip -\arraycolsep
  \let\@ifnextchar\new@ifnextchar
  \array{#1}}
\makeatother

\newcommand*\constant[1]{
  Each product has a different #1,
  but these are constant values not constrained by anything else in the process.
}

\newcommand*\genericconstraint[2]{
  Each product has a #1 constraint on #2.
}

\newcommand*\maximumconstraint[1]{
  \genericconstraint{maximum}{#1}
}

\newcommand*\minimumconstraint[1]{
  \genericconstraint{minimum}{#1}
}

\newcommand*\genericobjective[2]{
  The objective is to #1 #2.
}

\newcommand*\maximumobjective[1]{
  \genericobjective{maximum}{#1}
}

\newcommand*\minimumobjective[1]{
  \genericobjective{minimum}{#1}
}

\lstdefinelanguage{zimpl}{
  morekeywords={forall,in,maximize,minimize,param,set,subto,sum,var},
  sensitive=true,
  morecomment=[l]{\#},
  morestring=[b]",
}
\lstset{basicstyle=\scriptsize, frame=single, language=zimpl}

\setlength{\droptitle}{-10ex}

\preauthor{\begin{flushright}\large \lineskip 0.5em}
\postauthor{\par\end{flushright}}
\predate{\begin{flushright}\large}
\postdate{\par\end{flushright}}

\title{MAT 168 Modeling 3\vspace{-2ex}}
\author{Hardy Jones\\
        999397426\\
        Professor K\"{o}ppe\vspace{-2ex}}
\date{Spring 2015}

\begin{document}
  \maketitle

  \begin{enumerate}
    \item [r4-25)]

      Let's try to understand what the problem is saying.

      Calving is the process of a cow giving birth.
      Cows calve in a certain month, $c$.

      There is a maintenance cost for a cow calving in a certain month, $m_c$.

      The milk production of each cow is based on when it calved, $c$ and the month of the year it is, $d$.
      This production can be estimated by $p_{dc}$.
      For example, if a cow calved in month 3, then its milk production will be close to zero in month 1.
      The production is close to zero because, according to the problem,
      milk production drops almost to zero by the \nth{10} month of calving.

      Each month, $d$ has a certain demand for milk, $r_d$.
      So we can compute the excess amount of milk for each month $d$ as
      $y_d = \left(\sum\limits_{c = 1}^{12} x_c p_{cd}\right) - r_d$.
      Any excess milk produced must be sold at a loss of $b$ per pound.

      We're asked to minimize cost of a calving schedule,
      which is the same as maximizing the negative cost.
      The cost for some month $i$ is $m_i x_i + b y_i$

      We use the decision variables:

      \begin{itemize}
        \item $x_c$ := Number of cows calving in month $c$. These are non negative integers, since it doesn't make sense to have a fractional cow calving.
        \item $y_d$ := Lbs. of excess milk produced in month $d$.
      \end{itemize}

      We can now model this in standard form (after a little algebraic manipulation):

      \begin{alignat*}{5}
        \text{maximize}   \quad \mathrlap{-\sum_{i = 1}^{12}m_i x_i + b y_i}  \\
        \text{subject to} \quad &       &\left(\sum_{c = 1}^{12} x_c p_{cd}\right) & {}-{} & y_d & {}\leq{} &       & r_d            && \text{, for } d \in \{1, 2, \dots, 12\} \\
                                & {}-{} &\left(\sum_{c = 1}^{12} x_c p_{cd}\right) & {}+{} & y_d & {}\leq{} & {}-{} & r_d            && \text{, for } d \in \{1, 2, \dots, 12\} \\
                                &       &                                          &   &     &          &       & \mathclap{x_c} && \in  \{0, 1, 2, \dots\} \\
                                &       &                                          &   &     &          &       & \mathclap{y_d} && \in  [0, \infty) \\
      \end{alignat*}

      We should note that this problem feels incomplete since it doesn't mention the number of cows that are available to calve, among other things.
  \end{enumerate}
\end{document}
