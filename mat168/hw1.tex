\documentclass[12pt,letterpaper]{article}

\usepackage[margin=1in]{geometry}
\usepackage[round-mode=figures,round-precision=3,scientific-notation=false]{siunitx}
\usepackage[super]{nth}
\usepackage[title]{appendix}
\usepackage{amsfonts}
\usepackage{amsmath}
\usepackage{amsthm}
\usepackage{cancel}
\usepackage{caption}
\usepackage{color, colortbl}
\usepackage{dcolumn}
\usepackage{enumitem}
\usepackage{fp}
\usepackage{float}
\usepackage{listings}
\usepackage{mathtools}
\usepackage{pgfplots}
\usepackage{subcaption}
\usepackage{systeme}
\usepackage{tikz}
\usepackage{titling}

\usepgfplotslibrary{statistics}

\usetikzlibrary{intersections}
\usetikzlibrary{patterns}

\pgfplotsset{compat=1.8}

\definecolor{Gray}{gray}{0.8}
\newcolumntype{g}{>{\columncolor{Gray}}c}
\newcolumntype{d}{D{.}{.}{-1}}
\DeclarePairedDelimiter\ceil{\lceil}{\rceil}
\DeclarePairedDelimiter\floor{\lfloor}{\rfloor}

\newcommand*\circled[1]{
  \tikz[baseline=(char.base)]{
    \node[shape=circle,draw,inner sep=2pt] (char) {#1};
  }
}

\makeatletter
\renewcommand*\env@matrix[1][*\c@MaxMatrixCols c]{%
  \hskip -\arraycolsep
  \let\@ifnextchar\new@ifnextchar
  \array{#1}}
\makeatother

\newcommand*\constant[1]{
  Each product has a different #1,
  but these are constant values not constrained by anything else in the process.
}

\newcommand*\genericconstraint[2]{
  Each product has a #1 constraint on #2.
}

\newcommand*\maximumconstraint[1]{
  \genericconstraint{maximum}{#1}
}

\newcommand*\minimumconstraint[1]{
  \genericconstraint{minimum}{#1}
}

\newcommand*\genericobjective[2]{
  The objective is to #1 #2.
}

\newcommand*\maximumobjective[1]{
  \genericobjective{maximum}{#1}
}

\newcommand*\minimumobjective[1]{
  \genericobjective{minimum}{#1}
}

\lstdefinelanguage{zimpl}{
  morekeywords={forall,in,maximize,minimize,param,set,subto,sum,var},
  sensitive=true,
  morecomment=[l]{\#},
  morestring=[b]",
}
\lstset{basicstyle=\scriptsize, frame=single, language=zimpl}

\setlength{\droptitle}{-10ex}

\preauthor{\begin{flushright}\large \lineskip 0.5em}
\postauthor{\par\end{flushright}}
\predate{\begin{flushright}\large}
\postdate{\par\end{flushright}}

\title{MAT 168 HW 1\vspace{-2ex}}
\author{Hardy Jones\\
        999397426\\
        Professor K\"{o}eppe\vspace{-2ex}}
\date{Spring 2015}

\begin{document}
  \maketitle

  \begin{enumerate}
    \item [1.1]

      The first thing we should do is try to understand the domain.
      The steel company wants to know how many hours to allocate to each job.
      Since the description does not mention that each job must be a full hour--%
      or some minimum fraction of an hour--%
      we assume that hours can be real valued.
      Since the hours are what the problem explicitly asks for,
      these can be our decision variables.

      Say $x_1, x_2$. Next we look at the other information in the statement.
      \begin{itemize}
        \item
          \constant{rate}

          We have
          \begin{itemize}
            \item $r_1 = 200$
            \item $r_2 = 140$
          \end{itemize}

        \item
          \constant{profit}

          We have
          \begin{itemize}
            \item $p_1 = 25$
            \item $p_2 = 30$
          \end{itemize}
        \item
          \maximumconstraint{the weight that can be produced}
          Since the problem does not state otherwise,
          we assume this can be a real value.
        \item
          \maximumobjective{profit}

          We can find the profit of either product by taking the expression
          \[
            p_i \cdot r_i \cdot x_i, \text{ for } i \in \{1, 2\}
          \]
      \end{itemize}

      With this information, we can now formalize the problem.

      \begin{alignat*}{7}
        \text{maximize}   \quad \rlap{$5000 x_1 + 4200 x_2$}                                        \\
        \text{subject to} \quad & 200 & x_1 &       &     &                   && \leq 6000          \\
                                &     &     &       & 140 & x_2               && \leq 4000          \\
                                &     & x_1 & {}+{} &     & x_2               && \leq 40            \\
                                &     &     &       &     & \clap{$x_1, x_2$} && \geq 0             \\
                                &     &     &       &     & \clap{$x_1, x_2$} && \in  \mathbb{R}    \\
      \end{alignat*}

      It turns out what we have modeled is an instance of the fractional knapsack problem.
      We can see it better if we simplify the first two constraints to
      $x_1 \leq 30, x_2 \leq \frac{200}{7}$, respectively.

      Here, our ``knapsack'' is the number of hours to schedule.
      The maximum ``weight'' of the ``knapsack'' is the 40 hours available for the next week.
      The maximum amount of each ``item'' we want to take is the first two constraints.
      Finally, we have the amount of profit per ``weight'' (in this case hours).

      Importantly, the decision variables in the problem are positive real values.
      The variables being real values allows us to formalize the model as a fractional knapsack rather than 0-1 knapsack or some other version.

      Since we already know that fractional knapsack has optimal substructure and
      satisfies the greedy choice property,
      we can use a greedy approach to solve this.

      Let's catalog the steps.

      \begin{itemize}
        \item
          We have a system of constraints

          \sysdelim..
          \systeme{ 200x_1 \leq 6000
                  , 140x_2 \leq 4000
                  , x_1 + x_2 \leq 40
                  , x_1 \geq 0
                  , x_2 \geq 0
                  }

          Bands are worth more from the profit perspective, so we take as many as possible.
          Using the first constraint, this gives us a direct number for $x_1$.

          Namely $200 x_1 \leq 6000 \implies x_1 \leq 30$.

          This choice does not violate any other constraints,
          so we choose the maximum possible value $x_1$ can take which is 30.

        \item
          Our new system is

          \systeme{ 140x_2 \leq 4000
                  , x_2 \leq 10
                  , x_2 \geq 0
                  }

          So we take the maximum value possible for $x_2$ which is 10.
      \end{itemize}

      The profit can now be computed and we end up with the following result:

      With a choice of 30 hours making Bands and 10 hours making Coils,
      the company can make an optimally maximized profit of $\$192,000$.

      We can verify this by modeling it with ZIMPL:

      \lstinputlisting{hw1_steel_2.zpl}

      And then solving with scip:

      \lstinputlisting{hw1_steel_scip.txt}

    \item [1.2]

      We use as our decision variables the number of tickets of each flight combination,
      say $x_{ij}, i \in \{1, 2, 3\}, j \in \{1, 2, 3\}$.
      Where $i$ is the fare, and $j$ is the flight.

      Now we look at the rest of the information:

      \begin{itemize}
        \item
          Each flight has a maximum constraint on the number of passengers since the plane can only hold 30 people.

        \item
          Each flight has only so many seats available for each fare,
          say $s_{ij}, i \in \{1, 2, 3\}, j \in \{1, 2, 3\}$.

        \item
          Each flight fare has a ticket price associated with it,
          say $p_{ij}, i \in \{1, 2, 3\}, j \in \{1, 2, 3\}$.

        \item
          Our objective function is to maximize revenue for each flight.
      \end{itemize}

      So we can formalize this model as:

      \begin{alignat*}{4}
        \text{maximize}   \quad \mathrlap{\sum_{i = 1}^{3} \sum_{j = 1}^{3} p_{ij}x_{ij}}                                        \\
        \text{subject to} \quad & \forall i \in \{1, 2, 3\}, j \in \{1, 2, 3\}, \quad & x_{ij}                && \leq s_{ij}     \\
                                & \forall j \in \{1, 2, 3\},                    \quad & \sum_{i = 1}^3 x_{ij} && \leq 30         \\
                                & \forall i \in \{1, 2, 3\}, j \in \{1, 2, 3\}, \quad & x_{ij}                && \geq 0          \\
                                & \forall i \in \{1, 2, 3\}, j \in \{1, 2, 3\}, \quad & x_{ij}                && \in  \mathbb{Z} \\
      \end{alignat*}

      We model this in ZIMPL in order to arrive at a solution.

      \lstinputlisting{hw1_airline.zpl}

      Now we can use scip to find the solution:

      \lstinputlisting{hw1_airline_scip.txt}

      So we see that the optimal revenue is \$14,710,
      assuming the appropriate number of tickets are sold.
    \item [2-2]
      \begin{enumerate}
        \item

          As stated in the problem,
          we use the decision variables
          \begin{itemize}
            \item $x_1$ := dollars invested in domestic stocks (in millions)
            \item $x_2$ := dollars invested in foreign stocks (in millions)
          \end{itemize}

          And we can read the five constraints as follows:

          \begin{enumerate}
            \item The sum of $x_1$ and $x_2$ can be at most 12.
            \item $x_1$ can be at most 10.
            \item $x_2$ can be at most 7.
            \item $x_1$ must be at least twice as much as $x_2$.
            \item $x_2$ must be at least twice as much as $x_1$.
          \end{enumerate}

          N.B. It might seem like we need an additional constraint that $x_1, x_2 \geq 0$,
          however, the last two constraints together ensure this fact.

          We want to maximize the returns for domestic and foreign stocks,
          which are 0.11 and 0.17, respectively.

          Now we can create a model:

          \begin{alignat*}{4}
            \text{maximize}   \quad \mathrlap{0.11 x_1 + 0.17 x_2}                                 \\
            \text{subject to} \quad &   & x_1 & {}+{} &   & x_2                 && \leq 12         \\
                                    &   & x_1 &       &   &                     && \leq 10         \\
                                    &   &     &       &   & x_2                 && \leq  7         \\
                                    & 2 & x_1 & {}-{} &   & x_2                 && \geq  0         \\
                                    & - & x_1 & {}+{} & 2 & x_2                 && \geq  0         \\
                                    &   &     &       &   & \mathclap{x_1, x_2} && \in  \mathbb{R} \\
          \end{alignat*}

        \item
          We can formulate this model in ZIMPL:

          \lstinputlisting{hw1_invest.zpl}

          \pagebreak

          And solve it with SCIP:

          \lstinputlisting{hw1_invest_scip.txt}

          We see that the optimal solution is $x_1 = 5, x_2 = 7$,
          which returns \$1.74 million.

        \item
          \begin{figure}
          \centering
            \begin{minipage}{.5\textwidth}
              \centering

              \begin{tikzpicture}
                \begin{axis}[axis on top,smooth,
                    axis line style=very thick,
                    axis x line=bottom,
                    axis y line=left,
                    ymin=0,ymax=15,
                    xmin=0,xmax=15,
                    xlabel=$x_1$, ylabel=$x_2$,
                    grid=major,
                  ]
                  \addplot[name path global=c1,very thick, domain=0:15]{12 - x}
                    node[above, pos=0.1, sloped, style={font=\tiny}] {$x_1 + x_2 \leq 12$}
                  ;
                  \addplot[name path global=c2,very thick, domain=0:15]
                    coordinates { (10, 0) (10, 15) }
                    node[right, pos=0.7, style={font=\tiny}] {$x_1 \leq 10$}
                  ;
                  \addplot[name path global=c3,very thick, domain=0:15]
                    coordinates { (0, 7) (15, 7) }
                    node[above, pos=0.6, sloped, style={font=\tiny}] {$x_2 \leq 7$}
                  ;
                  \addplot[name path global=c4,very thick, domain=0:15]{2 * x}
                    node[above, pos=0.4, sloped, style={font=\tiny}] {$2x_1 - x_2 \geq 0$}
                  ;
                  \addplot[name path global=c5,very thick, domain=0:15]{0.5 * x}
                    node[below, pos=0.8, sloped, style={font=\tiny}] {$-x_1 + 2x_2 \geq 0$}
                  ;
                  \foreach \d in {34.8, 69.6, 104.4, 139.2, 174.0}{
                    \addplot[domain=0:15]{(\d / 17) - (11 / 17) * x};
                  }
                  \fill[ name intersections={of=c1 and c3, by=point1}
                       , name intersections={of=c1 and c5, by=point2}
                       , name intersections={of=c5 and c4, by=point3}
                       , name intersections={of=c4 and c3, by=point4}
                       , name intersections={of=c3 and c1, by=point5}
                       ]
                       [ very thick
                       , draw=orange
                       , pattern=crosshatch dots
                       , pattern color=green!60!white
                       ]
                    (point1)
                    --(point2)
                    --(point3)
                    --(point4)
                    --(point5)
                    --(point1)
                    ;
                  \addplot
                    coordinates { (5, 7) }
                    node[label={[style={font=\tiny}]30:(5, 7)}] {}
                  ;

                \end{axis}
              \end{tikzpicture}

              \captionof{figure}{Objective function: $0.11x_1 + 0.17x_2$}
              \label{fig:c}
            \end{minipage}%
            \begin{minipage}{.5\textwidth}
              \centering

              \begin{tikzpicture}
                \begin{axis}[axis on top,smooth,
                    axis line style=very thick,
                    axis x line=bottom,
                    axis y line=left,
                    ymin=0,ymax=15,
                    xmin=0,xmax=15,
                    xlabel=$x_1$, ylabel=$x_2$,
                    grid=major,
                  ]
                  \addplot[name path global=c1,very thick, domain=0:15]{12 - x}
                    node[above, pos=0.1, sloped, style={font=\tiny}] {$x_1 + x_2 \leq 12$}
                  ;
                  \addplot[name path global=c2,very thick, domain=0:15]
                    coordinates { (10, 0) (10, 15) }
                    node[right, pos=0.7, style={font=\tiny}] {$x_1 \leq 10$}
                  ;
                  \addplot[name path global=c3,very thick, domain=0:15]
                    coordinates { (0, 7) (15, 7) }
                    node[above, pos=0.6, sloped, style={font=\tiny}] {$x_2 \leq 7$}
                  ;
                  \addplot[name path global=c4,very thick, domain=0:15]{2 * x}
                    node[above, pos=0.4, sloped, style={font=\tiny}] {$2x_1 - x_2 \geq 0$}
                  ;
                  \addplot[name path global=c5,very thick, domain=0:15]{0.5 * x}
                    node[below, pos=0.8, sloped, style={font=\tiny}] {$-x_1 + 2x_2 \geq 0$}
                  ;
                  \foreach \d in {2.4, 4.8, 7.2, 9.6, 12.0}{
                    \addplot[domain=0:15]{\d - x};
                  }
                  \fill[ name intersections={of=c1 and c3, by=point1}
                       , name intersections={of=c1 and c5, by=point2}
                       , name intersections={of=c5 and c4, by=point3}
                       , name intersections={of=c4 and c3, by=point4}
                       , name intersections={of=c3 and c1, by=point5}
                       ]
                       [ very thick
                       , draw=orange
                       , pattern=crosshatch dots
                       , pattern color=green!60!white
                       ]
                    (point1)
                    --(point2)
                    --(point3)
                    --(point4)
                    --(point5)
                    --(point1)
                    ;
                  \addplot[lightgray, very thick]
                    coordinates { (5, 7) (8, 4) }
                  ;

                  \addplot[blue, mark=*]
                    coordinates { (5, 7) }
                    node[label={[style={font=\tiny}]30:(5, 7)}] {}
                  ;

                  \addplot[blue, mark=*]
                    coordinates { (8, 4) }
                    node[label={[style={font=\tiny}]90:(8, 4)}] {}
                  ;

                \end{axis}
              \end{tikzpicture}

              \captionof{figure}{Objective function: $x_1 + x_2$}
              \label{fig:d}
            \end{minipage}
          \end{figure}

          The hyperplanes in Figure ~\ref{fig:c} run orthogonal to the vector
          $\begin{bmatrix}11 \\ 17\end{bmatrix}$.

          From this we can find the hyperplane with the largest value.
          The largest value is at the point (5, 7).
          This gives the optimal solution of
          $0.11(5) + 0.17(7) = \$1.74$ million dollars.

        \item
          By making both rates of return equal in Figure ~\ref{fig:d},
          we see that the hyper planes now run parallel to the line $x_1 + x_2 = 12$.

          Since this line is a boundary line for our polygon,
          any solution on this line segment--from (5, 7) to (8, 4)--is an optimal solution.
      \end{enumerate}

    \item [4-2]
      \begin{enumerate}
        \item
          We use one decision variable for each combination of designer and project.
          The variable represents how many hours the designer spends on the said project.

          Say $x_{ij}, i \in \{1, 2, 3\}, j \in \{1, 2, 3, 4\}$

          Now lets look at the rest of the information

          \begin{itemize}
            \item We have that each designer can spend a maximum of 80 hours working.
            \item
              We have the productivity rating for each designer on each project.

              Say $p_{ij}, i \in \{1, 2, 3\}, j \in \{1, 2, 3, 4\}$
            \item
              We have some estimates--that we'll use as upper bounds--for how long each project takes.
              \begin{itemize}
                \item $e_1 = 70$
                \item $e_2 = 50$
                \item $e_3 = 85$
                \item $e_4 = 35$
              \end{itemize}
              So we know that for each project $i$,
              the sum of all the hours each designer spends on project $i$ must equal $e_i$.
          \end{itemize}

          Our objective function is to maximize productivity.
          We can calculate productivity as the sum of all $p_{ij} \cdot x_{ij}$.

          Now we can create a model:

          \begin{alignat*}{3}
            \text{maximize}   \quad \mathrlap{\sum_{i = 1}^3 \sum_{j = 1}^4 p_{ij} x_{ij}}                  \\
            \text{subject to} \quad & \forall i \in \{1, 2, 3\}    \sum_{j = 1}^4 x_{ij} && \leq 80         \\
                                    & \forall j \in \{1, 2, 3, 4\} \sum_{i = 1}^3 x_{ij} && \leq e_j        \\
                                    & \forall j \in \{1, 2, 3, 4\} \sum_{i = 1}^3 x_{ij} && \geq e_j        \\
                                    & x_{ij}                                             && \in  \mathbb{R} \\
          \end{alignat*}

        \pagebreak

        \item

          We can formulate this model in ZIMPL:

          \lstinputlisting{hw1_productivity.zpl}

          And solve it with SCIP:

          \lstinputlisting{hw1_productivity_scip.txt}
      \end{enumerate}
  \end{enumerate}
\end{document}
