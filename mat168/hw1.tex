\documentclass[12pt,letterpaper]{article}
\usepackage{amsmath}
\usepackage{amsfonts}
\usepackage{amsthm}
\usepackage{mathtools}
\usepackage{cancel}
\usepackage[margin=1in]{geometry}
\usepackage{titling}
\usepackage{fp}
\usepackage{enumitem}
\usepackage[super]{nth}
\usepackage{dcolumn}
\usepackage[title]{appendix}
\usepackage{pgfplots}
\usepackage{tikz}
\pgfplotsset{compat=1.8}
\usepgfplotslibrary{statistics}
\usepackage[round-mode=figures,round-precision=3,scientific-notation=false]{siunitx}
\usepackage{color, colortbl}
\usepackage{systeme}
\definecolor{Gray}{gray}{0.8}
\newcolumntype{g}{>{\columncolor{Gray}}c}
\newcolumntype{d}{D{.}{.}{-1}}
\DeclarePairedDelimiter\ceil{\lceil}{\rceil}
\DeclarePairedDelimiter\floor{\lfloor}{\rfloor}

\newcommand*\circled[1]{
  \tikz[baseline=(char.base)]{
    \node[shape=circle,draw,inner sep=2pt] (char) {#1};
  }
}

\makeatletter
\renewcommand*\env@matrix[1][*\c@MaxMatrixCols c]{%
  \hskip -\arraycolsep
  \let\@ifnextchar\new@ifnextchar
  \array{#1}}
\makeatother

\newcommand*\constant[1]{
  Each product has a different #1,
  but these are constant values not constrained by anything else in the process.
}

\newcommand*\genericconstraint[2]{
  Each product has a #1 constraint on #2.
}

\newcommand*\maximumconstraint[1]{
  \genericconstraint{maximum}{#1}
}

\newcommand*\minimumconstraint[1]{
  \genericconstraint{minimum}{#1}
}

\newcommand*\genericobjective[2]{
  The objective is to #1 #2.
}

\newcommand*\maximumobjective[1]{
  \genericobjective{maximum}{#1}
}

\newcommand*\minimumobjective[1]{
  \genericobjective{minimum}{#1}
}

\setlength{\droptitle}{-10ex}

\preauthor{\begin{flushright}\large \lineskip 0.5em}
\postauthor{\par\end{flushright}}
\predate{\begin{flushright}\large}
\postdate{\par\end{flushright}}

\title{MAT 168 HW 1\vspace{-2ex}}
\author{Hardy Jones\\
        999397426\\
        Professor K\"{o}eppe\vspace{-2ex}}
\date{Spring 2015}

\begin{document}
  \maketitle

  \begin{enumerate}
    \item [1.1]

      The first thing we should do is try to understand the domain.
      The steel company wants to know how many hours to allocate to each job.
      Since the description does not mention that each job must be a full hour--%
      or some minimum fraction of an hour--%
      we assume that hours can be real valued.
      Since the hours are what the problem explicitly asks for,
      these can be our decision variables.

      Call these
      \begin{itemize}
        \item $x_1$ := Band
        \item $x_2$ := Coil
      \end{itemize}

      Next we look at the other information in the statement.
      \begin{itemize}
        \item
          \constant{rate}

          We have
          \begin{itemize}
            \item $r_1 = 200 \frac{ton}{hour}$
            \item $r_2 = 140 \frac{ton}{hour}$
          \end{itemize}

        \item
          \constant{profit}

          We have
          \begin{itemize}
            \item $p_1 = 25 \frac{dollar}{ton}$
            \item $p_2 = 30 \frac{dollar}{ton}$
          \end{itemize}
        \item
          \maximumconstraint{the weight that can be produced}
          Since the problem does not state otherwise,
          we assume this can be a real value.
        \item
          \maximumobjective{profit}

          We can find the profit of either product by taking the expression
          \[
            p_i \cdot r_i \cdot x_i, \text{ for } i \in \{1, 2\}
          \]
      \end{itemize}

      With this information, we can now formalize the problem.

      \begin{alignat*}{7}
        \text{maximize}   \quad \rlap{$5000 x_1 + 4500 x_2$}                              \\
        \text{subject to} \quad & 200 & x_1 &       &     &                   && \leq 6000 \\
                                &     &     &       & 140 & x_2               && \leq 4000 \\
                                &     & x_1 & {}+{} &     & x_2               && \leq 40   \\
                                &     &     &       &     & \clap{$x_1, x_2$} && \geq 0    \\
      \end{alignat*}

      Since all of the variables in the problem are positive real values,
      we can use a greedy approach to solve this.

      Let's catalogue the steps.

      \begin{itemize}
        \item
          We have a system of constraints

          \sysdelim..
          \systeme{ 200x_1 \leq 6000
                  , 140x_2 \leq 4000
                  , x_1 + x_2 \leq 40
                  , x_1 \geq 0
                  , x_2 \geq 0
                  }

          Bands are worth more from the profit perspective, so we take as many as possible.
          Using the first constraint, this gives us a direct number for $x_1$.

          Namely $200 x_1 \leq 6000 \implies x_1 \leq 30$.

          This choice does not violate any other constraints,
          so we choose the maximum possible value $x_1$ can take which is 30.

        \item
          Our new system is

          \systeme{ 140x_2 \leq 4000
                  , x_2 \leq 10
                  , x_2 \geq 0
                  }

          So we take the maximum value possible for $x_2$ which is 10.
      \end{itemize}

      The profit can now be computed and we end up with the following result:

      With a choice of 30 hours making Bands and 10 hours making Coils,
      the company can make an optimally maximized profit of $\$192,000$.

    \item [1.2]

    \item [2-2]
      \begin{enumerate}
        \item
        \item
        \item
        \item
      \end{enumerate}

    \item [4-2]
      \begin{enumerate}
        \item
        \item
      \end{enumerate}
  \end{enumerate}
\end{document}
