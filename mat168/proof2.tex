\documentclass[12pt,letterpaper]{article}

\usepackage[margin=1in]{geometry}
\usepackage[round-mode=figures,round-precision=3,scientific-notation=false]{siunitx}
\usepackage[super]{nth}
\usepackage[title]{appendix}
\usepackage{amsfonts}
\usepackage{amsmath}
\usepackage{amsthm}
\usepackage{cancel}
\usepackage{caption}
\usepackage{color, colortbl}
\usepackage{dcolumn}
\usepackage{enumitem}
\usepackage{fp}
\usepackage{float}
\usepackage{listings}
\usepackage{mathtools}
\usepackage{pgfplots}
\usepackage{subcaption}
\usepackage{systeme}
\usepackage{tikz}
\usepackage{titling}

\usepgfplotslibrary{statistics}

\usetikzlibrary{intersections}
\usetikzlibrary{patterns}

\pgfplotsset{compat=1.8}

\definecolor{Gray}{gray}{0.8}
\newcolumntype{g}{>{\columncolor{Gray}}c}
\newcolumntype{d}{D{.}{.}{-1}}
\DeclarePairedDelimiter\ceil{\lceil}{\rceil}
\DeclarePairedDelimiter\floor{\lfloor}{\rfloor}

\newcommand*\circled[1]{
  \tikz[baseline=(char.base)]{
    \node[shape=circle,draw,inner sep=2pt] (char) {#1};
  }
}

\makeatletter
\renewcommand*\env@matrix[1][*\c@MaxMatrixCols c]{%
  \hskip -\arraycolsep
  \let\@ifnextchar\new@ifnextchar
  \array{#1}}
\makeatother

\newcommand*\constant[1]{
  Each product has a different #1,
  but these are constant values not constrained by anything else in the process.
}

\newcommand*\genericconstraint[2]{
  Each product has a #1 constraint on #2.
}

\newcommand*\maximumconstraint[1]{
  \genericconstraint{maximum}{#1}
}

\newcommand*\minimumconstraint[1]{
  \genericconstraint{minimum}{#1}
}

\newcommand*\genericobjective[2]{
  The objective is to #1 #2.
}

\newcommand*\maximumobjective[1]{
  \genericobjective{maximum}{#1}
}

\newcommand*\minimumobjective[1]{
  \genericobjective{minimum}{#1}
}

% Simplex commands

\newcommand*\seeconstraints{
  We look at the constraints and see:
}

\newcommand*\continueopt[1]{
  Now, we can continue optimizing since #1 has a positive coefficient.
}

\newcommand*\enterleave[2]{
  So we can let #1 enter and #2 leave.
}

\newcommand*\morerestrictive[2]{
  The more restrictive constraint is that #1 $\leq$ #2, so set #1 $=$ #2.
}

\newcommand*\newdict{
  So we have a new dictionary:
}

\newcommand*\newvalue[2]{
  This gives a new value for #1. #1 $ = $ #2
}

\newcommand*\opt{
  Since we have no more optimizable variables
  (all variable coefficients of $\zeta$ are non-positive),
  we can no longer maximize $\zeta$.
}

\newcommand*\optsolution[2]{
  Then we have an optimal solution with #1, and value #2.
}

\lstdefinelanguage{zimpl}{
  morekeywords={forall,in,maximize,minimize,param,set,subto,sum,var},
  sensitive=true,
  morecomment=[l]{\#},
  morestring=[b]",
}
\lstset{basicstyle=\scriptsize, frame=single, language=zimpl}

\setlength{\droptitle}{-10ex}

\preauthor{\begin{flushright}\large \lineskip 0.5em}
\postauthor{\par\end{flushright}}
\predate{\begin{flushright}\large}
\postdate{\par\end{flushright}}

\title{MAT 168 Proof Writing 2\vspace{-2ex}}
\author{Hardy Jones\\
        999397426\\
        Professor K\"{o}ppe\vspace{-2ex}}
\date{Spring 2015}

\begin{document}
  \maketitle

  \begin{enumerate}
    \item
      Given the dictionary:

      \begin{alignat*}{9}
        \zeta  & {}={} & 3 & {}+{} &   & x_1 & {}+{} & 2 & x_2 \\
        x_3    & {}={} & 4 & {}-{} &   & x_1 & {}-{} &   & x_2 \\
      \end{alignat*}

      We have non-basic variables $x_1, x_2$ and the basic variable $x_3$.

      We can let $x_1$ enter and $x_3$ leave to get:

      \begin{alignat*}{9}
        \zeta  & {}={} & 7 & {}-{} &   & x_3 & {}+{} &   & x_2 \\
        x_1    & {}={} & 4 & {}-{} &   & x_3 & {}-{} &   & x_2 \\
      \end{alignat*}

      And then let $x_2$ enter and $x_1$ leave to get:

      \begin{alignat*}{9}
        \zeta  & {}={} & 11 & {}-{} & 2 & x_3 & {}-{} &   & x_1 \\
        x_2    & {}={} & 4  & {}-{} &   & x_3 & {}-{} &   & x_1 \\
      \end{alignat*}
    \item
      \begin{proof}
        Given some dictionary:

        \begin{alignat*}{9}
          \zeta  & {}={} & \overline{\zeta} & {}+{} & \sum_{j \in N} \overline{c}_jx_j    &         \\
          x_i    & {}={} & \overline{b}_i   & {}-{} & \sum_{j \in N} \overline{c}_{ij}x_j & \text{ for } i \in B \\
        \end{alignat*}

        If $x_k, k \in B$ is chosen to leave the basis and become non-basic,
        and $x_l, l \in N$ is chosen to enter the basis and become basic.

        Then we know that $\overline{c}_l > 0$.

        We also know that $x_l = \overline{b}_l - \ldots - \overline{c}_{lk}x_k - \ldots$.

        When we go to pivot in the objective function $\zeta$, we substitute this new value for $x_l$.

        We end up with $\zeta = \overline{\zeta} + \ldots + \overline{c}_l\left(\overline{b}_l - \ldots - \overline{c}_{lk}x_k - \ldots\right) + \ldots$.

        After simplification, and letting $\overline{c}_{lk}' = \overline{c}_l\overline{c}_{lk}$, we have

        $\zeta = \overline{\zeta}' + \ldots - \overline{c}_{lk}'x_k + \ldots$

        Since $\overline{c}_{lk}' < 0$, it will not be chosen to become basic in the next iteration.

        Since our choice of entering and leaving variables was arbitrary,
        this holds for any such entering and leaving variables.

        Thus, we have shown that
        if a variable becomes non-basic in one iteration,
        then it cannot become basic in the next iteration.
      \end{proof}
    \item
      We're asked to show that:

      Given a linear program with all right hand sides equal to 0,
      either $x_j = 0$ is optimal for all $j$, or the problem is unbounded.

      We can rephrase this as:

      Given a linear program with all right hand sides equal to 0,
      either it's not the case that $x_j = 0$ is not optimal for all $j$, or the problem is unbounded.

      And this is equivalent to the rephrasing:

      Given a linear program with all right hand sides equal to 0,
      if $x_j = 0$ is not optimal for all $j$, then the problem is unbounded.

      \begin{proof}
        Given some linear program with all right hand sides equal to 0.

        Assume $x_j = 0$ is not optimal for all $j$.
        Then we should be able to increase some $x_j$ and arrive at an optimal solution.

        From the constraints, we see that there must exist some $a_{ij}$ for each $i$ less than or equal to 0.
        Otherwise $\sum\limits_{j = 1}^n a_{ij}x_j \leq 0$ would fail to hold for all $i$.

        If we choose an arbitrary $a_{kl} \leq 0$ and its corresponding $x_l$,
        then we can increase $x_l$ while still keeping the constraint valid.

        If increasing $x_l$ causes some other constraint to be invalidated--%
        say $\sum\limits_{j = 1}^n a_{pj}x_j \leq 0$--%
        then we simply increase the $x_q$ corresponding to the $a_{pq}$ that is less than or equal to 0.
        Repeating this process as necessary.

        What we find is that, increasing any arbitrary $x_l$ does not invalidate any constraints.
        However, the objective value becomes larger.
        Since we can increase any arbitrary $x_l$ to any amount,
        this problem is unbounded.

        Thus, (with some mental reformulation) we have shown that:

        Given a linear program with all right hand sides equal to 0,
        either $x_j = 0$ is optimal for all $j$, or the problem is unbounded.
      \end{proof}
    \item
      \begin{enumerate}
        \item
          We're asked to prove or disprove the feasible region of problem 3.4 is a convex cone.

          The feasible region of 3.4 is the set $\{x_j | x_j \geq 0\}$ for $x_j \in \mathbb{R}^n$.

          \begin{proof}
            Choose some $x, y \in C = \{x_j | x_j \geq 0\}$ for $x_j \in \mathbb{R}^n$.

            Choose some $\lambda, \mu \geq 0 \in \mathbb{R}$.

            Then $\lambda x + \mu y \geq 0 + 0 \geq 0$, so $\lambda x + \mu y \in C$.

            Since our choice of $x, y, \lambda, \mu$ were arbitrary,
            this result holds for any such values.

            So $C$ is a convex cone.

            Thus we have shown that the feasible region of problem 3.4 is a convex cone.
          \end{proof}
        \item
          We're asked to prove or disprove that every convex polyhedron is a convex cone.

          We can represent a convex polyhedron as $\{x \in \mathbb{R}^n | Ax = b\}$,
          where $A \in \mathbb{R}^{m \times n}, b \in \mathbb{R}^m$.

          \begin{proof}
            Let $C = \{x \in \mathbb{R}^n | Ax = b\}$,
            where $A \in \mathbb{R}^{m \times n}, b \in \mathbb{R}^m$, be a convex polyhedron.

            Then choose some arbitrary $x, y \in C$ and $\lambda, \mu > 0 \in \mathbb{R}$.

            Now $A \left(\lambda x + \mu y\right) = A \lambda x + A \mu y = \lambda A x + \mu A y = \lambda b + \mu b = \left(\lambda + \mu \right) b > b$.

            So $\lambda x + \mu y \notin C$.

            Since we chose arbitrary $x, y, \lambda, \mu$, and $\lambda x + \mu y \notin C$,
            $C$ is not a convex cone.

            Then there is at least one convex polyhedron that is not a convex cone.

            Thus, we have disproved that every convex polyhedron is a convex cone.
          \end{proof}
        \item
          We're asked to prove or disprove that every convex cone is convex.

          A set $S \subseteq \mathbb{R}^n$ is convex
          if for any $x, y \in S, \lambda \in \left[0, 1\right]$,
          then $\lambda x + \left(1 - \lambda\right) y \in S$.

          \begin{proof}
            Let $S \subseteq \mathbb{R}^n$ be some convex cone.

            Then choose $\lambda \in \left[0, 1\right]$.

            Let $\mu = 1 - \lambda$, then $\lambda, \mu \geq 0$.

            Now, since $S$ is a convex cone, we can choose $x, y \in S$.
            Then $\lambda x + \mu y \in S$.

            But $\lambda x + \mu y = \lambda x + \left(1 - \lambda\right) y \in S$.

            Since our choice of $x, y, \lambda$ was arbitrary,
            this result holds for all such values.

            So $S$ is convex.

            Now, since our choice of $S$ was arbitrary,
            this result holds for all such convex cones.

            Thus we have shown that all convex cones are convex.
          \end{proof}
      \end{enumerate}
  \end{enumerate}
\end{document}
